\documentclass[14pt]{book}
\usepackage{polyglossia}
\usepackage[normalem]{ulem}
\usepackage[noend,noeledsec,noledgroup]{reledmac}
%\usepackage[margin=1in]{geometry}

%%%
\usepackage{fancyhdr}
\pagestyle{fancy}
\setlength{\headheight}{14.49998pt} % as instructed by log file
% manual p.18
\fancyhf{}%
\renewcommand{\chaptermark}[1]{\markboth{#1}{}}
\renewcommand{\sectionmark}[1]{\markright{#1}}
%
\fancyhead[LE]{\thepage}
\fancyhead[RE]{\nouppercase{\emph{\leftmark}}}
\fancyhead[LO]{\nouppercase{\emph{\rightmark}}}
\fancyhead[RO]{\thepage}
%%%

\arrangementX[A]{paragraph}
\arrangementX[B]{paragraph}
\renewcommand*{\thefootnoteB}{\Roman{footnoteB}}
\arrangementX[C]{paragraph}
\renewcommand*{\thefootnoteC}{\roman{footnoteC}}

\newcommand*{\caesura}{\linebreak}

\Xarrangement[A]{paragraph}
\Xnotenumfont[A]{\bfseries}
\Xlemmafont[A]{\bfseries}

\setdefaultlanguage{sanskrit}
\setotherlanguage{english}
\setmainfont{Brill}
\newfontfamily\devanagarifont{Brill}
\newfontfamily{\devafont}{Pedantic Devanagari}

\usepackage[Devanagari,DevanagariExtended]{ucharclasses}

\makeatletter
\setTransitionsFor{Devanagari}%
 {\let\curfamily\f@family\let\curshape\f@shape\let\curseries\f@series\devafont}
 {\fontfamily{\curfamily}\fontshape{\curshape}\fontseries{\curseries}\selectfont}
\makeatother

\makeatletter
\setTransitionsFor{DevanagariExtended}%
 {\let\curfamily\f@family\let\curshape\f@shape\let\curseries\f@series\devafont}
 {\fontfamily{\curfamily}\fontshape{\curshape}\fontseries{\curseries}\selectfont}
\makeatother

\setcounter{secnumdepth}{-1}
\setcounter{page}{139}

\begin{document}
    \raggedright
\input{sanskrit-hyphenations}
\large

\lineation{page}
\begingroup
\beginnumbering

\chapter{Kalpasthāna 1: Protecting the King from Poison}
\pstart
                         \textsc{[1938 ed. 5.1.1]}
                        \caesura athāto 'nnapānarakṣākalpaṃ vyākhyāsyāmaḥ\edlabel{SS.5.1.1-3} \edtext{||}{
  \linenum{|\xlineref{SS.5.1.1-3}}\lemma{vyākhyāsyāmaḥ ||}\Afootnote{vyākhyāsyāmaḥ || K.}
}
\pend

 
\pstart

                         \textsc{[1938 ed. 5.1.2]}
                        \caesura yathovāca bhagavān dhanvantariḥ \edtext{||
}{
  \Afootnote{\textsc{[add]} \textbf{atha khalu vatsasuśrutaḥ | 1} H.}
}
\pend

 
\pstart

                         \textsc{[1938 ed. 5.1.3]}
                        \caesura divodāsakṣitipatis
 \edtext{tapodharmabhṛtāṃ}{
  \Afootnote{dhanvantariḥ kāśipatis A.}
} varaḥ | \caesura suśrutapramukhāñ \edtext{chiṣyāñ}{
  \Afootnote{\uwave{suśru}tapramu\uwave{khāñ} K; °prabhṛtīñ A.}
} \edtext{śaśāsāhata}{
  \Afootnote{chi\textsc{(l. 2)}ṣyāṃ K; chiṣyān H.}
} \edtext{śāsanaḥ}{
  \Afootnote{chaśā° A; \textsc{[add]} \uuline{sātanaḥ} H.}
} ||\edlabel{SS.5.1.3-8} 
\pend

 
\pstart

                         \textsc{[1938 ed. 5.1.4]}
                        \caesura ripavo \edtext{vikramākrāntāḥ}{
  \Afootnote{°krāntā A.}
} \edtext{sve}{
  \Afootnote{ye A.}
} \edtext{vā}{
  \Afootnote{ca A.}
} \edtext{syuḥ}{
  \Afootnote{sve A.}
} kṛtyatāṅ gatāḥ | \caesura sisṛkṣavaḥ krodhaviṣaṃ vivaraṃ prāpya tādṛśam || 
\pend

 
\pstart

                         \textsc{[1938 ed. 5.1.5]}
                        \caesura viṣair \edtext{hiṃsyur}{
  \Afootnote{nihanyur A.}
} \edtext{akiñcijjñaṃ}{
  \Afootnote{nipuṇaṃ A; akiñcijñaṃ Nep.}
} nṛpatiṃ duṣṭacetasaḥ\edlabel{SS.5.1.5ab-5} \edtext{|}{
  \linenum{|\xlineref{SS.5.1.5ab-5}}\lemma{duṣṭacetasaḥ |}\Afootnote{duṣṭavetasaḥ | H.}
}
\pend

 
\pstart
\edtext{}{
  \Afootnote{\textsc{[pre]} striyo vā vividhān yogān kadācit subhagecchayā || A.}
}
\pend

 
\pstart
\edtext{}{
  \Afootnote{\textsc{[pre]} viṣakanyopayogād vā kṣaṇāj jahyād asūn naraḥ | A.}
}
\pend

 
\pstart

                         \textsc{[1938 ed. 5.1.6]}
                        \caesura tasmād vaidyena satataṃ viṣād \edtext{rakṣyo}{
  \Afootnote{rakṣen H.}
} narādhipaḥ || 
\pend

 
\pstart

                         \textsc{[1938 ed. 5.1.7]}
                        \caesura \edtext{yasmāc}{
  \Afootnote{\textsc{[add]} ca A.}
} \edtext{cānityacittatvam}{
  \Afootnote{ceto 'nityatvam A; °ci\uuline{nna} \textbf{tta}tvam H.}
} aśvavat prathitaṃ nṛṣu\edlabel{SS.5.1.7-5} | \caesura \edtext{tasmān}{
  \linenum{|\xlineref{SS.5.1.7-5}}\lemma{nṛṣu\ldots tasmān}\Afootnote{nṛṇām |  A.}
} na \edtext{viśvased}{
  \Afootnote{viśvasyāt tato A.}
} rājā kadācid api kasya cit || 
\pend

 
\pstart

                         \textsc{[1938 ed. 5.1.8]}
                        \caesura kulīnaṃ dhārmikaṃ snigdham \edtext{akṛśaṃ}{
  \Afootnote{subhṛtaṃ A.}
} satatotthitam\edlabel{SS.5.1.8ab-5} \edtext{|}{
  \linenum{|\xlineref{SS.5.1.8ab-5}}\lemma{satatotthitam |}\Afootnote{saṃta° A.}
}
\pend

 
\pstart
\edtext{}{
  \Afootnote{\textsc{[pre]} alubdham aśaṭhaṃ bhaktaṃ kṛtajñaṃ priyadarśanam || A.}
}
\pend

 
\pstart
\edtext{}{
  \Afootnote{\textsc{[pre]} krodhapāruṣyamātsaryamāyālasyavivarjitam |  jitendriyaṃ kṣamāvantaṃ śuciṃ śīladayānvitam || A.}
}
\pend

 
\pstart
\edtext{}{
  \Afootnote{\textsc{[pre]} medhāvinamasaṃśrāntamanuraktaṃ hitaiṣiṇam |  paṭuṃ pragalbhaṃ nipuṇaṃ dakṣamālasyavarjitam || A.}
}
\pend

 
\pstart
\edtext{}{
  \Afootnote{\textsc{[pre]} pūrvoktaiś ca guṇair yuktaṃ nityaṃ sannihitāgadam | A.}
}
\pend

 
\pstart

                         \textsc{[1938 ed. 5.1.11]}
                        \caesura mahānase \edtext{niyuñjīta}{
  \Afootnote{prayuñ° A.}
} \edtext{vaidyan}{
  \Afootnote{vaidyaṃ A.}
} tadvidyapūjitam || 
\pend

 
\pstart

                         \textsc{[1938 ed. 5.1.12]}
                        \caesura praśastadigdeśakṛtaṃ \edtext{śucibhāṇḍam}{
  \Afootnote{sūci° H; śucibhāṇḍāṃ K.}
} mahacchuciḥ\edlabel{SS.5.1.12ab-3} \edtext{|}{
  \linenum{|\xlineref{SS.5.1.12ab-3}}\lemma{mahacchuciḥ |}\Afootnote{mahacchuci | A.}
}
\pend

 
\pstart
\edtext{}{
  \Afootnote{\textsc{[pre]} sajālakaṃ gavākṣāḍhyam āptavarganiṣevitam || A.}
}
\pend

 
\pstart
\edtext{}{
  \Afootnote{\textsc{[pre]} vikakṣasṛṣṭasaṃsṛṣṭaṃ savitānaṃ kṛtārcanam | A.}
}
\pend

 
\pstart

                         \textsc{[1938 ed. 5.1.13]}
                        \caesura parīkṣitastrīpuruṣaṃ bhavec cāpi mahānasam || 
\pend

 
\pstart
\edtext{}{
  \Afootnote{\textsc{[pre]} tatrādhyakṣaṃ niyuñjīta prāyo vaidyaguṇānvitam |  śucayo dakṣiṇā dakṣā vinītāḥ priyadarśanāḥ || A.}
}
\pend

 
\pstart
\edtext{}{
  \Afootnote{\textsc{[pre]} saṃvibhaktāḥ sumanaso nīcakeśanakhāḥ sthirāḥ |  snātā dṛḍhaṃ saṃyaminaḥ kṛtoṣṇīṣāḥ susaṃyatāḥ || A.}
}
\pend

 
\pstart
\edtext{}{
  \Afootnote{\textsc{[pre]} tasya cājñāvidheyāḥ syur vividhāḥ parikarmiṇaḥ |  āhārasthitayaś cāpi bhavanti prāṇino yataḥ || A.}
}
\pend

 
\pstart
\edtext{}{
  \Afootnote{\textsc{[pre]} tasmān mahānase vaidyaḥ pramādarahito bhavet | A.}
}
\pend

 
\pstart

                         \textsc{[1938 ed. 5.1.17]}
                        \caesura \edtext{mahānasikavoḍhāraḥ}{
  \Afootnote{māhā° A; °voḍhḍhāraḥ K.}
} \edtext{saupodanika}{
  \Afootnote{sūpo° H; saupauda° A.}
} pūpikāḥ\edlabel{SS.5.1.17cd-3} \edtext{||}{
  \linenum{|\xlineref{SS.5.1.17cd-3}}\lemma{pūpikāḥ ||}\Afootnote{paupikāḥ || A.}
}
\pend

 
\pstart

                         \textsc{[1938 ed. 5.1.18]}
                        \caesura bhaveyur vaidyavaśagā \edtext{ye}{
  \Afootnote{\uuline{}ye H.}
} cāpy anye 'tra kecana | \caesura iṅgitajño manuṣyāṇāṃ vākceṣṭāmukhavaikṛtaiḥ || 
\pend

 
\pstart

                         \textsc{[1938 ed. 5.1.19]}
                        \caesura \edtext{jānīyād}{
  \Afootnote{vidyād A.}
} \edtext{viṣadātāram}{
  \Afootnote{viṣasya dā° A.}
} ebhir liṅgaiś ca buddhimān | \caesura \edtext{vivakṣur}{
  \Afootnote{na A; vicakṣur H.}
} \edtext{muhyate}{
  \Afootnote{dadāty uttaraṃ A.}
} pṛṣṭo \edtext{nottaraṃ}{
  \Afootnote{vivakṣan moham A.}
} pratipadyate\edlabel{SS.5.1.19-12} \edtext{||}{
  \linenum{|\xlineref{SS.5.1.19-12}}\lemma{pratipadyate ||}\Afootnote{eti A.}
\lemma{||}  \Afootnote{\textsc{[add]} ca || A.}
}
\pend

 
\pstart

                         \textsc{[1938 ed. 5.1.20]}
                        \caesura apārthaṃ bahusaṅkīrṇaṃ bhāṣate cāpi mūḍhavat | \caesura \edtext{hasaty}{
  \Afootnote{sphoṭayaty A.}
} \edtext{akasmād}{
  \Afootnote{\textsc{[om]} A.}
} \edtext{aṅgulīḥ}{
  \Afootnote{\textsc{[add]} bhūmim A.}
} \edtext{sphoṭayed}{
  \Afootnote{akasmād A; sphoṭāyed K.}
} \edtext{vilikhed}{
  \Afootnote{vi\uwave{li}khe\textsc{(l. 5)}d K; vilikhen H.}
} mahīm\edlabel{SS.5.1.20-12} \edtext{||}{
  \linenum{|\xlineref{SS.5.1.20-12}}\lemma{mahīm ||}\Afootnote{dhaset || A.}
}
\pend

 
\pstart

                         \textsc{[1938 ed. 5.1.21]}
                        \caesura vepathuś \edtext{cāsya}{
  \Afootnote{jāyate A.}
} \edtext{bhavati}{
  \Afootnote{tasya A.}
} trastaś cānyo 'nyam īkṣate\edlabel{SS.5.1.21-7} \edtext{| \caesura}{
  \linenum{|\xlineref{SS.5.1.21-7}}\lemma{īkṣate | }\Afootnote{īkṣyate |  H.}
\lemma{| }  \Afootnote{\textsc{[add]} kṣāmo A.}
} vivarṇavaktro \edtext{dhyāmaś}{
  \Afootnote{\textsc{[om]} A.}
} ca nakhaiḥ kiñcic\edlabel{SS.5.1.21-13} \edtext{chinaty}{
  \linenum{|\xlineref{SS.5.1.21-13}}\lemma{kiñcic chinaty}\Afootnote{°natty A.}
} api || 
\pend

 
\pstart
\edtext{}{
  \Afootnote{\textsc{[pre]} ālabhetāsakṛddīnaḥ kareṇa ca śiroruhān |  niryiyāsurapadvārair vīkṣate ca punaḥ punaḥ || A.}
}
\pend

 
\pstart

                         \textsc{[1938 ed. 5.1.23]}
                        \caesura vartate \edtext{viparītaś}{
  \Afootnote{viparītaṃ A; viparītañ H.}
} \edtext{ca}{
  \Afootnote{tu A.}
} viṣadātā vicetanaḥ \edtext{|}{
  \Afootnote{\textsc{[add]} kecid bhayāt pārthivasya tvaritā vā tadājñayā || A.}
}
\pend

 
\pstart
\edtext{}{
  \Afootnote{\textsc{[pre]} asatām api santo 'pi ceṣṭāṃ kurvanti mānavāḥ |  tasmāt parīkṣaṇaṃ kāryaṃ bhṛtyānām ādṛtair nṛpaiḥ || A.}
}
\pend

 
\pstart
\edtext{
                         \textsc{[1938 ed. 5.1.25]}
                        \caesura}{
  \Afootnote{\textsc{[pre]} anne pāne A.}
} dantakāṣṭhe 'nnapāne\edlabel{SS.5.1.25-2} \edtext{ca}{
  \linenum{|\xlineref{SS.5.1.25-2}}\lemma{'nnapāne ca}\Afootnote{\textsc{[om]} A.}
} \edtext{tathābhyaṅge}{
  \Afootnote{tathā 'bhyaṅge A.}
} 'valekhane | \caesura \edtext{utsādane}{
  \Afootnote{\textsc{[add]} kaṣāye ca A.}
} \edtext{parīṣeke}{
  \Afootnote{pariṣeke A.}
} \edtext{kaṣāye}{
  \Afootnote{\textsc{[om]} A; kaṣāyaiḥ H.}
} sānulepane\edlabel{SS.5.1.25-10} \edtext{||}{
  \linenum{|\xlineref{SS.5.1.25-10}}\lemma{sānulepane ||}\Afootnote{'nulepane || A.}
}
\pend

 
\pstart

                         \textsc{[1938 ed. 5.1.26]}
                        \caesura srakṣu vastreṣu śayyāsu kavacābharaṇeṣu ca | \caesura pādukāpādapīṭheṣu pṛṣṭheṣu gajavājinām || 
\pend

 
\pstart

                         \textsc{[1938 ed. 5.1.27]}
                        \caesura \edtext{viṣajuṣṭeṣu}{
  \Afootnote{viṣaduṣṭeṣu H.}
} cānyeṣu nasyadhūmāñjanādiṣu\edlabel{SS.5.1.27-3} \edtext{| \caesura}{
  \linenum{|\xlineref{SS.5.1.27-3}}\lemma{nasyadhūmāñjanādiṣu | }\Afootnote{dhūmanasyāñja° Nep.}
} lakṣaṇāni pravakṣyāmi cikitsāñ \edtext{cāpy}{
  \Afootnote{apy A.}
} anantaram || 
\pend

 
\pstart

                         \textsc{[1938 ed. 5.1.28]}
                        \caesura nṛpabhaktād \edtext{balin}{
  \Afootnote{baliṃ A; valiṃn H.}
} \edtext{dattaṃ}{
  \Afootnote{nyastaṃ A.}
} saviṣaṃ bhakṣayanti ye | \caesura tatraiva te vinaśyanti makṣikāvāyasādayaḥ | 
\pend

 
\pstart

                         \textsc{[1938 ed. 5.1.29]}
                        \caesura hutabhuk tena \edtext{cānnena}{
  \Afootnote{(From 143v)\textsc{(l. 1)}cānne\textbf{na} K.}
} bhṛśañ caṭacaṭāyate | \caesura mayūrakaṇṭhapratimo jāyate cāpi duḥsahaḥ | 
\pend

 
\pstart
\edtext{}{
  \Afootnote{\textsc{[pre]} bhinnārcis tīkṣṇadhūmaś ca na cirāc copaśāmyati | A.}
}
\pend

 
\pstart

                         \textsc{[1938 ed. 1.30cd]}
                        \caesura cakorasyākṣivairāgyaṃ jāyate kṣipram eva tu | 
\pend

 
\pstart

                         \textsc{[1938 ed. 5.1.31]}
                        \caesura \edtext{dṛṣṭvānnaṃ}{
  \Afootnote{dṛṣṭvā 'nnaṃ A.}
} viṣasaṃsṛṣṭaṃ \edtext{mriyate}{
  \Afootnote{mriyante A.}
} jīvajīvakaḥ\edlabel{SS.5.1.31-4} \edtext{| \caesura}{
  \linenum{|\xlineref{SS.5.1.31-4}}\lemma{jīvajīvakaḥ | }\Afootnote{\uuline{jīva}jī° H; °vakāḥ |  A.}
} kokilaḥ svaravaikṛtyaṃ \edtext{kroñcas}{
  \Afootnote{krauñcas A H.}
} tu madam arcchati\edlabel{SS.5.1.31-11} \edtext{|}{
  \linenum{|\xlineref{SS.5.1.31-11}}\lemma{arcchati |}\Afootnote{ṛcchati || A.}
}
\pend

 
\pstart

                         \textsc{[1938 ed. 5.1.32]}
                        \caesura hṛṣyen \edtext{mayūras}{
  \Afootnote{mayūra A.}
} \edtext{tūdvigne}{
  \Afootnote{udvignaḥ A.}
} \edtext{krośete}{
  \Afootnote{krośataḥ A; \uwave{krośete} K; krośaite H.}
} śukasārike | \caesura \edtext{haṃsaḥ 
                            kṣvelati
                            kṣveḍati
                        }{
  \Afootnote{haṃsa K.}
} \edtext{cātyarthaṃ}{
  \Afootnote{kṣveḍati A; khelati H.}
} kūjate \edtext{bhṛṅgarājakaḥ}{
  \Afootnote{\textsc{[om]} A; kū\textsc{(gap of 1, insertion in the line above)}jate K.}
} |\edlabel{SS.5.1.32-11} 
\pend

 
\pstart
 \edtext{\textsc{[1938 ed. 5.1.33]}
                        \caesura                        
                            vṛṣabho
                            pṛṣato
                        }{
  \Afootnote{pṛṣato A; \uwave{pṛ}ṣabho K.}
} visṛjaty \edtext{asraṃ}{
  \Afootnote{aśruṃ A.}
} \edtext{muñcate}{
  \Afootnote{\textsc{[om]} A.}
} \edtext{viṭ}{
  \Afootnote{viṣṭhāṃ A.}
} \edtext{ca}{
  \Afootnote{muñcati A.}
} markaṭaḥ | 
\pend

 
\pstart
\edtext{}{
  \Afootnote{\textsc{[pre]} sannikṛṣṭāṃs tataḥ kuryād rājñas tān mṛgapakṣiṇaḥ || A.}
}
\pend

 
\pstart
\edtext{}{
  \Afootnote{\textsc{[pre]} veśmano 'tha vibhūṣārthaṃ rakṣārthaṃ cātmanaḥ sadā | A.}
}
\pend

 
\pstart

                         \textsc{[1938 ed. 5.1.34cd]}
                        \caesura upakṣiptasya cānnasya bāṣpeṇordhvam udīyatā\edlabel{SS.5.1.34cd-4} \edtext{|}{
  \linenum{|\xlineref{SS.5.1.34cd-4}}\lemma{udīyatā |}\Afootnote{prasarpatā || A; udīryatāṃ | H.}
}
\pend

 
\pstart

                         \textsc{[1938 ed. 5.1.35]}
                        \caesura hṛtpīḍā bhrāntanetratvaṃ śiroduḥkhañ ca jāyate | \caesura tatra \edtext{nasyāñjane}{
  \Afootnote{tasyāñ° H.}
} kuṣṭhaṃ lāmajjaṃ naladaṃ madhuḥ\edlabel{SS.5.1.35-12} \edtext{|}{
  \linenum{|\xlineref{SS.5.1.35-12}}\lemma{madhuḥ |}\Afootnote{madhu || A; madhū\uuline{ḥ}| H.}
}
\pend

 
\pstart
\edtext{}{
  \Afootnote{\textsc{[pre]} kuryāc chirīṣarajanīcandanaiś ca pralepanam | A.}
}
\pend

 
\pstart

                         \textsc{[1938 ed. 5.1.36cd]}
                        \caesura hṛdi candanalepaś \edtext{ca}{
  \Afootnote{tu A.}
} tathā sukham avāpnuyāt | 
\pend

 
\pstart

                         \textsc{[1938 ed. 5.1.37]}
                        \caesura pāṇiprāptaṃ pāṇidāhaṃ nakhaśātaṅ karoti ca\edlabel{SS.5.1.37-5} \edtext{| \caesura}{
  \linenum{|\xlineref{SS.5.1.37-5}}\lemma{ca | }\Afootnote{\textsc{[om]} A.}
} \edtext{tatra}{
  \Afootnote{ca |  atra A.}
} pralepaḥ śyāmendra \edtext{gopa}{
  \Afootnote{gopā A.}
} somotpalāni ca | 
\pend

 
\pstart

                         \textsc{[1938 ed. 5.1.38]}
                        \caesura sa cet pramādān mohād vā tad \edtext{bhuṃkte}{
  \Afootnote{\textsc{[om]} A.}
} \edtext{bhojanaṃ}{
  \Afootnote{annam A.}
} yadi\edlabel{SS.5.1.38-9} \edtext{| \caesura}{
  \linenum{|\xlineref{SS.5.1.38-9}}\lemma{yadi | }\Afootnote{upasevate |  A.}
} \edtext{tato}{
  \Afootnote{\textsc{[om]} A.}
} \edtext{'syāṣṭhīlavaj}{
  \Afootnote{aṣṭhīlāvat tato A.}
} jihvā \edtext{jāyate}{
  \Afootnote{bhavaty A.}
} 'rasavedanī\edlabel{SS.5.1.38-15} \edtext{|}{
  \linenum{|\xlineref{SS.5.1.38-15}}\lemma{'rasavedanī |}\Afootnote{arasavedinī || A.}
}
\pend

 
\pstart

                         \textsc{[1938 ed. 5.1.39]}
                        \caesura tudyate dahyate cāpi śleṣmā \edtext{cāsya}{
  \Afootnote{cāsyāt A.}
} prasicyate | \caesura tatra bāṣperitaṃ karma yac \edtext{ca}{
  \Afootnote{\textbf{ca} \uuline{cā}\textsc{(l. 3)} H.}
} syād dantakāṣṭhikam\edlabel{SS.5.1.39-14} \edtext{|}{
  \linenum{|\xlineref{SS.5.1.39-14}}\lemma{dantakāṣṭhikam |}\Afootnote{dānta° A.}
}
\pend

 
\pstart

                         \textsc{[1938 ed. 5.1.40]}
                        \caesura \edtext{mūrchāṃ}{
  \Afootnote{mūrcchā K.}
} chardiṃ \edtext{romaharṣam}{
  \Afootnote{atīsāram A.}
} ādhmānaṃ dāham\edlabel{SS.5.1.40-5} eva ca \edtext{| \caesura}{
  \linenum{|\xlineref{SS.5.1.40-5}}\lemma{dāham\ldots | }\Afootnote{dāhavepathū |  A.}
} indriyāṇāñ ca \edtext{vaikṛtyaṃ}{
  \Afootnote{\textsc{[add]} \textbf{dravadravye śaye gataṃ} H.}
} \edtext{kuryād}{
  \Afootnote{|3ku° H.}
} āmāśayaṃ gatam | 
\pend

 
\pstart

                         \textsc{[1938 ed. 5.1.41]}
                        \caesura tatrāśu madanālābubimbīkośātakīphalaiḥ | \caesura chardanaṃ \edtext{dadhyudaśvidbhyām}{
  \Afootnote{°śvibhyām Nep.}
} athavā taṇḍulāmbunā\edlabel{SS.5.1.41-7} \edtext{|}{
  \linenum{|\xlineref{SS.5.1.41-7}}\lemma{taṇḍulāmbunā |}\Afootnote{tāṇḍu° K.}
}
\pend

 
\pstart

                         \textsc{[1938 ed. 5.1.42]}
                        \caesura dāhaṃ mūrcchām atīsāraṃ tṛṣṇām indriyavaikṛtam | \caesura āṭopaṃ pāṇḍutāṃ kārśyaṃ kuryāt pakvāśayaṃ \edtext{gataṃ}{
  \Afootnote{gata\uwave{ṃ} K.}
}
\pend

 
\pstart

                         \textsc{[1938 ed. 5.1.43]}
                        \caesura \edtext{tatra}{
  \Afootnote{\textsc{[om]} A.}
} nīlīphalaṃ\edlabel{SS.5.1.43-2} \edtext{śreṣṭhaṃ}{
  \linenum{|\xlineref{SS.5.1.43-2}}\lemma{nīlīphalaṃ śreṣṭhaṃ}\Afootnote{virecanaṃ A.}
} \edtext{sasarpiṣkaṃ}{
  \Afootnote{\textsc{[add]} tatroktaṃ A.}
} virecanam\edlabel{SS.5.1.43-5} \edtext{| \caesura}{
  \linenum{|\xlineref{SS.5.1.43-5}}\lemma{virecanam | }\Afootnote{nīlinīphalam |  A.}
} dadhnā dūṣīviṣāriś ca \edtext{peyo}{
  \Afootnote{\textsc{[add]} vā A.}
} madhusamāyutaḥ\edlabel{SS.5.1.43-11} \edtext{|}{
  \linenum{|\xlineref{SS.5.1.43-11}}\lemma{madhusamāyutaḥ |}\Afootnote{madhusaṃyutaḥ || A.}
}
\pend

 
\pstart

                         \textsc{[1938 ed. 5.1.44]}
                        \caesura dravadravyeṣu sarveṣu kṣīramadyodakādiṣu | \caesura bhavanti vividhā rājyaḥ phenabudbudajanma ca | 
\pend

 
\pstart

                         \textsc{[1938 ed. 5.1.45]}
                        \caesura \edtext{chāyāś}{
  \Afootnote{chāyā Nep.}
} cātra na dṛśyante dṛśyante yadi vā punaḥ | \caesura bhavanti \edtext{vikṛtāś}{
  \Afootnote{yamalāś A; vikṛtā Nep; \textsc{[add]} \textbf{yamalā} H.}
} chidrās tanvyo vā vikṛtās tathā | 
\pend

 
\pstart

                         \textsc{[1938 ed. 1.46]}
                        \caesura \edtext{śākasūpānna}{
  \Afootnote{°pāṃṇa K.}
} māṃsāni klinnāni virasāni ca | \caesura sadyaḥ \edtext{paryuṣitānīva}{
  \Afootnote{paryuśitā° K.}
} \edtext{vigandhīni}{
  \Afootnote{vigandhāni A.}
} bhavanti ca | 
\pend

 
\pstart

                         \textsc{[1938 ed. 5.1.47]}
                        \caesura gandhavarṇṇarasair hīnāḥ sarve bhakṣyāḥ phalāni ca | \caesura pakvāny \edtext{āśu}{
  \Afootnote{āśuḥ K; āśu\uwave{ḥ} H.}
} \edtext{prakuthyante}{
  \Afootnote{viśīryante A.}
} pākam āmāni yānti ca | 
\pend

 
\pstart

                         \textsc{[1938 ed. 5.1.48]}
                        \caesura \edtext{viśīryante}{
  \Afootnote{viśīryate A.}
} \edtext{kūrcakas}{
  \Afootnote{\uwave{kū}rcaka\textsc{(l. 2)}s H.}
} tu dantakāṣṭhagate viṣe | \caesura \edtext{jihvādantauṣṭhamāṃseṣu}{
  \Afootnote{jihvādantoṣṭhamāṃ° H; °māṃsānāṃ A; °māṃ\uwave{seṣu} K.}
} śvayathuś copajāyate | 
\pend

 
\pstart

                         \textsc{[1938 ed. 5.1.49]}
                        \caesura athāsya dhātakīpuṣpa jambvāmrāsthi\edlabel{SS.5.1.49-3} harītakaiḥ\edlabel{SS.5.1.49-4} \edtext{| \caesura}{
  \linenum{|\xlineref{SS.5.1.49-3}}\lemma{jambvāmrāsthi\ldots | }\Afootnote{pathyājambūphalāsthibhiḥ |  A.}
  \linenum{|\xlineref{SS.5.1.49-4}}\lemma{harītakaiḥ | }\Afootnote{harīkakaiḥ |  K.}
} sakṣaudraiḥ pracchite śophe kartavyaṃ pratisāraṇam | 
\pend

 
\pstart

                         \textsc{[1938 ed. 5.1.50]}
                        \caesura \edtext{athavāṅkollamūlāni}{
  \Afootnote{athāvāṅkoṭhamū° A.}
} tvacaḥ saptachadasya vā | \caesura \edtext{śirīṣamāṣakā}{
  \Afootnote{°kād Nep.}
} \edtext{vāpi}{
  \Afootnote{vā 'pi A.}
} \edtext{kartavyaṃ}{
  \Afootnote{sakṣaudrāḥ A.}
} pratisāraṇam | 
\pend

 
\pstart

                         \textsc{[1938 ed. 5.1.51]}
                        \caesura jihvānirlekhakavalau dantakāṣṭhavad ādiśet | \caesura \edtext{picchalo}{
  \Afootnote{picchilo A H.}
} \edtext{bahalo}{
  \Afootnote{bahulo A.}
} 'bhyaṅgo vivarṇaś \edtext{ca}{
  \Afootnote{vā A.}
} viṣānvitaḥ | 
\pend

 
\pstart

                         \textsc{[1938 ed. 5.1.52]}
                        \caesura \edtext{sphoṭā}{
  \Afootnote{sphoṭa A.}
} \edtext{janmarujāsrāvas\edlabel{SS.5.1.52-2}}{
  \Afootnote{°srāvās H.}
} \edtext{tvakpākaḥ}{
  \linenum{|\xlineref{SS.5.1.52-2}}\lemma{janmarujāsrāvas tvakpākaḥ}\Afootnote{°srāvatvakpākaḥ A.}
} \edtext{sveda}{
  \Afootnote{svedanaṃ A; sveda\uwave{m} K.}
} \edtext{eva\edlabel{SS.5.1.52-5}}{
  \Afootnote{eva K; \uwave{e}va H.}
} ca \edtext{| \caesura}{
  \linenum{|\xlineref{SS.5.1.52-5}}\lemma{eva\ldots | }\Afootnote{jvaraḥ |  A.}
} \edtext{dāraṇañ}{
  \Afootnote{daraṇaṃ A; dāruṇañ H.}
} cāpi māṃsānām abhyaṅge viṣadūṣite\edlabel{SS.5.1.52-12} \edtext{|}{
  \linenum{|\xlineref{SS.5.1.52-12}}\lemma{viṣadūṣite |}\Afootnote{viṣasaṃyute || A.}
}
\pend

 
\pstart

                         \textsc{[1938 ed. 5.1.53]}
                        \caesura tatra śītāmbusiktasya kartavyam anulepanam | \caesura \edtext{candanan}{
  \Afootnote{candanaṃ A.}
} tagaraṃ kuṣṭham uśīraṃ veṇupatrikā | 
\pend

 
\pstart

                         \textsc{[1938 ed. 5.1.54]}
                        \caesura \edtext{somavalyamṛtā}{
  \Afootnote{somavallyamṛtā A.}
} śvetā padmaṃ kālīyakan tathā\edlabel{SS.5.1.54-5} \edtext{| \caesura}{
  \linenum{|\xlineref{SS.5.1.54-5}}\lemma{tathā | }\Afootnote{tvacam |  A.}
} \edtext{kapittharasapatrābhyāṃ}{
  \Afootnote{°samūtrābhyāṃ A.}
} \edtext{pānam}{
  \Afootnote{pāṇa\textbf{m} K.}
} \edtext{etac}{
  \Afootnote{etac K.}
} ca pūjyate\edlabel{SS.5.1.54-11} \edtext{|}{
  \linenum{|\xlineref{SS.5.1.54-11}}\lemma{pūjyate |}\Afootnote{yujyate || A.}
}
\pend

 
\pstart

                         \textsc{[1938 ed. 5.1.55]}
                        \caesura utsādane parīṣeke kaṣāye sānulepane\edlabel{SS.5.1.55-4} \edtext{| \caesura}{
  \linenum{|\xlineref{SS.5.1.55-4}}\lemma{sānulepane | }\Afootnote{cānu° A.}
} \edtext{śayyāvastratanutreṣu}{
  \Afootnote{°nutveṣu H.}
} \edtext{vidyād}{
  \Afootnote{jñeyam A.}
} abhyaṅgavad\edlabel{SS.5.1.55-8} bhiṣak \edtext{|}{
  \linenum{|\xlineref{SS.5.1.55-8}}\lemma{abhyaṅgavad\ldots |}\Afootnote{abhyaṅgalakṣaṇaiḥ || A.}
}
\pend

 
\pstart

                         \textsc{[1938 ed. 5.1.56]}
                        \caesura keśaśātaḥ śiroduḥkhaṃ khebhyaś ca rudhirāgamaḥ | \caesura \edtext{granthijanmottamāṅge}{
  \Afootnote{°geṣu A.}
} \edtext{ca}{
  \Afootnote{\textsc{[om]} A.}
} viṣajuṣṭe 'valekhane | 
\pend

 
\pstart

                         \textsc{[1938 ed. 5.1.57]}
                        \caesura \edtext{tatra}{
  \Afootnote{\textsc{[om]} A.}
} pralepo \edtext{bahuśo}{
  \Afootnote{\textsc{[add]} tatra A.}
} \edtext{bhāvitā}{
  \Afootnote{bhavitāḥ A.}
} kṛṣṇamṛttikā\edlabel{SS.5.1.57-5} \edtext{| \caesura}{
  \linenum{|\xlineref{SS.5.1.57-5}}\lemma{kṛṣṇamṛttikā | }\Afootnote{°ttikāḥ |  A.}
} \edtext{ṛkṣapittaghṛta}{
  \Afootnote{ṛṣyapi° A; ṛkṣapi\uwave{tta}ghṛ\textsc{(l. 4)}ta K.}
} śyāmāpālindī taṇḍulīyakaiḥ | 
\pend

 
\pstart

                         \textsc{[1938 ed. 5.1.58]}
                        \caesura gomayasvaraso \edtext{vāpi}{
  \Afootnote{vā 'pi A.}
} hito vā mālatīrasaḥ | \caesura raso \edtext{mūṣikakarṇyā}{
  \Afootnote{mūṣikaparṇyā A.}
} vā dhūmo vāgārasaṃjñitaḥ\edlabel{SS.5.1.58-11} \edtext{||}{
  \linenum{|\xlineref{SS.5.1.58-11}}\lemma{vāgārasaṃjñitaḥ ||}\Afootnote{vā 'gārasaṃbhavaḥ || A.}
}
\pend

 
\pstart

                         \textsc{[1938 ed. 5.1.59]}
                        \caesura śiro \edtext{'bhyaṅgaḥ}{
  \Afootnote{'bhyaṅga H.}
} śirastrāṇaṃ snānam uṣṇīṣam eva ca | \caesura srajaś ca viṣasaṃsṛṣṭāḥ sādhayed avalekhavat\edlabel{SS.5.1.59-13} \edtext{|}{
  \linenum{|\xlineref{SS.5.1.59-13}}\lemma{avalekhavat |}\Afootnote{°khanāt \uline{A} \uline{H}; °khanā\uuline{va}t | K.}
}
\pend

 
\pstart

                         \textsc{[1938 ed. 5.1.60]}
                        \caesura mukhālepe \edtext{mukhaśyāvaṃ}{
  \Afootnote{mukhaṃ śyāvaṃ A.}
} yuktam abhyaṅgalakṣaṇaiḥ | \caesura \edtext{padminīkaṇṭaka}{
  \Afootnote{°kaṇṭhaka H.}
} prakhyaiḥ \edtext{kaṇṭakaiś}{
  \Afootnote{kaṇṭhakaiś H.}
} copacīyate | 
\pend

 
\pstart

                         \textsc{[1938 ed. 5.1.61]}
                        \caesura tatra kṣaudraghṛtaṃ pānam \edtext{ālepaś}{
  \Afootnote{pralepaś A.}
} candanaṃ ghṛtam | \caesura payasyā \edtext{madhukā}{
  \Afootnote{madhukaṃ A.}
} phañjī \edtext{bandhujīvaḥ}{
  \Afootnote{vandhujīva Nep.}
} punarṇṇavā\edlabel{SS.5.1.61-12} \edtext{||}{
  \linenum{|\xlineref{SS.5.1.61-12}}\lemma{punarṇṇavā ||}\Afootnote{punarnavā || A.}
}
\pend

 
\pstart

                         \textsc{[1938 ed. 5.1.62]}
                        \caesura \edtext{asvāsthyaṃ}{
  \Afootnote{āsvāsthyaṃ H.}
} kuñjarādīnāṃ \edtext{lālāsravaṇam}{
  \Afootnote{lālāsrāvo 'kṣiraktatā |  A.}
} \edtext{eva}{
  \Afootnote{sphikpāyumeḍhramuṣkeṣu A.}
} ca\edlabel{SS.5.1.62-5} \edtext{| \caesura}{
  \linenum{|\xlineref{SS.5.1.62-5}}\lemma{ca | }\Afootnote{\textsc{[om]} A.}
} yātuś ca \edtext{sphoṭanāsrāvau}{
  \Afootnote{sphoṭasaṃbhavaḥ || A; sphoṭa\uwave{naṃ}māsrāvau K; sphoṭanaṃ srāvau H.}
} muṣka\edlabel{SS.5.1.62-10} meḍhra gudeṣv atha \edtext{|}{
  \linenum{|\xlineref{SS.5.1.62-10}}\lemma{muṣka\ldots |}\Afootnote{\textsc{[om]} A.}
}
\pend

 
\pstart

                         \textsc{[1938 ed. 5.1.63]}
                        \caesura tatrābhyaṅgavad \edtext{ācaṣṭe}{
  \Afootnote{eveṣṭā A.}
} \edtext{yātrivāhanayoḥ}{
  \Afootnote{yātṛvā° A.}
} kriyām\edlabel{SS.5.1.63-4} \edtext{| \caesura}{
  \linenum{|\xlineref{SS.5.1.63-4}}\lemma{kriyām | }\Afootnote{kriyā |  A.}
} śoṇitāgamanaṃ khebhyaḥ śiroruk kaphasaṃsravaḥ | 
\pend

 
\pstart

                         \textsc{[1938 ed. 5.1.64]}
                        \caesura nasyadhūmagate liṅgam\edlabel{SS.5.1.64-2} \edtext{indriyāṇāñ}{
  \linenum{|\xlineref{SS.5.1.64-2}}\lemma{liṅgam indriyāṇāñ}\Afootnote{liṅgamindri° H; liṅgamindiryāṇāṃ A.}
} ca vaikṛtam | \caesura tatra \edtext{sarpir\edlabel{SS.5.1.64-8}}{
  \Afootnote{dugdhair A.}
} \edtext{gavādīnāṃ}{
  \linenum{|\xlineref{SS.5.1.64-8}}\lemma{sarpir gavādīnāṃ}\Afootnote{sarpirgavā° K; sarpirgavādināṃ H.}
} \edtext{dugdhaiḥ}{
  \Afootnote{sarpiḥ A.}
} sātiviṣaiḥ śṛtam\edlabel{SS.5.1.64-12} \edtext{|}{
  \linenum{|\xlineref{SS.5.1.64-12}}\lemma{śṛtam |}\Afootnote{śritaṃ | K; sritaṃ | H.}
}
\pend

 
\pstart
\edtext{
                         \textsc{[1938 ed. 5.1.65]}
                        \caesura}{
  \Afootnote{\textsc{[pre]} pāne A.}
} \edtext{nasyaṃ}{
  \Afootnote{nasye A.}
} \edtext{pānañ}{
  \Afootnote{ca A.}
} ca\edlabel{SS.5.1.65-3} \edtext{vihitaṃ}{
  \linenum{|\xlineref{SS.5.1.65-3}}\lemma{ca vihitaṃ}\Afootnote{saśvetaṃ A.}
} \edtext{śītaṃ}{
  \Afootnote{hitaṃ A.}
} samadayantikam || \caesura gandhahānir vivarṇṇatvaṃ puṣpāṇāṃ mlānatā tathā\edlabel{SS.5.1.65-12} \edtext{|}{
  \linenum{|\xlineref{SS.5.1.65-12}}\lemma{tathā |}\Afootnote{bhavet || A.}
}
\pend

 
\pstart

                         \textsc{[1938 ed. 5.1.66]}
                        \caesura jighrataś ca śiroduḥkhaṃ vāripūrṇṇe ca locane | \caesura \edtext{tatreṣṭaṃ}{
  \Afootnote{tatra A.}
} \edtext{bāṣpikaṃ}{
  \Afootnote{bāṣperitaṃ A.}
} karma mukhālepe ca yat smṛtam || 
\pend

 
\pstart

                         \textsc{[1938 ed. 5.1.67]}
                        \caesura karṇṇatailagate śrotre\edlabel{SS.5.1.67-2} \edtext{vaikṛtyaṃ}{
  \linenum{|\xlineref{SS.5.1.67-2}}\lemma{śrotre vaikṛtyaṃ}\Afootnote{śrotravaiguṇyaṃ A.}
} śophavedanā\edlabel{SS.5.1.67-4} \edtext{| \caesura}{
  \linenum{|\xlineref{SS.5.1.67-4}}\lemma{śophavedanā | }\Afootnote{°dane |  A.}
} \edtext{karṇṇāsrāvaś}{
  \Afootnote{karṇasrāvaś A \uline{H}.}
} ca \edtext{tatrāśu}{
  \Afootnote{tatrāśuḥ K.}
} kartavyaṃ pratipūraṇam | 
\pend

 
\pstart

                         \textsc{[1938 ed. 5.1.68]}
                        \caesura svaraso bahuputrāyāḥ \edtext{saghṛtaṃ}{
  \Afootnote{saghṛtaḥ A; \textsc{(gap of 8, torn)}(From 144v)\textsc{(l. 1)}taṃ |  K.}
} kṣaudrasaṃyutam\edlabel{SS.5.1.68-4} \edtext{| \caesura}{
  \linenum{|\xlineref{SS.5.1.68-4}}\lemma{kṣaudrasaṃyutam | }\Afootnote{\textsc{[om]} K; °yutaḥ |  A.}
} \edtext{somavalkarasaś}{
  \Afootnote{somakalka° H.}
} cāpi suśīto \edtext{hitam}{
  \Afootnote{hita A.}
} iṣyate | 
\pend

 
\pstart

                         \textsc{[1938 ed. 5.1.69]}
                        \caesura \edtext{asropadehau}{
  \Afootnote{aśrūpadeho A.}
} dāhaś ca vedanā dṛṣṭivibhramaḥ | \caesura añjane viṣasaṃsṛṣṭe bhaved āndhyam athāpi vā\edlabel{SS.5.1.69-12} \edtext{|}{
  \linenum{|\xlineref{SS.5.1.69-12}}\lemma{vā |}\Afootnote{ca || A.}
}
\pend

 
\pstart

                         \textsc{[1938 ed. 5.1.70]}
                        \caesura tatra sadyo ghṛtaṃ peyaṃ \edtext{tarpaṇe}{
  \Afootnote{tarpaṇaṃ A; tarpa\uwave{ṇai} K.}
} ca samāgadham | \caesura añjanaṃ meṣaśṛṅgasya niryāso \edtext{varuṇasya}{
  \Afootnote{varaṇasya H.}
} ca | 
\pend

 
\pstart
\edtext{}{
  \Afootnote{\textsc{[pre]} muṣkakasyājakarṇasya pheno gopittasaṃyutaḥ | A.}
}
\pend

 
\pstart

                         \textsc{[1938 ed. 5.1.71]}
                        \caesura kapittha \edtext{meṣaśṛṅgābhyām\edlabel{SS.5.1.71cd-2}}{
  \Afootnote{meṣaśṛṅgyoś ca A; meṣaśrṛṅgābhyām H.}
} \edtext{puṣpam}{
  \linenum{|\xlineref{SS.5.1.71cd-2}}\lemma{meṣaśṛṅgābhyām puṣpam}\Afootnote{meṣaśṛṅga\textsc{(gap of 9, torn)} K.}
} bhallātakasya\edlabel{SS.5.1.71cd-4} \edtext{ca}{
  \linenum{|\xlineref{SS.5.1.71cd-4}}\lemma{bhallātakasya ca}\Afootnote{\textsc{[om]} K.}
\lemma{ca}  \Afootnote{vā || A.}
}
\pend

 
\pstart
\edtext{}{
  \Afootnote{\textsc{[pre]} ekaikaṃ kārayet puṣpaṃ bandhūkāṅkoṭayor api | A.}
}
\pend

 
\pstart

                         \textsc{[1938 ed. 5.1.72]}
                        \caesura śophaḥ \edtext{svāpas}{
  \Afootnote{srāvas A.}
} tathā \edtext{srāvaḥ}{
  \Afootnote{svāpaḥ A.}
} pādayoḥ sphoṭajanma ca | 
\pend

 
\pstart

                         \textsc{[1938 ed. 5.1.73]}
                        \caesura bhavanti \edtext{viṣaduṣṭābhyāṃ}{
  \Afootnote{viṣajuṣṭābhyāṃ A.}
} pādukābhyām asaṃśayam | \caesura upānatpādapīṭhāni \edtext{pādukābhyāṃ}{
  \Afootnote{pādukāvat A.}
} prasādhayet | 
\pend

 
\pstart

                         \textsc{[1938 ed. 5.1.74]}
                        \caesura \edtext{bhūṣaṇāni}{
  \Afootnote{bhūṣanāni H.}
} hatārcīṃṣi na vibhānti yathā purā | \caesura svāni sthānāni hanyuś ca dāhapākāvadāraṇaiḥ\edlabel{SS.5.1.74-12} \edtext{|}{
  \linenum{|\xlineref{SS.5.1.74-12}}\lemma{dāhapākāvadāraṇaiḥ |}\Afootnote{°dāruṇaiḥ | K; °dā\uuline{ru} \textbf{ra}ṇaiḥ | H.}
}
\pend

 
\pstart

                         \textsc{[1938 ed. 5.1.75]}
                        \caesura \edtext{pādukābhūṣaṇeṣūktam}{
  \Afootnote{pādukabhūṣaṇe\textbf{ṣū}ktam H.}
} abhyaṅgavidhim ācaret | \caesura viṣopasargo bāṣpādir bhūṣaṇānto ya īritaḥ | 
\pend

 
\pstart

                         \textsc{[1938 ed. 5.1.76]}
                        \caesura \edtext{upadravāṃs}{
  \Afootnote{samīkṣyopa° A; °vām̐s H.}
} \edtext{tatra}{
  \Afootnote{tasya A.}
} \edtext{vīkṣya}{
  \Afootnote{\textsc{[om]} A.}
} vidadhīta cikitsitam | \caesura \edtext{mahāsugandham}{
  \Afootnote{°gandhim A.}
} agadaṃ yaṃ pravakṣyāmi tam bhiṣak | 
\pend

 
\pstart

                         \textsc{[1938 ed. 5.1.77]}
                        \caesura pānālepananasyeṣu vidadhītāñjaneṣu ca || \caesura virecanāni tīkṣṇāni kuryāt pracchardanāni ca | 
\pend

 
\pstart

                         \textsc{[1938 ed. 5.1.78]}
                        \caesura \edtext{śirāś}{
  \Afootnote{sirāś A.}
} ca \edtext{vyadhayet}{
  \Afootnote{vedhayet H.}
} \edtext{prāptāḥ}{
  \Afootnote{kṣipraṃ A.}
} prāptaṃ visrāvaṇaṃ yadi | \caesura \edtext{mūṣikājaruhā}{
  \Afootnote{mūsikā 'ja° A; °rukā H.}
} \edtext{vāpi}{
  \Afootnote{vā 'pi A.}
} haste baddhā tu bhūpateḥ\edlabel{SS.5.1.78-14} \edtext{|}{
  \linenum{|\xlineref{SS.5.1.78-14}}\lemma{bhūpateḥ |}\Afootnote{bhūpate | H.}
}
\pend

 
\pstart

                         \textsc{[1938 ed. 5.1.79]}
                        \caesura karoti nirviṣaṃ sarvam annaṃ viṣasamāyutam | \caesura hṛdayāvaraṇaṃ nityaṃ kuryāc cāmitramadhyagaḥ
\footnoteC{ The reading of the parallel verse in the \emph{Aṣṭāṅgasaṅgraha} (1.8.89) supports the reading \emph{amitra}. }
 |\edlabel{SS.5.1.79-10} 
\pend

 
\pstart

                         \textsc{[1938 ed. 5.1.80]}
                        \caesura pibed ghṛtam ajeyākhyam \edtext{amṛtaṃ}{
  \Afootnote{amṛtākhyaṃ A.}
} \edtext{cāpy}{
  \Afootnote{ca A; vāpy H.}
} abhuktavān\edlabel{SS.5.1.80-6} \edtext{| \caesura}{
  \linenum{|\xlineref{SS.5.1.80-6}}\lemma{abhuktavān | }\Afootnote{buddhimān |  A.}
} sarpiḥ \edtext{kṣaudraṃ}{
  \Afootnote{\textsc{[om]} A.}
} dadhi \edtext{payaḥ}{
  \Afootnote{\textsc{[add]} kṣaudraṃ A.}
} pibed vā śītalañ jalam | 
\pend

 
\pstart

                         \textsc{[1938 ed. 5.1.81]}
                        \caesura \edtext{godhāmayūranakulān}{
  \Afootnote{mayūrān na° godhāḥ A; °kulāṃ K.}
} \edtext{pṛṣatān}{
  \Afootnote{pṛṣatā K.}
} hariṇān api | \caesura viṣaghnānāñ\edlabel{SS.5.1.81-6} \edtext{ca}{
  \linenum{|\xlineref{SS.5.1.81-6}}\lemma{viṣaghnānāñ ca}\Afootnote{\textsc{[om]} A.}
} \edtext{satataṃ}{
  \Afootnote{\textsc{[add]} bhakṣayec cāpi A.}
} \edtext{rasāṃs}{
  \Afootnote{rasās Nep.}
} teṣāṃ pibed api | 
\pend

 
\pstart

                         \textsc{[1938 ed. 5.1.82]}
                        \caesura godhānakulamāṃseṣu hariṇasya ca buddhimān | \caesura dadyāt supiṣṭāṃ pālindīṃ \edtext{madhukaṃ}{
  \Afootnote{madhūkaṃ H.}
} \edtext{śarkaran}{
  \Afootnote{śarkarāṃ A.}
} tathā | 
\pend

 
\pstart

                         \textsc{[1938 ed. 5.1.83]}
                        \caesura śarkarātiviṣe deye māyūre samahauṣadhe | \caesura pārṣate cāpi \edtext{deyāḥ}{
  \Afootnote{deyā K.}
} syuḥ pippalyaḥ samahauṣadhāḥ | 
\pend

 
\pstart

                         \textsc{[1938 ed. 5.1.84]}
                        \caesura sakṣaudraḥ saghṛtaḥ \edtext{śītaḥ}{
  \Afootnote{caiva A.}
} \edtext{nimbayūṣa}{
  \Afootnote{śimbīyūṣo A.}
} hitas tathā\edlabel{SS.5.1.84-6} \edtext{| \caesura}{
  \linenum{|\xlineref{SS.5.1.84-6}}\lemma{tathā | }\Afootnote{sadā |  A.}
} \edtext{viṣaghnāni}{
  \Afootnote{viṣāghnāni K.}
} ca seveta bhakṣyabhojyāni buddhimān | 
\pend

 
\pstart

                         \textsc{[1938 ed. 1.85]}
                        \caesura pippalīmadhukakṣaudraśarkarekṣurasāmbubhiḥ | \caesura chardayed guptahṛdayo yadi\edlabel{SS.5.1.85-5} \edtext{pītaṃ}{
  \linenum{|\xlineref{SS.5.1.85-5}}\lemma{yadi pītaṃ}\Afootnote{bhakṣitaṃ A.}
} \edtext{bhaved}{
  \Afootnote{yadi vai A.}
} \edtext{viṣam}{
  \Afootnote{\textsc{[add]} iti || Nep.}
}
\pend

 
\pstart
iti\emph{\edlabel{SS.5.1.colophon-0}} || \edtext{||}{
  \linenum{|\xlineref{SS.5.1.colophon-0}}\lemma{iti\ldots ||}\Afootnote{\textsc{[om]} Nep.}
\lemma{||}  \Afootnote{\textsc{[add]} suśrutasaṃhitāyāṃ A.}
} \edtext{kalpeṣu\edlabel{SS.5.1.colophon-3}}{
  \Afootnote{kalpasthāne 'nnapānarakṣākalpo nāma A.}
} prathamo 'dhyāyaḥ \edtext{||}{
  \linenum{|\xlineref{SS.5.1.colophon-3}}\lemma{kalpeṣu\ldots ||}\Afootnote{kalpe 1 || ❈ || K.}
}
\pend

  \chapter{Kalpasthāna 2: Poisonous Plants}
\pstart

                         \textsc{[1938 ed. 5.2.1]}
                        \caesura athātaḥ sthāvara \edtext{viṣavijñānīyaṃ}{
  \Afootnote{\textsc{[add]} adhyāyaṃ A.}
} vyākhyāsyāmaḥ\edlabel{SS.5.2.1-4} \edtext{||}{
  \linenum{|\xlineref{SS.5.2.1-4}}\lemma{vyākhyāsyāmaḥ ||}\Afootnote{vyākhyāsyāmaḥ || K.}
}
\pend

 
\pstart
\edtext{}{
  \Afootnote{\textsc{[pre]} yathovāca bhagavān dhanvantariḥ || A.}
}
\pend

 
\pstart

                         \textsc{[1938 ed. 5.2.3]}
                        \caesura sthāvaraṃ jaṅgamaṃ caiva dvividhaṃ viṣam ucyate | \caesura daśādhiṣṭhānam \edtext{ādyan}{
  \Afootnote{ādyaṃ A.}
} tu \edtext{dvitīyaṃ}{
  \Afootnote{dvitīyāṃ Nep.}
} ṣoḍaśāśrayam || 
\pend

 
\pstart

                         \textsc{[1938 ed. 5.2.4]}
                        \caesura mūlaṃ patraṃ phalaṃ puṣpaṃ tvak kṣīraṃ \edtext{sāram}{
  \Afootnote{sāra A.}
} eva ca | \caesura niryāso dhātavaś caiva \edtext{kandañ}{
  \Afootnote{kandaś A.}
} ca \edtext{daśamaṃ}{
  \Afootnote{daśamaḥ A.}
} smṛtam\edlabel{SS.5.2.4-17} \edtext{||}{
  \linenum{|\xlineref{SS.5.2.4-17}}\lemma{smṛtam ||}\Afootnote{smṛtaḥ || A.}
}
\pend

 
\pstart

                         \textsc{[1938 ed. 5.2.5a]}
                        \caesura tatra \edtext{klītakāśvamāraka}{
  \Afootnote{°ra A.}
} guñjā \edtext{subhaṅgurā}{
  \Afootnote{sugandha gargaraka A.}
} \edtext{karaṭā}{
  \Afootnote{karaghāṭa A.}
} \edtext{vidyucchikhānanta}{
  \Afootnote{°khā A.}
} vijayā\edlabel{SS.5.2.5a-7} \edtext{cety}{
  \linenum{|\xlineref{SS.5.2.5a-7}}\lemma{vijayā cety}\Afootnote{vijayānīty A.}
} aṣṭau mūlaviṣāṇi || 
\pend

 
\pstart

                         \textsc{[1938 ed. 5.2.5b]}
                        \caesura viṣapatrikā \edtext{lambaradā}{
  \Afootnote{lambā vara dāru A.}
} karambha \edtext{mahākarambhādīni}{
  \Afootnote{°bhāṇi pañca A.}
} patraviṣāṇi || 
\pend

 
\pstart

                         \textsc{[1938 ed. 5.2.5c]}
                        \caesura \edtext{kumudavati}{
  \Afootnote{kumudvatī A.}
} \edtext{reṇuka}{
  \Afootnote{veṇukā A.}
} \edtext{kuruvaka}{
  \Afootnote{karkoṭaka A; kurukaka H.}
} \edtext{veṇuka}{
  \Afootnote{reṇuka khadyotaka A.}
} \edtext{karambha\edlabel{SS.5.2.5c-5}}{
  \Afootnote{karañ ca K.}
} \edtext{mahākarambha}{
  \linenum{|\xlineref{SS.5.2.5c-5}}\lemma{karambha mahākarambha}\Afootnote{carmarī bhagandhā A.}
\lemma{mahākarambha}  \Afootnote{\textsc{[add]} sarpaghāti A.}
} \edtext{nandanā}{
  \Afootnote{nandana A.}
} \edtext{kākādīni}{
  \Afootnote{sārapākānīti A; kākādanī H.}
} \edtext{guñjāruṣkara}{
  \Afootnote{dvādaśa A; guñjārūṣkara K.}
} \edtext{viṣavedikādīnāṃ}{
  \Afootnote{\textsc{[om]} A.}
} phalāni\edlabel{SS.5.2.5c-11} \edtext{||}{
  \linenum{|\xlineref{SS.5.2.5c-11}}\lemma{phalāni ||}\Afootnote{phalaviṣāṇi; A.}
}
\pend

 
\pstart
\edtext{
                         \textsc{[1938 ed. 5.2.5d]}
                        \caesura}{
  \Afootnote{\textsc{[pre]} vetra kādamba A.}
} \edtext{ullika}{
  \Afootnote{vallīja A; ullijā H.}
} \edtext{reṇu\edlabel{SS.5.2.5d-2}}{
  \Afootnote{\textsc{[om]} A.}
} \edtext{karambha}{
  \linenum{|\xlineref{SS.5.2.5d-2}}\lemma{reṇu karambha}\Afootnote{reṇuka\textbf{ka}rambha H.}
} \edtext{mahākarambhādīnāṃ}{
  \Afootnote{°bhāṇi pañca A.}
} puṣpāṇi\edlabel{SS.5.2.5d-5} \edtext{|}{
  \linenum{|\xlineref{SS.5.2.5d-5}}\lemma{puṣpāṇi |}\Afootnote{puṣpaviṣāṇi; A.}
}
\pend

 
\pstart

                         \textsc{[1938 ed. 5.2.5e]}
                        \caesura \edtext{vallija}{
  \Afootnote{antrapācaka kartarīya saurīyaka A.}
} karaghāṭaka \edtext{karambha}{
  \Afootnote{rambha A.}
} \edtext{nārācakādīnāṃ}{
  \Afootnote{nārā\uwave{ca}kā° K; nandananārācakāni sapta A.}
} tvaksāraniryāsāḥ\edlabel{SS.5.2.5e-5} \edtext{|}{
  \linenum{|\xlineref{SS.5.2.5e-5}}\lemma{tvaksāraniryāsāḥ |}\Afootnote{tvkasāra° K; °ryāsaviṣāṇi; A.}
}
\pend

 
\pstart

                         \textsc{[1938 ed. 5.2.5f]}
                        \caesura \edtext{kumudavati}{
  \Afootnote{kumudaghnī A.}
} \edtext{dantī}{
  \Afootnote{\textsc{[om]} A.}
} snuhā\edlabel{SS.5.2.5f-3} \edtext{jālinī}{
  \linenum{|\xlineref{SS.5.2.5f-3}}\lemma{snuhā jālinī}\Afootnote{snuhājā° H.}
} \edtext{prabhṛtīnāṃ}{
  \linenum{|\xlineref{SS.5.2.5f-3}}\lemma{snuhā\ldots prabhṛtīnāṃ}\Afootnote{snuhījālakṣīrīṇi trīṇi A.}
} kṣīrāṇi\edlabel{SS.5.2.5f-6} \edtext{|}{
  \linenum{|\xlineref{SS.5.2.5f-6}}\lemma{kṣīrāṇi |}\Afootnote{kṣīraviṣāṇi; A.}
}
\pend

 
\pstart

                         \textsc{[1938 ed. 5.2.5g]}
                        \caesura \edtext{haritāla}{
  \Afootnote{\textsc{[om]} A.}
} phenāśma bhasma \edtext{rakta}{
  \Afootnote{haritālaṃ A; rakṣe K.}
} \edtext{prabhṛtīni}{
  \Afootnote{ca dve A.}
} dhātuviṣāṇi\edlabel{SS.5.2.5g-6} \edtext{|}{
  \linenum{|\xlineref{SS.5.2.5g-6}}\lemma{dhātuviṣāṇi |}\Afootnote{dhātuviṣe; A.}
}
\pend

 
\pstart

                         \textsc{[1938 ed. 5.2.5h]}
                        \caesura \edtext{kālakūṭā}{
  \Afootnote{kālakūṭa A; kālakuṭā Nep.}
} vatsanābha sarṣapa \edtext{kapālaka}{
  \Afootnote{pālaka A.}
} kardamaka vairāṭaka \edtext{mustakā}{
  \Afootnote{mustaka A H.}
} \edtext{mahāviṣa}{
  \Afootnote{śṛṅgīviṣa A.}
} \edtext{puṇḍarīka}{
  \Afootnote{prapuṇ° A; puṣḍa° K.}
} mūlaka hālāhala \edtext{śṛṅgī}{
  \Afootnote{\textsc{[om]} mahāviṣa A.}
} \edtext{markaṭādīnāṃ}{
  \Afootnote{karkaṭakānīti trayodaśa A.}
} kandāḥ\edlabel{SS.5.2.5h-14} \edtext{||}{
  \linenum{|\xlineref{SS.5.2.5h-14}}\lemma{kandāḥ ||}\Afootnote{kandaviṣāṇi; ity evaṃ pañcapañcāśat sthāvaraviṣāṇi bhavanti || A.}
}
\pend

 
\pstart
\edtext{}{
  \Afootnote{\textsc{[pre]} catvāri vatsanābhāni mustake dve prakīrtite  ṣaṭ caiva sarṣapāṇy āhuḥ śeṣāṇy ekaikam eva tu || A.}
}
\pend

 
\pstart

                         \textsc{[1938 ed. 5.2.7]}
                        \caesura udveṣṭanam mūlaviṣaiḥ pralāpo moha eva ca | \caesura \edtext{jṛmbhaṇodveṣṭanaśvāsā}{
  \Afootnote{jṛmbha\textsc{(l. 1)}nodve° H; jṛmbhāṅgodve° A; °ṣṭanaṃ śvāsāḥ K.}
} jñeyāḥ \edtext{patraviṣair}{
  \Afootnote{patraviṣeṇa A.}
} nṛṇāṃ\edlabel{SS.5.2.7-11} \edtext{||
\footnoteC{ Cf. \emph{Mādhavanidāna} (69.8cd–9ab) \emph{udveṣṭanaṃ mūlaviṣaiḥ pralāpo moha eva ca }| \emph{jṛmbhaṇaṃ vepanaṃ śvāso mohaḥ patraviṣeṇa tu } || }
}{
  \linenum{|\xlineref{SS.5.2.7-11}}\lemma{nṛṇāṃ ||}\Afootnote{tu || A.}
}
\pend

 
\pstart

                         \textsc{[1938 ed. 5.2.8]}
                        \caesura muṣkaśophaḥ phalaviṣaiḥ \edtext{dāhodveṣṭanam}{
  \Afootnote{dāho 'nnadveṣa A.}
} eva ca | \caesura bhavet puṣpaviṣaiś chardir ādhmānaṃ \edtext{svāpam}{
  \Afootnote{moha A.}
} eva ca
\footnoteC{ Cf. \emph{Mādhavanidāna} (69.9cd–10ab) \emph{muṣkaśo 'thaḥ phalaviṣair dāho 'nnadveṣa eva ca }| \emph{bhavet puṣpaviṣaiś chardir ādhmānaṃ śvāsa eva ca } || }
 || 
\pend

 
\pstart

                         \textsc{[1938 ed. 5.2.9]}
                        \caesura tvaksāraniryāsaviṣair upayuktair bhavanti ha\edlabel{SS.5.2.9-4} \edtext{| \caesura}{
  \linenum{|\xlineref{SS.5.2.9-4}}\lemma{ha | }\Afootnote{hi |  A H.}
} āsya daurgandhya pāruṣya śirorukkapha saṃsravāḥ
\footnoteC{ Cf. \emph{Mādhavanidāna} (69.10cd–11ab) \emph{tvaksāraniryāsaviṣair upayuktair bhavanti hi }| \emph{āsyadaurgandhyapāruṣyaśirorukkaphasaṃsravāḥ } || }
 || 
\pend

 
\pstart

                         \textsc{[1938 ed. 5.2.10]}
                        \caesura phenāgamaḥ kṣīraviṣair \edtext{viḍbhedo}{
  \Afootnote{viḍbhede K.}
} gurujihvatā | \caesura \edtext{hṛtpīḍanan}{
  \Afootnote{hṛtpīḍanaṃ A.}
} dhātuviṣair mūrcchā dāhaś ca tāluni
\footnoteC{ Cf. \emph{Mādhavanidāna} (69.11cd–12ab) \emph{ phenāgamaḥ kṣīraviṣair viḍbhedo gurugātratā }| \emph{hṛtpīḍanaṃ dhātuviṣair mūrcchā dāhaś ca tāluni } || }
 || 
\pend

 
\pstart

                         \textsc{[1938 ed. 5.2.11]}
                        \caesura \edtext{prāyeṇa}{
  \Afootnote{prāye\textbf{ṇa} K; prāyena H.}
} kālaghātīni viṣāṇy etāni nirdiśet
\footnoteC{ Cf. \emph{Mādhavanidāna} (gurujihv) \emph{ prāyeṇa kālaghātīni viṣāṇy etāni nirdiśet } | }
 || \caesura kandajāni tu tīkṣṇāni teṣām vakṣyāmi vistaram | 
\pend

 
\pstart

                         \textsc{[1938 ed. 5.2.12]}
                        \caesura sparśājñānaṃ \edtext{kālakūṭe}{
  \Afootnote{kālakūṭo H.}
} vepathuś \edtext{ca}{
  \Afootnote{stambha A.}
} sudāruṇaḥ\edlabel{SS.5.2.12-5} \edtext{| \caesura}{
  \linenum{|\xlineref{SS.5.2.12-5}}\lemma{sudāruṇaḥ | }\Afootnote{sadā° H; eva A.}
\lemma{| }  \Afootnote{\textsc{[add]} ca |  A.}
} grīvāstambho vatsanābhe \edtext{pītaviṇmūtratā}{
  \Afootnote{°tranetratā || A.}
} tathā\edlabel{SS.5.2.12-10} \edtext{|}{
  \linenum{|\xlineref{SS.5.2.12-10}}\lemma{tathā |}\Afootnote{\textsc{[om]} A.}
}
\pend

 
\pstart

                         \textsc{[1938 ed. 5.2.13]}
                        \caesura \edtext{sārṣape}{
  \Afootnote{sarṣape A.}
} \edtext{vāyuvaiguṇyam}{
  \Afootnote{vātavai° A H.}
} ānāho granthijanma ca | \caesura grīvādaurbalyavāksaṅgau \edtext{pālakena}{
  \Afootnote{pālake A.}
} bhavanti\edlabel{SS.5.2.13-9} ha\edlabel{SS.5.2.13-10} \edtext{|}{
  \linenum{|\xlineref{SS.5.2.13-9}}\lemma{bhavanti\ldots |}\Afootnote{'numatāv iha || A.}
  \linenum{|\xlineref{SS.5.2.13-10}}\lemma{ha |}\Afootnote{hi || H.}
}
\pend

 
\pstart

                         \textsc{[1938 ed. 5.2.14]}
                        \caesura \edtext{prasekaḥ}{
  \Afootnote{prase\textbf{kaḥ} K.}
} kardamākhyena viḍbhedo\edlabel{SS.5.2.14-3} \edtext{'kṣṇoś}{
  \linenum{|\xlineref{SS.5.2.14-3}}\lemma{viḍbhedo 'kṣṇoś}\Afootnote{viṅbhedo A.}
} \edtext{ca}{
  \Afootnote{netra A.}
} pītatā | \caesura \edtext{vairāṭakenāṅgaduḥkhaṃ}{
  \Afootnote{°duḥkha H.}
} śirorogaś ca jāyate | 
\pend

 
\pstart

                         \textsc{[1938 ed. 5.2.15]}
                        \caesura gātrastambho vepathuś \edtext{ca}{
  \Afootnote{\textsc{[add]} jāyate A.}
} \edtext{mustakena}{
  \Afootnote{mustākena K.}
} prakīrtitau\edlabel{SS.5.2.15-5} \edtext{| \caesura}{
  \linenum{|\xlineref{SS.5.2.15-5}}\lemma{prakīrtitau | }\Afootnote{tu |  A.}
} mahāviṣeṇāṅgasādadāhodaravivṛddhayaḥ\edlabel{SS.5.2.15-7} \edtext{|}{
  \linenum{|\xlineref{SS.5.2.15-7}}\lemma{mahāviṣeṇāṅgasādadāhodaravivṛddhayaḥ |}\Afootnote{mahāviṣeṇāṅgasādaḥ | dā° K; śṛṅgīviṣeṇāṅgasāda dā° A; °ṣeṇā\uuline{ṅgasādaḥ | dāhodaravivṛddhayaḥ} \textbf{hṛdaye granthiśūlodbhavo bhṛśaṃ | 1} || H.}
}
\pend

 
\pstart

                         \textsc{[1938 ed. 5.2.16]}
                        \caesura \edtext{puṇḍarīkeṇa}{
  \Afootnote{°ke na Nep.}
} raktatvam akṣṇor vṛddhis tathodare | \caesura \edtext{mūlakenāṅgavaivarṇyaṃ}{
  \Afootnote{\textsc{[om]} mūlakenāṅga° A; °varṇṇya H.}
} \edtext{gātrastambhaś}{
  \Afootnote{mūlakaiś A.}
} \edtext{ca}{
  \Afootnote{chardir A.}
} jāyate\edlabel{SS.5.2.16-10} \edtext{|}{
  \linenum{|\xlineref{SS.5.2.16-10}}\lemma{jāyate |}\Afootnote{hikkāśophapramūḍhatāḥ || A.}
}
\pend

 
\pstart

                         \textsc{[1938 ed. 5.2.17]}
                        \caesura \edtext{dhyāmaś}{
  \Afootnote{\textsc{[om]} A.}
} \edtext{cireṇocchvasiti}{
  \Afootnote{\textsc{[add]} śyāvo A.}
} naro hālāhalena tu\edlabel{SS.5.2.17-5} \edtext{| \caesura}{
  \linenum{|\xlineref{SS.5.2.17-5}}\lemma{tu | }\Afootnote{vai |  A.}
} \edtext{śṛṅgīviṣeṇa}{
  \Afootnote{mahāvi° A.}
} hṛdaye \edtext{granthiśūlodbhavo}{
  \Afootnote{°lodgamau A.}
} bhṛśam | 
\pend

 
\pstart

                         \textsc{[1938 ed. 5.2.18]}
                        \caesura \edtext{markaṭenotplavaty}{
  \Afootnote{markkaṭono° H; karkaṭenotpataty A.}
} ūrdhvaṃ \edtext{hasaty}{
  \Afootnote{hasan A.}
} \edtext{api}{
  \Afootnote{dantān A.}
} daśaty api | 
\pend

 
\pstart
 kandajāny ugravīryāṇi yāny\edlabel{SS.5.2.18cd-3} \edtext{uktāni}{
  \linenum{|\xlineref{SS.5.2.18cd-3}}\lemma{yāny uktāni}\Afootnote{pratyuktāni A.}
} trayodaśa\edlabel{SS.5.2.18cd-5} \edtext{| \caesura
                        \textsc{[1938 ed. 5.2.19]}
                        \caesura}{
  \linenum{|\xlineref{SS.5.2.18cd-5}}\lemma{trayodaśa |  \textsc{[1938 ed. 5.2.19]} }\Afootnote{trayodaśaḥ |  K.}
} \edtext{jñeyāny}{
  \Afootnote{sarvāṇi A.}
} \edtext{uktāni}{
  \Afootnote{\textsc{[om]} A.}
} kuśalair \edtext{yuktāni}{
  \Afootnote{jñeyāny etāni A.}
} \edtext{daśabhir}{
  \Afootnote{daśabhi\uwave{rguṇai}(\textbf{\uwave{}})(From 145v)\textsc{(l. 1)}r K.}
} guṇaiḥ | 
\pend

 
\pstart
 rūkṣam uṣṇañ \edtext{ca}{
  \Afootnote{tathā A.}
} tīkṣṇañ \edtext{ca}{
  \Afootnote{\textsc{[om]} A; \uwave{ca} K.}
} sūkṣmam \edtext{āśu}{
  \Afootnote{āśu\textbf{r} K.}
} vyavāyi ca | 
\pend

 
\pstart

                         \textsc{[1938 ed. 5.2.20]}
                        \caesura \edtext{vikāsi}{
  \Afootnote{vikāśi A.}
} viśadañ caiva laghv apāki ca \edtext{te}{
  \Afootnote{tat A.}
} daśa\edlabel{SS.5.2.20-8} \edtext{| \caesura}{
  \linenum{|\xlineref{SS.5.2.20-8}}\lemma{daśa | }\Afootnote{smṛtam |  A; daśaḥ |  K.}
} tad raukṣyāt kopayed \edtext{vātam}{
  \Afootnote{vāyum A.}
} auṣṇyāt pittaṃ saśoṇitam || 
\pend

 
\pstart

                         \textsc{[1938 ed. 5.2.21]}
                        \caesura \edtext{taikṣṇyān}{
  \Afootnote{\textsc{[om]} A.}
} \edtext{matim}{
  \Afootnote{\textsc{[add]} ca A.}
} \edtext{mohayati}{
  \Afootnote{mohayet taikṣṇyān A.}
} \edtext{marmabandhāñ}{
  \Afootnote{marmabandhān A.}
} chinatti ca | \caesura saukṣmyāc\edlabel{SS.5.2.21-8} \edtext{charīrāvayavān}{
  \linenum{|\xlineref{SS.5.2.21-8}}\lemma{saukṣmyāc charīrāvayavān}\Afootnote{śarī° A.}
\lemma{charīrāvayavān}  \Afootnote{\textsc{[add]} saukṣmyāt A.}
} praviśed \edtext{vikaroti}{
  \Afootnote{vikriyeta K; vikrayeta H.}
} ca | 
\pend

 
\pstart
\edtext{
                         \textsc{[1938 ed. 5.2.22]}
                        \caesura}{
  \Afootnote{\textsc{[pre]} \uuline{aviśuddhaṃm āśu tad dhinti} H.}
} \edtext{āśutvād}{
  \Afootnote{\textbf{ā}° H; aviśuddham K.}
} āśu tad \edtext{dhanti}{
  \Afootnote{dh\textbf{i}nt\textbf{i} K; dhanti H.}
} vyavāyāt prakṛtiṃ bhajet \edtext{| \caesura}{
  \Afootnote{\textsc{[add]} kṣapayec ca A.}
} \edtext{vikāsitvād}{
  \Afootnote{vikāśitvād A.}
} \edtext{anuviśed}{
  \Afootnote{\textsc{[om]} A.}
} doṣān dhātūn malān api | 
\pend

 
\pstart

                         \textsc{[1938 ed. 5.2.23]}
                        \caesura vaiśadyāc \edtext{cāsaktagatir}{
  \Afootnote{atiricyeta A.}
} \edtext{duścikitsañ}{
  \Afootnote{duścikitsyaṃ A; dduściki\uwave{tsa}ñ H.}
} ca lāghavāt | \caesura \edtext{durnirharam}{
  \Afootnote{durharaṃ A.}
} \edtext{apākitvāt}{
  \Afootnote{cāvipā° A.}
} tasmāt kleśayate ciram || 
\pend

 
\pstart

                         \textsc{[1938 ed. 5.2.24]}
                        \caesura sthāvaraṃ \edtext{jaṅgamam}{
  \Afootnote{\textsc{[add]} yac ca A.}
} \edtext{vāpi}{
  \Afootnote{\textsc{[om]} A.}
} kṛtrimam \edtext{vāpi}{
  \Afootnote{cāpi A.}
} yad viṣam | \caesura sadyo \edtext{mārayate}{
  \Afootnote{vyāpādayet tat A.}
} \edtext{martyaṃ}{
  \Afootnote{tu A.}
} \edtext{jñeyan}{
  \Afootnote{jñeyaṃ A.}
} \edtext{daśaguṇan\edlabel{SS.5.2.24-13}}{
  \Afootnote{daśaguṇa\uwave{n} K.}
} tu tat\edlabel{SS.5.2.24-15} \edtext{|}{
  \linenum{|\xlineref{SS.5.2.24-13}}\lemma{daśaguṇan\ldots |}\Afootnote{daśaguṇānvitam || A.}
  \linenum{|\xlineref{SS.5.2.24-15}}\lemma{tat |}\Afootnote{tat | K.}
}
\pend

 
\pstart
\edtext{}{
  \Afootnote{\textsc{[pre]} yat sthāvaraṃ jaṅgamakṛtrimaṃ vā dehād aśeṣaṃ yad anirgataṃ tat | A.}
}
\pend

 
\pstart

                         \textsc{[1938 ed. 5.2.25cd]}
                        \caesura jīrṇam viṣaghnauṣadhibhir hatam vā \caesura dāvāgnivātātapaśoṣitam vā | 
\pend

 
\pstart

                         \textsc{[1938 ed. 5.2.26]}
                        \caesura svabhāvato vā \edtext{guṇaviprahīṇaṃ \caesura}{
  \Afootnote{°hīnaṃ A.}
} \edtext{viṣaṃ}{
  \Afootnote{(\textbf{vi}) \uwave{} śaṃ K.}
} hi dūṣīviṣatām upaiti | \caesura \edtext{vīryālpabhāvād}{
  \Afootnote{vīryālpam āvānna nipātayet A.}
} \edtext{avibhāvanīyaṃ \caesura}{
  \Afootnote{api bhā° H; tat A.}
} kaphāvṛtatvāt\edlabel{SS.5.2.26-11} sucirānubandhi \edtext{|}{
  \linenum{|\xlineref{SS.5.2.26-11}}\lemma{kaphāvṛtatvāt\ldots |}\Afootnote{kaphāvṛtaṃ varṣagaṇānubandhi || A.}
}
\pend

 
\pstart

                         \textsc{[1938 ed. 5.2.27]}
                        \caesura tenārdito bhinna \edtext{purīṣa}{
  \Afootnote{purīśa K.}
} varṇo \caesura \edtext{vidagdha}{
  \Afootnote{vigandha A.}
} vairasya \edtext{yutaḥ}{
  \Afootnote{mukhaḥ A.}
} pipāsī | \caesura \edtext{mūrcchāṃ}{
  \Afootnote{mūrcchan A.}
} \edtext{bhramaṃ}{
  \Afootnote{vaman A.}
} \edtext{gadgadavākyamartyo \caesura}{
  \Afootnote{°vāgviṣaṇṇo  A.}
} viceṣṭamāno\edlabel{SS.5.2.27-13} ratim \edtext{āpnuyāc}{
  \linenum{|\xlineref{SS.5.2.27-13}}\lemma{viceṣṭamāno\ldots āpnuyāc}\Afootnote{bhavec A.}
} ca \edtext{|}{
  \Afootnote{\textsc{[add]} duṣyodaraliṅgajuṣṭaḥ || A.}
}
\pend

 
\pstart

                         \textsc{[1938 ed. 5.2.28]}
                        \caesura āmāśayasthe \edtext{kapha}{
  \Afootnote{phaka K.}
} vātarogī \caesura pakvāśayasthe 'nilapittarogī | \caesura \edtext{bhavet samudhvastaśiroruhāṅgo
                        \caesura
}{
  \Afootnote{bhaven A.}
} \edtext{vilūnapakṣas}{
  \Afootnote{naro dhva° A.}
} tu yathā vihaṅgaḥ | 
\pend

 
\pstart

                         \textsc{[1938 ed. 5.2.29]}
                        \caesura sthitaṃ rasādiṣv \edtext{ayathāyathoktān \caesura}{
  \Afootnote{athavā ya° A.}
} karoti dhātuprabhavān vikārān | \caesura kopañ ca \edtext{śītānila}{
  \Afootnote{śītānilar K.}
} durdineṣu \caesura yāty āśu pūrvaṃ śṛṇu \edtext{tasya}{
  \Afootnote{tatra A.}
} liṅgam\edlabel{SS.5.2.29-17} \edtext{|}{
  \linenum{|\xlineref{SS.5.2.29-17}}\lemma{liṅgam |}\Afootnote{rūpam || A.}
}
\pend

 
\pstart

                         \textsc{[1938 ed. 5.2.30]}
                        \caesura nidrāgurutvañ ca \edtext{vijṛmbhaṇañ}{
  \Afootnote{vijṛṇañ H.}
} ca \caesura viśleṣaharṣāv athavāṅgamardam\edlabel{SS.5.2.30-6} \edtext{| \caesura}{
  \linenum{|\xlineref{SS.5.2.30-6}}\lemma{athavāṅgamardam | }\Afootnote{°mardaḥ |  A.}
} tataḥ karoty annamadāvipākāv \caesura arocakaṃ maṇḍala \edtext{koṭhatāñ}{
  \Afootnote{koṭhamohān || A.}
} ca\edlabel{SS.5.2.30-14} \edtext{|}{
  \linenum{|\xlineref{SS.5.2.30-14}}\lemma{ca |}\Afootnote{\textsc{[om]} A.}
}
\pend

 
\pstart

                         \textsc{[1938 ed. 5.2.31]}
                        \caesura \edtext{māṃsakṣayaṃ}{
  \Afootnote{dhātukṣayaṃ A; mānsakṣayam H.}
} pādakarāsyaśophaṃ \caesura \edtext{pralepakañ\edlabel{SS.5.2.31-3}}{
  \Afootnote{dakodaraṃ A; pralepakaś H.}
} \edtext{chardim}{
  \linenum{|\xlineref{SS.5.2.31-3}}\lemma{pralepakañ chardim}\Afootnote{pralepakacchardim K.}
} athātisāram | \caesura dūṣīviṣaṃ\edlabel{SS.5.2.31-7} \edtext{śvāsatṛṣājvarāṃś}{
  \linenum{|\xlineref{SS.5.2.31-7}}\lemma{dūṣīviṣaṃ śvāsatṛṣājvarāṃś}\Afootnote{vaivarṇyamūrcchāviṣamajva° A.}
\lemma{śvāsatṛṣājvarāṃś}  \Afootnote{°rām̐ś H.}
} \edtext{ca \caesura}{
  \Afootnote{vā A.}
} kuryāt \edtext{pravṛddhiṃ}{
  \Afootnote{pravṛddhāṃ A.}
} \edtext{jaṭharasya}{
  \Afootnote{prabalāṃ tṛṣāṃ A.}
} cāpi\edlabel{SS.5.2.31-13} \edtext{|}{
  \linenum{|\xlineref{SS.5.2.31-13}}\lemma{cāpi |}\Afootnote{vā || A.}
}
\pend

 
\pstart

                         \textsc{[1938 ed. 5.2.32]}
                        \caesura unmādam anyaj janayet tathānyad \caesura ānāham anyat \edtext{kṣapayec}{
  \Afootnote{kṣapaye K.}
} ca śukram\edlabel{SS.5.2.32-9} \edtext{| \caesura}{
  \linenum{|\xlineref{SS.5.2.32-9}}\lemma{śukram | }\Afootnote{śuktraṃ |  K.}
} \edtext{kārśyan}{
  \Afootnote{gādgadyam A.}
} \edtext{tathānyaj}{
  \Afootnote{anyaj A.}
} \edtext{janayec}{
  \Afootnote{janaye K.}
} ca kuṣṭhaṃ \caesura \edtext{tāṃs}{
  \Afootnote{tān H.}
} \edtext{tān}{
  \Afootnote{tāṃ K.}
} \edtext{vikārāṃś}{
  \Afootnote{vikārām̐ś H.}
} ca bahuprakārān\edlabel{SS.5.2.32-20} \edtext{|}{
  \linenum{|\xlineref{SS.5.2.32-20}}\lemma{bahuprakārān |}\Afootnote{°kārāṃ | K.}
}
\pend

 
\pstart

                         \textsc{[1938 ed. 5.2.33]}
                        \caesura dūṣitaṃ \edtext{deśakālānnadivāsvapnair}{
  \Afootnote{deśakālā\uwave{nnu}di° K.}
} abhīkṣṇaśaḥ | \caesura yasmād \edtext{vā}{
  \Afootnote{\textsc{[om]} A.}
} \edtext{dūṣayed}{
  \Afootnote{dūṣayate A.}
} \edtext{dhātuṃ}{
  \Afootnote{dhātūn A.}
} tasmād dūṣīviṣaṃ smṛtam || 
\pend

 
\pstart

                         \textsc{[1938 ed. 2.34]}
                        \caesura sthāvarasyopayuktasya vege tu prathame nṛṇām | \caesura śyāvā jihvā bhavet stabdhā mūrcchā \edtext{trāsaś}{
  \Afootnote{śvāsaś A.}
} ca jāyate | 
\pend

 
\pstart

                         \textsc{[1938 ed. 5.2.35]}
                        \caesura dvitīye vepathuḥ sādo dāhaḥ kaṇṭharujas tathā | \caesura viṣañ \edtext{cāmāśaya}{
  \Afootnote{āmā° A.}
} prāptaṅ kurute hṛdi vedanām | 
\pend

 
\pstart

                         \textsc{[1938 ed. 5.2.36]}
                        \caesura \edtext{tāluśoṣas}{
  \Afootnote{tāluśoṣaṃ A.}
} tṛtīye tu \edtext{śūlañ}{
  \Afootnote{śūlaś H.}
} cāmāśaye bhṛśam \edtext{| \caesura}{
  \Afootnote{\textsc{[add]} bhṛśaṃ |  K.}
} \edtext{durbale}{
  \Afootnote{durvarṇe A.}
} \edtext{harite}{
  \Afootnote{\textsc{[add]} \uuline{stū} H.}
} \edtext{śūne}{
  \Afootnote{\uwave{śū}ne K; \textbf{śū}ne H.}
} \edtext{jāyete}{
  \Afootnote{jāyate H.}
} cāsya locane ||
 
\pend

 
\pstart

                         \textsc{[1938 ed. 5.2.37]}
                        \caesura pakvāmāśayayos \edtext{sādo}{
  \Afootnote{todo A.}
} hikkā \edtext{kāso}{
  \Afootnote{śvāso H.}
} 'ntrakūjanam | \caesura caturthe jāyate vege śirasaś cāpi\edlabel{SS.5.2.37-11} gauravam \edtext{|}{
  \linenum{|\xlineref{SS.5.2.37-11}}\lemma{cāpi\ldots |}\Afootnote{cātigauravam || A.}
}
\pend

 
\pstart

                         \textsc{[1938 ed. 5.2.38]}
                        \caesura kaphapraseko vaivarṇyaṃ \edtext{pārśvabhedaś}{
  \Afootnote{parvabhedaś A.}
} ca pañcame | \caesura sarvadoṣaprakopaś ca \edtext{pakvādhāne}{
  \Afootnote{pakvādhāṇe K.}
} \edtext{ca}{
  \Afootnote{\textbf{ca} H.}
} vedanā | 
\pend

 
\pstart

                         \textsc{[1938 ed. 5.2.39]}
                        \caesura ṣaṣṭhe \edtext{sañjñāpraṇāśaś}{
  \Afootnote{prajñā° A.}
} ca \edtext{bhṛśaṃ}{
  \Afootnote{bhṛṣa K.}
} \edtext{cāpy}{
  \Afootnote{\uwave{cāpy} K.}
} atisāryate\edlabel{SS.5.2.39-6} \edtext{|| \caesura}{
  \linenum{|\xlineref{SS.5.2.39-6}}\lemma{atisāryate || }\Afootnote{atisā\uwave{rya}te ||  K.}
} \edtext{skandha}{
  \Afootnote{skanda Nep.}
} pṛṣṭha \edtext{kaṭībhaṅgāḥ}{
  \Afootnote{kaṭībhaṅgaḥ A; kaṭībhāgāḥ Nep.}
} sannirodhaś ca saptame || 
\pend

 
\pstart

                         \textsc{[1938 ed. 5.2.40]}
                        \caesura prathame viṣavege tu \edtext{vāntaṃ}{
  \Afootnote{vānte A.}
} śītāṃbusecitam\edlabel{SS.5.2.40-5} \edtext{| \caesura}{
  \linenum{|\xlineref{SS.5.2.40-5}}\lemma{śītāṃbusecitam | }\Afootnote{śītāmbusevitaṃ ||  H.}
\lemma{| }  \Afootnote{\textsc{[add]} agadaṃ A.}
} \edtext{sarpirmadhubhyāṃ}{
  \Afootnote{madhusarpirbhyāṃ A.}
} saṃyuktam\edlabel{SS.5.2.40-8} \edtext{agadam}{
  \linenum{|\xlineref{SS.5.2.40-8}}\lemma{saṃyuktam agadam}\Afootnote{\textsc{[om]} A.}
} \edtext{pāyayen}{
  \Afootnote{pāyayeta A; pā\textbf{ya}yen K; pāyayet H.}
} naram\edlabel{SS.5.2.40-11} \edtext{||
}{
  \linenum{|\xlineref{SS.5.2.40-11}}\lemma{naram ||}\Afootnote{samāyutam || A; naraḥ | H.}
}
\pend

 
\pstart

                         \textsc{[1938 ed. 5.2.41]}
                        \caesura dvitīye pūrvavad vāntaṃ viriktañ\edlabel{SS.5.2.41-4} \edtext{cāpi}{
  \linenum{|\xlineref{SS.5.2.41-4}}\lemma{viriktañ cāpi}\Afootnote{\textsc{[om]} A.}
} pāyayet \edtext{| \caesura}{
  \Afootnote{\textsc{[add]} tu virecanam |  A.}
} tṛtīye \edtext{'gadapānan}{
  \Afootnote{'gadapānaṃ A.}
} tu \edtext{hitan}{
  \Afootnote{hitaṃ A.}
} nasyaṃ tathāñjanam || 
\pend

 
\pstart
\edtext{
                         \textsc{[1938 ed. 5.2.42]}
                        \caesura}{
  \Afootnote{\textsc{[pre]} \textbf{ caturthe 'lpasnehalavaṇam iti pāṭhaḥ |} H.}
} \edtext{sindhuṃ}{
  \Afootnote{\textsc{[om]} A; \textbf{2}sindhuñ H.}
} caturthe \edtext{'lpasneham}{
  \Afootnote{snehasaṃmiśraṃ A.}
} \edtext{agadam}{
  \Afootnote{\textsc{[om]} A.}
} \edtext{pāyayed}{
  \Afootnote{\textsc{[add]} āgadaṃ A.}
} bhiṣak |
                        \caesura
 pañcame kṣaudramadhukakvāthayuktaṃ pradāpayet || \caesura
                        \textsc{[1938 ed. 5.2.43AB]}
                        \caesura ṣaṣṭhe 'tisāravat \edtext{siddhir}{
  \Afootnote{'tīsā° A.}
} avasīdet \edtext{tu}{
  \Afootnote{avapīḍaś A; avasīden K.}
} \edtext{saptame}{
  \Afootnote{ca A.}
} | 
\pend

 
\pstart
\edtext{}{
  \Afootnote{\textsc{[pre]} mūrdhni kākapadaṃ kṛtvā sāsṛg vā piśitaṃ kṣipet || A.}
}
\pend

 
\pstart

                         \textsc{[1938 ed. 5.2.44]}
                        \caesura vegāntare tv anyatame kṛte \edtext{karmaṇi}{
  \Afootnote{marmmaṇi H.}
} śītalām | \caesura yavāgūṃ \edtext{saghṛtakṣaudrām}{
  \Afootnote{saghṛtaṃ kṣaudrām K.}
} imāṃ dadyād viṣāpahām\edlabel{SS.5.2.44-12} \edtext{|}{
  \linenum{|\xlineref{SS.5.2.44-12}}\lemma{viṣāpahām |}\Afootnote{viṣāpahaṃ || H.}
}
\pend

 
\pstart

                         \textsc{[1938 ed. 5.2.45]}
                        \caesura \edtext{kośavaty}{
  \Afootnote{koṣātakyo A.}
} \edtext{agnikaḥ}{
  \Afootnote{'gnikaḥ A.}
} \edtext{pāṭhā\edlabel{SS.5.2.45-3}}{
  \Afootnote{pā\textsc{(l. 1)}\uwave{ṭhā} H.}
} sūryavaly amṛtābhayā\edlabel{SS.5.2.45-5} \edtext{| \caesura}{
  \linenum{|\xlineref{SS.5.2.45-5}}\lemma{amṛtābhayā | }\Afootnote{amṛ° H.}
} \edtext{śeluḥ}{
  \linenum{|\xlineref{SS.5.2.45-3}}\lemma{pāṭhā\ldots śeluḥ}\Afootnote{pāṭhāsūryavallyamṛtābhayāḥ |  A.}
\lemma{śeluḥ}  \Afootnote{śelū K.}
} \edtext{śirīṣakiṇihī}{
  \Afootnote{śirīṣaḥ ki° A.}
} \edtext{haridre}{
  \Afootnote{śelur giryāhvā A.}
} bṛhatīdvayam\edlabel{SS.5.2.45-10} \edtext{|
}{
  \linenum{|\xlineref{SS.5.2.45-10}}\lemma{bṛhatīdvayam |}\Afootnote{rajanīdvayam || A.}
}
\pend

 
\pstart

                         \textsc{[1938 ed. 5.2.46]}
                        \caesura \edtext{punarṇṇavau}{
  \Afootnote{punarnave A; punarnnavau H.}
} hareṇuś ca \edtext{tryūṣaṇaṃ}{
  \Afootnote{trikaṭuḥ A.}
} śārivotpale\edlabel{SS.5.2.46-5} \edtext{| \caesura}{
  \linenum{|\xlineref{SS.5.2.46-5}}\lemma{śārivotpale | }\Afootnote{sārive balā |  A.}
} eṣāṃ \edtext{yavāgūr}{
  \Afootnote{yavāgū Nep.}
} niḥkvāthe kṛtā hanti viṣadvayam | 
\pend

 
\pstart

                         \textsc{[1938 ed. 5.2.47]}
                        \caesura \edtext{madhukaṃ}{
  \Afootnote{\uwave{ma}\textsc{(l. 2)}dhukan H.}
} tagaraṃ kuṣṭhaṃ bhadradāruhareṇavaḥ\edlabel{SS.5.2.47-4} \edtext{| \caesura}{
  \linenum{|\xlineref{SS.5.2.47-4}}\lemma{bhadradāruhareṇavaḥ | }\Afootnote{bhadradāru ha° A.}
} \edtext{mañjiṣṭhailailavālūni}{
  \Afootnote{punnāgailai° A; °la\uwave{vā}lū\uwave{}ni H.}
} nāgapuṣpotpalaṃ sitā | 
\pend

 
\pstart

                         \textsc{[1938 ed. 5.2.48]}
                        \caesura viḍaṅgaṃ candanaṃ patraṃ \edtext{priyaṅgu}{
  \Afootnote{priyaṅgur A.}
} dhyāmakan tathā | \caesura \edtext{haridre}{
  \Afootnote{\uwave{ha}° H.}
} dve bṛhatyau \edtext{ca}{
  \Afootnote{ca H.}
} \edtext{sārivāṃśumatī}{
  \Afootnote{sārive ca sthirā A; sāri\textsc{(l. 5)}\uwave{vāṅśumatī} K; \textsc{(l. 3)}\uwave{sāricāṃśumatī} H.}
} balā\edlabel{SS.5.2.48-13} \edtext{|}{
  \linenum{|\xlineref{SS.5.2.48-13}}\lemma{balā |}\Afootnote{sahā || A; valā | K.}
}
\pend

 
\pstart

                         \textsc{[1938 ed. 5.2.49]}
                        \caesura kalkair eṣāṃ ghṛtaṃ siddham ajeyam iti viśrutam | \caesura viṣāṇi hanti \edtext{sarvāṇi}{
  \Afootnote{sarvāni K.}
} śīghram \edtext{evājitan}{
  \Afootnote{evājitaṃ A.}
} \edtext{tu}{
  \Afootnote{\textsc{[om]} A.}
} tat\edlabel{SS.5.2.49-15} \edtext{||}{
  \linenum{|\xlineref{SS.5.2.49-15}}\lemma{tat ||}\Afootnote{kvacit || A.}
}
\pend

 
\pstart

                         \textsc{[1938 ed. 5.2.50]}
                        \caesura dūṣīviṣārttaṃ susvinnam ūrdhvañ \edtext{cādham}{
  \Afootnote{cādhaś A H.}
} ca śodhitam | \caesura pāyayed\edlabel{SS.5.2.50-8} \edtext{agadam}{
  \linenum{|\xlineref{SS.5.2.50-8}}\lemma{pāyayed agadam}\Afootnote{pāyayetāgadaṃ A.}
} \edtext{mukhyam}{
  \Afootnote{nityam A.}
} \edtext{idaṃ}{
  \Afootnote{imaṃ A.}
} dūṣīviṣāpaham\edlabel{SS.5.2.50-12} \edtext{||}{
  \linenum{|\xlineref{SS.5.2.50-12}}\lemma{dūṣīviṣāpaham ||}\Afootnote{dūṣivi° A.}
}
\pend

 
\pstart

                         \textsc{[1938 ed. 5.2.51]}
                        \caesura pippalyo dhyāmakaṃ māṃsī \edtext{lodhram}{
  \Afootnote{śāvaraḥ A; \uwave{lopram} K.}
} \edtext{elā}{
  \Afootnote{paripelavam |  A; elā K.}
} suvarccikā | \caesura \edtext{bālakaṅ}{
  \Afootnote{\textsc{[om]} A.}
} gairiko\edlabel{SS.5.2.51-9} \edtext{hemas}{
  \linenum{|\xlineref{SS.5.2.51-9}}\lemma{gairiko hemas}\Afootnote{sasūkṣmailā A.}
} \edtext{tathā}{
  \Afootnote{toyaṃ A.}
} \edtext{ca}{
  \Afootnote{kanakagairikam || A.}
} paripelavā\edlabel{SS.5.2.51-13} \edtext{|}{
  \linenum{|\xlineref{SS.5.2.51-13}}\lemma{paripelavā |}\Afootnote{\textsc{[om]} A.}
}
\pend

 
\pstart

                         \textsc{[1938 ed. 5.2.52]}
                        \caesura kṣaudrayukto gado hy eṣa dūṣīviṣam apohati \edtext{|| \caesura}{
  \Afootnote{\textsc{[add]} nāmnā A.}
} dūṣīviṣārir \edtext{nāmnā}{
  \Afootnote{\textsc{[om]} A.}
} tu na cānyatrāpi vāryate | 
\pend

 
\pstart

                         \textsc{[1938 ed. 5.2.53]}
                        \caesura jvare dāhe \edtext{ca}{
  \Afootnote{'tha H.}
} \edtext{hikkāyāṃ}{
  \Afootnote{hikkāyā\uwave{m} H.}
} \edtext{ānāhe}{
  \Afootnote{\uwave{srunāhe} K; ānāhe H.}
} śukrasaṃkṣaye\edlabel{SS.5.2.53-6} \edtext{| \caesura}{
  \linenum{|\xlineref{SS.5.2.53-6}}\lemma{śukrasaṃkṣaye | }\Afootnote{°kṣaya |  A.}
} śophe\edlabel{SS.5.2.53-8} \edtext{'tisāre}{
  \linenum{|\xlineref{SS.5.2.53-8}}\lemma{śophe 'tisāre}\Afootnote{śopheti° H.}
\lemma{'tisāre}  \Afootnote{ti sāre K.}
} \edtext{murcchāyāṃ}{
  \Afootnote{mūrcchāyāṃ A.}
} \edtext{tvagdoṣe}{
  \Afootnote{hṛdroge A.}
} jaṭhare pi ca | 
\pend

 
\pstart

                         \textsc{[1938 ed. 5.2.54]}
                        \caesura unmāde vepathau caiva ye \edtext{cāpy}{
  \Afootnote{cānye A; \uwave{cāpy} K.}
} \edtext{anya}{
  \Afootnote{syur A; anyā Nep.}
} upadravāḥ | \caesura yathāsvaṃ teṣu kurvīta viṣaghnair \edtext{eva}{
  \Afootnote{\textsc{[om]} A.}
} bheṣajaiḥ\edlabel{SS.5.2.54-14} \edtext{|}{
  \linenum{|\xlineref{SS.5.2.54-14}}\lemma{bheṣajaiḥ |}\Afootnote{auṣadhaiḥ A.}
\lemma{|}  \Afootnote{\textsc{[add]} kriyām || A.}
}
\pend

 
\pstart

                         \textsc{[1938 ed. 5.2.55]}
                        \caesura sādhyam ātmavataḥ sadyo yāpyaṃ samvatsarotthitam\edlabel{SS.5.2.55-5} \edtext{| \caesura}{
  \linenum{|\xlineref{SS.5.2.55-5}}\lemma{samvatsarotthitam | }\Afootnote{saṃvatsa° A.}
} dūṣīviṣaṃ \edtext{varjanīyam}{
  \Afootnote{asādhyaṃ A.}
} \edtext{ato}{
  \Afootnote{\textsc{[om]} A.}
} \edtext{'nyad}{
  \Afootnote{tu A.}
} ahitāśinaḥ\edlabel{SS.5.2.55-11} \edtext{||}{
  \linenum{|\xlineref{SS.5.2.55-11}}\lemma{ahitāśinaḥ ||}\Afootnote{kṣīṇasyāhitasevinaḥ || A; ahitāśina || Nep.}
}
\pend

 
\pstart
 \edtext{iti}{
  \Afootnote{\textsc{[add]} suśrutasaṃhitāyāṃ A.}
} \edtext{kalpasthāne}{
  \Afootnote{kalpe 2 || ❈ || K; \textsc{[add]} sthāvaraviṣavijñānīyo nāma A.}
} dvitīyo\edlabel{SS.5.2.end-3} 'dhyāyaḥ \edtext{||}{
  \linenum{|\xlineref{SS.5.2.end-3}}\lemma{dvitīyo\ldots ||}\Afootnote{\textsc{[om]} K.}
}
\pend

 \chapter{Kalpasthāna 3: Poisonous Insects and Animals}
\pstart

                         \textsc{[1938 ed. 5.3.1]}
                        \caesura athāto jaṅgamaviṣavijñānīyaṃ kalpaṃ vyākhyāsyāmaḥ\edlabel{SS.5.3.1-4} \edtext{||}{
  \linenum{|\xlineref{SS.5.3.1-4}}\lemma{vyākhyāsyāmaḥ ||}\Afootnote{vyākhyāsyāmaḥ || K.}
}
\pend

 
\pstart
\edtext{}{
  \Afootnote{\textsc{[pre]} yathovāca bhagavān dhanvantariḥ || A.}
}
\pend

 
\pstart

                         \textsc{[1938 ed. 5.3.3]}
                        \caesura jaṅgamasya viṣasyoktāny adhiṣṭhānāni ṣoḍaśa | \caesura samāsena mayā yāni vistaras teṣu vakṣyate | 
\pend

 
\pstart

                         \textsc{[1938 ed. 5.3.4]}
                        \caesura tatra dṛṣṭi \edtext{niśvāsa}{
  \Afootnote{niḥśvāsa A.}
} daṃṣṭrānakha \edtext{mukha}{
  \Afootnote{\textsc{[om]} A.}
} mūtra \edtext{purīṣārtava}{
  \Afootnote{purīṣa A; purīśārtava K.}
} śukra \edtext{lāṅgūla}{
  \Afootnote{\textsc{[om]} A; lā\textsc{(l. 3)}\uwave{ṅgū}la H.}
} \edtext{lālāsparśa}{
  \Afootnote{lālārtava A.}
} mukha \edtext{sandaṃśāvaśardhita}{
  \Afootnote{sandaṃśa viśa° A.}
} \edtext{gudāsthi}{
  \Afootnote{tuṇḍāsthi A.}
} pitta śūkaśavāni\edlabel{SS.5.3.4-15} \edtext{||}{
  \linenum{|\xlineref{SS.5.3.4-15}}\lemma{śūkaśavāni ||}\Afootnote{śūka śavāni || K; °vānīti || A.}
}
\pend

 
\pstart

                         \textsc{[1938 ed. 5.3.5(1)]}
                        \caesura tatra \edtext{niśvāsadṛṣṭiviṣāḥ}{
  \Afootnote{niśvāsa dṛ° H; dṛṣṭiniḥśvāsaviṣā A.}
} divyāḥ sarpāḥ | 
\pend

 
\pstart

                         \textsc{[1938 ed. 5.3.5(2)]}
                        \caesura bhaumās\edlabel{SS.5.3.5.2-1} tu daṃṣṭrāviṣāḥ \edtext{||}{
  \linenum{|\xlineref{SS.5.3.5.2-1}}\lemma{bhaumās\ldots ||}\Afootnote{bhaumādaṃ° Nep.}
}
\pend

 
\pstart

                         \textsc{[1938 ed. 5.3.5(3)]}
                        \caesura \edtext{mārjāraśva\edlabel{SS.5.3.5.3-1}}{
  \Afootnote{mārjāra śva \uline{A} \uline{H}.}
} \edtext{vānara}{
  \linenum{|\xlineref{SS.5.3.5.3-1}}\lemma{mārjāraśva vānara}\Afootnote{\uwave{mā}rjāra\textsc{(l. 2)} śva|vā° K.}
} \edtext{nara}{
  \Afootnote{\textsc{[om]} A.}
} makara maṇḍūka pākamatsya godhā śambūka \edtext{pracalāka}{
  \Afootnote{pracalākaḥ K.}
} \edtext{gṛhagoḍikāś}{
  \Afootnote{gṛhagodhikā A; \textsc{[add]} \textbf{ galagoḍikā | kutracit | 3} H.}
} \edtext{catuṣpadāś\edlabel{SS.5.3.5.3-11}}{
  \Afootnote{catuṣpādāś H.}
} ca \edtext{kīṭās}{
  \linenum{|\xlineref{SS.5.3.5.3-11}}\lemma{catuṣpadāś\ldots kīṭās}\Afootnote{catuṣpādakīṭās A.}
\lemma{kīṭās}  \Afootnote{kī\uwave{ṭā}s K.}
} \edtext{tathānye}{
  \Afootnote{tathā 'nye A.}
} nakhamukhadaṃṣṭrāviṣāḥ\edlabel{SS.5.3.5.3-15} \edtext{|}{
  \linenum{|\xlineref{SS.5.3.5.3-15}}\lemma{nakhamukhadaṃṣṭrāviṣāḥ |}\Afootnote{daṃṣṭrānakhaviṣāḥ, A.}
}
\pend

 
\pstart

                         \textsc{[1938 ed. 5.3.5(4)]}
                        \caesura \edtext{kiṭipa}{
  \Afootnote{cipiṭa A.}
} \edtext{picciṭā}{
  \Afootnote{picciṭaka A H.}
} kaṣāyavāsika sarṣapaka\edlabel{SS.5.3.5.4-4} toṭaka \edtext{varcaḥkīṭāḥ}{
  \linenum{|\xlineref{SS.5.3.5.4-4}}\lemma{sarṣapaka\ldots varcaḥkīṭāḥ}\Afootnote{sarṣapa katoṭa kavarcaḥ kīṭa A.}
} \edtext{kauṇḍinyā}{
  \Afootnote{kauṇḍinyakāḥ A.}
} mūtrapurīṣaviṣāḥ\edlabel{SS.5.3.5.4-8} \edtext{||}{
  \linenum{|\xlineref{SS.5.3.5.4-8}}\lemma{mūtrapurīṣaviṣāḥ ||}\Afootnote{śakṛnmūtraviṣāḥ, A; °rīsaviṣāḥ || K.}
}
\pend

 
\pstart

                         \textsc{[1938 ed. 5.3.5(5)]}
                        \caesura mūṣikāḥ śukraviṣāḥ | 
\pend

 
\pstart

                         \textsc{[1938 ed. 5.3.5(6)]}
                        \caesura vṛścika \edtext{viśvambhara}{
  \Afootnote{śvabhrara K.}
} \edtext{varaki}{
  \Afootnote{varaṭīrājīva A.}
} \edtext{matsyocciṭiṅga}{
  \Afootnote{°ṭiṅgāḥ A.}
} \edtext{patravṛścikāḥ}{
  \Afootnote{patraviṣāḥ | vṛ° H; samudravṛ° A.}
} śūlaviṣāḥ\edlabel{SS.5.3.5.6-6} \edtext{|}{
  \linenum{|\xlineref{SS.5.3.5.6-6}}\lemma{śūlaviṣāḥ |}\Afootnote{cāla(ra)viṣāḥ, A.}
}
\pend

 
\pstart

                         \textsc{[1938 ed. 5.3.5(7)]}
                        \caesura lūtā lālā \edtext{nakha}{
  \Afootnote{\textsc{[om]} A; \textsc{[add]} mukha H.}
} mūtra \edtext{purīṣārtava}{
  \Afootnote{purīṣa A.}
} \edtext{śukra}{
  \Afootnote{mukha A.}
} daṃṣṭrāviṣāḥ\edlabel{SS.5.3.5.7-7} \edtext{|}{
  \linenum{|\xlineref{SS.5.3.5.7-7}}\lemma{daṃṣṭrāviṣāḥ |}\Afootnote{sandaṃśa nakha śukrārtava viṣāḥ, A; daṃṣṭāviṣāḥ | K.}
}
\pend

 
\pstart

                         \textsc{[1938 ed. 5.3.5(8)]}
                        \caesura makṣikā kaṇabha \edtext{jalāyukā}{
  \Afootnote{jalāyu kā H.}
} mukhasandaṃśaviṣāḥ | 
\pend

 
\pstart

                         \textsc{[1938 ed. 5.3.5(9)]}
                        \caesura \edtext{citraśīrṣa}{
  \Afootnote{citraśiraḥ A.}
} \edtext{śarāvakurdi}{
  \Afootnote{sarāva kurdi A.}
} śata \edtext{dārukāri}{
  \Afootnote{dārūkāri K; dārukā|ari(From )\textsc{(l. 1)} H.}
} medaka \edtext{śārikā}{
  \Afootnote{sārikāmukhā A; śarārikā H.}
} mukha\edlabel{SS.5.3.5.9-7} \edtext{sandaṃśa}{
  \linenum{|\xlineref{SS.5.3.5.9-7}}\lemma{mukha sandaṃśa}\Afootnote{mukhasan° A K.}
} \edtext{daṃṣtrāsyarśā\edlabel{SS.5.3.5.9-9}}{
  \Afootnote{\textsc{[om]} A.}
} \edtext{viśarddhita}{
  \linenum{|\xlineref{SS.5.3.5.9-9}}\lemma{daṃṣtrāsyarśā viśarddhita}\Afootnote{daṃṣtrāsya sāvasadhita K.}
} \edtext{guda}{
  \linenum{|\xlineref{SS.5.3.5.9-9}}\lemma{daṃṣtrāsyarśā\ldots guda}\Afootnote{daṃṣṭrāsya \textbf{r}śāvasarddhita\uuline{ḥ}guda H.}
\lemma{guda}  \Afootnote{mūtra A.}
} \edtext{purīṣa\edlabel{SS.5.3.5.9-12}}{
  \Afootnote{purīśa K.}
} viṣāḥ \edtext{|}{
  \linenum{|\xlineref{SS.5.3.5.9-12}}\lemma{purīṣa\ldots |}\Afootnote{purīṣaviṣāḥ, A.}
} 
 
\pend

 
\pstart

                         \textsc{[1938 ed. 5.3.5(10)]}
                        \caesura viṣahatāsthi\edlabel{SS.5.3.5.10-1} \edtext{sarpakaṇṭakavarakimatsyāsthi\edlabel{SS.5.3.5.10-2}}{
  \linenum{|\xlineref{SS.5.3.5.10-1}}\lemma{viṣahatāsthi sarpakaṇṭakavarakimatsyāsthi}\Afootnote{viṣahatāsthisa° K.}
\lemma{sarpakaṇṭakavarakimatsyāsthi}  \Afootnote{sarpa kaṇṭaka varaṭī ma° A.}
} \edtext{cety\edlabel{SS.5.3.5.10-3}}{
  \linenum{|\xlineref{SS.5.3.5.10-2}}\lemma{sarpakaṇṭakavarakimatsyāsthi cety}\Afootnote{sarppa kaṇṭaka varaki matsyāsthicety H.}
} asthiviṣāṇi \edtext{||}{
  \linenum{|\xlineref{SS.5.3.5.10-3}}\lemma{cety\ldots ||}\Afootnote{cetyasthi° K.}
}
\pend

 
\pstart

                         \textsc{[1938 ed. 5.3.5(11)]}
                        \caesura \edtext{śakalimatsyaraktarājīvakimatsyāḥ}{
  \Afootnote{śakali matsya ra° H; śakulī matsya raktarājivarakī(ṭī)matsyāś ca A.}
} pittaviṣāḥ\edlabel{SS.5.3.5.11-2} \edtext{||}{
  \linenum{|\xlineref{SS.5.3.5.11-2}}\lemma{pittaviṣāḥ ||}\Afootnote{pittaviṣā || K.}
}
\pend

 
\pstart

                         \textsc{[1938 ed. 5.3.5(12)]}
                        \caesura sūkṣma tuṇḍocciṭiṅga \edtext{vāraṭi}{
  \Afootnote{varaṭī A.}
} \edtext{śatapadi}{
  \Afootnote{śatapadī A.}
} \edtext{valabhika\edlabel{SS.5.3.5.12-5}}{
  \Afootnote{śūkavala bhikā A.}
} \edtext{śṛṅga\edlabel{SS.5.3.5.12-6}}{
  \linenum{|\xlineref{SS.5.3.5.12-5}}\lemma{valabhika śṛṅga}\Afootnote{°kaśṛṅga K; °ka\uwave{śṛ}ṅga H.}
} \edtext{bhramarāḥ}{
  \linenum{|\xlineref{SS.5.3.5.12-6}}\lemma{śṛṅga bhramarāḥ}\Afootnote{śṛṅgibhra° A.}
} śūkaviṣāḥ\edlabel{SS.5.3.5.12-8} \edtext{|}{
  \linenum{|\xlineref{SS.5.3.5.12-8}}\lemma{śūkaviṣāḥ |}\Afootnote{śūkatuṇḍaviṣāḥ, A.}
}
\pend

 
\pstart

                         \textsc{[1938 ed. 5.3.5(13)]}
                        \caesura \edtext{kīṭasarpadehā}{
  \Afootnote{°hā\uwave{} K.}
} \edtext{vyasavaḥ}{
  \Afootnote{gatāsavaḥ A; vyasavaḥ K.}
} śavaviṣāḥ\edlabel{SS.5.3.5.13-3} \edtext{|}{
  \linenum{|\xlineref{SS.5.3.5.13-3}}\lemma{śavaviṣāḥ |}\Afootnote{śavaviṣā | K.}
}
\pend

 
\pstart

                         \textsc{[1938 ed. 5.3.5(14)]}
                        \caesura śeṣās tv \edtext{anuktā}{
  \Afootnote{a\uwave{nu}\textsc{(l. 5)}ktā K.}
} \edtext{mukhadaṃśaviṣeṣv}{
  \Afootnote{mukhasandaṃ° A; mukhasaṃdaṃśaviṣeśv H.}
} eva gaṇayitavyā\edlabel{SS.5.3.5.14-6} iti \edtext{||}{
  \linenum{|\xlineref{SS.5.3.5.14-6}}\lemma{gaṇayitavyā\ldots ||}\Afootnote{°tavyāḥ || A.}
}
\pend

 
\pstart
bhavanti cātra\edlabel{SS.5.3.6-1} \edtext{ślokāḥ \caesura
                        \textsc{[1938 ed. 5.3.6]}
                        \caesura}{
  \linenum{|\xlineref{SS.5.3.6-1}}\lemma{cātra ślokāḥ  \textsc{[1938 ed. 5.3.6]} }\Afootnote{\textsc{[om]} K.}
\lemma{ślokāḥ  \textsc{[1938 ed. 5.3.6]} }  \Afootnote{\textsc{[om]} A.}
} rājño 'rideśe ripavo \edtext{jalāni \caesura}{
  \Afootnote{tṛṇāmbu  A.}
} mārgāṃś\edlabel{SS.5.3.6-7} \edtext{ca}{
  \linenum{|\xlineref{SS.5.3.6-7}}\lemma{mārgāṃś ca}\Afootnote{mārgānna A.}
} \edtext{bhaktāni\edlabel{SS.5.3.6-9}}{
  \Afootnote{bha\uwave{ktā}ni K.}
} \edtext{ca}{
  \linenum{|\xlineref{SS.5.3.6-9}}\lemma{bhaktāni ca}\Afootnote{dhūma A.}
\lemma{ca}  \Afootnote{\textsc{[add]} śvasanān viṣeṇa |  A.}
} dūṣayanti\edlabel{SS.5.3.6-11} \edtext{| \caesura}{
  \linenum{|\xlineref{SS.5.3.6-11}}\lemma{dūṣayanti | }\Afootnote{saṃdū° A.}
} \edtext{tāni}{
  \Afootnote{ebhir A.}
} \edtext{praduṣṭāni}{
  \Afootnote{atipraduṣṭān  A.}
} \edtext{bhiṣag}{
  \Afootnote{vijñāya A.}
} \edtext{vipaścid \caesura}{
  \Afootnote{liṅgair A.}
} \edtext{viśodhayed}{
  \Afootnote{abhiśo° A.}
} āgamitārthaśuddhaḥ\edlabel{SS.5.3.6-18} \edtext{||}{
  \linenum{|\xlineref{SS.5.3.6-18}}\lemma{āgamitārthaśuddhaḥ ||}\Afootnote{tān || A.}
}
\pend

 
\pstart

                         \textsc{[1938 ed. 5.3.7]}
                        \caesura duṣṭañ jalaṃ picchilam \edtext{asragandhi \caesura}{
  \Afootnote{ugragandhi  A.}
} \edtext{phenāvṛtaṃ}{
  \Afootnote{phenānvitaṃ A.}
} rājibhir āvṛtañ ca | \caesura maṇḍūkamatsyaṃ \edtext{mriyate}{
  \Afootnote{priyate Nep.}
} vihaṅgā \caesura mattāś ca sānūpacarā bhramanti || 
\pend

 
\pstart

                         \textsc{[1938 ed. 5.3.8]}
                        \caesura majjanti ye cātra narāśvanāgās \caesura te cchardimohajvaraśophaśūlān\edlabel{SS.5.3.8-6} \edtext{| \caesura}{
  \linenum{|\xlineref{SS.5.3.8-6}}\lemma{cchardimohajvaraśophaśūlān | }\Afootnote{°radāhaśophān |  A.}
} \edtext{arcchanti}{
  \Afootnote{ṛcchanti(gacchanti) A.}
} teṣām \edtext{apahṛtya}{
  \Afootnote{apahatya H.}
} \edtext{rogāṃ \caesura}{
  \Afootnote{doṣān  A.}
} duṣṭaṃ jalaṃ śodhayituṃ yateta | 
\pend

 
\pstart

                         \textsc{[1938 ed. 5.3.9]}
                        \caesura dhavāśvakarṇāv\edlabel{SS.5.3.9-1} atha \edtext{pāribhadraṃ \caesura}{
  \linenum{|\xlineref{SS.5.3.9-1}}\lemma{dhavāśvakarṇāv\ldots pāribhadraṃ }\Afootnote{°karṇāsanapāribhadrān  A.}
} \edtext{sapāṭalaṃ}{
  \Afootnote{sapāṭalān A.}
} \edtext{sidhraka}{
  \Afootnote{siddhaka A.}
} \edtext{muṣkakau}{
  \Afootnote{mokṣakau A.}
} ca | \caesura dagdhvā sarājadrumasomavalkān \caesura \edtext{tad}{
  \Afootnote{taṃ K.}
} \edtext{bhasma}{
  \Afootnote{bha\uuline{smā}\textbf{sma} K.}
} śītaṃ \edtext{vikiret}{
  \Afootnote{vitaret A.}
} sarassu\edlabel{SS.5.3.9-15} \edtext{|}{
  \linenum{|\xlineref{SS.5.3.9-15}}\lemma{sarassu |}\Afootnote{saratsu | K.}
}
\pend

 
\pstart

                         \textsc{[1938 ed. 5.3.10]}
                        \caesura bhasmāñjaliñ cāpi \edtext{ghaṭe}{
  \Afootnote{ghaṭo H.}
} nidhāya \caesura viśodhayed īpsitam evam ambhaḥ || \caesura kṣitipradeśaṃ \edtext{viṣadūṣitan}{
  \Afootnote{°ṣitaṃ A.}
} \edtext{tu \caesura}{
  \Afootnote{\textsc{[add]} śilātalaṃ A.}
} tīrthaṃ śilām\edlabel{SS.5.3.10-14} \edtext{vāpy}{
  \linenum{|\xlineref{SS.5.3.10-14}}\lemma{śilām vāpy}\Afootnote{\textsc{[om]} A.}
} aribhiḥ\edlabel{SS.5.3.10-16} \edtext{sthalīm}{
  \linenum{|\xlineref{SS.5.3.10-16}}\lemma{aribhiḥ sthalīm}\Afootnote{atheriṇaṃ A.}
} vā | 
\pend

 
\pstart

                         \textsc{[1938 ed. 5.3.11]}
                        \caesura spṛśanti gātreṇa tu yena yena \caesura \edtext{govājināgāḥ}{
  \Afootnote{°nāgoṣṭrakharā narā A.}
} puruṣāḥ\edlabel{SS.5.3.11-7} \edtext{striyo}{
  \linenum{|\xlineref{SS.5.3.11-7}}\lemma{puruṣāḥ striyo}\Afootnote{\textsc{[om]} A.}
} vā \caesura \edtext{tad}{
  \Afootnote{tac chūnatāṃ A; d K.}
} \edtext{āśu}{
  \Afootnote{\textsc{[om]} A; ā\uwave{śu} K.}
} \edtext{śūyaty}{
  \Afootnote{yāsty A; śū(From 147r)\textsc{(l. 1)}ya\uwave{ty} K.}
} \edtext{atha}{
  \Afootnote{atha K.}
} dahyate ca \caesura \edtext{śīryanti}{
  \Afootnote{viśīryate A.}
} romāṇi\edlabel{SS.5.3.11-17} \edtext{nakhāś}{
  \linenum{|\xlineref{SS.5.3.11-17}}\lemma{romāṇi nakhāś}\Afootnote{romanakhaṃ A.}
} \edtext{ca}{
  \Afootnote{tathaiva || A.}
} tasmin\edlabel{SS.5.3.11-20} \edtext{|}{
  \linenum{|\xlineref{SS.5.3.11-20}}\lemma{tasmin |}\Afootnote{\textsc{[om]} A; tasmiṃ | K.}
}
\pend

 
\pstart

                         \textsc{[1938 ed. 5.3.12]}
                        \caesura tatrāpy \edtext{anantāṃ}{
  \Afootnote{anaṃtā H.}
} saha sarvagandhaiḥ \caesura piṣṭvā surābhiḥ \edtext{saha}{
  \Afootnote{\textsc{[om]} A.}
} yojyamārgān\edlabel{SS.5.3.12-8} \edtext{| \caesura}{
  \linenum{|\xlineref{SS.5.3.12-8}}\lemma{yojyamārgān | }\Afootnote{viniyojya mārgam |  A; yojyamārgāṃ |  K.}
} siñced \edtext{athādbhiś}{
  \Afootnote{payobhiḥ A.}
} \edtext{ca}{
  \Afootnote{\textsc{[om]} A.}
} \edtext{mṛdanvitābhir \caesura}{
  \Afootnote{sumṛdanvitais A.}
} \edtext{mārgo}{
  \Afootnote{\textsc{[om]} A.}
} \edtext{'sti}{
  \Afootnote{\textsc{[om]} A.}
} \edtext{cānyo}{
  \Afootnote{taṃ  viḍaṅgapāṭhākaṭabhījalair vā || A.}
} yadi\edlabel{SS.5.3.12-17} tena gacchet \edtext{||
}{
  \linenum{|\xlineref{SS.5.3.12-17}}\lemma{yadi\ldots ||}\Afootnote{\textsc{[om]} A.}
}
\pend

 
\pstart

                         \textsc{[1938 ed. 5.3.13]}
                        \caesura tṛṇeṣu bhakteṣu ca dūṣiteṣu \caesura sīdanti mūrcchanti vamanti cānye | \caesura viḍbhedam \edtext{arcchanty}{
  \Afootnote{ṛcchanty A.}
} athavā \edtext{mriyante \caesura}{
  \Afootnote{mri\uwave{ya}nte  K.}
} \edtext{teṣāṃ}{
  \Afootnote{\uwave{teṣā}ṃ\textsc{(l. 2)} K.}
} cikitsām \edtext{prayated}{
  \Afootnote{praṇayed A.}
} yathoktām\edlabel{SS.5.3.13-17} \edtext{|}{
  \linenum{|\xlineref{SS.5.3.13-17}}\lemma{yathoktām |}\Afootnote{yathoktaṃ || H.}
}
\pend

 
\pstart

                         \textsc{[1938 ed. 5.3.14]}
                        \caesura viṣāpahair \edtext{vāpy}{
  \Afootnote{vā 'py A; vyāpy K; vvyā\uuline{pa} \textbf{py} H.}
} \edtext{agadaiḥ}{
  \Afootnote{agadaiḥ H.}
} \edtext{pralipya \caesura}{
  \Afootnote{vilipya  A.}
} vādyāni citrāṇy upavādayeta\edlabel{SS.5.3.14-7} \edtext{| \caesura}{
  \linenum{|\xlineref{SS.5.3.14-7}}\lemma{upavādayeta | }\Afootnote{api vā° A.}
} \edtext{tārāvitāraḥ}{
  \Afootnote{tāraḥ sutāraḥ A.}
} sasurendragopas \caesura \edtext{tenaiva}{
  \Afootnote{sarvaiś ca A.}
} tulyaḥ kuruvindabhāgaḥ || 
\pend

 
\pstart

                         \textsc{[1938 ed. 5.3.15]}
                        \caesura pittena yuktaḥ \edtext{kapilāhvayena \caesura}{
  \Afootnote{kapilānvayena  A.}
} \edtext{vādyapralepo}{
  \Afootnote{°leyo K.}
} \edtext{'bhihitaḥ}{
  \Afootnote{vihitaḥ A.}
} praśastaḥ | \caesura vādyasya śabdena hi yānti nāśaṃ \caesura viṣāṇi ghorāṇy api yāni tatra\edlabel{SS.5.3.15-17} \edtext{||}{
  \linenum{|\xlineref{SS.5.3.15-17}}\lemma{tatra ||}\Afootnote{santi || A.}
}
\pend

 
\pstart

                         \textsc{[1938 ed. 5.3.16]}
                        \caesura dhūme 'nile vā viṣasamprayukte \caesura \edtext{khagā}{
  \Afootnote{khagāḥ A.}
} \edtext{bhramantaḥ}{
  \Afootnote{śramārtāḥ A.}
} prapatanti bhūmau | \caesura \edtext{kāsapratiśyāyaśirovikārān \caesura}{
  \Afootnote{°rorujaś A.}
} \edtext{archanti}{
  \Afootnote{ca  bhavanti A.}
} \edtext{tīvrān}{
  \Afootnote{tīvrā A.}
} \edtext{nayanāmayāṃś}{
  \Afootnote{°mayāś A.}
} ca || 
\pend

 
\pstart

                         \textsc{[1938 ed. 5.3.17]}
                        \caesura \edtext{lākṣāharidrātiviṣābhayāś}{
  \Afootnote{°bhayābda  A.}
} ca \caesura\edlabel{SS.5.3.17-2} savakrakuṣṭhailahareṇukābhiḥ\edlabel{SS.5.3.17-3} \edtext{| \caesura}{
  \linenum{|\xlineref{SS.5.3.17-2}}\lemma{ca \ldots | }\Afootnote{hareṇukailādalavakrakuṣṭham |  A.}
  \linenum{|\xlineref{SS.5.3.17-3}}\lemma{savakrakuṣṭhailahareṇukābhiḥ | }\Afootnote{savakra\textbf{kuṣṭhātiviṣāhareṇu |1}kuṣ° H; °kuṣṭhātiviṣāhareṇu |  K.}
} \edtext{priyaṅgavaś}{
  \Afootnote{priyaṅgukāṃ A.}
} cāpy\edlabel{SS.5.3.17-6} \edtext{anile}{
  \linenum{|\xlineref{SS.5.3.17-6}}\lemma{cāpy anile}\Afootnote{cāpyanale A; cāpyanile K.}
} nidhāya \caesura dhūmānilau \edtext{tena}{
  \Afootnote{cāpi A.}
} viśodhayīta\edlabel{SS.5.3.17-11} \edtext{||}{
  \linenum{|\xlineref{SS.5.3.17-11}}\lemma{viśodhayīta ||}\Afootnote{viśodhayeta || A.}
}
\pend

 
\pstart

                         \textsc{[1938 ed. 5.3.18]}
                        \caesura \edtext{prajā}{
  \Afootnote{prajām A \uline{K}.}
} \edtext{imāḥ}{
  \Afootnote{imām A; imāḥ vrahmaṇaḥ K.}
} \edtext{padmayoner}{
  \Afootnote{ātmayoner A.}
} \edtext{brahmaṇaḥ}{
  \Afootnote{\textsc{[om]} K.}
} sṛjataḥ kila\edlabel{SS.5.3.18-6} | \caesura \edtext{akarod}{
  \linenum{|\xlineref{SS.5.3.18-6}}\lemma{kila\ldots akarod}\Afootnote{kilaḥ |  aka° H.}
\lemma{akarod}  \Afootnote{\textsc{[add]} asuro A.}
} vighnam \edtext{asuraḥ}{
  \Afootnote{\textsc{[om]} A.}
} kaiṭabho\edlabel{SS.5.3.18-11} \edtext{nāma}{
  \linenum{|\xlineref{SS.5.3.18-11}}\lemma{kaiṭabho nāma}\Afootnote{kaiṭabho \textbf{vala}nāma H.}
} darpitaḥ || 
\pend

 
\pstart

                         \textsc{[1938 ed. 5.3.19]}
                        \caesura \edtext{tataḥ}{
  \Afootnote{tasya A.}
} \edtext{kruddhasya}{
  \Afootnote{\textsc{[add]} vai A.}
} \edtext{vadanād}{
  \Afootnote{vaktrād A.}
} brahmaṇas \edtext{tejasām}{
  \Afootnote{tejaso A.}
} nidheḥ | \caesura krodho \edtext{vigrahavān}{
  \Afootnote{vigrahavāṃ K.}
} bhūtvā niṣpapātātidāruṇaḥ\edlabel{SS.5.3.19-11} \edtext{||}{
  \linenum{|\xlineref{SS.5.3.19-11}}\lemma{niṣpapātātidāruṇaḥ ||}\Afootnote{nipapā° A; °tidāurṇaḥ || K.}
}
\pend

 
\pstart

                         \textsc{[1938 ed. 5.3.20]}
                        \caesura sa \edtext{tan}{
  \Afootnote{taṃ A.}
} dadāha garjantam antakābham mahāsuram\edlabel{SS.5.3.20-6} \edtext{| \caesura}{
  \linenum{|\xlineref{SS.5.3.20-6}}\lemma{mahāsuram | }\Afootnote{mahābalam |  A.}
} tato 'suraṃ ghātayitvā \edtext{tattejo}{
  \Afootnote{tatte\uwave{jo} K.}
} 'vardhatādbhutam\edlabel{SS.5.3.20-12} \edtext{|}{
  \linenum{|\xlineref{SS.5.3.20-12}}\lemma{'vardhatādbhutam |}\Afootnote{va° K.}
}
\pend

 
\pstart

                         \textsc{[1938 ed. 5.3.21]}
                        \caesura tato viṣādo \edtext{daityānām}{
  \Afootnote{devānām A.}
} abhavat \edtext{tan}{
  \Afootnote{taṃ A.}
} nirīkṣya vai | \caesura viṣādajananatvāc ca viṣam ity abhidhīyate || 
\pend

 
\pstart

                         \textsc{[1938 ed. 5.3.22]}
                        \caesura tataḥ sṛṣṭvā prajāḥ \edtext{paścāt}{
  \Afootnote{śeṣaṃ A.}
} tadā taṃ krodham īśvaraḥ | \caesura \edtext{vyāveśayata}{
  \Afootnote{vinyastavān sa A.}
} \edtext{bhūteṣu}{
  \Afootnote{bhūtesu A.}
} sthāvareṣu careṣu ca || 
\pend

 
\pstart

                         \textsc{[1938 ed. 5.3.23]}
                        \caesura \edtext{yathāvyaktarasan}{
  \Afootnote{yathāvya\textbf{kta}rasan K; °rasaṃ A.}
} toyam antarīkṣāt mahīgatam | \caesura teṣu teṣu pradeśeṣu rasaṃ \edtext{tan}{
  \Afootnote{taṃ A.}
} \edtext{tan}{
  \Afootnote{taṃ A.}
} nigacchati\edlabel{SS.5.3.23-12} \edtext{||}{
  \linenum{|\xlineref{SS.5.3.23-12}}\lemma{nigacchati ||}\Afootnote{niyacchati || A; ni\uuline{ya} \textbf{ga}cchati || H.}
}
\pend

 
\pstart

                         \textsc{[1938 ed. 5.3.24]}
                        \caesura evam \edtext{eva}{
  \Afootnote{eṣa K.}
} \edtext{viṣaṃ}{
  \Afootnote{\uwave{viṣaṃ} K.}
} \edtext{yad}{
  \Afootnote{sadya K; sadyo H.}
} \edtext{yad}{
  \Afootnote{\textsc{[om]} Nep.}
} dravyam prāpyāvatiṣṭhate\edlabel{SS.5.3.24-7} \edtext{| \caesura}{
  \linenum{|\xlineref{SS.5.3.24-7}}\lemma{prāpyāvatiṣṭhate | }\Afootnote{vyāpyā° A.}
} svabhāvād eva \edtext{tat}{
  \Afootnote{taṃ A; tan H.}
} tasya rasaṃ samanuvartate | 
\pend

 
\pstart

                         \textsc{[1938 ed. 5.3.25]}
                        \caesura viṣe yasmād guṇāḥ sarve tīkṣṇāḥ \edtext{prāyeṇa}{
  \Afootnote{prāyena K.}
} santi vai\edlabel{SS.5.3.25-8} \edtext{| \caesura}{
  \linenum{|\xlineref{SS.5.3.25-8}}\lemma{vai | }\Afootnote{hi |  A.}
} viṣaṃ sarvam ato jñeyaṃ sarvadoṣaprakopanam\edlabel{SS.5.3.25-14} \edtext{|}{
  \linenum{|\xlineref{SS.5.3.25-14}}\lemma{sarvadoṣaprakopanam |}\Afootnote{°paṇam || A.}
}
\pend

 
\pstart

                         \textsc{[1938 ed. 5.3.26]}
                        \caesura te tu \edtext{vṛttīḥ}{
  \Afootnote{vṛttiṃ A.}
} prakupitā jahati \edtext{svā}{
  \Afootnote{svāṃ A.}
} viṣārditāḥ \caesura nopayāti viṣam pākam ataḥ \edtext{prāṇān}{
  \Afootnote{prāṇāṃ K.}
} ruṇaddhi ca | 
\pend

 
\pstart

                         \textsc{[1938 ed. 3.27]}
                        \caesura śleṣmaṇāvṛtamārgatvād \edtext{ucchvāso}{
  \Afootnote{ucchvāso 'sya A.}
} vinivāryate\edlabel{SS.5.3.27-3} \edtext{| \caesura}{
  \linenum{|\xlineref{SS.5.3.27-3}}\lemma{vinivāryate | }\Afootnote{nirudhyate |  A.}
} visaṃjñaḥ sati jīve 'pi tasmāt tiṣṭhati mānavaḥ | 
\pend

 
\pstart

                         \textsc{[1938 ed. 5.3.28]}
                        \caesura śukravat sarvasarpāṇām viṣaṃ sarvaśarīragam | \caesura kruddhānām eti cāṅgebhyaḥ śukraṃ \edtext{nirmathanād}{
  \Afootnote{nirmantha° A.}
} iva || 
\pend

 
\pstart

                         \textsc{[1938 ed. 5.3.29]}
                        \caesura teṣām baḍiśavad \edtext{daṃṣṭrā}{
  \Afootnote{daṃṣṭrās A; draṃṣṭrā H.}
} tāsu sajjati cāgatam | \caesura \edtext{anudvṛttam}{
  \Afootnote{anudvṛttā A.}
} viṣaṃ tasmān na \edtext{vimuñcati}{
  \Afootnote{muñcanti ca A.}
} bhoginaḥ | 
\pend

 
\pstart

                         \textsc{[1938 ed. 5.3.30]}
                        \caesura yasmād atyartham uṣṇañ ca tīkṣṇañ ca paṭhitaṃ viṣam | \caesura ataḥ \edtext{sarvaviṣeṣūktaḥ}{
  \Afootnote{sarvaviśeṣūktaḥ K.}
} pariṣekaḥ suśītalaḥ\edlabel{SS.5.3.30-13} \edtext{|}{
  \linenum{|\xlineref{SS.5.3.30-13}}\lemma{suśītalaḥ |}\Afootnote{tu śītalaḥ || A.}
}
\pend

 
\pstart

                         \textsc{[1938 ed. 5.3.31]}
                        \caesura \edtext{kīṭeṣu}{
  \Afootnote{\textsc{[om]} A.}
} \edtext{mandaṃ}{
  \Afootnote{\textsc{[add]} kīṭeṣu A.}
} nātyuṣṇam bahuvātakapham viṣam | \caesura ataḥ kīṭaviṣe cāpi svedo na pratiṣidhyate | 
\pend

 
\pstart
\edtext{}{
  \Afootnote{\textsc{[pre]} kīṭair daṣṭān ugraviṣaiḥ sarpavat samupācaret | A.}
}
\pend

 
\pstart

                         \textsc{[1938 ed. 5.3.32cd]}
                        \caesura svabhāvād \edtext{avatiṣṭheta}{
  \Afootnote{eva tiṣṭhet tu A.}
} \edtext{prahārādaṃśayor}{
  \Afootnote{praharā\textbf{t$\^$daṃ° H.}
}} viṣam ||
 \pend

 
\pstart
                         \textsc{[1938 ed. 5.3.33]}
                        \caesura \edtext{prakhyāpya}{
  \Afootnote{vyāpya sāvayavaṃ A; pra\uwave{khyā}\textsc{(l. 2)}pya K.}
} deham \edtext{mṛtayor}{
  \Afootnote{\textsc{[om]} A.}
} digdhaviddhāhidaṣṭayoḥ\edlabel{SS.5.3.33-4} \edtext{| \caesura}{
  \linenum{|\xlineref{SS.5.3.33-4}}\lemma{digdhaviddhāhidaṣṭayoḥ | }\Afootnote{°ddhābhidaṣṭayoḥ |  K; °ddhātidaṣṭayoḥ ||  H.}
} laulyād \edtext{viṣārditam}{
  \Afootnote{viṣānvitaṃ A.}
} māṃsaṃ yaḥ khāden mṛtamātrayoḥ | 
\pend

 
\pstart

                         \textsc{[1938 ed. 5.3.34]}
                        \caesura \edtext{yathāviṣaṃ}{
  \Afootnote{yathāvi\uuline{ā}\textbf{a}ṣaṃ K.}
} sa rogeṇa kliśyate mriyate pi vā | \caesura ataś cāpy anayor māṃsam abhakṣyam mṛtamātrayoḥ | 
\pend

 
\pstart

                         \textsc{[1938 ed. 5.3.35]}
                        \caesura muhūrtāt tad upādeyam prahārādaṃśavarjitam
 |\edlabel{SS.5.3.35ab-4} 
\pend

 
\pstart

                         \textsc{[5.3.35.1]}
                        \caesura kṣīṇakṣate \edtext{}{
  \Afootnote{garbbhiṇi H.}
} garbhiṇī kuṣṭhimehirūkṣeṣu deheṣv abaleṣu caiva | 
 
\pend

 
\pstart

                         \textsc{[5.3.35.2]}
                        \caesura
                        \caesura \edtext{saran}{
  \Afootnote{\textsc{[om]} K.}
} \edtext{tu}{
  \Afootnote{\textsc{[om]} K.}
} \edtext{saukṣmyataikṣṇyoṣṇyād}{
  \Afootnote{\textbf{\#} \textbf{\textbf{\#}\textsc{(gap of 7 chars, 
  damaged)}\uwave{taikṣṇyoṣṇyād} } K.}
} vikāsitvāt tathaiva ca \edtext{|
}{
  \Afootnote{\textsc{[add]} K.}
}
\pend

 
\pstart

                         \textsc{[5.3.35.3]}
                        \caesura viṣam etair guṇair yuktaṃ kṣate samanudhāvati \caesura vātātapābhyāṃ nihataṃ \edtext{nirvīryam}{
  \Afootnote{nirvvīyam H.}
} upajāyate | \caesura tasmād viṣahataṃ sarvam bhakṣitan tu na mārayet
 
\pend

 
\pstart

                         \textsc{[1938 ed. 5.3.36ab]}
                        \caesura savātaṃ gṛhadhūmābhaṃ purīṣaṃ yo 'tisāryate | 
\pend

 
\pstart
\edtext{}{
  \Afootnote{\textsc{[pre]} ādhmāto 'tyartham uṣṇāsro vivarṇaḥ sādapīḍitaḥ | A.}
}
\pend

 
\pstart

                         \textsc{[1938 ed. 5.3.36cd]}
                        \caesura \edtext{phenam}{
  \Afootnote{\textsc{[om]} A.}
} \edtext{udvamate}{
  \Afootnote{udvamaty atha phenaṃ A.}
} \edtext{cāpi}{
  \Afootnote{ca A.}
} \edtext{viṣapītan}{
  \Afootnote{viṣapītaṃ A.}
} tam ādiśet\edlabel{SS.5.3.36cd-6} \edtext{||}{
  \linenum{|\xlineref{SS.5.3.36cd-6}}\lemma{ādiśet ||}\Afootnote{āviśet || K.}
}
\pend

 
\pstart

                         \textsc{[1938 ed. 5.3.37]}
                        \caesura \edtext{viṣavyāptam}{
  \Afootnote{na cāsya A.}
} \edtext{ato}{
  \Afootnote{hṛdayaṃ A.}
} \edtext{hy}{
  \Afootnote{\textsc{[om]} A.}
} \edtext{agnir}{
  \Afootnote{vahnir A; agni Nep.}
} \edtext{hṛdayaṃ}{
  \Afootnote{viṣajuṣṭaṃ A.}
} \edtext{nirdahaty}{
  \Afootnote{dahaty A.}
} api | \caesura tad dhi sthānañ cetanāyāḥ svabhāvād vyāpya tiṣṭhati || 
\pend

 
\pstart

                         \textsc{[1938 ed. 5.3.38]}
                        \caesura aśvatthadevāyatanaśmaśāna \caesura valmīkasandhyāsu catuṣpatheṣu | \caesura yāmye \edtext{ca}{
  \Afootnote{sapitrye A.}
} \edtext{daṣṭāḥ}{
  \Afootnote{\textsc{[om]} A; dṛṣṭāḥ H.}
} \edtext{parivarjanīyāḥ \caesura}{
  \Afootnote{°nīyā  A; °nīyā\uwave{ḥ}\textsc{(l. 3)}  H.}
} ṛkṣe narā marmasu ye ca daṣṭāḥ\edlabel{SS.5.3.38-14} \edtext{|
}{
  \linenum{|\xlineref{SS.5.3.38-14}}\lemma{daṣṭāḥ |}\Afootnote{\uuline{da}\textbf{dṛ}ṣṭāḥ || H.}
}
\pend

 
\pstart

                         \textsc{[1938 ed. 5.3.39]}
                        \caesura darvīkarāṇām viṣam \edtext{āśughāti \caesura}{
  \Afootnote{āśu\uuline{ghā} \textbf{haṃ}ti  H.}
} sarvāṇi \edtext{corjjadviguṇaṃ\edlabel{SS.5.3.39-5}}{
  \Afootnote{corjjaṃ dvi° H.}
} labhante \edtext{| \caesura}{
  \linenum{|\xlineref{SS.5.3.39-5}}\lemma{corjjadviguṇaṃ\ldots | }\Afootnote{coṣṇe dviguṇībhavanti |  A.}
} \edtext{ajīrṇapittānilapīḍiteṣu \caesura}{
  \Afootnote{°ttātapapīḍiteṣu  A.}
} \edtext{vṛddheṣu}{
  \Afootnote{\textsc{[om]} A.}
} \edtext{bāleṣu}{
  \Afootnote{bālapramehiṣv atha A.}
} bubhukṣiteṣu\edlabel{SS.5.3.39-11} \edtext{|
}{
  \linenum{|\xlineref{SS.5.3.39-11}}\lemma{bubhukṣiteṣu |}\Afootnote{garbhiṇīṣu || A.}
}
\pend

 
\pstart

                         \textsc{[5.3.39 add 1]}
                        \caesura unmattamatteṣu
 bhayārditeṣu \caesura tīkṣī \edtext{bhavet}{
  \Afootnote{tīkṣṇī H.}
} bhinnavalāsaheṣu \caesura kṣīṇakṣate garbhiṇi\edlabel{SS.5.3.39.add-1-6} kuṣṭhim ehi \caesura rūkṣeṣu deheṣv avaleṣu caiva
 
\pend

 
\pstart

                         \textsc{[SS.5.3.39.add-2]}
                        \caesura saran tu \edtext{saukṣmyatair}{
  \Afootnote{\uwave{sau}° H.}
} \edtext{ślakṣṇyād}{
  \Afootnote{ślakṣṇyād H.}
} vikāsitvāt tathaiva ca | \caesura \edtext{viṣamatair}{
  \Afootnote{viṣam atair H.}
} gguṇair yuktaṃ kṣate samanudhāvati || \caesura vātātapābhyān nihatann ivīyam upajāyate | \caesura tasmād viṣahataṃ sarvvam bhakṣitan tu na mārayet || 
 
\pend

 
\pstart
\edtext{}{
  \Afootnote{\textsc{[pre]} vṛddhāturakṣīṇabubhukṣiteṣu  rūkṣeṣu bhīruṣv atha durdineṣu | A.}
}
\pend

 
\pstart

                         \textsc{[1938 ed. 5.3.40cd]}
                        \caesura \edtext{śastrakṣate}{
  \Afootnote{śastraṃ kṣate H.}
} yasya na raktam \edtext{asti \caesura}{
  \Afootnote{eti  A.}
} rājyo latābhiś ca na saṃbhavanti |
 
\pend

 
\pstart

                         \textsc{[1938 ed. 5.3.41]}
                        \caesura śītābhir adbhiś ca na romaharṣo \caesura \edtext{viṣābhibhūtam}{
  \Afootnote{viṣā\uwave{bhi}\textsc{(l. 5)}bhūtaṃ K.}
} parivarjayet tam | \caesura \edtext{jihmam}{
  \Afootnote{jihvā A.}
} \edtext{mukhaṃ}{
  \Afootnote{sitā A.}
} yasya ca keśaśāto \caesura nā\edlabel{SS.5.3.41-15} \edtext{sāvasādaś}{
  \linenum{|\xlineref{SS.5.3.41-15}}\lemma{nā sāvasādaś}\Afootnote{nāsāvabhaṅgaś A.}
} ca sakaṇṭhabhaṅgaḥ\edlabel{SS.5.3.41-18} \edtext{|
}{
  \linenum{|\xlineref{SS.5.3.41-18}}\lemma{sakaṇṭhabhaṅgaḥ |}\Afootnote{°bhaṅgāḥ | Nep.}
}
\pend

 
\pstart

                         \textsc{[1938 ed. 5.3.42]}
                        \caesura \edtext{kṛṣṇaḥ}{
  \Afootnote{\textsc{(l. 1)}kṛṣṇa H.}
} saraktaḥ śvayathuś ca daṃśe \caesura \edtext{hanvoḥ}{
  \Afootnote{hanvo Nep.}
} \edtext{sthiratvaṃ}{
  \Afootnote{\textsc{[add]} ca A.}
} sa visarjanīyaḥ\edlabel{SS.5.3.42-9} \edtext{| \caesura}{
  \linenum{|\xlineref{SS.5.3.42-9}}\lemma{visarjanīyaḥ | }\Afootnote{vivarjja° H; varja° A.}
} vartir ghanā yasya nireti vaktrād \caesura raktaṃ sraved ūrdhvam adhaś ca yasya |
 
\pend

 
\pstart

                         \textsc{[1938 ed. 5.3.43ab]}
                        \caesura \edtext{daṃṣṭrānipātaś}{
  \Afootnote{°pātāḥ A.}
} \edtext{caturaś}{
  \Afootnote{sakalāś A.}
} ca yasya \caesura tañ cāpi vaidyaḥ parivarjayīteti\edlabel{SS.5.3.43ab-8} \edtext{||}{
  \linenum{|\xlineref{SS.5.3.43ab-8}}\lemma{parivarjayīteti ||}\Afootnote{°rjayet tu | A.}
} 
 
\pend

 
\pstart
\edtext{}{
  \Afootnote{\textsc{[pre]} unmattam atyartham upadrutaṃ vā  hīnasvaraṃ vā 'py athavā vivarṇam || A.}
}
\pend

 
\pstart
\edtext{}{
  \Afootnote{\textsc{[pre]} sāriṣṭam atyartham aveginaṃ ca  jahyān naraṃ tatra na karma kuryāt || A.}
}
\pend

\chapter{Kalpasthāna 4: Snakes and Envenomation}
 
\pstart

                         \textsc{[1938 ed. 5.4.1]}
                        \caesura athātaḥ sarpa daṣṭa \edtext{viṣa}{
  \Afootnote{\textsc{[om]} H.}
} vijñānīyaṃ kalpaṃ vyākhyāsyāmaḥ\edlabel{SS.5.4.1-7} \edtext{||}{
  \linenum{|\xlineref{SS.5.4.1-7}}\lemma{vyākhyāsyāmaḥ ||}\Afootnote{vyākhyāsyāmaḥ || K.}
}
\pend

 
\pstart
\edtext{}{
  \Afootnote{\textsc{[pre]} yathovāca bhagavān dhanvanatariḥ || A.}
}
\pend

 
\pstart

                         \textsc{[1938 ed. 5.4.3]}
                        \caesura dhanvantariṃ mahāprājñaṃ sarvaśāstraviśāradam | \caesura \edtext{caraṇāv}{
  \Afootnote{pādayor A.}
} upasaṅgṛhya suśrutaḥ paripṛcchati || 
\pend

 
\pstart

                         \textsc{[1938 ed. 5.4.4]}
                        \caesura sarpasaṃkhyāṃ\edlabel{SS.5.4.4-1} \edtext{vibhāgaṃ}{
  \linenum{|\xlineref{SS.5.4.4-1}}\lemma{sarpasaṃkhyāṃ vibhāgaṃ}\Afootnote{sarpasaṃkhyāvi° H.}
} ca daṣṭalakṣaṇam eva ca | \caesura jñānañ ca viṣavegānāṃ \edtext{bhagavan}{
  \Afootnote{bhagavaṃ K.}
} vaktum arhasi | 
\pend

 
\pstart

                         \textsc{[1938 ed. 5.4.5]}
                        \caesura tasya tad \edtext{vacanaṃ}{
  \Afootnote{va\uuline{va}canaṃ K.}
} śrutvā prābravīd bhiṣajām varaḥ | \caesura \edtext{asaṃkhyeyā}{
  \Afootnote{asaṃkhyā vāsukiśreṣṭhā A.}
} \edtext{mahātmāno}{
  \Afootnote{vikhyātās A; mahātmā\uwave{no} H.}
} vāsukītakṣakādayaḥ\edlabel{SS.5.4.5-11} \edtext{|
}{
  \linenum{|\xlineref{SS.5.4.5-11}}\lemma{vāsukītakṣakādayaḥ |}\Afootnote{\textsc{[om]} vāsukī° A.}
}
\pend

 
\pstart

                         \textsc{[1938 ed. 5.4.6]}
                        \caesura mahīdharāś ca nāgendrāḥ hutāgni samavarcasaḥ\edlabel{SS.5.4.6-5} \edtext{| \caesura}{
  \linenum{|\xlineref{SS.5.4.6-5}}\lemma{samavarcasaḥ | }\Afootnote{samatejasaḥ |  A.}
} ye cāpy ajasraṃ garjanti varṣanti ca tapanti ca | 
\pend

 
\pstart

                         \textsc{[1938 ed. 5.4.7]}
                        \caesura sasāgaragiridvīpā yaiś \edtext{ca}{
  \Afootnote{iyaṃ A.}
} \edtext{sandhāryate}{
  \Afootnote{dhāryate A.}
} mahī | \caesura kruddhā \edtext{niśvāsadṛṣṭibhyāṃ}{
  \Afootnote{niḥśvā° A.}
} ye hanyur akhilaṃ jagat | 
\pend

 
\pstart

                         \textsc{[1938 ed. 5.4.8]}
                        \caesura namas \edtext{tebhyo}{
  \Afootnote{tebhyo 'sti A; tebhyo(From 148r)\textsc{(l. 1)}\uwave{} K.}
} \edtext{na}{
  \Afootnote{no A.}
} \edtext{taiḥ}{
  \Afootnote{\textsc{[om]} A.}
} \edtext{kiñcit}{
  \Afootnote{teṣāṃ A; kiñcit | K.}
} kāryam \edtext{atra}{
  \Afootnote{kiñcic A.}
} cikitsayā | \caesura ye tu daṃṣṭrāviṣā bhaumā ye \edtext{daśanti}{
  \Afootnote{daśaṃte K.}
} ca mānavān\edlabel{SS.5.4.8-17} \edtext{||}{
  \linenum{|\xlineref{SS.5.4.8-17}}\lemma{mānavān ||}\Afootnote{mānuṣān || A.}
}
\pend

 
\pstart

                         \textsc{[1938 ed. 5.4.9]}
                        \caesura teṣāṃ saṃkhyāṃ pravakṣyāmi yathāvad anupūrvaśaḥ | \caesura \edtext{aśītir}{
  \Afootnote{\textsc{[add]} tv A.}
} eva sarpāṇāṃ bhidyate te\edlabel{SS.5.4.9-11} \edtext{tu}{
  \linenum{|\xlineref{SS.5.4.9-11}}\lemma{te tu}\Afootnote{\textsc{[om]} A.}
} pañcadhā \edtext{|}{
  \Afootnote{\textsc{[add]} tu sā || A.}
}
\pend

 
\pstart

                         \textsc{[1938 ed. 5.4.10]}
                        \caesura darvīkarā \edtext{maṇḍalino}{
  \Afootnote{maṇḍalinā H.}
} \edtext{rājīmantas}{
  \Afootnote{rājimantas A; rājīmanta\textbf{\uwave{}}s K.}
} tathaiva ca || \caesura nirviṣā vaikarañjāś ca trividhās te punaḥ smṛtāḥ | 
\pend

 
\pstart

                         \textsc{[1938 ed. 5.4.11]}
                        \caesura \edtext{viṃśatiḥ}{
  \Afootnote{darvīkarā A.}
} \edtext{phaṇinas}{
  \Afootnote{maṇḍalino A; phaninas K.}
} teṣāṃ\edlabel{SS.5.4.11-3} \edtext{ṣaṭ}{
  \linenum{|\xlineref{SS.5.4.11-3}}\lemma{teṣāṃ ṣaṭ}\Afootnote{rājimantaś A.}
\lemma{ṣaṭ}  \Afootnote{\uwave{ṣaṭ} H.}
} \edtext{ca}{
  \Afootnote{vā H.}
} \edtext{maṇḍalinaḥ}{
  \Afootnote{pannagāḥ |  A.}
} punaḥ\edlabel{SS.5.4.11-7} \edtext{| \caesura}{
  \linenum{|\xlineref{SS.5.4.11-7}}\lemma{punaḥ | }\Afootnote{teṣu A; puna |  H.}
} tāvanta\edlabel{SS.5.4.11-9} \edtext{eva}{
  \linenum{|\xlineref{SS.5.4.11-9}}\lemma{tāvanta eva}\Afootnote{darvīkarā A.}
} \edtext{vijñeyā}{
  \Afootnote{jñeyā A.}
} \edtext{rājīmantas}{
  \Afootnote{viṃśatiḥ A.}
} trayodaśa\edlabel{SS.5.4.11-13} \edtext{|}{
  \linenum{|\xlineref{SS.5.4.11-13}}\lemma{trayodaśa |}\Afootnote{ṣaṭ ca pannagāḥ || A.}
}
\pend

 
\pstart
\edtext{
                         \textsc{[1938 ed. 5.4.12]}
                        \caesura}{
  \Afootnote{\textsc{[pre]} dvāviṃśatir maṇḍalino rājimantas tathā daśa |  A.}
} nirviṣā dvādaśa \edtext{proktā}{
  \Afootnote{jñeyā A.}
} vaikarañjās \edtext{trayaḥ}{
  \Afootnote{traya K.}
} smṛtāḥ\edlabel{SS.5.4.12-6} \edtext{|}{
  \linenum{|\xlineref{SS.5.4.12-6}}\lemma{smṛtāḥ |}\Afootnote{tathā || A.}
}
\pend

 
\pstart
\edtext{
                         \textsc{[1938 ed. 5.4.13]}
                        \caesura}{
  \Afootnote{\textsc{[pre]} vaikarañjodbhavāḥ sapta citrā maṇḍalirājilāḥ |  A.}
} pādābhimṛṣṭā duṣṭā vā kruddhā grāsārthino 'pi vā | 
\pend

 
\pstart

                         \textsc{[1938 ed. 5.4.14]}
                        \caesura te daśanti mahākrodhās \edtext{tac}{
  \Afootnote{\textsc{[om]} A.}
} \edtext{ca}{
  \Afootnote{\textsc{[om]} A.}
} trividham ucyate\edlabel{SS.5.4.14-7} \edtext{|| \caesura}{
  \linenum{|\xlineref{SS.5.4.14-7}}\lemma{ucyate || }\Afootnote{bhīmadarśanāḥ |  A.}
} \edtext{sarpitan}{
  \Afootnote{sarpitaṃ A.}
} \edtext{daritam}{
  \Afootnote{raditaṃ A.}
} \edtext{vāpi}{
  \Afootnote{cāpi A.}
} tṛtīyam atha nirviṣam | \caesura \edtext{sarpagātrāhataṅ}{
  \Afootnote{sarpa <grā> /<srā> \textsc{(l. 3)}trā° K; sarpāṅgābhihataṃ A.}
} kecid icchanti khalu tadvidaḥ | 
\pend

 
\pstart

                         \textsc{[1938 ed. 5.4.15]}
                        \caesura padāni yatra \edtext{dantānām}{
  \Afootnote{dantānā\uuline{ṃ}m K.}
} \edtext{ekaṃ}{
  \Afootnote{eka\uwave{} K.}
} \edtext{dve}{
  \Afootnote{dve K.}
} vā bahūni vā | \caesura nimagnāny alparaktāni yāny \edtext{udvṛttaḥ}{
  \Afootnote{udvṛtya A.}
} karoti ca\edlabel{SS.5.4.15-15} \edtext{|}{
  \linenum{|\xlineref{SS.5.4.15-15}}\lemma{ca |}\Afootnote{hi || A; caḥ | K.}
}
\pend

 
\pstart

                         \textsc{[1938 ed. 5.4.16]}
                        \caesura \edtext{cuñcumālakayuktāni}{
  \Afootnote{cañcu° A.}
} vaikṛtyakaraṇāni ca | \caesura saṃkṣiptāni saśophāni vidyāt tat sarpitaṃ bhiṣak | 
\pend

 
\pstart

                         \textsc{[1938 ed. 5.4.17]}
                        \caesura \edtext{rājyāḥ}{
  \Afootnote{rājyaḥ A.}
} salohitā yatra \edtext{nīlā}{
  \Afootnote{nīlāḥ A.}
} \edtext{vā}{
  \Afootnote{pītāḥ A.}
} yadi\edlabel{SS.5.4.17-6} \edtext{vā}{
  \linenum{|\xlineref{SS.5.4.17-6}}\lemma{yadi vā}\Afootnote{sitās A.}
} sitā\edlabel{SS.5.4.17-8} \edtext{| \caesura}{
  \linenum{|\xlineref{SS.5.4.17-8}}\lemma{sitā | }\Afootnote{tathā |  A.}
} \edtext{vijñeyan}{
  \Afootnote{vijñeyaṃ A.}
} \edtext{daritan}{
  \Afootnote{raditaṃ A.}
} tat tu bhiṣajālpaviṣānvitam\edlabel{SS.5.4.17-14} \edtext{|}{
  \linenum{|\xlineref{SS.5.4.17-14}}\lemma{bhiṣajālpaviṣānvitam |}\Afootnote{viṣa° H; jñeyam alpaviṣaṃ A.}
\lemma{|}  \Afootnote{\textsc{[add]} ca tat || A.}
}
\pend

 
\pstart

                         \textsc{[1938 ed. 5.4.18]}
                        \caesura aśopham alpaduṣṭāsṛk prakṛtisthasya dehinaḥ | \caesura padaṃ padāni vā vidyād \edtext{aviṣāṇi}{
  \Afootnote{aviṣaṃ tac A.}
} cikitsakaḥ | 
\pend

 
\pstart

                         \textsc{[1938 ed. 5.4.19]}
                        \caesura sarpaspṛṣṭasya bhīror hi bhayena kupito 'nilaḥ | \caesura kasyacit kurute śophaṃ \edtext{sarpagātrāhatan}{
  \Afootnote{sarpāṅgābhihataṃ A.}
} tu tat | 
\pend

 
\pstart

                         \textsc{[1938 ed. 4.20]}
                        \caesura vyādhitodvignadaṣṭāni jñeyāny alpaviṣāṇi tu || \caesura tathātibāla\edlabel{SS.5.4.20-6} vṛddhānāṃ\edlabel{SS.5.4.20-7} \edtext{daṣṭam}{
  \linenum{|\xlineref{SS.5.4.20-6}}\lemma{tathātibāla\ldots daṣṭam}\Afootnote{tathātivṛddha bālābhidaṣṭam A.}
  \linenum{|\xlineref{SS.5.4.20-7}}\lemma{vṛddhānāṃ daṣṭam}\Afootnote{vṛddhadaṣṭam K.}
} alpaviṣaṃ smṛtam || 
\pend

 
\pstart

                         \textsc{[1938 ed. 4.21]}
                        \caesura suparṇṇadevabrahmarṣi \edtext{bhūta}{
  \Afootnote{yakṣa A.}
} siddhaniṣevite | \caesura \edtext{viṣaghnauṣadhajuṣṭe}{
  \Afootnote{viṣaghnauṣadhiyukte A.}
} ca deśe na kramate viṣam | 
\pend

 
\pstart

                         \textsc{[1938 ed. 4.22]}
                        \caesura rathāṅgalāṅgalacchatrasvastikāṃkuśadhāriṇaḥ | \caesura jñeyā darvīkarāḥ sarpāḥ phaṇinaḥ śīghragāminaḥ | 
\pend

 
\pstart

                         \textsc{[1938 ed. 5.4.23]}
                        \caesura maṇḍalair vividhaiś \edtext{cittrāḥ}{
  \Afootnote{ci\textsc{(l. 6)}\uwave{ttrāḥ} K; citrā H.}
} pṛthavo mandagāminaḥ | \caesura jñeyā maṇḍalinaś \edtext{cāpi}{
  \Afootnote{sarpā A.}
} jvalitāgnisamā\edlabel{SS.5.4.23-10} viṣaiḥ \edtext{|}{
  \linenum{|\xlineref{SS.5.4.23-10}}\lemma{jvalitāgnisamā\ldots |}\Afootnote{jvalanārkasamaprabhāḥ || A.}
}
\pend

 
\pstart

                         \textsc{[1938 ed. 5.4.24]}
                        \caesura snigdhā vividhavarṇṇābhis tiryag \edtext{ūrdhvañ}{
  \Afootnote{ūrddhañ H.}
} ca rājibhiḥ | \caesura \edtext{vicitrā}{
  \Afootnote{citritā A.}
} iva ye bhānti \edtext{rājīmantas}{
  \Afootnote{rājimantas A.}
} tu te smṛtāḥ | 
\pend

 
\pstart

                         \textsc{[1938 ed. 5.4.25]}
                        \caesura muktārūpyaprabhā ye ca kapilā ye ca pannagāḥ | \caesura \edtext{suvarṇṇābhāḥ}{
  \Afootnote{sugandhayaḥ A.}
} \edtext{sugandhāś}{
  \Afootnote{suvarṇābhās A.}
} \edtext{ca}{
  \Afootnote{\textsc{[om]} A.}
} te jātyā brāhmaṇāḥ smṛtāḥ 
\pend

 
\pstart

                         \textsc{[1938 ed. 5.4.26]}
                        \caesura kṣatriyāḥ snigdhavarṇṇās tu pannagā bhṛśakopanāḥ | \caesura sūryaś\edlabel{SS.5.4.26-7} candraḥ kṣitiś chatraṃ \edtext{lakṣyaṃ}{
  \linenum{|\xlineref{SS.5.4.26-7}}\lemma{sūryaś\ldots lakṣyaṃ}\Afootnote{sūryacandrākṛticchatralakṣma A.}
} \edtext{teṣān}{
  \Afootnote{teṣāṃ A.}
} tathādrijam\edlabel{SS.5.4.26-13} \edtext{||}{
  \linenum{|\xlineref{SS.5.4.26-13}}\lemma{tathādrijam ||}\Afootnote{tathā 'mbujam || A.}
} \caesura 
\pend

 
\pstart

                         \textsc{[1938 ed. 5.4.27]}
                        \caesura kṛṣṇā \edtext{vajraprabhā}{
  \Afootnote{vajranibhā A.}
} ye ca lohitā varṇṇatas tathā | \caesura dhūmrāḥ pārāvatābhāś ca vaiśyās te pannagāḥ smṛtāḥ || 
\pend

 
\pstart

                         \textsc{[1938 ed. 4.28]}
                        \caesura \edtext{mahiṣadvīpivarṇṇābhās}{
  \Afootnote{(From 148v)\textsc{(l. 1)}mahiṣa\uuline{dvi}\textbf{dvī}pi° K; °pi\textbf{dvija}varṇṇābhās H.}
} tathaiva paruṣatvacaḥ |
                        \caesura
 bhinnavarṇṇāś ca ye kecic chūdrās\edlabel{SS.5.4.28-8} \edtext{te}{
  \linenum{|\xlineref{SS.5.4.28-8}}\lemma{chūdrās te}\Afootnote{kecic chūdrā A.}
} parikīrttitāḥ || 
\pend

 
\pstart

                         \textsc{[1938 ed. 5.4.31]}
                        \caesura rajanyāḥ \edtext{prathame}{
  \Afootnote{paścime A.}
} yāme \edtext{sarvāś}{
  \Afootnote{sarpāś A.}
} citrāś caranti ha |
                        \caesura\edlabel{SS.5.4.31-7}
 \edtext{śeṣās}{
  \linenum{|\xlineref{SS.5.4.31-7}}\lemma{ha |  śeṣās}\Afootnote{hi |  A.}
} tv\edlabel{SS.5.4.31-9} atho \edtext{maṇḍalino}{
  \linenum{|\xlineref{SS.5.4.31-9}}\lemma{tv\ldots maṇḍalino}\Afootnote{śeṣeṣūktā A.}
} divā darvīkarās tathā |\edlabel{SS.5.4.31-15} 
\pend

 
\pstart

                         \textsc{[1938 ed. 5.4.29]}
                        \caesura kopayanty anilañ jantoḥ phaṇinaḥ sarva eva tu | \caesura pittaṃ maṇḍalinaś cāpi kaphaṃ cānekarājayaḥ | 
\pend

 
\pstart

                         \textsc{[1938 ed. 4.30]}
                        \caesura \edtext{atyalpasamavarṇṇābhyāṃ}{
  \Afootnote{apatyam asava° A.}
} dvidoṣakaralakṣaṇam \edtext{| \caesura}{
  \Afootnote{\textsc{[add]} jñeyau doṣaiś ca A.}
} \edtext{dampatyayogād}{
  \Afootnote{dampatyor A.}
} \edtext{vijñeyaṃ}{
  \Afootnote{viśeṣaś A.}
} \edtext{paravādañ}{
  \Afootnote{\textsc{[om]} A.}
} \edtext{ca}{
  \Afootnote{cātra A.}
} vakṣyati\edlabel{SS.5.4.30-8} \edtext{|}{
  \linenum{|\xlineref{SS.5.4.30-8}}\lemma{vakṣyati |}\Afootnote{vakṣyate || A.}
}
\pend

 
\pstart
\edtext{}{
  \Afootnote{\textsc{[pre]} darvīkarās tu taruṇā vṛddhā maṇḍalinas tathā |  rājimanto vayomadhyā jāyante mṛtyuhetavaḥ || A.}
}
\pend

 
\pstart
\edtext{}{
  \Afootnote{\textsc{[pre]} nakulākulitā bālā vāriviprahatāḥ kṛśāḥ |  vṛddhāmuktatvaco bhītāḥ sarpās tv alpaviṣāḥ smṛtāḥ || A.}
}
\pend

 
\pstart

                         \textsc{[1938 ed. 5.4.34.add-1]}
                        \caesura tatra darvīkarāḥ kṛṣṇasarpo mahākṛṣṇaḥ kṛṣṇodaraḥ\edlabel{SS.5.4.34.add-1-5} \edtext{|}{
  \linenum{|\xlineref{SS.5.4.34.add-1-5}}\lemma{kṛṣṇodaraḥ |}\Afootnote{\uwave{kṛṣṇo}daraḥ | H.}
} \edtext{sarvakṛṣṇaḥ}{
  \Afootnote{\textsc{[om]} A.}
} \edtext{śvetaḥ}{
  \Afootnote{śvetakapoto, A.}
} \edtext{kapoto}{
  \Afootnote{mahāka° A.}
} \edtext{valāhako}{
  \Afootnote{balā° A.}
} mahāsarpaḥ \edtext{śaṃkhapālo}{
  \Afootnote{śaṅkhakapālo, A.}
} lohitākṣo gavedhukaḥ parisarpaḥ khaṇḍaphaṇaḥ \edtext{kūkuṭaḥ}{
  \Afootnote{kakudaḥ, A; ku\uuline{kū}\textbf{kku}ṭaḥ H.}
} padmo mahāpadmaḥ \edtext{darbhapuṣpo}{
  \Afootnote{darvbhapuṣpo H.}
} dadhimukhaḥ \edtext{puṇḍarīkamukho}{
  \Afootnote{puṇḍarīko, A.}
} \edtext{babhrūkuṭīmukho}{
  \Afootnote{\textsc{[om]} ba° A.}
} \edtext{vicitraḥ}{
  \Afootnote{viṣkiraḥ, A; vicitra Nep.}
} \edtext{puṣpābhikīrṇṇābho}{
  \Afootnote{°kīrṇo, A.}
} \edtext{girisarpo}{
  \Afootnote{girisarpaḥ, A; girisarppo H.}
} \edtext{ṛjusarpaḥ}{
  \Afootnote{riju° Nep.}
} \edtext{śvetadaro}{
  \Afootnote{śvetodaro, A; svetodaro H.}
} mahāśīrṣo\edlabel{SS.5.4.34.add-1-29} \edtext{'lagardaś}{
  \linenum{|\xlineref{SS.5.4.34.add-1-29}}\lemma{mahāśīrṣo 'lagardaś}\Afootnote{mahāśīrṣāla° H; mahāśirā, alagarda, aśīviṣa A.}
} ceti\edlabel{SS.5.4.34.add-1-31} \edtext{||}{
  \linenum{|\xlineref{SS.5.4.34.add-1-31}}\lemma{ceti ||}\Afootnote{iti (1); A.}
}
\pend

 
\pstart

                         \textsc{[1938 ed. 5.4.34.add-2]}
                        \caesura maṇḍalinas tu ādarśamaṇḍalaḥ śvetamaṇḍalo \edtext{raktamaṇḍalaḥ}{
  \Afootnote{\textsc{[add]} citramaṇḍalaḥ, A.}
} pṛṣato \edtext{devadinnaḥ}{
  \Afootnote{rodhrapuṣpo, A; devadi\uwave{nnaḥ} H.}
} \edtext{pilindako}{
  \Afootnote{milin° gonaso, A.}
} vṛddhagonasaḥ \edtext{panasako}{
  \Afootnote{panaso, A.}
} \edtext{mahāpanasakaḥ}{
  \Afootnote{°naso, A.}
} veṇupatrakaḥ śiśuko \edtext{madanakaḥ}{
  \Afootnote{madanaḥ, A.}
} \edtext{pālindakaḥ}{
  \Afootnote{pālindiraḥ, piṅgalaḥ, A.}
} tantukaḥ puṣpapāṇḍuḥ \edtext{ṣaḍaṅgo\edlabel{SS.5.4.34.add-2-18}}{
  \Afootnote{ṣaḍaṅgā K.}
} \edtext{'gniko}{
  \linenum{|\xlineref{SS.5.4.34.add-2-18}}\lemma{ṣaḍaṅgo 'gniko}\Afootnote{ṣaḍaṅgo, agniko A.}
} \edtext{babhru}{
  \Afootnote{babhruḥ, A.}
} kaṣāyaḥ \edtext{khaluṣaḥ}{
  \Afootnote{kaluṣaḥ A.}
} pārāvato \edtext{hastābharaṇakaḥ}{
  \Afootnote{°raṇaḥ, A.}
} \edtext{tatraś}{
  \Afootnote{\textsc{[om]} A; tatrakaś H.}
} citrakaḥ \edtext{eṇīpadaś}{
  \Afootnote{eṇīpada A.}
} ceti\edlabel{SS.5.4.34.add-2-28} \edtext{||}{
  \linenum{|\xlineref{SS.5.4.34.add-2-28}}\lemma{ceti ||}\Afootnote{iti (2); A.}
}
\pend

 
\pstart

                         \textsc{[1938 ed. 5.4.34.add-3]}
                        \caesura \edtext{rājīmantas}{
  \Afootnote{rājimantas A.}
} tu \edtext{puṇḍarīko}{
  \Afootnote{\textsc{[add]} hi H.}
} \edtext{rājicitro}{
  \Afootnote{rājorā° H.}
} \edtext{aṅgulirājiḥ}{
  \Afootnote{aṅgularājiḥ, A.}
} dvyaṅgulirājiḥ\edlabel{SS.5.4.34.add-3-6} \edtext{|}{
  \linenum{|\xlineref{SS.5.4.34.add-3-6}}\lemma{dvyaṅgulirājiḥ |}\Afootnote{\textsc{[om]} A.}
} bindurājiḥ \edtext{kardamas}{
  \Afootnote{kardamakaḥ, A.}
} \edtext{tṛṇaśoṣakaḥ}{
  \Afootnote{\textsc{[add]} sarṣapakaḥ A.}
} śvetahanur \edtext{darbhapuṣpo}{
  \Afootnote{ddarvbhapuṣpo H.}
} \edtext{lohitākṣaś}{
  \Afootnote{\textsc{[om]} A.}
} \edtext{cakrakaḥ}{
  \Afootnote{ca\uuline{kra}\textbf{ndra}kaḥ H; \textsc{[add]} godhūmakaḥ, A.}
} \edtext{kikkisādaś}{
  \Afootnote{kikkisāda A.}
} ceti\edlabel{SS.5.4.34.add-3-16} \edtext{||}{
  \linenum{|\xlineref{SS.5.4.34.add-3-16}}\lemma{ceti ||}\Afootnote{iti (3); A.}
}
\pend

 
\pstart

                         \textsc{[1938 ed. 5.4.34.add-4]}
                        \caesura nirviṣās \edtext{tu}{
  \Afootnote{\textsc{[add]} galagolī, A.}
} valāhako\edlabel{SS.5.4.34.add-4-3} \edtext{'hipatākaḥ}{
  \linenum{|\xlineref{SS.5.4.34.add-4-3}}\lemma{valāhako 'hipatākaḥ}\Afootnote{\textsc{[om]} A.}
\lemma{'hipatākaḥ}  \Afootnote{hi pa° Nep.}
} śukapatro\edlabel{SS.5.4.34.add-4-5} \edtext{'jagaro}{
  \linenum{|\xlineref{SS.5.4.34.add-4-5}}\lemma{śukapatro 'jagaro}\Afootnote{śūkapatro, aja° A.}
} dīpyakaḥ\edlabel{SS.5.4.34.add-4-7} | ilikinī \edtext{|}{
  \linenum{|\xlineref{SS.5.4.34.add-4-7}}\lemma{dīpyakaḥ\ldots |}\Afootnote{divyako, A.}
} \edtext{varṣāhīko}{
  \Afootnote{varṣāhikaḥ, puṣpaśakalī, A.}
} dvyāhikaḥ\edlabel{SS.5.4.34.add-4-12} \edtext{|}{
  \linenum{|\xlineref{SS.5.4.34.add-4-12}}\lemma{dvyāhikaḥ |}\Afootnote{jyotīrathaḥ, A; \uwave{dvyā}hikaḥ | K; pyāhikaḥ | H.}
} \edtext{kṣīrikāpuṣpaḥ}{
  \Afootnote{°puṣpako, A; °puṣpakaḥ | H.}
} \edtext{puṣpasakalī}{
  \Afootnote{ahipatāko, andhāhiko, A.}
} \edtext{jyotīratho}{
  \Afootnote{gaurāhiko, A; jyot\uuline{i}\textbf{ī}ratho K.}
} \edtext{vṛkṣakaś}{
  \Afootnote{vṛkṣeśaya A.}
} ceti\edlabel{SS.5.4.34.add-4-18} \edtext{||}{
  \linenum{|\xlineref{SS.5.4.34.add-4-18}}\lemma{ceti ||}\Afootnote{iti (4); A.}
}
\pend

 
\pstart

                         \textsc{[1938 ed. 5.4.34.add-5]}
                        \caesura vaikarañjās tu trayāṇāṃ \edtext{varṇṇānāṃ}{
  \Afootnote{darvīkarādīnāṃ A.}
} \edtext{vyatirekajās}{
  \Afootnote{vyatikarāj jātāḥ, A.}
} tad yathā | mākuliḥ poṭagalaḥ snigdharājiś ceti\edlabel{SS.5.4.34.add-5-12} \edtext{||}{
  \linenum{|\xlineref{SS.5.4.34.add-5-12}}\lemma{ceti ||}\Afootnote{iti | A.}
} tatra kṛṣṇasarpeṇa gonasyāṃ \edtext{vaiparītyena}{
  \Afootnote{vaiparītena K.}
} vā jāto mākuliḥ | rājilena gonasyāṃ vaiparītyena \edtext{vā}{
  \Afootnote{\textsc{[om]} K.}
} jātaḥ poṭagalaḥ\edlabel{SS.5.4.34.add-5-27} \edtext{||}{
  \linenum{|\xlineref{SS.5.4.34.add-5-27}}\lemma{poṭagalaḥ ||}\Afootnote{poṭa\uwave{gala}ḥ || K.}
} \edtext{kṛṣṇasarpeṇa}{
  \Afootnote{\uwave{kṛṣṇa}\textsc{(l. 6)}sa° K.}
} rājimatyāṃ vaiparītyena vā jātaḥ snigdharājir iti || \edtext{teṣāṃ}{
  \Afootnote{\textsc{[add]} ādyasya A.}
} pitṛvad viṣam\edlabel{SS.5.4.34.add-5-39} \edtext{utkarṣād}{
  \linenum{|\xlineref{SS.5.4.34.add-5-39}}\lemma{viṣam utkarṣād}\Afootnote{viṣotkarṣo, A.}
} \edtext{dvayor}{
  \Afootnote{dva ꣸ ꣸ ꣸ ꣸ ꣸ ꣸ yor K.}
} \edtext{mātṛvad}{
  \Afootnote{mātṛ\textbf{va}d K.}
} ity eke | 
\pend

 
\pstart
evam \edtext{eṣāṃ}{
  \Afootnote{eteṣāṃ A.}
} sarpāṇām \edtext{aśītir}{
  \Afootnote{aśīti H.}
} vyākhyātā\edlabel{SS.5.4.34.5a-4} \edtext{||}{
  \linenum{|\xlineref{SS.5.4.34.5a-4}}\lemma{vyākhyātā ||}\Afootnote{vyākhyātāḥ || K; vyā\textsc{(l. 6)}𑑛 khyātāḥ | H.}
}
\pend

 
 
\pstart

                         \textsc{[1938 ed. 5.4.35]}
                        \caesura tatra \edtext{mahānetrajihvāśirasaḥ}{
  \Afootnote{°hvāsyaśirasaḥ A.}
} pumāṃsaḥ\edlabel{SS.5.4.35-3} \edtext{|}{
  \linenum{|\xlineref{SS.5.4.35-3}}\lemma{pumāṃsaḥ |}\Afootnote{pumānsaḥ | H.}
} \edtext{sūkṣmanetrajihvāśirasaḥ}{
  \Afootnote{°hvāsyaśirasaḥ A.}
} striyaḥ | ubhayalakṣaṇā \edtext{mandaceṣṭākrodhā}{
  \Afootnote{mandaviṣā akrodhā A.}
} napuṃsakā\edlabel{SS.5.4.35-10} iti \edtext{||}{
  \linenum{|\xlineref{SS.5.4.35-10}}\lemma{napuṃsakā\ldots ||}\Afootnote{napunsakā iti | H.}
}
\pend

 
\pstart

                         \textsc{[1938 ed. 5.4.36]}
                        \caesura tatra sarveṣām \edtext{eva}{
  \Afootnote{\textsc{[om]} A.}
} sarpāṇāṃ sāmānyata eva daṣṭalakṣaṇam upadekṣyāmaḥ\edlabel{SS.5.4.36-8} \edtext{|}{
  \linenum{|\xlineref{SS.5.4.36-8}}\lemma{upadekṣyāmaḥ |}\Afootnote{vakṣyāmaḥ | A.}
} kiṃ kāraṇam | viṣaṃ hi \edtext{huta\edlabel{SS.5.4.36-15}}{
  \Afootnote{\uwave{huta} H.}
} \edtext{hutavaha}{
  \linenum{|\xlineref{SS.5.4.36-15}}\lemma{huta hutavaha}\Afootnote{\textsc{[om]} A.}
\lemma{hutavaha}  \Afootnote{hutavaha H.}
} niśita \edtext{nistriṃśāśani}{
  \Afootnote{\textsc{[add]} hutavaha A.}
} \edtext{kalpam}{
  \Afootnote{deśyam A.}
} āśukāri muhūrttam apy upekṣitam āturam atipātayati | na cāvakāśo \edtext{'sti}{
  \Afootnote{sti\uwave{ḥ} K.}
} vāksamūham anusartum\edlabel{SS.5.4.36-31} \edtext{|}{
  \linenum{|\xlineref{SS.5.4.36-31}}\lemma{anusartum |}\Afootnote{upasartuṃ A.}
} pratyekam api \edtext{ca daṣṭalakṣaṇe}{
  \Afootnote{\textsc{[om]} A.}
} \edtext{'bhihite}{
  \Afootnote{duṣṭala° Nep.}
} sarpa traividhyāt\edlabel{SS.5.4.36-38} \edtext{kriyātraividhyaṃ}{
  \linenum{|\xlineref{SS.5.4.36-38}}\lemma{traividhyāt kriyātraividhyaṃ}\Afootnote{sa\uuline{rva}rpatrai° K; sarvvatrai° H; sarvatra A.}
} \edtext{bhavati}{
  \Afootnote{\textsc{[om]} kriyā° A.}
} | tasmāt traividhyena \edtext{vakṣyāmaḥ}{
  \Afootnote{traividhyam eva A.}
} | etad dhy āturahitam asaṃmohakarañ \edtext{cāsminn}{
  \Afootnote{asammohe karañ H.}
} eva\edlabel{SS.5.4.36-51} \edtext{ca sarvavyañjanāvarodha}{
  \linenum{|\xlineref{SS.5.4.36-51}}\lemma{eva ca sarvavyañjanāvarodha}\Afootnote{ca, api A.}
} \edtext{iti}{
  \Afootnote{cātraiva A.}
} |\edlabel{SS.5.4.36-54} 
\pend

 
\pstart

                         \textsc{[1938 ed. 5.4.37]}
                        \caesura tatra \edtext{darvīkaraviṣeṇa}{
  \Afootnote{dar\uuline{va}\textbf{vī}ka° K.}
} \edtext{tvaṅ}{
  \Afootnote{\textsc{[add]} nayana A.}
} nakha \edtext{nayana}{
  \Afootnote{daśana A.}
} vadana mūtra \edtext{purīṣa}{
  \Afootnote{purīśa K.}
} daṃśa kṛṣṇatvaṃ \edtext{raukṣyaṃ}{
  \Afootnote{\textsc{[add]} śiraso gauravaṃ A.}
} sandhivedanā \edtext{śirogauravaṃ}{
  \Afootnote{\textsc{[om]} A.}
} kaṭīpṛṣṭhagrīvādaurbalyaṃ \edtext{jṛmbhaṇaṃ}{
  \Afootnote{\textsc{[add]} vepathuḥ A.}
} svarāvasādaḥ khurakhurako
 \edtext{jaḍatā}{
  \Afootnote{ghurghurako A; kharukharuko Nep.}
} śuṣkodgāraḥ kāsaḥ śvāso\edlabel{SS.5.4.37-20} \edtext{hikkā}{
  \linenum{|\xlineref{SS.5.4.37-20}}\lemma{śvāso hikkā}\Afootnote{kāsaśvāsau A.}
} vāyor urdhvagamanaṃ \edtext{śūlodveṣṭanaṃ}{
  \Afootnote{ūrdhva° A; ūrddhaga° H.}
} kṛṣṇalālāsravaṇaṃ \edtext{phenāgamanaṃ}{
  \Afootnote{tṛṣṇā lālāsrāvaḥ A.}
} \edtext{srotovarodhas}{
  \Afootnote{pheṇāga° H.}
} \edtext{tās}{
  \Afootnote{srotrova° H; sroto 'va° A.}
} tāś ca vātavedanā bhavanti || maṇḍaliviṣeṇa \edtext{tu}{
  \Afootnote{maṇḍalavi° K.}
} \edtext{tvaṅ}{
  \Afootnote{\textsc{[om]} A.}
} \edtext{nakha}{
  \Afootnote{tvagādīnāṃ A.}
} nayana\edlabel{SS.5.4.37-38} daśana \edtext{vadana}{
  \Afootnote{\textsc{[om]} H.}
} mūtra purīṣa \edtext{daṃśa}{
  \Afootnote{purīśa K.}
} \edtext{pītatvaṃ}{
  \linenum{|\xlineref{SS.5.4.37-38}}\lemma{nayana\ldots pītatvaṃ}\Afootnote{\textsc{[om]} A.}
\lemma{pītatvaṃ}  \Afootnote{daṃ\textbf{śa} K.}
} śītābhilāṣaḥ paridhūpāyanaṃ \edtext{dāhas}{
  \Afootnote{paridhūpanaṃ A.}
} tṛṣṇā\edlabel{SS.5.4.37-48} \edtext{mado}{
  \linenum{|\xlineref{SS.5.4.37-48}}\lemma{tṛṣṇā mado}\Afootnote{dāhastṛṣṇā A.}
} mūrcchā jvaraḥ śoṇitāgamanam ūrdhvam \edtext{adhaś}{
  \Afootnote{ūrddham H.}
} ca māṃsavasāvasādaḥ \edtext{śvayathur}{
  \Afootnote{māsā° H; māṃsānām avaśātanaṃ A.}
} daṃśakotho viparītadarśanam \edtext{āturakopas}{
  \Afootnote{pītarūpada° A.}
} \edtext{tās}{
  \Afootnote{āśukopas A.}
} tāś ca pittavedanā\edlabel{SS.5.4.37-64} \edtext{bhavanti}{
  \linenum{|\xlineref{SS.5.4.37-64}}\lemma{pittavedanā bhavanti}\Afootnote{capi° H.}
} || rājīmadviṣeṇa
 tu \edtext{tvaṅ}{
  \Afootnote{\uwave{rā}\textsc{(l. 5)}jī° K; rājima° A.}
} nakha\edlabel{SS.5.4.37-69} \edtext{nayana}{
  \linenum{|\xlineref{SS.5.4.37-69}}\lemma{nakha nayana}\Afootnote{śuklatvaṃ A.}
} daśana\edlabel{SS.5.4.37-71} vadana \edtext{mūtra}{
  \linenum{|\xlineref{SS.5.4.37-71}}\lemma{daśana\ldots mūtra}\Afootnote{\textsc{[om]} A.}
\lemma{mūtra}  \Afootnote{da\uuline{ṃ}śana K.}
} \edtext{purīṣa}{
  \Afootnote{\textsc{[om]} A.}
} \edtext{daṃśa}{
  \Afootnote{\textsc{[om]} A.}
} \edtext{pāṇḍutvaṃ}{
  \Afootnote{\textsc{[om]} A; purīśā K; purīṣā H.}
} \edtext{śītajvaro}{
  \Afootnote{\textsc{[om]} A.}
} \edtext{romaharṣaḥ}{
  \Afootnote{tvagādīnāṃ A.}
} stabdhatvaṃ \edtext{gātrāṇām}{
  \Afootnote{romaharṣa Nep.}
} ādaṃśaśophaḥ sāndrakaphaprasekaś chardir akṣṇoḥ \edtext{kaṇḍū}{
  \Afootnote{\textsc{[add]} abhīkṣṇam A.}
} khurakhurakaḥ \edtext{ucchvāsanirodhas}{
  \Afootnote{kaṇḍūḥ kaṇṭhe śvayathur A; \textsc{[om]} H.}
} \edtext{tās}{
  \Afootnote{ghurghuraka A; khurukhurukaḥ | H.}
} \edtext{tāś}{
  \Afootnote{ucchvāsa ni° K; \textsc{[add]} tamaḥ praveśas A.}
} ca kaphavedanā bhavanti || 
\pend

 
\pstart

                         \textsc{[1938 ed. 5.4.38]}
                        \caesura \edtext{tatra}{
  \Afootnote{\textsc{[om]} A.}
} puruṣeṇa\edlabel{SS.5.4.38-2} \edtext{daṣṭa}{
  \linenum{|\xlineref{SS.5.4.38-2}}\lemma{puruṣeṇa daṣṭa}\Afootnote{puruṣābhidaṣṭa A.}
\lemma{daṣṭa}  \Afootnote{da\uwave{ṣṭa} K.}
} \edtext{ūrdhvaṃ}{
  \Afootnote{urddhvam H.}
} prekṣate \edtext{|}{
  \Afootnote{\textsc{[add]} adhastāt A.}
} \edtext{striyā}{
  \Afootnote{striyās Nep; \textsc{[add]} sirāś A.}
} \edtext{tiryaṅ}{
  \Afootnote{cottiṣṭhanti lalāṭe, A.}
} \edtext{napuṃsakenādha}{
  \Afootnote{napunsake° H; napuṃsakābhidaṣṭas tiryakprekṣī A.}
} iti\edlabel{SS.5.4.38-10} \edtext{|}{
  \linenum{|\xlineref{SS.5.4.38-10}}\lemma{iti |}\Afootnote{bhavati, A.}
} \edtext{garbhiṇyā}{
  \Afootnote{garbbhiṇyāḥ H.}
} pāṇḍumukho \edtext{ādhmātaś}{
  \Afootnote{\textsc{[om]} ā° A.}
} ca bhavati\edlabel{SS.5.4.38-16} \edtext{|}{
  \linenum{|\xlineref{SS.5.4.38-16}}\lemma{bhavati |}\Afootnote{\textsc{[om]} A.}
} sūtikayā kukṣiśūlārttaḥ sarudhiraṃ mehati \edtext{| grāsārthinānnam
}{
  \Afootnote{\textsc{[add]} upajihvikā cāsya bhavati, A.}
} \edtext{ākāṃkṣati}{
  \Afootnote{grāsārthinā 'nnaṃ A; grāsārthimānnam K; grāsārthino 'nnam H.}
} |\edlabel{SS.5.4.38-24} \edtext{vṛddhena}{
  \linenum{|\xlineref{SS.5.4.38-24}}\lemma{| vṛddhena}\Afootnote{\textsc{[om]} ā° A.}
} cirān mandāś ca vegā bhavanti |\edlabel{SS.5.4.38-31} \edtext{bālenāśus}{
  \linenum{|\xlineref{SS.5.4.38-31}}\lemma{| bālenāśus}\Afootnote{\textsc{[om]} A.}
} \edtext{tīkṣṇaś}{
  \Afootnote{bālenāśu A.}
} \edtext{ca}{
  \Afootnote{mṛdavaś A.}
} | nirviṣeṇāviṣaliṅgam | andhāhikenāndhatvam \edtext{eke}{
  \Afootnote{\textsc{[add]} ity A.}
} | grasanād \edtext{ajagaraḥ}{
  \Afootnote{grasa\uwave{nā}\textsc{(l. 7)}d K.}
} prāṇaharo \edtext{na}{
  \Afootnote{śarīraprā° A.}
} viṣād \edtext{iti}{
  \Afootnote{visāt | A.}
} ||\edlabel{SS.5.4.38-47} 
\pend

 
\pstart

                         \textsc{[1938 ed. 5.4.39]}
                        \caesura tatra \edtext{sarvasarpaviṣāṇāṃ}{
  \Afootnote{sarveṣāṃ sarpāṇāṃ viṣasya A.}
} sapta\edlabel{SS.5.4.39-3} \edtext{viṣavegā}{
  \linenum{|\xlineref{SS.5.4.39-3}}\lemma{sapta viṣavegā}\Afootnote{saptavi° Nep; saptavegā A.}
} bhavanti | tatra darvīkarāṇāṃ prathame vege\edlabel{SS.5.4.39-10} \edtext{viṣaṃ}{
  \linenum{|\xlineref{SS.5.4.39-10}}\lemma{vege viṣaṃ}\Afootnote{vegaviṣaṃ K.}
} śoṇitaṃ dūṣayati | tatpraduṣṭaṃ kṛṣṇatām upaiti | tena \edtext{kārṣṇyaṃ}{
  \Afootnote{kārṣṇya Nep.}
} \edtext{pipīlikāparisarpaṇam}{
  \Afootnote{°rpa \uwave{\uwave{ñca}} (\textbf{ṇa}) m K.}
} \edtext{iva}{
  \Afootnote{i\uuline{vā}\textbf{va} H.}
} cāṅge bhavanti\edlabel{SS.5.4.39-24} \edtext{||}{
  \linenum{|\xlineref{SS.5.4.39-24}}\lemma{bhavanti ||}\Afootnote{bhavati; A; bhava\uwave{nti} | H.}
} dvitīye \edtext{māṃsaṃ}{
  \Afootnote{mānsan H.}
} dūṣayati | \edtext{tenātyarthakṛṣṇatā}{
  \Afootnote{tenātyarthaṃ kṛ° śopho A.}
} granthayaś \edtext{ca}{
  \Afootnote{cāṅge A.}
} bhavanti || tṛtīye \edtext{medo}{
  \Afootnote{me\textbf{do} K.}
} dūṣayati | tena \edtext{daṃśakledaḥ}{
  \Afootnote{daṃśa\uuline{ḥ}kledaḥ H.}
} \edtext{śirogauravaṃ}{
  \Afootnote{\textsc{[add]} svedaś A.}
} \edtext{cakṣurgrahaṇañ}{
  \Afootnote{cakṣugrahaṇañ H.}
} ca bhavati\edlabel{SS.5.4.39-44} \edtext{|}{
  \linenum{|\xlineref{SS.5.4.39-44}}\lemma{bhavati |}\Afootnote{\textsc{[om]} A.}
} caturthe koṣṭham anupraviśati\edlabel{SS.5.4.39-48} | \edtext{tataḥ}{
  \linenum{|\xlineref{SS.5.4.39-48}}\lemma{anupraviśati\ldots tataḥ}\Afootnote{°viśya A.}
} \edtext{kaphaprabhavān}{
  \Afootnote{kaphapradhānān A.}
} \edtext{doṣān}{
  \Afootnote{doṣāṃ K.}
} \edtext{kopayati}{
  \Afootnote{dūṣayati, A.}
} tena tandrīkaphaprasekaḥ\edlabel{SS.5.4.39-55} sandhiviśleṣaś \edtext{ca}{
  \linenum{|\xlineref{SS.5.4.39-55}}\lemma{tandrīkaphaprasekaḥ\ldots ca}\Afootnote{tandrāprasekasandhiviśleṣā A.}
} bhavati\edlabel{SS.5.4.39-58} \edtext{||}{
  \linenum{|\xlineref{SS.5.4.39-58}}\lemma{bhavati ||}\Afootnote{bhavanti; A.}
} \edtext{pañcame}{
  \Afootnote{pa\uwave{ñca}\textsc{(l. 2)}me K.}
} 'sthīny \edtext{anupraviśati}{
  \Afootnote{\textsc{[add]} prāṇam agniṃ ca dūṣayati, A.}
} \edtext{tena}{
  \Afootnote{\textsc{[add]} \uuline{sarppa} H.}
} parvabhedo hikkā dāhaś ca bhavati || ṣaṣṭhe \edtext{majjām}{
  \Afootnote{majjānam A H.}
} anupraviśati | \edtext{tena}{
  \Afootnote{\textsc{[om]} A.}
} \edtext{grahaṇīdoṣā}{
  \Afootnote{grahaṇīṃ cātyarthaṃ dūṣayati, tena A.}
} \edtext{gātragauravam}{
  \Afootnote{gātrāṇāṃ gau° A.}
} atīsāro hṛtpīḍā mūrcchā ca bhavati || saptame śukram anupraviśati vyānañ cātyarthaṃ kopayati kaphañ ca sūkṣmaṃ\edlabel{SS.5.4.39-91} \edtext{srotobhyaḥ}{
  \linenum{|\xlineref{SS.5.4.39-91}}\lemma{sūkṣmaṃ srotobhyaḥ}\Afootnote{sūkṣmasro° A.}
} pracyāvayati | tena \edtext{śleṣmaprādurbhāvaḥ}{
  \Afootnote{śleṣmavartiprā° A; °durvbhāvaḥ H.}
} \edtext{kaṭīpṛṣṭhaskandabhaṅgaḥ}{
  \Afootnote{kaṭī\textbf{stambha}pṛṣṭhaskande\textsc{(l. 4)} bhaṅgāḥ H; °ṣṭhabhaṅgaḥ A.}
} \edtext{sarvaceṣṭhāvighātaḥ}{
  \Afootnote{sarvvaceṣṭāvi° H; sarvaceṣṭāvighāto lālāsvedayor atipravṛttir A.}
} \edtext{ucchvāsavirodho}{
  \Afootnote{ucchvāsanirodhaś ca A.}
} bhavatīti\edlabel{SS.5.4.39-100} \edtext{||}{
  \linenum{|\xlineref{SS.5.4.39-100}}\lemma{bhavatīti ||}\Afootnote{bhavati \uline{A} \uline{H}.}
} \edtext{maṇḍalinān}{
  \Afootnote{maṇḍalināṃ A.}
} \edtext{tu}{
  \Afootnote{\textsc{[om]} A.}
} prathame vege \edtext{viṣaḥ}{
  \Afootnote{viṣaṃ A.}
} śoṇitaṃ dūṣayati | tatpraduṣṭaṃ pītatām upaiti | \edtext{tena}{
  \Afootnote{tatra paridāhaḥ A.}
} pītāvabhāsatā \edtext{paridāhaś}{
  \Afootnote{\textsc{[om]} A.}
} \edtext{ca}{
  \Afootnote{cāṅgānāṃ A.}
} bhavati || dvitīye māṃsaṃ dūṣayati | \edtext{tena\edlabel{SS.5.4.39-124}}{
  \Afootnote{tenātyarthapītāṅgatā H.}
} \edtext{cātyarthapītāṅgatā}{
  \linenum{|\xlineref{SS.5.4.39-124}}\lemma{tena cātyarthapītāṅgatā}\Afootnote{tenātyarthaṃ pītatā A.}
\lemma{cātyarthapītāṅgatā}  \Afootnote{°tā\uwave{} K.}
} \edtext{cātyarthaparidāho}{
  \Afootnote{\textsc{[om]} cātyartha° A; śvarthapa° K.}
} \edtext{daṃśaśvayathur}{
  \Afootnote{daṃśa\uuline{ś ca} śva° K; daṃśe śva° A; \textsc{[add]} ca A H.}
} bhavati || tṛtīye medo dūṣayati \edtext{tena}{
  \Afootnote{\textsc{[add]} pūrvavac cakṣurgrahaṇaṃ A.}
} \edtext{kṛṣṇādaṃśakledaḥ}{
  \Afootnote{tṛṣṇā daṃ° A.}
} svedaś ca bhavati\edlabel{SS.5.4.39-137} \edtext{||}{
  \linenum{|\xlineref{SS.5.4.39-137}}\lemma{bhavati ||}\Afootnote{\textsc{[om]} A.}
} caturthe \edtext{pūrvavadanupraviśya}{
  \Afootnote{pūrvavad anu° Nep; koṣṭham anu° A.}
} jvaram āpādayati || pañcame \edtext{dāhaṃ}{
  \Afootnote{paridāhaṃ A.}
} sarvagātreṣu karoti | ṣaṣṭhasaptamayoḥ pūrvavad iti\edlabel{SS.5.4.39-151} \edtext{|}{
  \linenum{|\xlineref{SS.5.4.39-151}}\lemma{iti |}\Afootnote{\textsc{[om]} A.}
} \edtext{rājīmatāṃ}{
  \Afootnote{rājimatāṃ A.}
} \edtext{tu}{
  \Afootnote{\textsc{[om]} A.}
} prathame \edtext{vege}{
  \Afootnote{\textsc{[add]} viṣaṃ A.}
} \edtext{śoṇitan}{
  \Afootnote{śoṇitaṃ A.}
} dūṣayati || tatpraduṣṭaṃ pāṇḍutām upaiti tena romaharṣaḥ \edtext{pāṇḍvāvabhāsaś}{
  \Afootnote{śuklāva° A.}
} ca puruṣo bhavati || dvitīye māṃsaṃ dūṣayati tena \edtext{pāṇḍur}{
  \Afootnote{pāṇḍutā A.}
} \edtext{atyarthajāḍyañ}{
  \Afootnote{'tyarthaṃ jāḍyaṃ śiraḥśophaś A.}
} ca bhavati | tṛtīye medo \edtext{dūṣayati}{
  \Afootnote{\textbf{dū}ṣa° K.}
} \edtext{tena}{
  \Afootnote{\textsc{[add]} cakṣurgrahaṇaṃ A.}
} \edtext{daṃśakledo}{
  \Afootnote{daṃśakledaḥ svedo A.}
} \edtext{'kṣināsāsrāvaś}{
  \Afootnote{ghrāṇākṣisrāvaś A.}
} ca bhavati || caturthe \edtext{pūrvavad}{
  \Afootnote{koṣṭham A.}
} anupraviśya \edtext{manyāstambhaśirogauravañ}{
  \Afootnote{manyāstambhaṃ śi° A.}
} cāpādayati || pañcame \edtext{vāksaṅgaḥ śītajvaraś
}{
  \Afootnote{vāksaṅgaṃ A.}
} \edtext{ca}{
  \Afootnote{śītajvaraṃ A; śītajvarañ Nep.}
} || \edtext{ṣaṣṭhasaptamayoḥ}{
  \Afootnote{\textsc{[add]} karoti; A.}
} pūrvavad \edtext{iti}{
  \Afootnote{pūvad K.}
} || 
\pend

 
\pstart

                         \textsc{[1938 ed. 5.4.40]}
                        \caesura \edtext{bhavanti}{
  \Afootnote{bhavati || K.}
} cātra\edlabel{SS.5.4.40-2} ślokāḥ\edlabel{SS.5.4.40-3} \edtext{|| \caesura}{
  \linenum{|\xlineref{SS.5.4.40-2}}\lemma{cātra\ldots || }\Afootnote{\textsc{[om]} K.}
  \linenum{|\xlineref{SS.5.4.40-3}}\lemma{ślokāḥ || }\Afootnote{\textsc{[om]} A.}
} dhātvantareṣu yāḥ sapta kalāḥ saṃparikīrttitāḥ \caesura tāsv
 \edtext{ekaikam}{
  \Afootnote{\uwave{tāḥsv} K; tāḥsv H.}
} \edtext{atikramya}{
  \Afootnote{ekaikām A.}
} vegaṃ prakurute viṣaḥ
 || 
\pend

 
\pstart

                         \textsc{[1938 ed. 5.4.41]}
                        \caesura yenāntareṇa tu \edtext{kalāḥ}{
  \Afootnote{kalāṃ A; ka\uwave{lā}ḥ H.}
} kālakalpaṃ \edtext{bhinatti}{
  \Afootnote{\textsc{[add]} \textbf{vadanti } H.}
} ha\edlabel{SS.5.4.41-6} \edtext{| \caesura}{
  \linenum{|\xlineref{SS.5.4.41-6}}\lemma{ha | }\Afootnote{hi |  A.}
} samīraṇenohyamānaṃ tat tu vegāntaraṃ matam\edlabel{SS.5.4.41-12} \edtext{||}{
  \linenum{|\xlineref{SS.5.4.41-12}}\lemma{matam ||}\Afootnote{smṛtam || A.}
}
\pend

 
\pstart

                         \textsc{[1938 ed. 5.4.42]}
                        \caesura śūnāṅgaḥ prathame vege paśuḥ \edtext{pradhyāti}{
  \Afootnote{dhyāyati A.}
} duḥkhitaḥ \edtext{|| \caesura}{
  \Afootnote{\textsc{[add]} lālāsrāvo A.}
} dvitīye \edtext{lālimān}{
  \Afootnote{tu A.}
} \edtext{kiñcid dhṛṣṭāṅgaḥ\edlabel{SS.5.4.42-10}
}{
  \Afootnote{\textsc{[om]} A.}
} \edtext{pīḍyate}{
  \linenum{|\xlineref{SS.5.4.42-10}}\lemma{kiñcid dhṛṣṭāṅgaḥ pīḍyate}\Afootnote{kiñci\uwave{ddhṛ}ṣṭāṅgaḥ K; kiñci\uuline{ddhṛ}\textbf{ddṛ}ṣṭāṅgāḥ H.}
\lemma{pīḍyate}  \Afootnote{kṛṣṇāṅgaḥ A.}
} hṛdi | 
\pend

 
\pstart

                         \textsc{[1938 ed. 5.4.43]}
                        \caesura \edtext{tṛtīyasya}{
  \Afootnote{tṛtīye ca A.}
} śiroduḥkhaṃ \edtext{karṇṇagrīvāñ}{
  \Afootnote{kaṇṭhagrīvaṃ A; \uuline{karṇṇa}\textbf{kaṇṭha}grīvāñ H.}
} ca bhajyate | \caesura caturthe vepate mūḍhaḥ khādan \edtext{dantāñ}{
  \Afootnote{dantān A H.}
} \edtext{jahāty}{
  \Afootnote{jahaty H.}
} asūn | 
\pend

 
\pstart

                         \textsc{[1938 ed. 5.4.44]}
                        \caesura kecid \edtext{vegatrayaṃ}{
  \Afootnote{ve\uuline{da}\textbf{ga}trayam H.}
} prāhur \edtext{antaḥsvedeṣu}{
  \Afootnote{antaḥ sve° H; antaṃ caiteṣu A; anta\uwave{sve}deṣu K.}
} tadvidaḥ\edlabel{SS.5.4.44-5} \edtext{|| \caesura}{
  \linenum{|\xlineref{SS.5.4.44-5}}\lemma{tadvidaḥ || }\Afootnote{tadviduḥ |  H.}
} \edtext{vege}{
  \Afootnote{\textsc{[om]} A; vede K.}
} \edtext{tu}{
  \Afootnote{dhyāyati A.}
} \edtext{prathame}{
  \Afootnote{\textsc{[add]} vege A.}
} pakṣī \edtext{dhyāti}{
  \Afootnote{\textsc{[om]} A.}
} muhyaty ataḥ param ||
 
\pend

 
\pstart

                         \textsc{[1938 ed. 5.4.45]}
                        \caesura dvitīye vihvalaḥ \edtext{kūjan}{
  \Afootnote{proktas A.}
} \edtext{pakṣī\edlabel{SS.5.4.45-4}}{
  \Afootnote{tṛtīye A.}
} \edtext{maraṇam}{
  \linenum{|\xlineref{SS.5.4.45-4}}\lemma{pakṣī maraṇam}\Afootnote{pakṣīma° K.}
\lemma{maraṇam}  \Afootnote{mṛtyum A.}
} arcchati\edlabel{SS.5.4.45-6} \edtext{| \caesura}{
  \linenum{|\xlineref{SS.5.4.45-6}}\lemma{arcchati | }\Afootnote{ṛcchati |  A.}
} kecid ekaṃ vihaṅgeṣu \edtext{viṣavegam uśanti
}{
  \Afootnote{viṣaveṣam K; viṣavegem H.}
} \edtext{vai}{
  \Afootnote{uṣanti Nep.}
} || \caesura\edlabel{SS.5.4.45-13} \edtext{mārjāranakulādīnāṃ}{
  \linenum{|\xlineref{SS.5.4.45-13}}\lemma{||  mārjāranakulādīnāṃ}\Afootnote{hi |  A.}
} viṣaṃ nātipravartata \edtext{iti}{
  \Afootnote{°rtate || A.}
} ||\edlabel{SS.5.4.45-18} 
\pend

 \chapter{Kalpasthāna 5: Therapy for those Bitten by Snakes}
\pstart

                         \textsc{[1938 ed. 5.5.1]}
                        \caesura athātaḥ sarpa \edtext{daṣṭa}{
  \Afootnote{\textsc{[add]} viṣa A.}
} cikitsitaṃ \edtext{kalpaṃ\edlabel{SS.5.5.1-5}}{
  \Afootnote{\textsc{[om]} H.}
} vyākhyāsyāmaḥ \edtext{||}{
  \linenum{|\xlineref{SS.5.5.1-5}}\lemma{kalpaṃ\ldots ||}\Afootnote{kalpaṃvyākhyāsyāmaḥ || K.}
}
\pend

 
\pstart
\edtext{}{
  \Afootnote{\textsc{[pre]} yathovāca bhagavān dhanvantariḥ || A.}
}
\pend

 
\pstart

                         \textsc{[1938 ed. 5.5.3]}
                        \caesura sarvair evāditaḥ sarpaiḥ śākhādaṣṭasya dehinaḥ \edtext{| \caesura}{
  \Afootnote{\textsc{[add]} daṃśasyopari A.}
} badhnīyād \edtext{gāḍham}{
  \Afootnote{\textsc{[om]} A.}
} \edtext{upari}{
  \Afootnote{ariṣṭāś A.}
} daṃśāt\edlabel{SS.5.5.3-10} \edtext{tu}{
  \linenum{|\xlineref{SS.5.5.3-10}}\lemma{daṃśāt tu}\Afootnote{\textsc{[om]} A.}
} caturaṅgulam\edlabel{SS.5.5.3-12} \edtext{|}{
  \linenum{|\xlineref{SS.5.5.3-12}}\lemma{caturaṅgulam |}\Afootnote{°gule || A.}
}
\pend

 
\pstart

                         \textsc{[1938 ed. 5.5.4]}
                        \caesura plotacarmāntavalkānāṃ \edtext{mṛdunānyatamena}{
  \Afootnote{mṛdunā 'nya° A.}
} vā\edlabel{SS.5.5.4-3} \edtext{| \caesura}{
  \linenum{|\xlineref{SS.5.5.4-3}}\lemma{vā | }\Afootnote{vai |  A.}
} \edtext{na}{
  \Afootnote{\textsc{[om]} H.}
} \edtext{paryeti}{
  \Afootnote{gacchati A; \uwave{pa}ryeti ca H.}
} \edtext{viṣaṃ}{
  \Afootnote{\uuline{cā}viṣaṃ K.}
} deham ariṣṭābhir nivāritam\edlabel{SS.5.5.4-10} \edtext{|}{
  \linenum{|\xlineref{SS.5.5.4-10}}\lemma{nivāritam |}\Afootnote{nnivārataḥ || H.}
}
\pend

 
\pstart

                         \textsc{[1938 ed. 5.5.5]}
                        \caesura dahed daṃśam \edtext{athoddhṛtya}{
  \Afootnote{athotkṛtya A.}
} yatra bandho na jāyate | \caesura \edtext{ācūṣaṇacchedadāhāḥ}{
  \Afootnote{\uwave{ācūṣaṇaccheda}\textsc{(l. 4)}dāhaḥ H.}
} sarvatraiva tu pūjitāḥ | 
\pend

 
\pstart

                         \textsc{[1938 ed. 5.5.6]}
                        \caesura pratipūrya mukhaṃ \edtext{pāṃśor}{
  \Afootnote{vastrair A; pāṃśo\uwave{r} K; pāṃśo\uwave{r} H.}
} hitam ācūṣaṇaṃ bhavet | \caesura \edtext{sandaṣṭavyo}{
  \Afootnote{sa da° A.}
} 'thavā sarpo \edtext{daṣṭamātreṇa}{
  \Afootnote{loṣṭo vā 'pi hi tat A.}
} jānatā\edlabel{SS.5.5.6-12} \edtext{||}{
  \linenum{|\xlineref{SS.5.5.6-12}}\lemma{jānatā ||}\Afootnote{kṣaṇam || A.}
}
\pend

 
\pstart

                         \textsc{[1938 ed. 5.5.7]}
                        \caesura atha \edtext{maṇḍalidaṣṭan}{
  \Afootnote{maṇḍalinā daṣṭaṃ A; maṇḍali\uwave{daṣṭan} H.}
} \edtext{tu}{
  \Afootnote{\textsc{[om]} A.}
} \edtext{na}{
  \Afootnote{na\textsc{(l. 5)} H.}
} \edtext{kathañ}{
  \Afootnote{\textsc{[add]} cana A.}
} cit\edlabel{SS.5.5.7-6} \edtext{tu}{
  \linenum{|\xlineref{SS.5.5.7-6}}\lemma{cit tu}\Afootnote{\textsc{[om]} A; ca K.}
} dāhayet | \caesura sa \edtext{pittaviṣabāhulyād}{
  \Afootnote{pittabāhulyaviṣād A.}
} daṃśo dāhād vināśayet\edlabel{SS.5.5.7-14} \edtext{|}{
  \linenum{|\xlineref{SS.5.5.7-14}}\lemma{vināśayet |}\Afootnote{visarpate || A.}
}
\pend

 
\pstart

                         \textsc{[1938 ed. 5.5.8]}
                        \caesura \edtext{ariṣṭām}{
  \Afootnote{ariṣṭāv Nep.}
} api mantrais \edtext{tu}{
  \Afootnote{ca A.}
} badhnīyāt mantrakovidaḥ | \caesura sā tu \edtext{rajjvādibhir}{
  \Afootnote{rajvādibhir Nep.}
} baddhā \edtext{viṣapūtikarī}{
  \Afootnote{viṣaprati° A; viṣa\textsc{(l. 6)}pūtīkarī H.}
} matā | 
\pend

 
\pstart

                         \textsc{[1938 ed. 5.5.9]}
                        \caesura \edtext{devabrahmarṣivihitā}{
  \Afootnote{°rṣibhiḥ proktā A.}
} mantrāḥ satyatapomayāḥ | \caesura bhavanty \edtext{anatyayāḥ}{
  \Afootnote{nānyathā A.}
} kṣipraṃ viṣaṃ hanyuś \edtext{ca}{
  \Afootnote{\textsc{[om]} A.}
} dustaram\edlabel{SS.5.5.9-11} \edtext{|}{
  \linenum{|\xlineref{SS.5.5.9-11}}\lemma{dustaram |}\Afootnote{sudustaram || A.}
}
\pend

 
\pstart

                         \textsc{[1938 ed. 5.5.10]}
                        \caesura viṣaṃ tejomayair mantraiḥ satyabrahmatapomayaiḥ\edlabel{SS.5.5.10-4} \edtext{| \caesura}{
  \linenum{|\xlineref{SS.5.5.10-4}}\lemma{satyabrahmatapomayaiḥ | }\Afootnote{\uwave{satyavrahmatapomayaiḥ}  H.}
} yathā nivāryate kṣipraṃ \edtext{prayuktair}{
  \Afootnote{prayuktan H.}
} na tathauṣadhaiḥ | 
\pend

 
\pstart

                         \textsc{[1938 ed. 5.5.11]}
                        \caesura mantrāṇāṃ grahaṇaṃ kāryaṃ strīmāṃsamadhuvarjinā | \caesura \edtext{yatāhāreṇa}{
  \Afootnote{mitā° A.}
} śucinā kuśāstaraṇaśāyinā | 
\pend

 
\pstart

                         \textsc{[1938 ed. 5.5.12]}
                        \caesura gandhamālyopahāraiś ca balibhiś cāpi devatām\edlabel{SS.5.5.12-5} \edtext{| \caesura}{
  \linenum{|\xlineref{SS.5.5.12-5}}\lemma{devatām | }\Afootnote{devatāḥ |  A; devatā |  K.}
} pūjayet mantrasiddhyarthaṃ \edtext{japahomaiś}{
  \Afootnote{jāpa° H.}
} ca yatnataḥ || 
\pend

 
\pstart

                         \textsc{[1938 ed. 5.5.13]}
                        \caesura mantrās tv avidhinā proktā hīnā vā svaravarṇṇataḥ\edlabel{SS.5.5.13-7} \edtext{| \caesura}{
  \linenum{|\xlineref{SS.5.5.13-7}}\lemma{svaravarṇṇataḥ | }\Afootnote{svaravarṇṇitaḥ |  H.}
} yasmān na siddhim āyānti tasmād yojyo 'gadakramaḥ | 
\pend

 
\pstart

                         \textsc{[1938 ed. 5.5.14]}
                        \caesura \edtext{daṃśāt}{
  \Afootnote{\textsc{[om]} A.}
} \edtext{samantāc}{
  \Afootnote{samantataḥ A.}
} \edtext{ca}{
  \Afootnote{\textsc{[om]} A.}
} \edtext{sirāṃ}{
  \Afootnote{sirā daṃśād A.}
} \edtext{vyadhayet}{
  \Afootnote{vidhyet tu A.}
} kuśalo bhiṣak | \caesura \edtext{śākhāśrayāṃ}{
  \Afootnote{śākhāgre vā A.}
} \edtext{lalāṭe}{
  \Afootnote{lalāṭo K.}
} \edtext{ca}{
  \Afootnote{vā A.}
} \edtext{veddhavyā}{
  \Afootnote{vyadhyās tā A.}
} \edtext{visṛte}{
  \Afootnote{viṛte A.}
} viṣe | 
\pend

 
\pstart

                         \textsc{[1938 ed. 5.5.15]}
                        \caesura \edtext{raktan}{
  \Afootnote{rakte A; \uwave{rakta} K.}
} \edtext{nirhriyamānan}{
  \Afootnote{ni\uwave{rhri}yamāṇan H; °māṇe A.}
} tu kṛtsnaṃ \edtext{nirharate}{
  \Afootnote{nirhriyate A.}
} viṣam | \caesura tasmād visrāvayed raktaṃ sā hy asya \edtext{paramā}{
  \Afootnote{parama H.}
} kriyā | 
\pend

 
\pstart

                         \textsc{[1938 ed. 5.5.16]}
                        \caesura \edtext{daṃśaṃ}{
  \Afootnote{\textsc{[om]} A.}
} samantād \edtext{agadaiḥ}{
  \Afootnote{\textsc{[add]} daṃśaṃ A.}
} pracchayitvā ca\edlabel{SS.5.5.16-5} lepayet \edtext{| \caesura}{
  \linenum{|\xlineref{SS.5.5.16-5}}\lemma{ca\ldots | }\Afootnote{prale° A.}
} \edtext{candanośīrasiktena}{
  \Afootnote{°rayuktena A.}
} vāriṇā cāpi\edlabel{SS.5.5.16-10} secayet \edtext{|}{
  \linenum{|\xlineref{SS.5.5.16-10}}\lemma{cāpi\ldots |}\Afootnote{pariṣecayet || A.}
}
\pend

 
\pstart

                         \textsc{[1938 ed. 5.5.17]}
                        \caesura \edtext{pāyayec}{
  \Afootnote{pāyaye A.}
} \edtext{cāgadāṃs}{
  \Afootnote{tā ga° A; cāgadāṃ K.}
} \edtext{tāṃs}{
  \Afootnote{tās K.}
} tān dadhikṣaudraghṛtādibhiḥ\edlabel{SS.5.5.17-5} \edtext{| \caesura}{
  \linenum{|\xlineref{SS.5.5.17-5}}\lemma{dadhikṣaudraghṛtādibhiḥ | }\Afootnote{kṣīrakṣau° A.}
} tadalābhe hitā vā syāt kṛṣṇavalmīkamṛttikā\edlabel{SS.5.5.17-11} \edtext{||}{
  \linenum{|\xlineref{SS.5.5.17-11}}\lemma{kṛṣṇavalmīkamṛttikā ||}\Afootnote{kṛṣṇā va° A.}
\lemma{||}  \Afootnote{\textsc{[add]} \textsc{(l. 7)}\textbf{|}\uuline{ḥ} K.}
}
\pend

 
\pstart

                         \textsc{[1938 ed. 5.5.18]}
                        \caesura kovidāraśirīṣārkaṃ\edlabel{SS.5.5.18-1} \edtext{kaṭabhīṃ}{
  \linenum{|\xlineref{SS.5.5.18-1}}\lemma{kovidāraśirīṣārkaṃ kaṭabhīṃ}\Afootnote{°ṣārkakaṭabhīr A.}
\lemma{kaṭabhīṃ}  \Afootnote{kaṭabhīr K.}
} \edtext{vāpi}{
  \Afootnote{vā 'pi A.}
} bhakṣayet | \caesura na pibet tailakaulatthaṃ\edlabel{SS.5.5.18-8} madyaṃ\edlabel{SS.5.5.18-9} \edtext{sauvīrakaṃ}{
  \linenum{|\xlineref{SS.5.5.18-8}}\lemma{tailakaulatthaṃ\ldots sauvīrakaṃ}\Afootnote{°latthamadyasauvīrakāṇi A.}
  \linenum{|\xlineref{SS.5.5.18-9}}\lemma{madyaṃ sauvīrakaṃ}\Afootnote{madyasau° H.}
} \edtext{ca}{
  \Afootnote{na H.}
} na\edlabel{SS.5.5.18-12} \edtext{|}{
  \linenum{|\xlineref{SS.5.5.18-12}}\lemma{na |}\Afootnote{\textsc{[om]} A; ca || H.}
}
\pend

 
\pstart

                         \textsc{[1938 ed. 5.5.19]}
                        \caesura dravam \edtext{anyat}{
  \Afootnote{anyan H.}
} tu yat kiñcit pītvā pītvā tad uddharet\edlabel{SS.5.5.19-9} \edtext{| \caesura}{
  \linenum{|\xlineref{SS.5.5.19-9}}\lemma{uddharet | }\Afootnote{udvamet |  A.}
} prāyo hi vamanenaiva \edtext{sukhaṃ}{
  \Afootnote{mukhaṃ A.}
} nirhriyate viṣam | 
\pend

 
\pstart

                         \textsc{[1938 ed. 5.5.20]}
                        \caesura phaṇināṃ viṣavege tu \edtext{prathamaṃ}{
  \Afootnote{prathame A.}
} śoṇitaṃ haret | \caesura \edtext{dvitīye}{
  \Afootnote{dvitī K.}
} madhusarpirbhyām \edtext{agadaṃ}{
  \Afootnote{pāyayetāgadaṃ A.}
} \edtext{saha\edlabel{SS.5.5.20-11}}{
  \Afootnote{sa\uwave{ha} K.}
} pāyayet\edlabel{SS.5.5.20-12} \edtext{|}{
  \linenum{|\xlineref{SS.5.5.20-11}}\lemma{saha\ldots |}\Afootnote{bhisak || A.}
  \linenum{|\xlineref{SS.5.5.20-12}}\lemma{pāyayet |}\Afootnote{pā(From 150v)\textsc{(l. 1)}yayet | K.}
}
\pend

 
\pstart

                         \textsc{[1938 ed. 5.5.21]}
                        \caesura \edtext{nastaḥ}{
  \Afootnote{nasya A; nasta H.}
} karmāñjane yuñjyāt tṛtīye viṣanāśane
 | \caesura\edlabel{SS.5.5.21-5} \edtext{vānte}{
  \linenum{|\xlineref{SS.5.5.21-5}}\lemma{|  vānte}\Afootnote{°śanaṃ |  Nep.}
} \edtext{caturthe}{
  \Afootnote{vāntaṃ A.}
} viṣaghnāṃ \edtext{yavāgūṃ}{
  \Afootnote{pūrvoktāṃ A.}
} \edtext{pāyayed bhiṣak
}{
  \Afootnote{\textsc{[add]} atha A.}
} \edtext{||}{
  \Afootnote{dāpayet || A.}
}
\pend

 
\pstart

                         \textsc{[1938 ed. 5.5.22]}
                        \caesura \edtext{śītopacāraṃ}{
  \Afootnote{\textbf{śī}° K.}
} \edtext{puruṣaṃ}{
  \Afootnote{kṛtvādau A.}
} \edtext{vegayoḥ}{
  \Afootnote{bhiṣak A.}
} pañcaṣaṣṭhayoḥ\edlabel{SS.5.5.22-4} \edtext{| \caesura}{
  \linenum{|\xlineref{SS.5.5.22-4}}\lemma{pañcaṣaṣṭhayoḥ | }\Afootnote{pañcamaṣa° A.}
} pāyayec\edlabel{SS.5.5.22-6} \edtext{chodhanaṃ}{
  \linenum{|\xlineref{SS.5.5.22-6}}\lemma{pāyayec chodhanaṃ}\Afootnote{pāyayet cho° K.}
} tīkṣṇaṃ yavāgūñ cāpi kīrttitām\edlabel{SS.5.5.22-11} \edtext{|}{
  \linenum{|\xlineref{SS.5.5.22-11}}\lemma{kīrttitām |}\Afootnote{kīrttitā K.}
}
\pend

 
\pstart

                         \textsc{[1938 ed. 5.5.23]}
                        \caesura \edtext{saptame}{
  \Afootnote{\textbf{sa}° K.}
} tv avapīḍena śiras \edtext{tīkṣṇena}{
  \Afootnote{tīkṣṇeṇa H.}
} śodhayet \edtext{|}{
  \Afootnote{\textsc{[add]} tīkṣṇam evāñjanaṃ dadyāt tīkṣṇaśastreṇa mūrdhni ca || A; \textsc{[add]} K.}
}
\pend

 
\pstart
\edtext{
                         \textsc{[1938 ed. 5.5.24]}
                        \caesura}{
  \Afootnote{\textsc{[pre]} kṛtvā kākapadaṃ carma sāsṛg vā piśitaṃ kṣipet |  A.}
} \edtext{pūrvo}{
  \Afootnote{\textbf{pūrvo} K.}
} maṇḍalināṃ vego darvīkaravad ācaret\edlabel{SS.5.5.24-5} \edtext{|}{
  \linenum{|\xlineref{SS.5.5.24-5}}\lemma{ācaret |}\Afootnote{ācaret K.}
}
\pend

 
\pstart

                         \textsc{[1938 ed. 5.5.25]}
                        \caesura \edtext{dvitīye}{
  \Afootnote{agadaṃ A.}
} \edtext{sarppirmmadhunī}{
  \Afootnote{madhusarpirbhyāṃ dvitīye A.}
} \edtext{pāyayitvā}{
  \Afootnote{pāyayeta ca |  vāmayitvā yavāgūṃ A.}
} \edtext{ca}{
  \Afootnote{\textsc{[add]} pūrvoktām atha A.}
} vāmayet\edlabel{SS.5.5.25-5} \edtext{|}{
  \linenum{|\xlineref{SS.5.5.25-5}}\lemma{vāmayet |}\Afootnote{dāpayet || A.}
}
\pend

 
\pstart

                         \textsc{[1938 ed. 5.5.26]}
                        \caesura \edtext{tṛtīye}{
  \Afootnote{\textbf{tṛ}° K; \textsc{[add]} śodhitaṃ A.}
} \edtext{ca\edlabel{SS.5.5.26-2}}{
  \Afootnote{tīkṣṇair A.}
} \edtext{viriktasya}{
  \linenum{|\xlineref{SS.5.5.26-2}}\lemma{ca viriktasya}\Afootnote{\uwave{\uuline{sa}}\textbf{su}vi° H.}
\lemma{viriktasya}  \Afootnote{\textsc{[om]} A; \textsc{[add]}  K.}
} \edtext{yavāgūn}{
  \Afootnote{yavāgūṃ A; yavāgum K.}
} \edtext{dāpayed}{
  \Afootnote{pāyayed A.}
} dhitām | \caesura caturthe pañcame \edtext{cāpi}{
  \Afootnote{vāpi\textsc{(l. 4)} H.}
} darvvīkaravad ācaret | 
\pend

 
\pstart

                         \textsc{[1938 ed. 5.5.27]}
                        \caesura kākolyādir hitaḥ ṣaṣṭhe peyaś ca madhuro 'gadaḥ | \caesura hito 'vapīḍe tv agadaḥ saptame viṣanāśanaḥ || 
\pend

 
\pstart

                         \textsc{[1938 ed. 5.5.28]}
                        \caesura \edtext{atha}{
  \Afootnote{pūrve A.}
} rājimatāṃ vege \edtext{prathame}{
  \Afootnote{'lābubhiḥ A.}
} śoṇitaṃ haret \edtext{|}{
  \Afootnote{\textsc{[add]} agadaṃ madhusarpirbhyāṃ saṃyuktaṃ pāyayeta ca || A.}
}
\pend

 
\pstart

                         \textsc{[1938 ed. 5.5.29]}
                        \caesura vāntaṃ dvitīye tv agadaṃ pāyayed viṣanāśanam | \caesura tṛtīyādiṣu triṣv \edtext{eva}{
  \Afootnote{evaṃ A.}
} vidhir dārvīkaro hitaḥ | 
\pend

 
\pstart

                         \textsc{[1938 ed. 5.5.30]}
                        \caesura ṣaṣṭhe 'ñjanaṃ tīkṣṇatamam avapīḍaś ca saptame | \caesura \edtext{garbhiṇībālavṛddhānāṃ}{
  \Afootnote{garvbhiṇī° H.}
} sirāvedhavivarjitam\edlabel{SS.5.5.30-9} \edtext{|}{
  \linenum{|\xlineref{SS.5.5.30-9}}\lemma{sirāvedhavivarjitam |}\Afootnote{sirāvyadhanavarjitam || A.}
}
\pend

 
\pstart

                         \textsc{[1938 ed. 5.5.31]}
                        \caesura \edtext{viṣārttiṣu}{
  \Afootnote{viṣārtānāṃ A.}
} yathoddiṣṭaṃ vidhānaṃ \edtext{mṛdu}{
  \Afootnote{\textsc{[om]} A.}
} śasyate \edtext{| \caesura}{
  \Afootnote{\textsc{[add]} mṛdu |  A.}
} raktāvasekāñjanāni \edtext{naratulyāny\edlabel{SS.5.5.31-8}}{
  \Afootnote{°lyā\uuline{}\textbf{ny} K.}
} ajāvike\edlabel{SS.5.5.31-9} \edtext{|}{
  \linenum{|\xlineref{SS.5.5.31-8}}\lemma{naratulyāny\ldots |}\Afootnote{naratulyāñjanāvike || H.}
  \linenum{|\xlineref{SS.5.5.31-9}}\lemma{ajāvike |}\Afootnote{ajāvike | K.}
}
\pend

 
\pstart
\edtext{
                         \textsc{[1938 ed. 5.5.32]}
                        \caesura}{
  \Afootnote{\textsc{[pre]} triguṇaṃ mahiṣe soṣṭre A.}
} \edtext{gavāśvayos}{
  \Afootnote{\textsc{[om]} A.}
} \edtext{tad}{
  \Afootnote{gavāśve A; ta K.}
} \edtext{dviguṇaṃ}{
  \Afootnote{dviguṇāṃ K.}
} \edtext{triguṇaṃ}{
  \Afootnote{tu A.}
} mahiṣoṣṭrayoḥ\edlabel{SS.5.5.32-5} \edtext{| \caesura}{
  \linenum{|\xlineref{SS.5.5.32-5}}\lemma{mahiṣoṣṭrayoḥ | }\Afootnote{tat |  A.}
} caturguṇaṃ tu nāgānāṃ kevalaṃ sarvapakṣiṇām | 
\pend

 
\pstart
\edtext{}{
  \Afootnote{\textsc{[pre]} pariṣekān pradehāṃś ca suśītān avacārayet |  māṣakaṃ tv añjanasyeṣṭaṃ dviguṇaṃ nasyato hitam |  pāne caturguṇaṃ pathyaṃ vamane 'ṣṭaguṇaṃ punaḥ || A.}
}
\pend

 
\pstart

                         \textsc{[1938 ed. 5.5.34]}
                        \caesura \edtext{deśa}{
  \Afootnote{deśe K.}
} prakṛti sātmya \edtext{rtu viṣavegabalābalam
}{
  \Afootnote{ntu Nep.}
} | \caesura pradhārya nipuṇaṃ \edtext{budhyā}{
  \Afootnote{nipunaṃ Nep.}
} tataḥ karma samācaret | 
\pend

 
\pstart
\edtext{}{
  \Afootnote{\textsc{[pre]} vegānupūrvyā karmoktam idaṃ viṣavināśanam |  karmāvasthāviśeṣeṇa viṣayor ubhayoḥ śṛṇu || A.}
}
\pend

 
\pstart
\edtext{}{
  \Afootnote{\textsc{[pre]} vivarṇe kaṭhine śūne saruje 'ṅge viṣānvite |  tūrṇaṃ visravaṇaṃ kāryam uktena vidhinā tataḥ || A.}
}
\pend

 
\pstart
\edtext{}{
  \Afootnote{\textsc{[pre]} kṣudhārtam anilaprāyaṃ tad viṣārtaṃ samāhitaḥ |  pāyayeta rasaṃ sarpiḥ śuktaṃ kṣaudraṃ tathā dadhi || A.}
}
\pend

 
\pstart
\edtext{}{
  \Afootnote{\textsc{[pre]} tṛḍdāhadharmasaṃmohe paittaṃ paittaviṣāturam |  śītaiḥ saṃvāhanasnānapradehaiḥ samupācaret || A.}
}
\pend

 
\pstart
\edtext{}{
  \Afootnote{\textsc{[pre]} śīte śītaprasekārtaṃ ślaiṣmikaṃ kaphakṛdviṣam |  vāmayed vamanais tīkṣṇais tathā mūrcchāmadānvitam || A.}
}
\pend

 
\pstart
\edtext{}{
  \Afootnote{\textsc{[pre]} koṣṭhadāharujādhmānamūtrasaṅgaruganvitam |  virecayec chakṛdvāyusaṅgapittāturaṃ naram || A.}
}
\pend

 
\pstart
\edtext{}{
  \Afootnote{\textsc{[pre]} śūnākṣikūṭaṃ nidrārtaṃ vivarṇāvilalocanam |  vivarṇaṃ cāpi paśyan tam añjanaiḥ samupācaret || A.}
}
\pend

 
\pstart
\edtext{}{
  \Afootnote{\textsc{[pre]} śiroruggauravālasya hanustambhagalagrahe |  śiro virecayet kṣipraṃ manyāstambhe ca dāruṇe || A.}
}
\pend

 
\pstart
\edtext{}{
  \Afootnote{\textsc{[pre]} naṣṭasaṃjñaṃ vivṛttākṣaṃ bhagnagrīvaṃ virecanaiḥ |  cūrṇaiḥ pradhamanais tīkṣṇair viṣārtaṃ samupācaret || A.}
}
\pend

 
\pstart
\edtext{}{
  \Afootnote{\textsc{[pre]} tāḍayec ca sirāḥ kṣipraṃ tasya śākhālalāṭajāḥ |  tāsv aprasicyamānāsu mūrdhni śastreṇa śastravit || A.}
}
\pend

 
\pstart
\edtext{}{
  \Afootnote{\textsc{[pre]} kuryāt kākapadākāraṃ vraṇam evaṃ sravanti tāḥ |  saraktaṃ carma māṃsaṃ vā nikṣipec cāsya mūrdhani || A.}
}
\pend

 
\pstart
\edtext{}{
  \Afootnote{\textsc{[pre]} carmavṛkṣakaṣāyaṃ vā kalkaṃ vā kuśalo bhiṣak |  vādayec cāgadair liptvā dundubhīṃs tasya pārśvayoḥ || A.}
}
\pend

 
\pstart
\edtext{}{
  \Afootnote{\textsc{[pre]} labdhasaṃjñaṃ punaś cainam ūrdhvaṃ cādhaś ca śodhayet | A.}
}
\pend

 
\pstart

                         \textsc{[1938 ed. 5.5.47]}
                        \caesura niḥśeṣaṃ \edtext{nirharec}{
  \Afootnote{nirha\textbf{re}c H.}
} \edtext{cainaṃ}{
  \Afootnote{caivaṃ A.}
} viṣaṃ paramadurjayam | 
\pend

 
\pstart

                         \textsc{[1938 ed. 5.5.48]}
                        \caesura \edtext{svalpam}{
  \Afootnote{alpam A.}
} \edtext{apy}{
  \Afootnote{a\uwave{py} K.}
} \edtext{avatiṣṭhaṃ}{
  \Afootnote{a\textsc{(l. 4)}va° K; avaśiṣṭaṃ A.}
} hi bhūyo vegāya kalpate | \caesura kuryād vā sādavaivarṇṇyajvarakāsaśirorujaḥ\edlabel{SS.5.5.48-11} \edtext{|}{
  \linenum{|\xlineref{SS.5.5.48-11}}\lemma{sādavaivarṇṇyajvarakāsaśirorujaḥ |}\Afootnote{°rujāḥ | H.}
}
\pend

 
\pstart

                         \textsc{[1938 ed. 5.5.49]}
                        \caesura \edtext{śoṣaśophapratiśyāyatimirāruci}{
  \Afootnote{śophaśoṣapra° A.}
} jāḍyatām\edlabel{SS.5.5.49-2} \edtext{| \caesura}{
  \linenum{|\xlineref{SS.5.5.49-2}}\lemma{jāḍyatām | }\Afootnote{pīnasān |  A.}
} \edtext{tāsu}{
  \Afootnote{teṣu A.}
} cāpi \edtext{yathāyogaṃ}{
  \Afootnote{yathādoṣaṃ A.}
} \edtext{pratikarma}{
  \Afootnote{prati\textbf{karmma} H.}
} prayojayet | 
\pend

 
\pstart

                         \textsc{[1938 ed. 5.5.50]}
                        \caesura viṣārttopadravāṃś cāpi yathāsvaṃ samupācaret || \caesura \edtext{athāriṣṭāṃ}{
  \Afootnote{yathā° H.}
} \edtext{vimucyāśu}{
  \Afootnote{vimucyāśuḥ K.}
} pracchayitvāṅkitaṃ tayā | 
\pend

 
\pstart

                         \textsc{[1938 ed. 5.5.51]}
                        \caesura \edtext{vidyāt}{
  \Afootnote{dahyāt A.}
} tatra viṣaṃ skannaṃ bhūyo vegāya kalpate \edtext{|}{
  \Afootnote{\textsc{[add]} evam auṣadhibhir mantraiḥ kriyāyogaiś ca yatnataḥ || A.}
}
\pend

 
\pstart
\edtext{}{
  \Afootnote{\textsc{[pre]} viṣe hṛtaguṇe dehād yadā doṣaḥ prakupyati |  tadā pavanam udvṛttaṃ snehādyaiḥ samupācaret || A.}
}
\pend

 
\pstart

                         \textsc{[1938 ed. 5.5.52.add-1]}
                        \caesura viṣāpāye 'nilaṃ kruddhaṃ jayed anilavāraṇaiḥ | 
\pend

 
\pstart

                         \textsc{[1938 ed. 5.5.53]}
                        \caesura taila \edtext{madya}{
  \Afootnote{matsya A.}
} \edtext{kulatthāmla}{
  \Afootnote{kulatthāmbla K.}
} \edtext{varjair}{
  \Afootnote{varjyair A H.}
} viṣaharāyutaiḥ | \caesura \edtext{pittaṃ}{
  \Afootnote{\textsc{[om]} A.}
} \edtext{pittajvaraharaiḥ}{
  \Afootnote{pittajvraharaiḥ pittaṃ A.}
} kaṣāyasneharecanaiḥ\edlabel{SS.5.5.53-9} \edtext{|}{
  \linenum{|\xlineref{SS.5.5.53-9}}\lemma{kaṣāyasneharecanaiḥ |}\Afootnote{°habastibhiḥ || A.}
}
\pend

 
\pstart

                         \textsc{[1938 ed. 5.5.54]}
                        \caesura kapham āragvadhādyena sakṣaudreṇa gaṇena tu \edtext{||}{
  \Afootnote{\textsc{[add]} śleṣmaghnair agadaiś caiva tiktai rūkṣaiś ca bhojanaiḥ || A.}
}
\pend

 
\pstart
\edtext{}{
  \Afootnote{\textsc{[pre]} vṛkṣaprapātaviṣam apatitaṃ mṛtam ambhasi |  udbaddhaṃ ca mṛtaṃ sadyaś cikitsen naṣṭasaṃjñavat || A.}
}
\pend

 
\pstart

                         \textsc{[1938 ed. 5.5.56]}
                        \caesura gāḍhaṃ baddhe 'riṣṭayā pracchite vā | \caesura tīkṣṇair lepair \edtext{viṣaśeṣeṇa}{
  \Afootnote{tadvidhair A.}
} vāpi\edlabel{SS.5.5.56-10} \edtext{| \caesura}{
  \linenum{|\xlineref{SS.5.5.56-10}}\lemma{vāpi | }\Afootnote{vāvaśiṣṭaiḥ |  A.}
} śūne gātre klinnam atyarthapūti \caesura \edtext{śīrṇṇaṃ}{
  \Afootnote{jñeyaṃ A.}
} \edtext{māṃsaṃ}{
  \Afootnote{mānsam H.}
} \edtext{viṣapūti}{
  \Afootnote{tadviṣāt pūti A.}
} pradiṣṭaṃ\edlabel{SS.5.5.56-19} \edtext{||}{
  \linenum{|\xlineref{SS.5.5.56-19}}\lemma{pradiṣṭaṃ ||}\Afootnote{kaṣṭam || A.}
}
\pend

 
\pstart

                         \textsc{[1938 ed. 5.5.57]}
                        \caesura \edtext{sadyaḥ}{
  \Afootnote{sadya K.}
} \edtext{kṣataṃ}{
  \Afootnote{viddhaṃ nisravet A.}
} pacyate\edlabel{SS.5.5.57-3} yasya jantoḥ\edlabel{SS.5.5.57-5} \edtext{| \caesura}{
  \linenum{|\xlineref{SS.5.5.57-3}}\lemma{pacyate\ldots | }\Afootnote{\textsc{[om]} A.}
  \linenum{|\xlineref{SS.5.5.57-5}}\lemma{jantoḥ | }\Afootnote{janto H.}
} kṛṣṇaṃ\edlabel{SS.5.5.57-7} \edtext{raktaṃ}{
  \linenum{|\xlineref{SS.5.5.57-7}}\lemma{kṛṣṇaṃ raktaṃ}\Afootnote{kṛṣṇaraktaṃ A.}
\lemma{raktaṃ}  \Afootnote{\textsc{[add]} pākaṃ yāyād A.}
} \edtext{sravate}{
  \Afootnote{\textsc{[om]} A.}
} dahyate ca\edlabel{SS.5.5.57-11} \edtext{| \caesura}{
  \linenum{|\xlineref{SS.5.5.57-11}}\lemma{ca | }\Afootnote{cāpy abhīkṣṇam |  A.}
} \edtext{śyāvībhūtaṃ}{
  \Afootnote{kṛṣṇībhūtaṃ A.}
} klinnam atyarthapūti | \caesura \edtext{kṣatāt}{
  \Afootnote{śīrṇaṃ A.}
} māṃsaṃ \edtext{śīryate}{
  \Afootnote{yāty A.}
} \edtext{yasya}{
  \Afootnote{ajasraṃ kṣatāc A.}
} cāpi\edlabel{SS.5.5.57-21} \edtext{|}{
  \linenum{|\xlineref{SS.5.5.57-21}}\lemma{cāpi |}\Afootnote{ca || A.}
}
\pend

 
\pstart

                         \textsc{[1938 ed. 5.5.58]}
                        \caesura tṛṣṇā mūrcchā \edtext{jvaradāhau}{
  \Afootnote{bhrāntidāhau jvaraś A.}
} ca yasya \edtext{| \caesura}{
  \Afootnote{\textsc{[add]} syus taṃ A.}
} \edtext{digdhāhataṃ}{
  \Afootnote{digdhaviddhaṃ vyavasyet |  pūrvoddiṣṭaṃ A.}
} \edtext{taṃ}{
  \Afootnote{lakṣaṇaṃ sarvam A.}
} \edtext{manujam}{
  \Afootnote{etaj juṣṭaṃ A.}
} vyavasyet\edlabel{SS.5.5.58-10} \edtext{||
}{
  \linenum{|\xlineref{SS.5.5.58-10}}\lemma{vyavasyet ||}\Afootnote{vraṇāḥ syuḥ || A.}
}
\pend

 
\pstart

                         \textsc{[1938 ed. 5.5.58.add-1]}
                        \caesura liṅgāny etāny eva vā yasya vidyād | \caesura vraṇe viṣaṃ \edtext{yasya}{
  \Afootnote{tasya H.}
} dattaṃ pramādāt | \caesura digdhāhataṃ \edtext{viṣajuṣṭaṃ}{
  \Afootnote{viṣaṃ juṣṭaṃ K.}
} vraṇañ ca | \caesura ye cāpy anye viṣapūtivraṇārttāḥ\edlabel{SS.5.5.58.add-1-22} \edtext{||}{
  \linenum{|\xlineref{SS.5.5.58.add-1-22}}\lemma{viṣapūtivraṇārttāḥ ||}\Afootnote{viṣapū\textbf{ti}° K.}
\lemma{||}  \Afootnote{\textsc{[add]} || K.}
}
\pend

 
\pstart
\edtext{
                         \textsc{[1938 ed. 5.5.59cd]}
                        \caesura}{
  \Afootnote{\textsc{[pre]} \textsc{[pre]} \textbf{} K.}
} teṣāṃ \edtext{dhīmān}{
  \Afootnote{yuktyā A; dhīmāṃn K.}
} \edtext{adhimāṃsāny}{
  \Afootnote{pūtimāṃ° A; adhimānsāny H.}
} apohya\edlabel{SS.5.5.59cd-4} \edtext{| \caesura}{
  \linenum{|\xlineref{SS.5.5.59cd-4}}\lemma{apohya | }\Afootnote{apo\uwave{hya} H.}
} \edtext{jalaukābhiḥ}{
  \Afootnote{vāryokobhiḥ A; jallaukābhiḥ K.}
} śoṇitaṃ cāpahṛtvā\edlabel{SS.5.5.59cd-8} \edtext{||}{
  \linenum{|\xlineref{SS.5.5.59cd-8}}\lemma{cāpahṛtvā ||}\Afootnote{cāpahṛtya || A.}
\lemma{||}  \Afootnote{\textsc{[add]} K.}
}
\pend

 


 
\pstart

                         \textsc{[1938 ed. 5.5.60]}
                        \caesura hṛtvā \edtext{doṣān}{
  \Afootnote{\textsc{[add]} kṣipram A.}
} \edtext{ūrdhvamadhaś}{
  \Afootnote{ūrdhvaṃ tv adhaś A; ūrddhvam adhaś H.}
} ca samyak | \caesura siñcec\edlabel{SS.5.5.60-7} \edtext{chītaiḥ}{
  \linenum{|\xlineref{SS.5.5.60-7}}\lemma{siñcec chītaiḥ}\Afootnote{siñcet A; siñcec chītaiḥ K.}
} kṣīriṇāṃ tvakaṣāyaiḥ\edlabel{SS.5.5.60-10} \edtext{| \caesura}{
  \linenum{|\xlineref{SS.5.5.60-10}}\lemma{tvakaṣāyaiḥ | }\Afootnote{tvakkaṣāyaiḥ |  A H.}
} \edtext{vastrāntarān}{
  \Afootnote{antarvastraṃ A.}
} dāpayec ca pradehāñ\edlabel{SS.5.5.60-15} \edtext{| \caesura}{
  \linenum{|\xlineref{SS.5.5.60-15}}\lemma{pradehāñ | }\Afootnote{pradehān A.}
} \edtext{cchītair}{
  \Afootnote{śītair A.}
} \edtext{dravyair}{
  \Afootnote{dravyai K.}
} \edtext{ghṛtayuktair}{
  \Afootnote{ājyayuktair A.}
} viṣaghnaiḥ | 
\pend

 
\pstart

                         \textsc{[1938 ed. 5.5.61]}
                        \caesura \edtext{kṣate 'sthani
}{
  \Afootnote{bhinne tv A.}
} \edtext{sa}{
  \Afootnote{asthnā A; sthitiḥ Nep.}
} \edtext{viṣair}{
  \Afootnote{\textsc{[om]} A.}
} eṣa\edlabel{SS.5.5.61-4} \edtext{eva}{
  \linenum{|\xlineref{SS.5.5.61-4}}\lemma{eṣa eva}\Afootnote{duṣṭajātena A.}
\lemma{eva}  \Afootnote{e\uwave{ṣa} H.}
} | \caesura\edlabel{SS.5.5.61-6} \edtext{vidhiḥ}{
  \linenum{|\xlineref{SS.5.5.61-6}}\lemma{|  vidhiḥ}\Afootnote{kāryaḥ  A.}
} \edtext{kāryaḥ}{
  \Afootnote{pūrvo A.}
} \edtext{pittaviṣe}{
  \Afootnote{mārgaḥ A.}
} \edtext{tathaiva}{
  \Afootnote{paittike yo viṣe A.}
} || \caesura\edlabel{SS.5.5.61-11} \edtext{trivṛd}{
  \linenum{|\xlineref{SS.5.5.61-11}}\lemma{||  trivṛd}\Afootnote{ca |  A.}
} \edtext{viśalyā\edlabel{SS.5.5.61-13}}{
  \Afootnote{tṛvṛd Nep.}
} \edtext{madhukaṃ}{
  \linenum{|\xlineref{SS.5.5.61-13}}\lemma{viśalyā madhukaṃ}\Afootnote{trivṛdviśalye A.}
} haridre | \caesura mañjiṣṭha \edtext{vakrau}{
  \Afootnote{raktā A.}
} \edtext{lavaṇaś}{
  \Afootnote{narendro A; va\uwave{krau} K; \textbf{yukto}varggo H.}
} ca sarvaḥ |\edlabel{SS.5.5.61-22}
 
\pend

 
\pstart

                         \textsc{[1938 ed. 5.5.62]}
                        \caesura kaṭutrikaṃ \edtext{caiva}{
  \Afootnote{cai\textbf{va} H.}
} \edtext{vicūrṇṇitāni \caesura}{
  \Afootnote{sucū° A.}
} śṛṅge nidadhyāt madhusaṃyutāni | \caesura eṣo 'gado hanti viṣaṃ prayuktaḥ \caesura pānāñjanābhyañjana nasya yogaiḥ | 
\pend

 
\pstart

                         \textsc{[1938 ed. 5.5.63]}
                        \caesura avāryavīryo
 \edtext{viṣavegahantā \caesura}{
  \Afootnote{avāravīryo Nep.}
} mahāgado nāma mahāprabhāvaḥ || \caesura viḍaṅga pāṭhā \edtext{triphalājamodā- \caesura}{
  \Afootnote{pāṭha K.}
} \edtext{hiṅgūni}{
  \Afootnote{°moda K.}
} \edtext{vakraṃ}{
  \Afootnote{hiṃguni K.}
} trikaṭuṃ \edtext{tathaiva}{
  \Afootnote{trikaṭūni A; trikaṭūn H.}
} |\edlabel{SS.5.5.63-13} 
\pend

 
\pstart

                         \textsc{[1938 ed. 5.5.64]}
                        \caesura sarvaś ca vargo lavaṇaḥ \edtext{susūkṣmaḥ \caesura}{
  \Afootnote{sasūkṣmaḥ K.}
} \edtext{sacitrakakṣaudrayuto}{
  \Afootnote{sacitrakaḥ kṣau° A.}
} nidheyaḥ\edlabel{SS.5.5.64-7} \edtext{| \caesura}{
  \linenum{|\xlineref{SS.5.5.64-7}}\lemma{nidheyaḥ | }\Afootnote{vidheyaḥ |  H.}
} śṛṅge gavāṃ śṛṅgamayena caiva \caesura pracchāditaḥ pakṣam upekṣitaś ca | 
\pend

 
\pstart

                         \textsc{[1938 ed. 5.5.65]}
                        \caesura eṣo \edtext{'gadaḥ}{
  \Afootnote{gada K.}
} sthāvarajaṅgamānāñ \caesura jetā viṣāṇām \edtext{ajito}{
  \Afootnote{aji\uwave{to} H.}
} hi nāmnā || \caesura prapauṇḍarīkaṃ suradāru \edtext{rāsnā \caesura}{
  \Afootnote{mustā  A; \uwave{rāsnā} H.}
} \edtext{kālānusārī}{
  \Afootnote{°sāryā A.}
} \edtext{kaṭurohaṇīś}{
  \Afootnote{kaṭurohiṇī A; kaṭurohiṇīñ H.}
} ca | 
\pend

 
\pstart

                         \textsc{[1938 ed. 5.5.66]}
                        \caesura sthauṇeyakadhyāmaka \edtext{padmakāni \caesura}{
  \Afootnote{guggulūni  A.}
} punnāga \edtext{tālīsa}{
  \Afootnote{tālīśa A.}
} \edtext{suvarcikāś}{
  \Afootnote{suvarccikāñ H.}
} ca | \caesura \edtext{kuṭannaṭailāsitasindhuvārāḥ \caesura}{
  \Afootnote{°sinduvārāḥ  K.}
} śaileyakuṣṭhe tagaraṃ priyaṅguḥ\edlabel{SS.5.5.66-11} \edtext{|}{
  \linenum{|\xlineref{SS.5.5.66-11}}\lemma{priyaṅguḥ |}\Afootnote{priyaṃṅguḥ | K.}
}
\pend

 
\pstart

                         \textsc{[1938 ed. 5.5.67]}
                        \caesura \edtext{lodhraṃ}{
  \Afootnote{rodhraṃ A.}
} \edtext{tathā}{
  \Afootnote{\textsc{[om]} A.}
} \edtext{guggula}{
  \Afootnote{jalaṃ kāñcana A; guggulu H.}
} gairikañ \edtext{ca \caesura}{
  \Afootnote{\textsc{[add]} samāgadhaṃ A.}
} sasaindhave\edlabel{SS.5.5.67-6} \edtext{pippalināgare}{
  \linenum{|\xlineref{SS.5.5.67-6}}\lemma{sasaindhave pippalināgare}\Afootnote{candanasaindhavaṃ A.}
} ca | \caesura sūkṣmāṇi cūrṇṇāni samāni kṛtvā \caesura śṛṅge nidadhyāt madhusaṃyutāni | 
\pend

 
\pstart

                         \textsc{[1938 ed. 5.5.68]}
                        \caesura eṣo 'gadas \edtext{tārkṣya}{
  \Afootnote{tā\uwave{rkṣya} H.}
} iti pradiṣṭo \caesura \edtext{viṣan}{
  \Afootnote{viṣaṃ A.}
} nihanyād api\edlabel{SS.5.5.68-8} takṣakasya \edtext{|| \caesura}{
  \linenum{|\xlineref{SS.5.5.68-8}}\lemma{api\ldots || }\Afootnote{apita° H.}
} \edtext{māṃsīhareṇutriphalāmuruṅgī \caesura}{
  \Afootnote{māṃsīhareṇutriḥpha° K; mānsīhareṇus tṛpha° H; °muraṅgī  A.}
} \edtext{mañjiṣṭha}{
  \Afootnote{raktālatā A.}
} \edtext{yaṣṭyāhvaya}{
  \Afootnote{yaṣṭika A; ya\uwave{ṣṭyāhvaya} H.}
} padmakāni | 
\pend

 
\pstart

                         \textsc{[1938 ed. 5.5.69]}
                        \caesura viḍaṅga\edlabel{SS.5.5.69-1} tālīsa \edtext{sugandhikailā \caesura}{
  \linenum{|\xlineref{SS.5.5.69-1}}\lemma{viḍaṅga\ldots sugandhikailā }\Afootnote{viḍaṅgatālīsasu° H; viḍaṅgatālīsasu\uwave{ga}\textsc{(l. 5)}ndhi° K.}
} \edtext{tvakkuṣṭhavakrāṇi}{
  \linenum{|\xlineref{SS.5.5.69-1}}\lemma{viḍaṅga\ldots tvakkuṣṭhavakrāṇi}\Afootnote{viḍaṅgatālīśasugandhikailātvakkuṣṭhapatrāṇi A.}
\lemma{tvakkuṣṭhavakrāṇi}  \Afootnote{tvakkuṣṭha\uwave{va}krāṇi H.}
} sacandanāni | \caesura bhārgī \edtext{paṭolīkiṇihī}{
  \Afootnote{paṭolaṃ ki° A; °ṇi hī Nep.}
} sapāṭhā \caesura mṛgādanī \edtext{kroṣṭakamekhalā}{
  \Afootnote{karkaṭikā puraś A.}
} ca | 
\pend

 
\pstart

                         \textsc{[1938 ed. 5.5.70]}
                        \caesura pālindyaśokau kramukaṃ \edtext{surasyā \caesura}{
  \Afootnote{surasyāḥ A \uline{H}.}
} prasūnam āruṣkarajañ ca puṣpam \edtext{| \caesura}{
  \Afootnote{\textsc{[add]} sūkṣmāni A.}
} cūrṇṇāny \edtext{athaiṣāṃ}{
  \Afootnote{\textsc{[om]} A.}
} \edtext{nihitāni}{
  \Afootnote{samāni A.}
} śṛṅge \caesura \edtext{deyāni}{
  \Afootnote{nyaset A.}
} \edtext{pittāni}{
  \Afootnote{sapittāni A.}
} samākṣikāni\edlabel{SS.5.5.70-15} \edtext{|}{
  \linenum{|\xlineref{SS.5.5.70-15}}\lemma{samākṣikāni |}\Afootnote{°kāṇi || A.}
}
\pend

 
\pstart

                         \textsc{[1938 ed. 5.5.71]}
                        \caesura \edtext{varāhagodhāśikhiśalyakānāṃ \caesura}{
  \Afootnote{°śallakīnāṃ A.}
} mārjārajaṃ pārṣatanākule ca | \caesura yasyāgado 'yaṃ sukṛto \edtext{gṛhastho \caesura}{
  \Afootnote{gṛhe syān A.}
} \edtext{nāmnārṣabho}{
  \Afootnote{nāmnarṣabho A.}
} nāma nararṣabhasya | 
\pend

 
\pstart

                         \textsc{[1938 ed. 5.5.72]}
                        \caesura na tatra sarpāḥ kuta eva kīṭās \caesura tyajanti vīryāṇi viṣāṇi caiva | \caesura etena bheryaḥ paṭahāś ca digdhāḥ \caesura nānadyamānā viṣam āśu hanyuḥ | 
\pend

 
\pstart

                         \textsc{[1938 ed. 5.5.73]}
                        \caesura digdhāḥ patākāś ca nirīkṣya sadyo \caesura \edtext{viṣābhibhūtāḥ}{
  \Afootnote{°bhūtā A.}
} \edtext{sukhino}{
  \Afootnote{hy aviṣā A.}
} bhavanti || \caesura lākṣā \edtext{hareṇvau}{
  \Afootnote{hareṇur A.}
} \edtext{naladapriyaṅgvau \caesura}{
  \Afootnote{nalada pri° H; naladaṃ priyaṅguḥ A; °priyagvau K.}
} \edtext{mañjiṣṭhayaṣṭyāhvayapṛthvikāś}{
  \Afootnote{śigrudvayaṃ yaṣṭikapṛ° A.}
} ca | 
\pend

 
\pstart

                         \textsc{[1938 ed. 5.5.74]}
                        \caesura cūrṇṇīkṛto 'yaṃ \edtext{rajanīvimiśro \caesura}{
  \Afootnote{rajanāvi° K; \textsc{[add]} sarpirmadhubhyāṃ A.}
} \edtext{vargo}{
  \Afootnote{sahito A.}
} nidheyo madhusarpiṣāktaḥ\edlabel{SS.5.5.74-6} \edtext{| \caesura}{
  \linenum{|\xlineref{SS.5.5.74-6}}\lemma{madhusarpiṣāktaḥ | }\Afootnote{\textsc{[om]} A; °ṣā\uwave{ktaḥ} ||  H.}
} śṛṅge gavāṃ pūrvavad ā \edtext{pidhānas \caesura}{
  \Afootnote{pidhāṇas K.}
} tataḥ prayojyo 'ñjana \edtext{pāna}{
  \Afootnote{\textsc{[om]} A.}
} nasyaiḥ\edlabel{SS.5.5.74-17} \edtext{|}{
  \linenum{|\xlineref{SS.5.5.74-17}}\lemma{nasyaiḥ |}\Afootnote{nasya pānaiḥ || A.}
}
\pend

 
\pstart

                         \textsc{[1938 ed. 5.5.75]}
                        \caesura sañjīvano nāma \edtext{gatāsukalpān \caesura}{
  \Afootnote{gatā\uuline{śu}\textbf{su}kalpa\textsc{(gap of 1)}m H; °lpā\uwave{n} K.}
} \edtext{eṣo}{
  \Afootnote{\textsc{[om]} K.}
} \edtext{'gado}{
  \Afootnote{(From 151v)\textsc{(l. 1)}gado K.}
} jīvayatīha martyān\edlabel{SS.5.5.75-7} \edtext{| \caesura}{
  \linenum{|\xlineref{SS.5.5.75-7}}\lemma{martyān | }\Afootnote{martyaḥ ||  H.}
} \edtext{śleṣmātakīkaṭphalamātuluṅga \caesura}{
  \Afootnote{śleṣmāntakīkaṭphalamātuluṅgā H; °luṅgyaḥ A.}
} śvetā \edtext{girihvā}{
  \Afootnote{giri\uwave{hvā} K.}
} kiṇihī sitā ca || 
\pend

 
\pstart

                         \textsc{[1938 ed. 5.5.76]}
                        \caesura sataṇḍulīyo 'gada \edtext{eṣa}{
  \Afootnote{eva H.}
} \edtext{mukhyo \caesura}{
  \Afootnote{mu\uwave{khyo} K.}
} \edtext{viṣeṣu}{
  \Afootnote{viśeṣu K.}
} darvīkararājilānām || \caesura \edtext{drākṣāśvagandhā}{
  \Afootnote{drākṣā sugandhā A.}
} \edtext{gajavṛttikā}{
  \Afootnote{nagavṛ° A.}
} ca \caesura śvetā \edtext{ca}{
  \Afootnote{samaṅgā A; \textsc{(gap of 1)} H.}
} \edtext{piṣṭā}{
  \Afootnote{\textsc{[om]} A.}
} samabhāgayuktāḥ\edlabel{SS.5.5.76-14} \edtext{||}{
  \linenum{|\xlineref{SS.5.5.76-14}}\lemma{samabhāgayuktāḥ ||}\Afootnote{°yuktā || A.}
}
\pend

 
\pstart

                         \textsc{[1938 ed. 5.5.77]}
                        \caesura deyo dvibhāgaḥ \edtext{surasacchadasya \caesura}{
  \Afootnote{surasāccha° A.}
} kapitthabilvād api \edtext{dāḍimāc}{
  \Afootnote{\uwave{dā}° K.}
} ca\edlabel{SS.5.5.77-7} \edtext{| \caesura}{
  \linenum{|\xlineref{SS.5.5.77-7}}\lemma{ca | }\Afootnote{ca  \textsc{(l. 2)} K.}
} tathā\edlabel{SS.5.5.77-9} \edtext{ca}{
  \linenum{|\xlineref{SS.5.5.77-9}}\lemma{tathā ca}\Afootnote{tathārdha A.}
} \edtext{bhāgo}{
  \Afootnote{bhāgaḥ A.}
} \edtext{sitasinduvārād \caesura}{
  \Afootnote{sitasindhuvā° A.}
} \edtext{aṅkollabījād}{
  \Afootnote{aṅkoṭhamūlād A.}
} api gairikāc ca | 
\pend

 
\pstart

                         \textsc{[1938 ed. 5.5.78ab]}
                        \caesura eṣo \edtext{'gadaḥ}{
  \Afootnote{gado K.}
} kṣaudrayuto nihanti \caesura viśeṣato maṇḍalinām viṣāṇi || 
\pend

 
\pstart
\edtext{}{
  \Afootnote{\textsc{[pre]} vaṃśatvagārdrāmalakaṃ kapitthaṃ kaṭutrikaṃ haimavatī sakuṣṭhā || A.}
}
\pend

 
\pstart
\edtext{}{
  \Afootnote{\textsc{[pre]} karañjabījaṃ tagaraṃ śirīṣapuṣpaṃ ca gopittayutaṃ nihanti |  viṣāṇi lūtondurapannagānāṃ kaiṭaṃ ca lepāñjananasyapānaiḥ || A.}
}
\pend

 
\pstart
\edtext{}{
  \Afootnote{\textsc{[pre]} pūrīṣamūtrānilagarbhasaṅgānnihanti vartyañjananābhilepaiḥ |  kācārmakothān paṭalāṃś ca ghorān puṣpaṃ ca hantyañjananasyayogaiḥ || A.}
}
\pend

 
\pstart
\edtext{}{
  \Afootnote{\textsc{[pre]} samūlapuṣpāṅkuravalkabījāt kvāthaḥ śirīṣāt trikaṭupragāḍhaḥ |  salāvaṇaḥ kṣaudrayuto 'tha pīto viśeṣataḥ kīṭaviṣaṃ nihanti || A.}
}
\pend

 
\pstart
\edtext{}{
  \Afootnote{\textsc{[pre]} kuṣṭhaṃ trikaṭukaṃ dārvī madhukaṃ lavaṇadvayam |  mālatī nāgapuṣpaṃ ca sarvāṇi madhurāṇi ca || A.}
}
\pend

 
\pstart
\edtext{}{
  \Afootnote{\textsc{[pre]} kapittharasapiṣṭo 'yaṃ śarkarākṣaudrasaṃyutaḥ |  viṣaṃ hantyagadaḥ sarvaṃ mūṣikāṇāṃ viśeṣataḥ || A.}
}
\pend

 
\pstart

                         \textsc{[1938 ed. 5.5.84ab]}
                        \caesura \edtext{somarājīyabahulā}{
  \Afootnote{°jīphalaṃ puṣpaṃ A; °vaphalā H.}
} \edtext{kadalī}{
  \Afootnote{kaṭabhī A.}
} sinduvārakaḥ\edlabel{SS.5.5.84ab-3} \edtext{|}{
  \linenum{|\xlineref{SS.5.5.84ab-3}}\lemma{sinduvārakaḥ |}\Afootnote{sindhuvārakaḥ | A.}
}
\pend

 
\pstart
\edtext{}{
  \Afootnote{\textsc{[pre]} punarnavā śirīṣasya puṣpamāragvadhārkajam | A.}
}
\pend

 
\pstart

                         \textsc{[1938 ed. 5.5.85cd]}
                        \caesura śyāmāmbaṣṭhā \edtext{tālapatrī}{
  \Afootnote{viḍaṅgāni A.}
} tathāmrāśmantako\edlabel{SS.5.5.85cd-3} \edtext{'pi}{
  \linenum{|\xlineref{SS.5.5.85cd-3}}\lemma{tathāmrāśmantako 'pi}\Afootnote{°takāni A.}
} ca | 
\pend

 
\pstart

                         \textsc{[1938 ed. 5.5.85.add-1]}
                        \caesura maṇḍūkaparṇṇī varuṇaḥ saptalā sa punarṇṇavā || 
\pend

 
\pstart

                         \textsc{[1938 ed. 5.5.84cd]}
                        \caesura corako \edtext{nāgavinnā}{
  \Afootnote{\textsc{[om]} A; nāgavi\uwave{nnā} K.}
} \edtext{ca}{
  \Afootnote{varuṇaḥ A.}
} \edtext{tathā}{
  \Afootnote{kuṣṭhaṃ A.}
} sarpasugandhikā\edlabel{SS.5.5.84cd-5} \edtext{|}{
  \linenum{|\xlineref{SS.5.5.84cd-5}}\lemma{sarpasugandhikā |}\Afootnote{sarpagandhā A.}
\lemma{|}  \Afootnote{\textsc{[add]} sasaptalā || A.}
}
\pend

 
\pstart

                         \textsc{[1938 ed. 5.5.86]}
                        \caesura bhūmī \edtext{kuravakaś}{
  \Afootnote{kurabakaś A.}
} caiva gaṇa \edtext{ekarasaḥ}{
  \Afootnote{ekasaraḥ A; ekarasa K.}
} smṛtaḥ\edlabel{SS.5.5.86-6} \edtext{|| \caesura}{
  \linenum{|\xlineref{SS.5.5.86-6}}\lemma{smṛtaḥ || }\Afootnote{smṛtāḥ ||  K.}
} \edtext{ekaikaśo}{
  \Afootnote{ekaśo A.}
} \edtext{dvandvaśo}{
  \Afootnote{dvitriśo A.}
} \edtext{vā}{
  \Afootnote{\textsc{[add]} 'pi A.}
} prayoktavyo viṣāpahaḥ\edlabel{SS.5.5.86-12} \edtext{||}{
  \linenum{|\xlineref{SS.5.5.86-12}}\lemma{viṣāpahaḥ ||}\Afootnote{viṣāpaham K; viṣāpaha iti || H.}
\lemma{||}  \Afootnote{\textsc{[add]} iti || K.}
}
\pend

 
\pstart
||\emph{\edlabel{SS.5.5.trailer-0}} \edtext{iti}{
  \linenum{|\xlineref{SS.5.5.trailer-0}}\lemma{|| iti}\Afootnote{\textsc{[om]} Nep.}
\lemma{iti}  \Afootnote{\textsc{[add]} suśrutasaṃhitāyāṃ A.}
} kalpe\edlabel{SS.5.5.trailer-2} 5 \edtext{||}{
  \linenum{|\xlineref{SS.5.5.trailer-2}}\lemma{kalpe\ldots ||}\Afootnote{kalpasthāne A \uline{H}.}
\lemma{||}  \Afootnote{\textsc{[add]} sarpadaṣṭaviṣacikitsitaṃ nāma pañcamo 'dhyāyaḥ ||5 || A; \textsc{[add]} pañcamo 'dhyāyaḥ || H.}
}
\pend

 \chapter{Kalpasthāna 6: Rats and Rabies}
\pstart

                         \textsc{[1938 ed. 5.7.1]}
                        \caesura athāto \edtext{mūṣikākalpaṃ}{
  \Afootnote{mūṣikakalpaṃ A.}
} vyākhyāsyāmaḥ\edlabel{SS.5.7.1-3} \edtext{||
}{
  \linenum{|\xlineref{SS.5.7.1-3}}\lemma{vyākhyāsyāmaḥ ||}\Afootnote{vyākhyāsyāmaḥ || K.}
}
\pend

 
\pstart
\edtext{}{
  \Afootnote{\textsc{[pre]} yathovāca bhagavān dhanvantariḥ || A.}
}
\pend

 
\pstart

                         \textsc{[1938 ed. 5.7.3]}
                        \caesura pūrvam \edtext{uktāḥ}{
  \Afootnote{\textsc{[om]} A.}
} \edtext{śukraviṣāḥ}{
  \Afootnote{°ṣā\uuline{ḥ} K; \textsc{[add]} uktā A.}
} mūṣikā ye samāsataḥ | \caesura nāmalakṣaṇabhaiṣajyair aṣṭādaśa nibodha tāḥ\edlabel{SS.5.7.3-11} \edtext{||}{
  \linenum{|\xlineref{SS.5.7.3-11}}\lemma{tāḥ ||}\Afootnote{me || A; tāṃ || H.}
}
\pend

 
\pstart

                         \textsc{[1938 ed. 5.7.4]}
                        \caesura lālanaḥ putrakaḥ kṛṣṇo\edlabel{SS.5.7.4-3} \edtext{vasiraś}{
  \linenum{|\xlineref{SS.5.7.4-3}}\lemma{kṛṣṇo vasiraś}\Afootnote{kṛṣṇava° K.}
\lemma{vasiraś}  \Afootnote{haṃsiraś A.}
} \edtext{cikkiras}{
  \Afootnote{cikvi(kki)ras A.}
} tathā \edtext{| \caesura}{
  \Afootnote{\textsc{[add]} jāyate ga H.}
} \edtext{cchucchundaro}{
  \Afootnote{\textsc{[om]} H.}
} \edtext{'ralaś}{
  \Afootnote{'lasaś A.}
} caiva kaṣāyadaśano 'pi ca | 
\pend

 
\pstart

                         \textsc{[1938 ed. 5.7.5]}
                        \caesura \edtext{kuliṅgaś}{
  \Afootnote{kuliṅgāś H.}
} \edtext{cājitaś}{
  \Afootnote{cājivaś H.}
} caiva capalaḥ kapilas tathā | \caesura kokilo\edlabel{SS.5.7.5-8} \edtext{'ruṇa}{
  \linenum{|\xlineref{SS.5.7.5-8}}\lemma{kokilo 'ruṇa}\Afootnote{kokilāruṇa Nep.}
} \edtext{saṃjñaś}{
  \Afootnote{saṃjñāś H.}
} ca mahākṛṣṇas tathonduruḥ\edlabel{SS.5.7.5-13} \edtext{|
}{
  \linenum{|\xlineref{SS.5.7.5-13}}\lemma{tathonduruḥ |}\Afootnote{tathonduraḥ || A.}
}
\pend

 
\pstart

                         \textsc{[1938 ed. 5.7.6]}
                        \caesura śvetaś\edlabel{SS.5.7.6-1} \edtext{ca}{
  \linenum{|\xlineref{SS.5.7.6-1}}\lemma{śvetaś ca}\Afootnote{śvetena A.}
} mahatā sārdhaṃ kapilenākhunā tathā | \caesura mūṣikaś ca \edtext{kapotābhas}{
  \Afootnote{\uwave{ka}po° K.}
} \edtext{tathaivāṣṭādaśa}{
  \Afootnote{°daśaḥ K.}
} smṛtāḥ || 
\pend

 
\pstart

                         \textsc{[1938 ed. 5.7.7]}
                        \caesura śukraṃ patati \edtext{yatraiṣāṃ}{
  \Afootnote{yatraiṣā H.}
} śukraspṛṣṭaiḥ spṛśanti vā | \caesura nakhadantādibhis \edtext{tasmiṃ}{
  \Afootnote{tasmin A H.}
} gātre raktaṃ praduṣyati | 
\pend

 
\pstart

                         \textsc{[1938 ed. 5.7.8]}
                        \caesura \edtext{jāyante}{
  \Afootnote{jāya\uwave{ṃ}te K; jāyate H.}
} granthayaḥ śophāḥ \edtext{karṇṇikā}{
  \Afootnote{\uuline{ki}ka° K; kaṇi\textbf{rṇṇi}kā H.}
} maṇḍalāni ca | \caesura \edtext{piṭakopacayāś}{
  \Afootnote{pīḍakopacayaś A.}
} \edtext{cogrā}{
  \Afootnote{cogro A.}
} \edtext{visarpāḥ}{
  \Afootnote{visarggāḥ H.}
} kiṭibhāni ca | 
\pend

 
\pstart

                         \textsc{[1938 ed. 5.7.9]}
                        \caesura parvabhedo rujaś cāpi\edlabel{SS.5.7.9-3} \edtext{jvaro}{
  \linenum{|\xlineref{SS.5.7.9-3}}\lemma{cāpi jvaro}\Afootnote{tīvrā A.}
} mūrcchā \edtext{ca}{
  \Afootnote{'ṅgasadanaṃ A.}
} dāruṇāḥ\edlabel{SS.5.7.9-7} \edtext{| \caesura}{
  \linenum{|\xlineref{SS.5.7.9-7}}\lemma{dāruṇāḥ | }\Afootnote{jvaraḥ |  A.}
} daurbalyam aruciḥ \edtext{sādo}{
  \Afootnote{śvāso A.}
} vamathur lomaharṣaṇam | 
\pend

 
\pstart

                         \textsc{[1938 ed. 5.7.10]}
                        \caesura daṣṭarūpaṃ samāsoktam etad vyāsam ataḥ śṛṇu || \caesura lālāsrāvo \edtext{lālanena}{
  \Afootnote{lālanā K; lāla\textsc{(l. 2)}nā ca H; \textsc{[add]} hikkā A.}
} cchardi\edlabel{SS.5.7.10-10} \edtext{hikkā}{
  \linenum{|\xlineref{SS.5.7.10-10}}\lemma{cchardi hikkā}\Afootnote{chardiś A.}
\lemma{hikkā}  \Afootnote{hikkāñ H.}
} ca jāyate |
 
\pend

 
\pstart

                         \textsc{[1938 ed. 5.7.11]}
                        \caesura taṇḍulīyakakalkaṃ tu lihyāt tatra samākṣikam || \caesura \edtext{putrakeṇāṅgasaṃsādaḥ}{
  \Afootnote{°gasādaś ca A.}
} \edtext{pāṇḍu}{
  \Afootnote{\uwave{pāṇḍu} K.}
} \edtext{valguś}{
  \Afootnote{varṇaś A.}
} ca jāyate \edtext{|}{
  \Afootnote{\textsc{[add]} | K.}
}
\pend

 
\pstart

                         \textsc{[1938 ed. 5.7.12]}
                        \caesura \edtext{cīyate}{
  \Afootnote{\uwave{cīyate} K.}
} granthibhiś cāṅgaṃ śiśur\edlabel{SS.5.7.12-4} mūṣikasaṃsthitaiḥ |
                        \caesura
 \edtext{śirīṣeṅgudipatraṃ}{
  \linenum{|\xlineref{SS.5.7.12-4}}\lemma{śiśur\ldots śirīṣeṅgudipatraṃ}\Afootnote{ākhuśāvakasannibhaiḥ |  A.}
} \edtext{tu}{
  \Afootnote{śirīśeṅgu° K; °gudakalkaṃ A.}
} lihyāt tatra samākṣikam || 
\pend

 
\pstart

                         \textsc{[1938 ed. 5.7.13]}
                        \caesura kṛṣṇenāsṛk\edlabel{SS.5.7.13-1} \edtext{chardayati}{
  \linenum{|\xlineref{SS.5.7.13-1}}\lemma{kṛṣṇenāsṛk chardayati}\Afootnote{kṛṣṇena daṃśe śopho 'sṛkchardiḥ prāyaś ca A.}
} durdine tu\edlabel{SS.5.7.13-4} viśeṣataḥ \edtext{| \caesura}{
  \linenum{|\xlineref{SS.5.7.13-4}}\lemma{tu\ldots | }\Afootnote{\textsc{[om]} A.}
} \edtext{śirīṣapatraṃ\edlabel{SS.5.7.13-7}}{
  \Afootnote{\uwave{śirīṣapatra} K.}
} \edtext{kuṣṭhailāḥ}{
  \linenum{|\xlineref{SS.5.7.13-7}}\lemma{śirīṣapatraṃ kuṣṭhailāḥ}\Afootnote{śirīṣaphalakuṣṭhaṃ tu A.}
\lemma{kuṣṭhailāḥ}  \Afootnote{kuṣṭhailā Nep.}
} pibet kiṃśukabhasmanā ||
 
\pend

 
\pstart

                         \textsc{[1938 ed. 5.7.14]}
                        \caesura \edtext{vasireṇānnavidveṣo}{
  \Afootnote{haṃsi° A; vasiroṇān° K.}
} \edtext{jṛmbho}{
  \Afootnote{jṛmbhā A.}
} \edtext{romnāñ}{
  \Afootnote{romṇāṃ A.}
} ca kuṣṭhatā\edlabel{SS.5.7.14-5} \edtext{| \caesura}{
  \linenum{|\xlineref{SS.5.7.14-5}}\lemma{kuṣṭhatā | }\Afootnote{harṣaṇam |  A; \uwave{kuṣṭhatā}  H.}
} pibed \edtext{āragvadhādin}{
  \Afootnote{°dhādiṃ A.}
} tu \edtext{vāntas}{
  \Afootnote{suvāntas A.}
} \edtext{tatrāśu}{
  \Afootnote{tatra A.}
} mānavaḥ || 
\pend

 
\pstart
\edtext{
                         \textsc{[1938 ed. 5.7.15]}
                        \caesura}{
  \Afootnote{\textsc{[pre]} cikvireṇa A.}
} \edtext{cikkireṇa}{
  \Afootnote{(From 152r)\textsc{(l. 1)}ci\uwave{kki}reṇa K.}
} śiroduḥkhaṃ śopho hikkā \edtext{vamis}{
  \Afootnote{va\uwave{mis} K.}
} tathā\edlabel{SS.5.7.15-6} \edtext{| \caesura}{
  \linenum{|\xlineref{SS.5.7.15-6}}\lemma{tathā | }\Afootnote{tathā |  K.}
} \edtext{suvānto}{
  \Afootnote{\textsc{[om]} A; \uwave{suvānto} K; vāsanto H.}
} \edtext{jālinīkvāthaiḥ}{
  \Afootnote{jālinīmadanāṅkoṭhakaṣāyair A.}
} \edtext{sāram}{
  \Afootnote{vāmayet tu A; sā\uwave{ram} K.}
} \edtext{aṅkollajam}{
  \Afootnote{tam || A.}
} pibet\edlabel{SS.5.7.15-12} \edtext{||
}{
  \linenum{|\xlineref{SS.5.7.15-12}}\lemma{pibet ||}\Afootnote{\textsc{[om]} A.}
\lemma{||}  \Afootnote{\textsc{[add]} | K.}
}
\pend

 
\pstart

                         \textsc{[1938 ed. 5.7.16cd]}
                        \caesura chucchundareṇa \edtext{viṭsaṅga}{
  \Afootnote{taṭ chardir jvaro A; \uwave{viṭchaṅgaḥ} K; vi\uwave{ṭcha}ṅgā\uuline{ḥ} H.}
} \edtext{grīvāstambha}{
  \Afootnote{daurbalyam eva A.}
} vijṛmbhikāḥ\edlabel{SS.5.7.16cd-4} \edtext{|
}{
  \linenum{|\xlineref{SS.5.7.16cd-4}}\lemma{vijṛmbhikāḥ |}\Afootnote{ca || A.}
\lemma{|}  \Afootnote{\textsc{[add]} | K.}
}
\pend

 
\pstart

                         \textsc{[1938 ed. 5.7.16ab]}
                        \caesura \edtext{yavanālārṣabhīkṣāraṃ}{
  \Afootnote{yavanā\uwave{lā}rṣa° Nep; yavanālarṣa° A.}
} \edtext{vṛhatyau}{
  \Afootnote{bṛhatyoś A; vṛhatyau\uwave{\textsc{(gap of 5, faded)}}\textsc{(l. 2)} K; vṛha\textsc{(l. 6)}tyoś H.}
} cātra\edlabel{SS.5.7.16ab-3} dāpayet \edtext{||
}{
  \linenum{|\xlineref{SS.5.7.16ab-3}}\lemma{cātra\ldots ||}\Afootnote{\textsc{[om]} K.}
}
\pend

 
\pstart
\edtext{}{
  \Afootnote{\textsc{[pre]} grīvāstambhaḥ pṛṣṭhaśopho gandhājñānaṃ visūcikā |  cavyaṃ harītakī śuṇṭhī viḍaṅgaṃ pippalī madhu || A; \textsc{[pre]} \uwave{\textsc{(gap of 32, faded)}} K.}
}
\pend

 
\pstart
\edtext{
}{
  \Afootnote{\textsc{[pre]} aṅkoṭhabījaṃ ca tathā pibed atra viṣāpaham | A.}
}
\pend

 
\pstart

                         \textsc{[1938 ed. 5.7.18cd]}
                        \caesura grīvāstambho \edtext{'ralenātha}{
  \Afootnote{'lasenordhvavāyurdaṃśe A.}
} \edtext{rujāś}{
  \Afootnote{rujā A.}
} cādaṃśamaṇḍale\edlabel{SS.5.7.18cd-4} \edtext{|}{
  \linenum{|\xlineref{SS.5.7.18cd-4}}\lemma{cādaṃśamaṇḍale |}\Afootnote{jvaraḥ || A.}
}
\pend

 
\pstart

                         \textsc{[1938 ed. 5.7.19]}
                        \caesura mahāgadam\edlabel{SS.5.7.19-1} \edtext{mahāvīryaṃ}{
  \Afootnote{sasarpiṣkaṃ A.}
} \edtext{lihyāt}{
  \linenum{|\xlineref{SS.5.7.19-1}}\lemma{mahāgadam\ldots lihyāt}\Afootnote{\textsc{[om]} K.}
} \edtext{tatra}{
  \Afootnote{\uwave{\textsc{(gap of 6, faded)}}tatra K.}
} samākṣikam || \caesura nidrā kaṣāyadantena \edtext{jāyate}{
  \Afootnote{hṛcchoṣaḥ A.}
} kārśyam eva ca | 
\pend

 
\pstart

                         \textsc{[1938 ed. 5.7.20]}
                        \caesura \edtext{lihyāt}{
  \Afootnote{kṣaudropetāḥ A.}
} \edtext{tatra}{
  \Afootnote{\textsc{[om]} A.}
} śirīṣasya \edtext{madhunā}{
  \Afootnote{lihyāt A.}
} sāramāṣakān\edlabel{SS.5.7.20-5} \edtext{|| \caesura}{
  \linenum{|\xlineref{SS.5.7.20-5}}\lemma{sāramāṣakān || }\Afootnote{sāraphalatvacaḥ |  A.}
} kuliṅgena rujaḥ śopho rājyaś \edtext{cā}{
  \Afootnote{ca A.}
} \edtext{daṃśa}{
  \Afootnote{da\uwave{mśa} K.}
} maṇḍale | 
\pend

 
\pstart

                         \textsc{[1938 ed. 5.7.21]}
                        \caesura sahe \edtext{sasinduvāre}{
  \Afootnote{sasindhuvāre A.}
} ca lihyāt tatra samākṣike || \caesura ajitena\edlabel{SS.5.7.21-8} \edtext{vamī}{
  \linenum{|\xlineref{SS.5.7.21-8}}\lemma{ajitena vamī}\Afootnote{ajitenāṅgakṛṣṇatvaṃ chardir A.}
} \edtext{mūrcchā}{
  \Afootnote{\textsc{[add]} ca A.}
} hṛdgrahaḥ kṛṣṇagātratā\edlabel{SS.5.7.21-12} \edtext{|}{
  \linenum{|\xlineref{SS.5.7.21-12}}\lemma{kṛṣṇagātratā |}\Afootnote{\textsc{[om]} A; \uwave{\textsc{(gap of 5, faded)}} K.}
}
\pend

 
\pstart
\edtext{tatra}{
  \Afootnote{\textsc{[om]} A K.}
} \edtext{snuhākṣīrayuktāṃ\edlabel{SS.5.7.22-1}}{
  \Afootnote{\uwave{\textsc{(gap of 5, faded)}}\textsc{(l. 3)}ra° K; snukkṣīrapiṣṭāṃ pālindīṃ A.}
} \edtext{mañjiṣṭhāṃ}{
  \linenum{|\xlineref{SS.5.7.22-1}}\lemma{snuhākṣīrayuktāṃ mañjiṣṭhāṃ}\Afootnote{°ktāṃmañjiṣṭhāṃ H.}
} madhunā lihet || \caesura capalena bhavec\edlabel{SS.5.7.22-7} \edtext{chardir}{
  \linenum{|\xlineref{SS.5.7.22-7}}\lemma{bhavec chardir}\Afootnote{bhavec ccharddir H.}
} mūrcchā ca saha tṛṣṇayā | 
\pend

 
\pstart

                         \textsc{[1938 ed. 5.7.23]}
                        \caesura \edtext{sabhasmakāṣṭhāṃ\edlabel{SS.5.7.23-1}}{
  \Afootnote{°kāṣṭhā K.}
} \edtext{sajaṭāṃ}{
  \linenum{|\xlineref{SS.5.7.23-1}}\lemma{sabhasmakāṣṭhāṃ sajaṭāṃ}\Afootnote{\textsc{[om]} A.}
} kṣaudreṇa \edtext{triphalāṃ}{
  \Afootnote{triphalaṃ K.}
} pibet\edlabel{SS.5.7.23-5} \edtext{|| \caesura}{
  \linenum{|\xlineref{SS.5.7.23-5}}\lemma{pibet || }\Afootnote{lihyād A.}
\lemma{|| }  \Afootnote{\textsc{[add]} bhadrakāṣṭhajaṭānvitām |  A.}
} kapilena \edtext{vraṇaṃ}{
  \Afootnote{vraṇe A H.}
} \edtext{kothaṃ}{
  \Afootnote{kotho A.}
} jvaro \edtext{granthyudgamas}{
  \Afootnote{graṃnthyu° K; granthy udgamaḥ A.}
} tathā\edlabel{SS.5.7.23-12} \edtext{|}{
  \linenum{|\xlineref{SS.5.7.23-12}}\lemma{tathā |}\Afootnote{satṛṭ || A.}
}
\pend

 
\pstart

                         \textsc{[1938 ed. 5.7.24]}
                        \caesura \edtext{kṣaudreṇa}{
  \Afootnote{\textsc{[om]} A.}
} lihyāc\edlabel{SS.5.7.24-2} \edtext{chvetātra}{
  \linenum{|\xlineref{SS.5.7.24-2}}\lemma{lihyāc chvetātra}\Afootnote{lihyān madhuyutāṃ śvetāṃ A.}
} \edtext{śvetā}{
  \Afootnote{śvetāṃ A.}
} \edtext{vāpi}{
  \Afootnote{cāpi A.}
} punarṇṇavā\edlabel{SS.5.7.24-6} \edtext{|| \caesura}{
  \linenum{|\xlineref{SS.5.7.24-6}}\lemma{punarṇṇavā || }\Afootnote{punarnavām |  A.}
} granthayaḥ \edtext{kokilenoktā}{
  \Afootnote{°nogrā A.}
} jvaro\edlabel{SS.5.7.24-10} \edtext{dāhaś}{
  \linenum{|\xlineref{SS.5.7.24-10}}\lemma{jvaro dāhaś}\Afootnote{\uwave{\textsc{(gap of 5, faded)}}\textsc{(l. 4)}haś K.}
} ca dāruṇaḥ | 
\pend

 
\pstart

                         \textsc{[1938 ed. 5.7.25]}
                        \caesura nīlāvarṣābhuniḥkvāthaiḥ\edlabel{SS.5.7.25-1} \edtext{siddhaṃ}{
  \linenum{|\xlineref{SS.5.7.25-1}}\lemma{nīlāvarṣābhuniḥkvāthaiḥ siddhaṃ}\Afootnote{varṣābhūnīlinīkvāthakalkasiddhaṃ A.}
} \edtext{tatra}{
  \Afootnote{ghṛtaṃ A.}
} pibed ghṛtam\edlabel{SS.5.7.25-5} \edtext{|| \caesura}{
  \linenum{|\xlineref{SS.5.7.25-5}}\lemma{ghṛtam || }\Afootnote{\textsc{[om]} A.}
} aruṇenānilaḥ kruddho \edtext{vātajāṃ}{
  \Afootnote{vātajān A.}
} kurute gadān\edlabel{SS.5.7.25-11} \edtext{|}{
  \linenum{|\xlineref{SS.5.7.25-11}}\lemma{gadān |}\Afootnote{gadāṃ | Nep.}
}
\pend

 
\pstart

                         \textsc{[1938 ed. 5.7.26]}
                        \caesura mahākṛṣṇena pittañ ca śvetena\edlabel{SS.5.7.26-4} \edtext{kapha}{
  \linenum{|\xlineref{SS.5.7.26-4}}\lemma{śvetena kapha}\Afootnote{śvete\textbf{na}kapha K.}
} eva ca | \caesura mahatā\edlabel{SS.5.7.26-9} \edtext{kapilenāsṛk}{
  \linenum{|\xlineref{SS.5.7.26-9}}\lemma{mahatā kapilenāsṛk}\Afootnote{mahatā\textsc{(l. 5)}ka° H.}
} kapotena catuṣṭayam || 
\pend

 
\pstart

                         \textsc{[1938 ed. 5.7.27]}
                        \caesura bhavanti \edtext{caiṣān}{
  \Afootnote{caiṣāṃ A.}
} daṃśeṣu granthimaṇḍalakarṇṇikāḥ\edlabel{SS.5.7.27-4} \edtext{|| \caesura}{
  \linenum{|\xlineref{SS.5.7.27-4}}\lemma{granthimaṇḍalakarṇṇikāḥ || }\Afootnote{°rṇṇikā ||  K.}
} \edtext{piḍakopacayāś}{
  \Afootnote{\uwave{piḍakopacayā}\textsc{(l. 4)}ś K; °cayaś A.}
} \edtext{cāṅge}{
  \Afootnote{cograḥ A.}
} \edtext{śophāś}{
  \Afootnote{śophaś A.}
} ca \edtext{bhṛśa}{
  \Afootnote{bhṛṣa K.}
} dāruṇāḥ\edlabel{SS.5.7.27-11} \edtext{|}{
  \linenum{|\xlineref{SS.5.7.27-11}}\lemma{dāruṇāḥ |}\Afootnote{dāruṇaḥ \uline{A} \uline{K}.}
}
\pend

 
\pstart

                         \textsc{[1938 ed. 5.7.28]}
                        \caesura dadhikṣīraghṛtaprasthās\edlabel{SS.5.7.28-1} \edtext{trayaḥ}{
  \linenum{|\xlineref{SS.5.7.28-1}}\lemma{dadhikṣīraghṛtaprasthās trayaḥ}\Afootnote{°ghṛtamprasthāstra\textsc{(l. 6)}yaḥ H; °prasthāstrayaḥ A K.}
} \edtext{pratyekaśo}{
  \Afootnote{pratyeka\textbf{śo} H.}
} mitāḥ\edlabel{SS.5.7.28-4} \edtext{| \caesura}{
  \linenum{|\xlineref{SS.5.7.28-4}}\lemma{mitāḥ | }\Afootnote{matāḥ |  A; mitā\textbf{1}sammitāḥ ||  H.}
} \edtext{karañjāragvadhaṃ}{
  \Afootnote{°gvadha A.}
} \edtext{vyoṣaṃ}{
  \Afootnote{vyoṣa \uline{A} K.}
} \edtext{bṛhaty}{
  \Afootnote{vṛhyaty H.}
} \edtext{aṃśumatī}{
  \Afootnote{aṅśumatī H.}
} sthirāḥ\edlabel{SS.5.7.28-10} \edtext{|}{
  \linenum{|\xlineref{SS.5.7.28-10}}\lemma{sthirāḥ |}\Afootnote{sthirā | K.}
}
\pend

 
\pstart

                         \textsc{[1938 ed. 5.7.29]}
                        \caesura \edtext{niṣkvāthya}{
  \Afootnote{ni\uwave{ṣkvā}thya K.}
} \edtext{tasya}{
  \Afootnote{caiṣāṃ A.}
} kvāthasya \edtext{caturthāṃśaḥ}{
  \Afootnote{caturtho 'ṃśaḥ A.}
} punarbhavet\edlabel{SS.5.7.29-5} \edtext{|| \caesura}{
  \linenum{|\xlineref{SS.5.7.29-5}}\lemma{punarbhavet || }\Afootnote{punar bhavet ||  K; punar vbhavet ||  H.}
} \edtext{tṛvṛt}{
  \Afootnote{trivṛd A.}
} \edtext{tilvāmṛtā}{
  \Afootnote{gojy amṛtā A.}
} \edtext{vakra}{
  \Afootnote{vakrā H.}
} sarvagandhāgamṛttikāḥ\edlabel{SS.5.7.29-10} \edtext{|
}{
  \linenum{|\xlineref{SS.5.7.29-10}}\lemma{sarvagandhāgamṛttikāḥ |}\Afootnote{sarpagandhāḥ samṛttikāḥ || A; \uwave{sarvāgandhāgamṛttikā |} K; °ttikā || H.}
}
\pend

 
\pstart

                         \textsc{[1938 ed. 5.7.30]}
                        \caesura kapitthadāḍimatvak ca \edtext{ślakṣṇapiṣṭāni}{
  \Afootnote{\uwave{śla}° K; ślakṣṇapiṣṭāḥ A; ślakṣṇapiṣṭhāni H.}
} dāpayet\edlabel{SS.5.7.30-4} \edtext{| \caesura}{
  \linenum{|\xlineref{SS.5.7.30-4}}\lemma{dāpayet | }\Afootnote{pradā° A; dā\textsc{(l. 5)}pa° K.}
} tat sarvam\edlabel{SS.5.7.30-7} \edtext{ekataḥ}{
  \linenum{|\xlineref{SS.5.7.30-7}}\lemma{sarvam ekataḥ}\Afootnote{sarvvamekataḥ H.}
} kṛtvā śanair\edlabel{SS.5.7.30-10} \edtext{mṛdvagninā}{
  \linenum{|\xlineref{SS.5.7.30-10}}\lemma{śanair mṛdvagninā}\Afootnote{śanairmmṛdva° H.}
\lemma{mṛdvagninā}  \Afootnote{mṛdv agninā A.}
} pacet || 
\pend

 
\pstart

                         \textsc{[1938 ed. 5.7.31]}
                        \caesura pañcānām aruṇādīnāṃ \edtext{viṣam}{
  \Afootnote{viṣa\textbf{m} K.}
} \edtext{etad}{
  \Afootnote{etad K.}
} vyapohati | \caesura \edtext{kākādanīkākamācī}{
  \Afootnote{°mācyoḥ A.}
} \edtext{svarasveṣv}{
  \Afootnote{svaraseṣv A H.}
} athavā kṛtam | 
\pend

 
\pstart

                         \textsc{[1938 ed. 5.7.32]}
                        \caesura sirāś ca \edtext{vyadhayet}{
  \Afootnote{srāvayet A.}
} \edtext{prāptāḥ}{
  \Afootnote{prā\uwave{ptāḥ} K.}
} kuryāt saṃśodhanāni ca | \caesura \edtext{sarvveṣu}{
  \Afootnote{sarveṣāṃ A; |  K.}
} \edtext{vā\edlabel{SS.5.7.32-10}}{
  \Afootnote{ca A.}
} \edtext{vidhiḥ}{
  \linenum{|\xlineref{SS.5.7.32-10}}\lemma{vā vidhiḥ}\Afootnote{\textsc{[om]} K.}
} \edtext{kāryo}{
  \Afootnote{\uwave{\textsc{(gap of 6, faded)}}kāryo K.}
} mūṣikāṇām viṣeṣv ayaṃ\edlabel{SS.5.7.32-15} \edtext{||}{
  \linenum{|\xlineref{SS.5.7.32-15}}\lemma{ayaṃ ||}\Afootnote{a\uwave{yaṃ}|| K.}
}
\pend

 
\pstart

                         \textsc{[1938 ed. 5.7.33]}
                        \caesura dagdhvā visrāvayed daṃśaṃ \edtext{pracchitañ}{
  \Afootnote{pracchitaṃñ K.}
} ca pralepayet | \caesura śirīṣarajanīvakraṃ\edlabel{SS.5.7.33-8} \edtext{kuṃkumair}{
  \linenum{|\xlineref{SS.5.7.33-8}}\lemma{śirīṣarajanīvakraṃ kuṃkumair}\Afootnote{°nīkuṣṭhakuṅkumair A.}
} amṛtāyutaiḥ | 
\pend

 
\pstart

                         \textsc{[1938 ed. 5.7.34ab]}
                        \caesura cchardanaṃ \edtext{nīlinīkvāthaiḥ}{
  \Afootnote{jāli° A; °kvā\textbf{thaṃ}thaiḥ H.}
} \edtext{śukākhyāṅkollayor}{
  \Afootnote{°koṭhayor A.}
} api | 
\pend

 
\pstart
\edtext{}{
  \Afootnote{\textsc{[pre]} devadālīphalaṃ caiva dadhnā pītvā viṣaṃ vamet |  sarvamūṣikadaṣṭānām eṣa yogaḥ sukhāvahaḥ || A.}
}
\pend

 
\pstart
\edtext{}{
  \Afootnote{\textsc{[pre]} phalaṃ vacā devadālī kuṣṭhaṃ gomūtrapeṣitam |  pūrvakalpena yojyāḥ syuḥ sarvonduruviṣacchidaḥ || A.}
}
\pend

 
\pstart

                         \textsc{[1938 ed. 5.7.37]}
                        \caesura virecane \edtext{tṛvṛddantītriphalākalka}{
  \Afootnote{trivṛ° A.}
} iṣyate | \caesura śirovirecane sāraḥ \edtext{śirīṣasya}{
  \Afootnote{śirīśasya K.}
} phalāni vā\edlabel{SS.5.7.37-9} \edtext{|}{
  \linenum{|\xlineref{SS.5.7.37-9}}\lemma{vā |}\Afootnote{ca || A.}
}
\pend

 
\pstart
\edtext{
                         \textsc{[1938 ed. 5.7.38]}
                        \caesura}{
  \Afootnote{\textsc{[pre]} hitas A.}
} \edtext{kaṭutrikāḍhyaś}{
  \Afootnote{trikaṭukāḍhyaś A.}
} ca \edtext{hito}{
  \Afootnote{\textsc{[om]} A; hi\uwave{to} K.}
} \edtext{gomayasvaraso}{
  \Afootnote{go\textsc{(l. 2)}mayaḥ sva° K; gomayaḥ svara\uwave{so} H.}
} 'ñjane | \caesura kapitthagomayarasau \edtext{sakṣaudrau}{
  \Afootnote{\textsc{[om]} A; sakṣaudrair H.}
} \edtext{leha}{
  \Afootnote{lihyān A.}
} iṣyate\edlabel{SS.5.7.38-10} \edtext{|
}{
  \linenum{|\xlineref{SS.5.7.38-10}}\lemma{iṣyate |}\Afootnote{mākṣikasaṃyutau || A.}
}
\pend

 
\pstart
\edtext{}{
  \Afootnote{\textsc{[pre]} rasāñjanaharidrendrayavakaṭvīṣu vā kṛtam |  prātaḥ sātiviṣaṃ kalkaṃ lihyān mākṣikasaṃyutam || A.}
}
\pend

 
\pstart

                         \textsc{[1938 ed. 5.7.40]}
                        \caesura taṇḍulīyakamūleṣu sarpiḥ siddhaṃ piben naraḥ | \caesura āsphotamūlasiddham vā \edtext{pañcakāpittam}{
  \Afootnote{°pittham A; °pi\uwave{tta}m Nep.}
} eva vā | 
\pend

 
\pstart

                         \textsc{[1938 ed. 5.7.41]}
                        \caesura \edtext{mūṣikāṇāṃ}{
  \Afootnote{mūṣikānām H.}
} viṣaṃ prāyaḥ kupyaty \edtext{abhreṣu}{
  \Afootnote{abhreṣv A.}
} nirhṛtam\edlabel{SS.5.7.41-6} \edtext{| \caesura}{
  \linenum{|\xlineref{SS.5.7.41-6}}\lemma{nirhṛtam | }\Afootnote{ani° A.}
} tatrāpy eṣa vidhiḥ kāryaḥ yaś ca dūṣīviṣāpahaḥ || 
\pend

 
\pstart

                         \textsc{[1938 ed. 5.7.42]}
                        \caesura \edtext{sthirāṃ\edlabel{SS.5.7.42-1}}{
  \Afootnote{sthirāṇāṃ A; sthirān H.}
} \edtext{mandarujāś}{
  \linenum{|\xlineref{SS.5.7.42-1}}\lemma{sthirāṃ mandarujāś}\Afootnote{sthirāṃ\uwave{manda}\textsc{(l. 3)}rujaś K.}
\lemma{mandarujāś}  \Afootnote{rujatāṃ A; manda\textsc{(l. 6)}rujāñ H.}
} \edtext{cāpi}{
  \Afootnote{vā 'pi vraṇānāṃ A.}
} karṇṇikāṃ \edtext{pracchayed}{
  \Afootnote{\textsc{[om]} A.}
} bhiṣak | \caesura \edtext{sarvasminn}{
  \Afootnote{pāṭayitvā yathādoṣaṃ A.}
} eva\edlabel{SS.5.7.42-9} tu \edtext{viṣe}{
  \linenum{|\xlineref{SS.5.7.42-9}}\lemma{eva\ldots viṣe}\Afootnote{\textsc{[om]} A.}
} vraṇavac \edtext{cācaret}{
  \Afootnote{cāpi śodhayet || A.}
} kriyām\edlabel{SS.5.7.42-14} \edtext{||}{
  \linenum{|\xlineref{SS.5.7.42-14}}\lemma{kriyām ||}\Afootnote{\textsc{[om]} A.}
}
\pend

 
\pstart

                         \textsc{[1938 ed. 5.7.43]}
                        \caesura śvaśṛgālavṛkavyāghratarakṣvāder\edlabel{SS.5.7.43-1} \edtext{viṣaṃ}{
  \linenum{|\xlineref{SS.5.7.43-1}}\lemma{śvaśṛgālavṛkavyāghratarakṣvāder viṣaṃ}\Afootnote{°latarakṣvṛkṣavyāghrādīnāṃ A.}
} yadā \edtext{| \caesura}{
  \Afootnote{\textsc{[add]} 'nilaḥ |  A.}
} \edtext{śleṣmā}{
  \Afootnote{śleṣma A.}
} \edtext{praduṣṭo}{
  \Afootnote{praduṣṭā H.}
} \edtext{muṣṇāti}{
  \Afootnote{\uwave{mu}ṣṇāti K; puṣṇāti H.}
} \edtext{saṃjñāṃ}{
  \Afootnote{saṃ\uwave{jñaṃ} K.}
} saṃjñāvahāśritaḥ\edlabel{SS.5.7.43-9} \edtext{||
}{
  \linenum{|\xlineref{SS.5.7.43-9}}\lemma{saṃjñāvahāśritaḥ ||}\Afootnote{\uwave{srotrovalāśritaḥ} || K.}
}
\pend

 
\pstart

                         \textsc{[1938 ed. 5.7.44]}
                        \caesura tadā prasrastalāṅgūlahanuskandho 'bhilālimān\edlabel{SS.5.7.44-3} \edtext{| \caesura}{
  \linenum{|\xlineref{SS.5.7.44-3}}\lemma{'bhilālimān | }\Afootnote{\uwave{bhi}lā° K; 'kilā° H; 'tilālavān |  A.}
} \edtext{avyaktavadhiro}{
  \Afootnote{atyarthabadhiro A.}
} 'ndhaś ca so 'nyonyam abhidhāvati | 
\pend

 
\pstart

                         \textsc{[1938 ed. 5.7.45]}
                        \caesura \edtext{tena}{
  \Afootnote{tenonmattena A.}
} daṣṭasya \edtext{cāṅge syuḥ
}{
  \Afootnote{daṃṣṭriṇā saviṣeṇa tu |  A.}
} \edtext{suptaḥ}{
  \Afootnote{\textsc{[om]} A.}
} \edtext{kṛṣṇaṃ}{
  \Afootnote{suptatā jāyate daṃśe A.}
} kṣaraty \edtext{asṛk}{
  \Afootnote{cātisravaty A.}
} || 
\pend

 
\pstart

                         \textsc{[1938 ed. 5.7.46]}
                        \caesura digdhaviddhasya liṅgena \edtext{prāyaśaś}{
  \Afootnote{prāyaśa\textbf{ś} cāpyaś H.}
} cābhiliṅgitaḥ\edlabel{SS.5.7.46-4} \edtext{| \caesura}{
  \linenum{|\xlineref{SS.5.7.46-4}}\lemma{cābhiliṅgitaḥ | }\Afootnote{copalakṣitaḥ |  A.}
} yena cāpi bhaved daṣṭas tasya \edtext{ceṣṭārutan}{
  \Afootnote{ceṣṭāṃ rutaṃ A.}
} naraḥ | 
\pend

 
\pstart

                         \textsc{[1938 ed. 5.7.47]}
                        \caesura bahuśaḥ pratikurvāṇaḥ kriyāhīno vinaśyati\edlabel{SS.5.7.47-4} \edtext{|| \caesura}{
  \linenum{|\xlineref{SS.5.7.47-4}}\lemma{vinaśyati || }\Afootnote{viśyati ||  K.}
} daṃṣṭriṇā yena \edtext{daṣṭas}{
  \Afootnote{\textsc{[add]} ca A.}
} \edtext{tu}{
  \Afootnote{tad A.}
} \edtext{taṃ}{
  \Afootnote{rūpaṃ A.}
} \edtext{daṣṭo}{
  \Afootnote{yas A.}
} \edtext{yadi}{
  \Afootnote{tu A.}
} paśyati | 
\pend

 
\pstart

                         \textsc{[1938 ed. 5.7.48]}
                        \caesura apsu vā yadi \edtext{vādarśe\edlabel{SS.5.7.48-4}}{
  \Afootnote{vādarśo A.}
} \edtext{'riṣṭaṃ}{
  \linenum{|\xlineref{SS.5.7.48-4}}\lemma{vādarśe 'riṣṭaṃ}\Afootnote{vādarśeriṣṭaṃ K.}
} tasya vinirdiśet | \caesura \edtext{yadi}{
  \Afootnote{\textsc{[om]} A.}
} \edtext{trasyaty\edlabel{SS.5.7.48-10}}{
  \Afootnote{\textsc{[add]} akasmād yo A.}
} \edtext{adaṣṭo}{
  \linenum{|\xlineref{SS.5.7.48-10}}\lemma{trasyaty adaṣṭo}\Afootnote{trasya\textbf{tya}daṣṭo K.}
\lemma{adaṣṭo}  \Afootnote{'bhīkṣṇaṃ dṛṣṭvā spṛṣṭvā A.}
} 'pi śabdasparśanadarśanaiḥ\edlabel{SS.5.7.48-13} \edtext{|}{
  \linenum{|\xlineref{SS.5.7.48-13}}\lemma{śabdasparśanadarśanaiḥ |}\Afootnote{śabdasyarśa° K; vā A; \uwave{śabdasyarśanadarśamaiḥ} H.}
\lemma{|}  \Afootnote{\textsc{[add]} jalam || A; \textsc{[add]} || H.}
}
\pend

 
\pstart

                         \textsc{[1938 ed. 5.7.49ab]}
                        \caesura jalatrāsaṃ tu \edtext{taṃ}{
  \Afootnote{\textsc{[om]} A.}
} \edtext{vidyād}{
  \Afootnote{vidyāt taṃ A.}
} \edtext{ṛṣṭaṃ}{
  \Afootnote{riṣṭaṃ A; daṣṭaṃ K; da\textsc{(l. 4)}ṣṭan H.}
} tad api kīrttitam | 
\pend

 
\pstart
\edtext{}{
  \Afootnote{\textsc{[pre]} adaṣṭo vā jalatrāsī na kathañcana sidhyati || A.}
}
\pend

 
\pstart
\edtext{}{
  \Afootnote{\textsc{[pre]} prasupto 'thotthito vā 'pi svasthas trasto na sidhyati | A.}
}
\pend

 
\pstart
\edtext{
                         \textsc{[1938 ed. 5.7.50cd]}
                        \caesura}{
  \Afootnote{\textsc{[pre]} daṃśaṃ A.}
} \edtext{visrāvya}{
  \Afootnote{visrā\textbf{vya} H.}
} \edtext{daṃśaṃ}{
  \Afootnote{\textsc{[om]} A.}
} \edtext{taṃ}{
  \Afootnote{tair A.}
} daṣṭe sarpiṣā paridāhitam | 
\pend

 
\pstart

                         \textsc{[1938 ed. 5.7.51]}
                        \caesura pradihyād agadaiḥ sarpiḥ purāṇaṃ \edtext{cāpi}{
  \Afootnote{\textsc{[om]} A.}
} pāyayet\edlabel{SS.5.7.51-6} \edtext{| \caesura}{
  \linenum{|\xlineref{SS.5.7.51-6}}\lemma{pāyayet | }\Afootnote{pāyayeta A.}
} \edtext{arkakṣīrayutañ}{
  \Afootnote{ca |  arka° A.}
} \edtext{cāpi}{
  \Afootnote{hy A.}
} \edtext{śīghran}{
  \Afootnote{asya A.}
} \edtext{dadyād}{
  \Afootnote{\textsc{[add]} cāpi A.}
} virecanam\edlabel{SS.5.7.51-12} \edtext{|
}{
  \linenum{|\xlineref{SS.5.7.51-12}}\lemma{virecanam |}\Afootnote{viśodhanam || A.}
}
\pend

 
\pstart

                         \textsc{[1938 ed. 5.7.52ab]}
                        \caesura śvetāṃ \edtext{punarṇṇavāñ}{
  \Afootnote{punarnavāṃ A.}
} \edtext{cāsyai}{
  \Afootnote{cāsya A; cāsye H.}
} dadyād dhuttūrakāyutām\edlabel{SS.5.7.52ab-5} \edtext{|}{
  \linenum{|\xlineref{SS.5.7.52ab-5}}\lemma{dhuttūrakāyutām |}\Afootnote{dhattū° A.}
}
\pend

 
\pstart
\edtext{}{
  \Afootnote{\textsc{[pre]} palalaṃ tilatailaṃ ca rūpikāyāḥ payo guḍaḥ || A.}
}
\pend

 
\pstart
\edtext{}{
  \Afootnote{\textsc{[pre]} nihanti viṣamālarkaṃ meghavṛndam ivānilaḥ |  mūlasya śarapuṅkhāyāḥ karṣaṃ dhattūrakārdhikam || A.}
}
\pend

 
\pstart
\edtext{}{
  \Afootnote{\textsc{[pre]} taṇḍulodakam ādāya peṣayet taṇḍulaiḥ saha |  unmattakasya patrais tu saṃveṣṭyāpūpakaṃ pacet || A.}
}
\pend

 
\pstart
\edtext{}{
  \Afootnote{\textsc{[pre]} khāded auṣadhakāle tamalarkaviṣadūṣitaḥ |  karoti śvavikārāṃs tu tasmiñ jīryati cauṣadhe || A.}
}
\pend

 
\pstart
\edtext{}{
  \Afootnote{\textsc{[pre]} vikārāḥ śiśire yāpyā gṛhe vārivivarjite |  tataḥ śāntavikāras tu snātvā caivāpare 'hani || A.}
}
\pend

 
\pstart
\edtext{}{
  \Afootnote{\textsc{[pre]} śāliṣāṣṭikayor bhaktaṃ kṣīreṇoṣṇena bhojayet |  dinatraye pañcame vā vidhir eṣo 'rdhamātrayā || A.}
}
\pend

 
\pstart
\edtext{}{
  \Afootnote{\textsc{[pre]} kartavyo bhiṣajā 'vaśyam alarkaviṣanāśanaḥ |  kupyet svayaṃ viṣaṃ yasya na sa jīvati mānavaḥ || A.}
}
\pend

 
\pstart
\edtext{}{
  \Afootnote{\textsc{[pre]} tasmāt prakopayed āśu svayaṃ yāvat prakupyati |  bījaratnauṣadhīgarbhaiḥ kumbhaiḥ śītāmbupūritaiḥ || A.}
}
\pend

 
\pstart

                         \textsc{[1938 ed. 5.7.60ab]}
                        \caesura snāpayet taṃ nadītīre samantrair vā catuṣpathe | 
\pend

 
\pstart
\edtext{}{
  \Afootnote{\textsc{[pre]} baliṃ nivedya tatrāpi piṇyākaṃ palalaṃ dadhi || A.}
}
\pend

 
\pstart

                         \textsc{[1938 ed. 5.7.60.add-1]}
                        \caesura \edtext{bījaratnauṣadhīgarbhaiḥ}{
  \Afootnote{°gavbhaiḥ H.}
} kumbhaiḥ śītāmbupūritaiḥ || 
\pend

 
\pstart
\edtext{}{
  \Afootnote{\textsc{[pre]} mālyāni ca vicitrāṇi māṃsaṃ pakvāmakaṃ tathā | A.}
}
\pend

 
\pstart

                         \textsc{[1938 ed. 5.7.61cd]}
                        \caesura \edtext{alarkādhipate}{
  \Afootnote{alakādhi° A H.}
} yakṣa sārameyagaṇādhipa | 
\pend

 
\pstart

                         \textsc{[1938 ed. 5.7.62]}
                        \caesura alarkajuṣṭam etan me nirviṣaṃ kuru mā cirāt || svāhā\edlabel{SS.5.7.62-9} \edtext{|| \caesura}{
  \linenum{|\xlineref{SS.5.7.62-9}}\lemma{svāhā || }\Afootnote{\textsc{[om]} A.}
} dadyāt \edtext{saṃśodhanan}{
  \Afootnote{saśo° K; saṃśodhanaṃ A.}
} tīkṣṇam \edtext{evaṃ\edlabel{SS.5.7.62-14}}{
  \Afootnote{avasyan H.}
} \edtext{snātasya}{
  \linenum{|\xlineref{SS.5.7.62-14}}\lemma{evaṃ snātasya}\Afootnote{avasyāṃn tasyaṃ K.}
\lemma{snātasya}  \Afootnote{tasya H.}
} dehinaḥ |
 
\pend

 
\pstart

                         \textsc{[1938 ed. 5.7.63]}
                        \caesura aśuddhasya surūḍhe 'pi vraṇe kupyati tad viṣam | 
\pend

 
\pstart
\edtext{}{
  \Afootnote{\textsc{[pre]} śvādayo 'bhihitā vyālā ye 'tra daṃṣṭrāviṣā mayā || A.}
}
\pend

 
\pstart

                         \textsc{[5.7.63.1]}
                        \caesura prasupto votthito vāpi svasthaḥ \edtext{trasto}{
  \Afootnote{\uwave{trasto} K; svastho H.}
} na sidhyati | \caesura jalatrāsī ca yo martyo \edtext{daṣṭe}{
  \Afootnote{\textbf{daṣṭo} H.}
} yaś \edtext{ca}{
  \Afootnote{\textsc{[add]} darśayaś ca H.}
} prakupyatīti \edtext{||}{
  \Afootnote{\textsc{[add]} || H.}
}
\pend

 
\pstart
\edtext{}{
  \Afootnote{\textsc{[pre]} ataḥ karoti daṣṭas tu teṣāṃ ceṣṭāṃ rutaṃ naraḥ |  bahuśaḥ pratikurvāṇo na cirān mriyate ca saḥ || A.}
}
\pend

 
\pstart
\edtext{}{
  \Afootnote{\textsc{[pre]} nakhadantakṣataṃ vyālair yat kṛtaṃ tad dhi mardayet |  siñcet tailena koṣṇena te hi vātaprakopakāḥ || A.}
}
\pend

 
\pstart
\edtext{kalpe\emph{\edlabel{SS.5.7.trailer-0}}}{
  \Afootnote{\textsc{[pre]} iti suśrutasaṃhitāyāṃ A.}
} 6 \edtext{||}{
  \linenum{|\xlineref{SS.5.7.trailer-0}}\lemma{kalpe\ldots ||}\Afootnote{kalpasthāne A \uline{H}.}
} o\edlabel{SS.5.7.trailer-3} \edtext{||}{
  \linenum{|\xlineref{SS.5.7.trailer-3}}\lemma{o ||}\Afootnote{mūṣikakalpo A; \uwave{6} || ❈ || K; ṣaṣṭho H.}
\lemma{||}  \Afootnote{\textsc{[add]} nāma saptamo 'dhyāyaḥ ||7 || A; \textsc{[add]} '\textsc{(l. 2)}dhyāyaḥ || H.}
}
\pend

  \chapter{Kalpasthāna 7: Beating Drums}
\pstart

                         \textsc{[1938 ed. 5.6.1]}
                        \caesura athāto dundubhisvanīyaṃ \edtext{kalpaṃ}{
  \Afootnote{kaṃlpa K.}
} vyākhyāsyāmaḥ\edlabel{SS.5.6.1-4} \edtext{||}{
  \linenum{|\xlineref{SS.5.6.1-4}}\lemma{vyākhyāsyāmaḥ ||}\Afootnote{vyākhyāsyāmaḥ || K.}
} 
 
\pend

 
\pstart
\edtext{}{
  \Afootnote{\textsc{[pre]} yathovāca bhagavān dhanvantariḥ || A.}
}
\pend

 
\pstart

                         \textsc{[1938 ed. 5.6.3]}
                        \caesura \edtext{dhavāśvakarṇṇa}{
  \Afootnote{\textsc{[add]} śirīṣa A.}
} \edtext{tiniśa}{
  \Afootnote{\textsc{[add]} palāśa A.}
} \edtext{picumarda}{
  \Afootnote{\textsc{[add]} picumardda H.}
} \edtext{pāṭalī}{
  \Afootnote{pāṭali A.}
} \edtext{pāribhadrakodumbara}{
  \Afootnote{°drakāmrodumbara A.}
} \edtext{karaghāṭakārjuna}{
  \Afootnote{karahāṭa° kakubha A; karaghātakā° K; karaghāṭārjjuna H.}
} sarjja kapītana \edtext{śleṣmātakāṅkoṭha}{
  \Afootnote{śleṣmānta° H; śleṣmāta kāṅkoṭhāmalakapragraha A.}
} kuṭaja śamī kapitthāśmantakārka \edtext{ciribilva}{
  \Afootnote{cirabilva A.}
} \edtext{mahāvṛkṣārala}{
  \Afootnote{°kṣāruṣkarāralu A; °ralu H.}
} madhuka \edtext{madhukaśigru}{
  \Afootnote{madhuśigru A H; madhukaṃśrigru K.}
} śāka \edtext{gojī}{
  \Afootnote{\textsc{[add]} mūrvā A.}
} \edtext{bhūja}{
  \Afootnote{bhūrja A.}
} \edtext{tilvakekṣuraka gopaghoṇṭārimedānāṃ
}{
  \Afootnote{tilvake\uwave{kṣu}raka K.}
} bhasmāny āhṛtya gavāṃ mūtreṇa kṣārakalpena parisrāvya vipacet | dadyāc cātra pippalī \edtext{pippalīmūla}{
  \Afootnote{\textsc{[om]} A H.}
} taṇḍulīyaka varāṅga coraka \edtext{mañjiṣṭhā}{
  \Afootnote{coca A; cocaka H.}
} karañjikā hastipippalī \edtext{viḍaṅgā}{
  \Afootnote{\textsc{[add]} marica A.}
} \edtext{gṛhadhūmānanta}{
  \Afootnote{viḍaṅga A.}
} \edtext{soma}{
  \Afootnote{°nantā A.}
} sarala \edtext{bāhlīka}{
  \Afootnote{saralā A.}
} \edtext{kuśāmra}{
  \Afootnote{vā\uwave{vālhīka} K.}
} \edtext{sarṣapa}{
  \Afootnote{guhākośāmra śveta A.}
} varuṇa \edtext{plakṣa}{
  \Afootnote{\textsc{[add]} lavaṇa A.}
} nicula \edtext{vardhamāna}{
  \Afootnote{niculaka vañjula vakrāla A.}
} vañjula \edtext{putraśreṇī}{
  \Afootnote{\textsc{[om]} A.}
} saptaparṇṇa ṭuṇṭukailavāluka \edtext{nāgadanty}{
  \Afootnote{°lu H.}
} ativiṣā \edtext{bhadradāru}{
  \Afootnote{\textsc{[add]} bhayā A.}
} marica \edtext{kuṣṭha}{
  \Afootnote{\textsc{[om]} A.}
} \edtext{vacā}{
  \Afootnote{\textsc{[add]} haridrā A.}
} cūrṇṇāni lohānāṃ \edtext{samabhāgāni}{
  \Afootnote{\textsc{[add]} ca A.}
} tataḥ kṣāravad āgatapākam avatārya lohakumbhe nidadhyāt | 
\pend

 
\pstart

                         \textsc{[1938 ed. 5.6.4]}
                        \caesura \edtext{etena}{
  \Afootnote{anena A.}
} dundubhiṃ limpet \edtext{patākāstaraṇāni}{
  \Afootnote{patākāṃ tora° A.}
} ca ||
                        \caesura
 darśanāc chravaṇāc\edlabel{SS.5.6.4-7} \edtext{cāpi}{
  \linenum{|\xlineref{SS.5.6.4-7}}\lemma{chravaṇāc cāpi}\Afootnote{śrava° A.}
\lemma{cāpi}  \Afootnote{\textsc{[add]} darśanāt sparśāt A.}
} \edtext{viṣān}{
  \Afootnote{\textsc{[om]} A.}
} \edtext{sarvān}{
  \Afootnote{viṣāt A.}
} pra\edlabel{SS.5.6.4-11} \edtext{mucyate}{
  \linenum{|\xlineref{SS.5.6.4-11}}\lemma{pra mucyate}\Afootnote{saṃprati A.}
} | 
\pend

 
\pstart

                         \textsc{[1938 ed. 5.6.5]}
                        \caesura eṣa kṣārāgado nāma śarkarāsv aśmarīṣu ca | \caesura arśassu vātagulmeṣu kāsaśūlodareṣu ca | 
\pend

 
\pstart

                         \textsc{[1938 ed. 5.6.6]}
                        \caesura ajīrṇṇe \edtext{grahaṇīdoṣe}{
  \Afootnote{grahaṇedoṣe K.}
} bhaktadveṣe ca dāruṇe | \caesura śophe sarvasare cāpi deyaḥ śvāse \edtext{ca}{
  \Afootnote{\textbf{ca} H.}
} dustare\edlabel{SS.5.6.6-13} \edtext{|}{
  \linenum{|\xlineref{SS.5.6.6-13}}\lemma{dustare |}\Afootnote{dāruṇe || A.}
}
\pend

 
\pstart

                         \textsc{[1938 ed. 5.6.7]}
                        \caesura \edtext{eṣa}{
  \Afootnote{sadā A.}
} sarvaviṣārttānāṃ sarvathaivopayujyate | \caesura \edtext{tathā}{
  \Afootnote{eṣa A.}
} takṣakamukhyānām api \edtext{sarpāṃkuśo}{
  \Afootnote{darpāṅ° A.}
} 'gadaḥ || 
\pend

 
\pstart
\edtext{}{
  \Afootnote{\textsc{[pre]} viḍaṅgatriphalādantībhadradāruhareṇavaḥ |  tālīśapatramañjiṣṭhākeśarotpalapadmakam || A.}
}
\pend

 
\pstart
\edtext{}{
  \Afootnote{\textsc{[pre]} dāḍimaṃ mālatīpuṣpaṃ rajanyau sārive sthire |  priyaṅgus tagaraṃ kuṣṭhaṃ bṛhatyau cailavālukam || A.}
}
\pend

 
\pstart
\edtext{}{
  \Afootnote{\textsc{[pre]} sacandanagavākṣībhir etaiḥ siddhaṃ viṣāpaham |  sarpiḥ kalyāṇakaṃ hy etad grahāpasmāranāśanam || A.}
}
\pend

 
\pstart
\edtext{}{
  \Afootnote{\textsc{[pre]} pāṇḍvāmayagaraśvāsamandāgnijvarakāsanut |  śoṣiṇām alpaśukrāṇāṃ vandhyānāṃ ca praśasyate || A.}
}
\pend

 
\pstart

                         \textsc{[1938 ed. 5.6.12]}
                        \caesura \edtext{apāmārgasya}{
  \Afootnote{\textsc{[add]} \uuline{ca} H.}
} bījāni śirīṣasya ca māṣakām\edlabel{SS.5.6.12-5} \edtext{| \caesura}{
  \linenum{|\xlineref{SS.5.6.12-5}}\lemma{māṣakām | }\Afootnote{māṣakān |  A.}
} \edtext{śvete}{
  \Afootnote{śve\uwave{te} K.}
} \edtext{dve}{
  \Afootnote{dve\textsc{(l. 6)} K.}
} kākamācīñ ca gavām mūtreṇa pīṣayet\edlabel{SS.5.6.12-13} \edtext{|
}{
  \linenum{|\xlineref{SS.5.6.12-13}}\lemma{pīṣayet |}\Afootnote{peṣayet || A.}
}
\pend

 
\pstart

                         \textsc{[1938 ed. 5.6.13]}
                        \caesura sarpir \edtext{eteṣu}{
  \Afootnote{etais tu A.}
} saṃsiddhaṃ viṣasaṃśamanaṃ param | \caesura amṛtaṃ nāma vikhyātam api sañjīvayet mṛtam || 
\pend

 
\pstart

                         \textsc{[1938 ed. 5.6.14]}
                        \caesura \edtext{candanāguruṇī}{
  \Afootnote{candanāgaruṇī H.}
} kuṣṭhaṃ tagaraṃ tailaparṇṇikam\edlabel{SS.5.6.14-4} \edtext{| \caesura}{
  \linenum{|\xlineref{SS.5.6.14-4}}\lemma{tailaparṇṇikam | }\Afootnote{tila° A; °rṇṇikāṃ |  H.}
} \edtext{prapauṇḍarīkan}{
  \Afootnote{°rīkaṃ A.}
} naladaṃ saralaṃ devadāru ca | 
\pend

 
\pstart

                         \textsc{[1938 ed. 5.6.15]}
                        \caesura bhadraśriyaṃ yavaphalāṃ \edtext{bhārgīn}{
  \Afootnote{bhārgīṃ A.}
} nīlīṃ sugandhikām | \caesura kāleyakaṃ padmakañ ca madhukaṃ \edtext{sanakhāṃ}{
  \Afootnote{nāgaraṃ A; sana\uwave{khaṃ} K.}
} jaṭām\edlabel{SS.5.6.15-12} \edtext{|}{
  \linenum{|\xlineref{SS.5.6.15-12}}\lemma{jaṭām |}\Afootnote{jaṭāṃ | K.}
}
\pend

 
\pstart

                         \textsc{[1938 ed. 5.6.16]}
                        \caesura punnāgailailavālūni gairikaṃ dhyāmakaṃ tathā\edlabel{SS.5.6.16-4} \edtext{| \caesura}{
  \linenum{|\xlineref{SS.5.6.16-4}}\lemma{tathā | }\Afootnote{balām |  A.}
} toyaṃ sarjarasaṃ māṃsīṃ śatapuṣpāṃ hareṇukām | 
\pend

 
\pstart

                         \textsc{[1938 ed. 5.6.17]}
                        \caesura \edtext{tālīsapatraṃ}{
  \Afootnote{tālīśapatraṃ A.}
} kṣudrailāṃ \edtext{priyaṅgū}{
  \Afootnote{priyaṅguṃ A; priyaṃgūṃ H.}
} sakuṭannaṭām\edlabel{SS.5.6.17-4} \edtext{| \caesura}{
  \linenum{|\xlineref{SS.5.6.17-4}}\lemma{sakuṭannaṭām | }\Afootnote{sakuṭaṃnnaṭāṃ |  K; °naṭam |  A.}
} \edtext{tilapuṣpaṃ}{
  \Afootnote{śilāpuṣpaṃ A.}
} saśaileyam patraṃ kālānusārivām | 
\pend

 
\pstart

                         \textsc{[1938 ed. 5.6.18]}
                        \caesura kaṭutrikaṃ śītaśivaṃ\edlabel{SS.5.6.18-2} \edtext{kāśmaryaṃ}{
  \linenum{|\xlineref{SS.5.6.18-2}}\lemma{śītaśivaṃ kāśmaryaṃ}\Afootnote{°vaṃ\uwave{kāśmaryaṃ} K.}
} kaṭurohiṇīm | \caesura somarājīm\edlabel{SS.5.6.18-6} \edtext{ativiṣāṃ}{
  \linenum{|\xlineref{SS.5.6.18-6}}\lemma{somarājīm ativiṣāṃ}\Afootnote{somarājīmati° A K.}
\lemma{ativiṣāṃ}  \Afootnote{ativiṣā H.}
} \edtext{pṛthvīkām\edlabel{SS.5.6.18-8}}{
  \Afootnote{pṛthvikām A.}
} indravāruṇīm \edtext{|}{
  \linenum{|\xlineref{SS.5.6.18-8}}\lemma{pṛthvīkām\ldots |}\Afootnote{pṛthvīkāmindra° K.}
}
\pend

 
\pstart

                         \textsc{[1938 ed. 5.6.19]}
                        \caesura uśīre\edlabel{SS.5.6.19-1} \edtext{dve}{
  \linenum{|\xlineref{SS.5.6.19-1}}\lemma{uśīre dve}\Afootnote{uśīraṃ A.}
} \edtext{varuṇakaṃ}{
  \Afootnote{varuṇaṃ mustaṃ A.}
} \edtext{kustumburyo}{
  \Afootnote{kustumburu A.}
} nakhāni\edlabel{SS.5.6.19-5} ca | \caesura \edtext{tvacaṃ}{
  \linenum{|\xlineref{SS.5.6.19-5}}\lemma{nakhāni\ldots tvacaṃ}\Afootnote{nakhaṃ A.}
} \edtext{taskarasāhvañ}{
  \Afootnote{\textsc{[om]} A.}
} \edtext{ca}{
  \Afootnote{\textsc{[om]} A.}
} \edtext{granthilāṃ}{
  \Afootnote{\textsc{[om]} A.}
} saharītakīm\edlabel{SS.5.6.19-12} \edtext{| \caesura}{
  \linenum{|\xlineref{SS.5.6.19-12}}\lemma{saharītakīm | }\Afootnote{tathā |  A.}
} śvete haridre sthauṇeyaṃ lākṣāñ ca \edtext{lavaṇāni}{
  \Afootnote{lavanāni K.}
} ca | 
\pend

 
\pstart

                         \textsc{[1938 ed. 5.6.20]}
                        \caesura kumudotpalapadmāni puṣpañ cāpi tathārjakam\edlabel{SS.5.6.20-4} \edtext{| \caesura}{
  \linenum{|\xlineref{SS.5.6.20-4}}\lemma{tathārjakam | }\Afootnote{tathā 'rkajam |  A.}
} campakāśoka \edtext{sumanā}{
  \Afootnote{sumanas A; sumanās H.}
} \edtext{tilaka}{
  \Afootnote{tilvaka A.}
} prasavāni ca | 
\pend

 
\pstart

                         \textsc{[1938 ed. 5.6.21]}
                        \caesura pāṭalīśālmalī \edtext{śelū}{
  \Afootnote{śailu A.}
} \edtext{śirīṣāṇān}{
  \Afootnote{śirīṣāṇāṃ A; śirīśāṇān K.}
} tathaiva ca | \caesura \edtext{surasyās}{
  \Afootnote{kusumaṃ A.}
} \edtext{tṛṇamūlyasya}{
  \Afootnote{tṛṇamūlyāś ca A; tṛṇa\uwave{mū}lyasya K; tṛṇaśū\textsc{(l. 3)}lyasya H.}
} \edtext{sinduvārasya}{
  \Afootnote{surabhīsindhuvārajam || A.}
} yāni\edlabel{SS.5.6.21-10} ca \edtext{|}{
  \linenum{|\xlineref{SS.5.6.21-10}}\lemma{yāni\ldots |}\Afootnote{\textsc{[om]} A.}
}
\pend

 
\pstart

                         \textsc{[1938 ed. 5.6.22]}
                        \caesura dhavāśvakarṇṇayoś\edlabel{SS.5.6.22-1} \edtext{cāpi}{
  \linenum{|\xlineref{SS.5.6.22-1}}\lemma{dhavāśvakarṇṇayoś cāpi}\Afootnote{°rṇapārthānāṃ A.}
} \edtext{puṣpāṇi}{
  \Afootnote{puṣpāni K.}
} tiniśasya ca \edtext{|}{
  \Afootnote{\textsc{[add]} gugguluṃ kuṅkumaṃ bimbīṃ sarpākṣīṃ gandhanākulīm || A.}
}
\pend

 
\pstart

                         \textsc{[1938 ed. 5.6.23]}
                        \caesura etat sambhṛtya sambhāraṃ sūkṣmaṃ\edlabel{SS.5.6.23-4} \edtext{cūrṇṇaṃ}{
  \linenum{|\xlineref{SS.5.6.23-4}}\lemma{sūkṣmaṃ cūrṇṇaṃ}\Afootnote{śūkṣmacūrṇṇan H.}
} \edtext{tu}{
  \linenum{|\xlineref{SS.5.6.23-4}}\lemma{sūkṣmaṃ\ldots tu}\Afootnote{sūkṣmacūrṇāni A.}
} kārayet | \caesura gopittamadhusarpirbhir yuktaṃ śṛṅge nidhāpayet | 
\pend

 
\pstart

                         \textsc{[1938 ed. 6.24]}
                        \caesura \edtext{bhagnaskandha}{
  \Afootnote{bhagnaskandhaṃ A \uline{H}.}
} vivṛttākṣaṃ mṛtyor daṃṣṭrāntaraṃ gatam | \caesura anenāgadamukhyena manuṣyaṃ punar ānayet\edlabel{SS.5.6.24-10} \edtext{|}{
  \linenum{|\xlineref{SS.5.6.24-10}}\lemma{ānayet |}\Afootnote{āharet || A.}
}
\pend

 
\pstart

                         \textsc{[1938 ed. 5.6.25]}
                        \caesura eṣo 'gnikalpaṃ durvāraṃ kruddhasyāmitatejasaḥ | \caesura \edtext{sarvanāga}{
  \Afootnote{viṣaṃ nāga A.}
} \edtext{pater}{
  \Afootnote{gater Nep.}
} hanyād api\edlabel{SS.5.6.25-9} \edtext{vā}{
  \linenum{|\xlineref{SS.5.6.25-9}}\lemma{api vā}\Afootnote{prasabhaṃ A.}
} vāsuker viṣam\edlabel{SS.5.6.25-12} \edtext{|
}{
  \linenum{|\xlineref{SS.5.6.25-12}}\lemma{viṣam |}\Afootnote{api || A.}
}
\pend

 
\pstart

                         \textsc{[1938 ed. 5.6.26]}
                        \caesura mahāsugandho\edlabel{SS.5.6.26-1} \edtext{nāmnāyaṃ}{
  \linenum{|\xlineref{SS.5.6.26-1}}\lemma{mahāsugandho nāmnāyaṃ}\Afootnote{°gandhināmā 'yaṃ A.}
} pañcāśītyaṅgasaṃbhṛtaḥ\edlabel{SS.5.6.26-3} \edtext{| \caesura}{
  \linenum{|\xlineref{SS.5.6.26-3}}\lemma{pañcāśītyaṅgasaṃbhṛtaḥ | }\Afootnote{°saṃyutaḥ |  A.}
} \edtext{rājāgadānāṃ}{
  \Afootnote{rājā 'ga° A.}
} sarveṣāṃ rājño haste bhavet sadā | 
\pend

 
\pstart

                         \textsc{[1938 ed. 5.6.27]}
                        \caesura \edtext{tenānuliptaś}{
  \Afootnote{snātānu° A.}
} \edtext{ca}{
  \Afootnote{tu A.}
} nṛpo bhavet sarvajanapriyaḥ | \caesura bhrājiṣṇutāñ ca labhate śatrumadhyagato 'pi saḥ\edlabel{SS.5.6.27-12} \edtext{|}{
  \linenum{|\xlineref{SS.5.6.27-12}}\lemma{saḥ |}\Afootnote{san || A.}
}
\pend

 
\pstart

                         \textsc{[1938 ed. 5.6.28]}
                        \caesura uṣṇavarjyo vidhiḥ kāryo viṣārttānāṃ vijānatā | \caesura \edtext{tyaktvā}{
  \Afootnote{muktvā A.}
} kīṭaviṣaṃ tad dhi śītenābhipravardhate | 
\pend

 
\pstart
\edtext{}{
  \Afootnote{\textsc{[pre]} annapānavidhāv uktam upadhārya śubhāśubham |  śubhaṃ deyaṃ viṣārtebhyo viruddhebhyaś ca vārayet || A.}
}
\pend

 
\pstart
\edtext{}{
  \Afootnote{\textsc{[pre]} phāṇitaṃ śugrusauvīram ajīrṇādhyaśanaṃ tathā |  varjayec ca samāsena navadhānyādikaṃ gaṇam || A.}
}
\pend

 
\pstart

                         \textsc{[1938 ed. 5.6.31]}
                        \caesura divāsvapnaṃ vyavāyañ ca vyāyāmaṃ krodham\edlabel{SS.5.6.31-5} ātapam \edtext{| \caesura}{
  \linenum{|\xlineref{SS.5.6.31-5}}\lemma{krodham\ldots | }\Afootnote{krodhamātapam |  A.}
} \edtext{surātilakulatthāṃś}{
  \Afootnote{°latthāś H.}
} ca \edtext{varjayīta}{
  \Afootnote{varjayed dhi A.}
} viṣāturaḥ || 
\pend

 
\pstart

                         \textsc{[1938 ed. 5.6.32]}
                        \caesura prasannadoṣaṃ prakṛtisthadhātum \edtext{annābhikāmaṃ samamūtraviṭkam
}{
  \Afootnote{annābhikāṅkṣaṃ A.}
} | \caesura\edlabel{SS.5.6.32-4} \edtext{prasannasarvendriyacittaceṣṭaṃ}{
  \linenum{|\xlineref{SS.5.6.32-4}}\lemma{|  prasannasarvendriyacittaceṣṭaṃ}\Afootnote{samasūtrajihvam |  A; °vi\uwave{ṭkaṃ} |  Nep.}
} \edtext{vaidyo}{
  \Afootnote{prasannavarṇendri° A.}
} 'vagacched aviṣaṃ manuṣyam || 
\pend

 
\pstart
\edtext{iti}{
  \Afootnote{\textsc{[add]} suśrutasaṃhitāyāṃ A.}
} kalpe\edlabel{SS.5.6.trailer-1} \edtext{7}{
  \linenum{|\xlineref{SS.5.6.trailer-1}}\lemma{kalpe 7}\Afootnote{kalpasthāne A H.}
\lemma{7}  \Afootnote{\textsc{[add]} dundubhisvanīyakalpo nāma ṣaṣṭho 'dhyāyaḥ ||6 || A; \textsc{[add]} saptamo 'dhyāyaḥ || H.}
}
\pend

  \chapter{Kalpasthāna 8: Poisonous Insects}
\pstart

                         \textsc{[1938 ed. 5.8.1]}
                        \caesura athātaḥ kīṭakalpaṃ vyākhyāsyāmaḥ\edlabel{SS.5.8.1-3} \edtext{||}{
  \linenum{|\xlineref{SS.5.8.1-3}}\lemma{vyākhyāsyāmaḥ ||}\Afootnote{vyākhyāsyāmaḥ || K.}
}
\pend

 
\pstart
\edtext{}{
  \Afootnote{\textsc{[pre]} yathovāca bhagavān dhanvantariḥ || A.}
}
\pend

 
\pstart

                         \textsc{[1938 ed. 5.8.3]}
                        \caesura sarpāṇāṃ śukraviṇmūtraśavapūtyaṇḍasambhavāḥ | \caesura vāyvagnyambuprakṛtayaḥ kīṭās tu \edtext{trividhāḥ}{
  \Afootnote{vividhāḥ A H.}
} smṛtāḥ | 
\pend

 
\pstart

                         \textsc{[1938 ed. 5.8.4]}
                        \caesura sarvadoṣaprakṛtibhir \edtext{yuktāś}{
  \Afootnote{yukāś H.}
} \edtext{cāpy}{
  \Afootnote{te A.}
} apare\edlabel{SS.5.8.4-4} matāḥ \edtext{| \caesura}{
  \linenum{|\xlineref{SS.5.8.4-4}}\lemma{apare\ldots | }\Afootnote{pariṇāmataḥ |  A.}
} kīṭās\edlabel{SS.5.8.4-7} \edtext{te}{
  \linenum{|\xlineref{SS.5.8.4-7}}\lemma{kīṭās te}\Afootnote{kīṭatve A.}
} 'pi sughorās \edtext{te}{
  \Afootnote{syuḥ A H; \uwave{te} K.}
} sarva eva caturvidhāḥ\edlabel{SS.5.8.4-14} \edtext{|}{
  \linenum{|\xlineref{SS.5.8.4-14}}\lemma{caturvidhāḥ |}\Afootnote{caturvidhā | K.}
}
\pend

 
\pstart

                         \textsc{[1938 ed. 5.8.5]}
                        \caesura \edtext{uṇḍunābhas}{
  \Afootnote{kumbhīnasas A.}
} tuṇḍikerī śṛṅgī śatakulimbhakāḥ\edlabel{SS.5.8.5-4} \edtext{| \caesura}{
  \linenum{|\xlineref{SS.5.8.5-4}}\lemma{śatakulimbhakāḥ | }\Afootnote{śatakulīrakaḥ |  A.}
} \edtext{ucciṭiṅgāgnyalpavācaḥ}{
  \Afootnote{ucciṭiṅga\uwave{s tya}lpa° K; ucciṭiṅgo 'gnināmā ca A; °gā\uwave{s ty}alpavāco\textsc{(l. 5)} H.}
} viciṭiṅgamasūrikāḥ\edlabel{SS.5.8.5-7} \edtext{|
}{
  \linenum{|\xlineref{SS.5.8.5-7}}\lemma{viciṭiṅgamasūrikāḥ |}\Afootnote{cicciṭiṅgo mayūrikā || A; viciṭiṅgā ma\uwave{sū}rikāḥ || H.}
}
\pend

 
\pstart

                         \textsc{[1938 ed. 5.8.6]}
                        \caesura āvarttakas tathorabhraḥ śārikāmukhavaidalau\edlabel{SS.5.8.6-3} \edtext{| \caesura}{
  \linenum{|\xlineref{SS.5.8.6-3}}\lemma{śārikāmukhavaidalau | }\Afootnote{sāri° A.}
} \edtext{śatakurdo}{
  \Afootnote{śarāvakurdo A.}
} \edtext{'bhirājīva}{
  \Afootnote{'bhīrājiḥ A; 'bhirājī ca K; 'bhirājī va H.}
} paruṣaś citraśīrṣakaḥ | 
\pend

 
\pstart
\edtext{}{
  \Afootnote{\textsc{[pre]} śatabāhuś ca yaś cāpi raktarājiś ca kīrtitaḥ | A.}
}
\pend

 
\pstart

                         \textsc{[1938 ed. 5.8.7cd]}
                        \caesura \edtext{aṣṭādaśaite}{
  \Afootnote{aṣṭādaśeti A.}
} vāyavyāḥ kīṭāḥ vātaprakopanāḥ\edlabel{SS.5.8.7cd-4} \edtext{|}{
  \linenum{|\xlineref{SS.5.8.7cd-4}}\lemma{vātaprakopanāḥ |}\Afootnote{pavanakopanāḥ || A.}
}
\pend

 
\pstart

                         \textsc{[1938 ed. 5.8.8]}
                        \caesura tair bhavantīha daṣṭānāṃ rogā vātanimittajāḥ || \caesura \edtext{kauṇḍinyaḥ}{
  \Afootnote{kauṇḍinyakaḥ A.}
} kaṇabhaḥ\edlabel{SS.5.8.8-8} \edtext{svargo}{
  \linenum{|\xlineref{SS.5.8.8-8}}\lemma{kaṇabhaḥ svargo}\Afootnote{kaṇabhako A.}
} \edtext{vāraṇī\edlabel{SS.5.8.8-10}}{
  \Afootnote{varaṭī A.}
} patravṛścikaḥ \edtext{|}{
  \linenum{|\xlineref{SS.5.8.8-10}}\lemma{vāraṇī\ldots |}\Afootnote{vāraṇīpa° K.}
}
\pend

 
\pstart

                         \textsc{[1938 ed. 5.8.9]}
                        \caesura vināsikā \edtext{brahmaṇīkā}{
  \Afootnote{brāhmaṇikā A.}
} bindulo bhramaras tathā | \caesura \edtext{bāhyakaḥ}{
  \Afootnote{bāhyakī A.}
} \edtext{piccaṭāḥ}{
  \Afootnote{picciṭaḥ A.}
} kumbhīvarcaḥ \edtext{kīro}{
  \Afootnote{kīṭo A.}
} 'rimedakaḥ | 
\pend

 
\pstart

                         \textsc{[1938 ed. 5.8.10]}
                        \caesura padmakīṭo \edtext{dundubhako}{
  \Afootnote{dundubhiko A; dundubheko H.}
} \edtext{maśakaḥ}{
  \Afootnote{makaraḥ A.}
} śatapādakaḥ | \caesura pañcālakaḥ \edtext{pākamatsyaḥ}{
  \Afootnote{pākamatsya K.}
} kṛṣṇatuṇḍo 'tha garddabhī\edlabel{SS.5.8.10-10} \edtext{|}{
  \linenum{|\xlineref{SS.5.8.10-10}}\lemma{garddabhī |}\Afootnote{garbhabhī | K.}
}
\pend

 
\pstart

                         \textsc{[1938 ed. 5.8.11]}
                        \caesura \edtext{kīṭāḥ}{
  \Afootnote{klītaḥ A.}
} \edtext{krimisarāvī}{
  \Afootnote{kṛmisarārī A.}
} ca yaś \edtext{cānyaḥ}{
  \Afootnote{cāpy A.}
} \edtext{śleṣmakaḥ}{
  \Afootnote{utkleśakas A.}
} smṛtaḥ\edlabel{SS.5.8.11-7} \edtext{| \caesura}{
  \linenum{|\xlineref{SS.5.8.11-7}}\lemma{smṛtaḥ | }\Afootnote{tathā |  A.}
} ete hy \edtext{agniprakṛtayaś}{
  \Afootnote{agni pra° H.}
} \edtext{caturviṃśatir}{
  \Afootnote{caturviṅśatir H; \textsc{[add]} eva ca || A.}
} īritāḥ\edlabel{SS.5.8.11-13} \edtext{|}{
  \linenum{|\xlineref{SS.5.8.11-13}}\lemma{īritāḥ |}\Afootnote{\textsc{[om]} A.}
}
\pend

 
\pstart

                         \textsc{[1938 ed. 5.8.12]}
                        \caesura tair bhavantīha daṣṭānāṃ \edtext{rogāḥ}{
  \Afootnote{vegā K; vegāḥ H.}
} pittanimittajāḥ |
                        \caesura
 vaiśvambharaḥ \edtext{pañcaśuklaḥ}{
  \Afootnote{viśvam° A.}
} pañcakṛṣṇo 'tha kokilaḥ | 
\pend

 
\pstart

                         \textsc{[1938 ed. 5.8.13]}
                        \caesura \edtext{śairyakaḥ}{
  \Afootnote{saireyakaḥ A.}
} \edtext{pravalākaś}{
  \Afootnote{pracalako A; pracalākaś H.}
} \edtext{ca}{
  \Afootnote{\textsc{[om]} A.}
} \edtext{bhaṭābhaḥ}{
  \Afootnote{valabhaḥ A.}
} \edtext{kiṭibho\edlabel{SS.5.8.13-5}}{
  \Afootnote{kiṭibhas A.}
} 'ṭakī\edlabel{SS.5.8.13-6} \edtext{| \caesura}{
  \linenum{|\xlineref{SS.5.8.13-5}}\lemma{kiṭibho\ldots | }\Afootnote{kiṭibhoṭa\uuline{jī}\textbf{kī\textbf{3}} ||  H.}
  \linenum{|\xlineref{SS.5.8.13-6}}\lemma{'ṭakī | }\Afootnote{tathā |  A; ṭajī |  K.}
} sūcīmukhaḥ \edtext{kṛṣṇagodhā}{
  \Afootnote{kṛṣṇagodhāḥ H.}
} \edtext{kuṣṭaḥ\edlabel{SS.5.8.13-10}}{
  \Afootnote{yaś ca A; \uwave{kuṣṭaḥ} K.}
} kāṣāyavāsikaḥ \edtext{|}{
  \linenum{|\xlineref{SS.5.8.13-10}}\lemma{kuṣṭaḥ\ldots |}\Afootnote{\uwave{kukuḥ}kā° H.}
}
\pend

 
\pstart
\edtext{
                         \textsc{[1938 ed. 5.8.14]}
                        \caesura}{
  \Afootnote{\textsc{[pre]} kīṭo gardabhakaś caiva tathā troṭaka eva ca |  A.}
} trayodaśaite saumyās \edtext{tu}{
  \Afootnote{syuḥ A.}
} kīṭāḥ śleṣmaprakopanāḥ\edlabel{SS.5.8.14-5} \edtext{|}{
  \linenum{|\xlineref{SS.5.8.14-5}}\lemma{śleṣmaprakopanāḥ |}\Afootnote{°paṇāḥ || A; °pajāḥ || H.}
}
\pend

 
\pstart

                         \textsc{[1938 ed. 5.8.15]}
                        \caesura tair bhavantīha daṣṭānāṃ rogāḥ śleṣmanimittajāḥ\edlabel{SS.5.8.15-5} \edtext{|| \caesura}{
  \linenum{|\xlineref{SS.5.8.15-5}}\lemma{śleṣmanimittajāḥ || }\Afootnote{kaphani° A.}
} \edtext{tuṅganāso}{
  \Afootnote{tuṅgīnāso A.}
} valabhikaḥ\edlabel{SS.5.8.15-8} \edtext{tolako}{
  \linenum{|\xlineref{SS.5.8.15-8}}\lemma{valabhikaḥ tolako}\Afootnote{vicilakastālako A.}
} \edtext{nāhanas}{
  \Afootnote{vāhakas A.}
} tathā | 
\pend

 
\pstart

                         \textsc{[1938 ed. 58.16]}
                        \caesura \edtext{koṇṭāgīrī\edlabel{SS.5.8.16-1}}{
  \Afootnote{koṣṭhāgārī A; ko\uwave{ṇṭā}gīrī K.}
} \edtext{krimikaro}{
  \linenum{|\xlineref{SS.5.8.16-1}}\lemma{koṇṭāgīrī krimikaro}\Afootnote{ko\uwave{ṇṭā\textbf{ñ cā}}gārī\textsc{(l. 5)}kri° H.}
} yaś ca maṇḍalapuṣpakaḥ\edlabel{SS.5.8.16-5} \edtext{| \caesura}{
  \linenum{|\xlineref{SS.5.8.16-5}}\lemma{maṇḍalapuṣpakaḥ | }\Afootnote{maṇḍalapucchakaḥ |  A.}
} \edtext{tuṇḍavaktraḥ}{
  \Afootnote{tuṇḍanābhaḥ (tuṅganābhaḥ) A.}
} \edtext{sarṣapakaḥ}{
  \Afootnote{sarṣapiko A; sarṣapaka K.}
} \edtext{sphoṭakaḥ}{
  \Afootnote{valguliḥ A; sphoṭākaḥ H.}
} śambukaś \edtext{ca}{
  \Afootnote{tathā || A.}
} yaḥ\edlabel{SS.5.8.16-12} \edtext{|}{
  \linenum{|\xlineref{SS.5.8.16-12}}\lemma{yaḥ |}\Afootnote{\textsc{[om]} A.}
}
\pend

 
\pstart

                         \textsc{[1938 ed. 5.8.17]}
                        \caesura \edtext{agnikīṭāś}{
  \Afootnote{agnikīṭaś A.}
} \edtext{ca}{
  \Afootnote{\textsc{[add]} vijñeyā A.}
} \edtext{ghorāḥ\edlabel{SS.5.8.17-3}}{
  \Afootnote{ghorā K; \uwave{ghorāḥ} \textbf{kīṭā} H.}
} \edtext{syur}{
  \linenum{|\xlineref{SS.5.8.17-3}}\lemma{ghorāḥ syur}\Afootnote{\textsc{[om]} A.}
\lemma{syur}  \Afootnote{syu K.}
} \edtext{dvādaśaite}{
  \Afootnote{dvādaśa A.}
} tridoṣajāḥ\edlabel{SS.5.8.17-6} \edtext{| \caesura}{
  \linenum{|\xlineref{SS.5.8.17-6}}\lemma{tridoṣajāḥ | }\Afootnote{prāṇanāśanāḥ |  A.}
} tair bhavanti\edlabel{SS.5.8.17-9} \edtext{ha}{
  \linenum{|\xlineref{SS.5.8.17-9}}\lemma{bhavanti ha}\Afootnote{bhavantīha A \uline{H}.}
} daṣṭānāṃ vegajñānāni sarpavat |
 \pend

 
\pstart
\edtext{}{
  \Afootnote{\textsc{[pre]} tās tāś ca vedanās tīvrā rogā vai sānnipātikāḥ |  kṣarāgnidagdhavad daṃśo raktapītasitāruṇaḥ || A.}
}%
\edtext{}{
  \Afootnote{\textsc{[pre]} jvarāṅgamardaromāñ ca vedanābhiḥ samanvitaḥ |  chardyatīsāratṛṣṇāś ca dāho mūrcchā vijṛmbhikā || A.}
}%
\edtext{}{
  \Afootnote{\textsc{[pre]} vepathuśvāsahikkāś ca dāhaḥ śītaṃ ca dāruṇam | A.}
}%
\textsc{[1938 ed. 5.8.20cd]}
                        \caesura \edtext{piṭakopacayaḥ}{
  \Afootnote{piḍako° A.}
} \edtext{śophāḥ}{
  \Afootnote{śopho A.}
} granthayo maṇḍalāni ca | 
\pend

 
\pstart

                         \textsc{[1938 ed. 5.8.21ab]}
                        \caesura \edtext{dardruś}{
  \Afootnote{dadravaḥ A; da\uwave{rddru}ś H.}
} \edtext{ca}{
  \Afootnote{\textsc{[om]} A.}
} karṇṇikāś caiva visarpāḥ kiṭibhāni ca | 
\pend

 


 
\pstart

                         \textsc{[1938 ed. 5.8.21.add-1]}
                        \caesura bhavanti daṃśaparyante dehe vāpi viṣākule | 
\pend

 
\pstart
\edtext{}{
  \Afootnote{\textsc{[pre]} ye 'nye teṣāṃ viśeṣās tu tūrṇaṃ teṣāṃ samādiśet |  dūṣīviṣaprakopāc ca tathaiva viṣalepanāt || A.}
}
\pend

 
\pstart
\edtext{}{
  \Afootnote{\textsc{[pre]} liṅgaṃ tīkṣṇaviṣeṣv etac chṛṇu mandaviṣeṣv ataḥ |  prasekārocakacchardiśirogauravaśītakāḥ || A.}
}
\pend

 
\pstart
\edtext{}{
  \Afootnote{\textsc{[pre]} piḍakākoṭhakaṇḍūnāṃ janma doṣavibhāgataḥ |  yogair nānāvidhair eṣāṃ cūrṇāni garam ādiśet || A.}
}
\pend

 
\pstart
\edtext{}{
  \Afootnote{\textsc{[pre]} dūṣīviṣaprakārāṇāṃ tathā cāpy anulepanāt | A.}
}
\pend

 
\pstart

                         \textsc{[1938 ed. 5.8.25cd]}
                        \caesura ekajātīn atas tūrdhvaṃ kīṭān \edtext{bhedena}{
  \Afootnote{\textsc{[om]} A.}
} vakṣyate\edlabel{SS.5.8.25cd-6} \edtext{|
}{
  \linenum{|\xlineref{SS.5.8.25cd-6}}\lemma{vakṣyate |}\Afootnote{vakṣyāmi A.}
\lemma{|}  \Afootnote{\textsc{[add]} bhedataḥ || A.}
}
\pend

 
\pstart

                         \textsc{[1938 ed. 5.8.26]}
                        \caesura sāmānyato daṣṭaliṅgaiḥ \edtext{sādhyāsādhyakrameṇa}{
  \Afootnote{°mena K.}
} ca | \caesura \edtext{trikaṇṭakaḥ}{
  \Afootnote{trikaṇṭaḥ A.}
} \edtext{kunī}{
  \Afootnote{kariṇī A; kuṇī H.}
} cāpi \edtext{hastikakṣyo}{
  \Afootnote{hastikakṣo A H.}
} 'parājitaḥ | \caesura catvāra ete kaṇabhāḥ\edlabel{SS.5.8.26-14} \edtext{vyākhyātās}{
  \linenum{|\xlineref{SS.5.8.26-14}}\lemma{kaṇabhāḥ vyākhyātās}\Afootnote{kaṇabhāḥvyā° H.}
} tīvravedanāḥ | 
\pend

 
\pstart

                         \textsc{[1938 ed. 5.8.27 verse]}
                        \caesura ebhir \edtext{daṣṭasya}{
  \Afootnote{daṣṭeti K; ddaṣṭeti H.}
} gurutā gātrāṇām aṅgavedanā |
                        \caesura\edlabel{SS.5.8.27-verse-5}
 \edtext{lālāsrāvaś}{
  \linenum{|\xlineref{SS.5.8.27-verse-5}}\lemma{aṅgavedanā |  lālāsrāvaś}\Afootnote{\uuline{aṅga}\textbf{daṃśa}ve° H.}
} ca bhavati gātrabhedaś ca dāruṇaḥ |
 
\pend

 
\pstart
\edtext{}{
  \Afootnote{\textsc{[pre]} tair daṣṭasya śvayathur aṅgamardo gurutā gātrāṇāṃ daṃśaḥ kṛṣṇaś ca bhavati || A.}
}
\pend

 
\pstart

                         \textsc{[ 5.8.28 verse 1]}
                        \caesura pratisūryaḥ piṅgabhāso bahuvarṇṇo mahāśirāḥ | \caesura tathā nirupamaś cāpi pañca godherakāḥ smṛtāḥ |
 
\pend

 
\pstart

                         \textsc{[ 5.8.28 verse 2]}
                        \caesura tair bhavantīha daṣṭānāṃ \edtext{vegajñānāni}{
  \Afootnote{vegajñā\textbf{nā}ni H.}
} sarpavat | \caesura rujaś ca vividhākārā granthayaś ca sudāruṇāḥ | 
\pend

 
\pstart
\edtext{}{
  \Afootnote{\textsc{[pre]} pratisūryakaḥ, piṅgābhāso, bahuvarṇo, nirūpamo godhereka iti pañca godherakāḥ; tair daṣṭasya śopho dāharujau ca bhavataḥ, godherakeṇaitad eva granthiprādurbhāvo jvaraś ca || A.}
}
\pend

 
\pstart

                         \textsc{[ 5.8.29 verse 1]}
                        \caesura śvetā kṛṣṇā kṛṣṇarājī raktā raktaiś ca maṇḍalaiḥ | \caesura sarvaśvetā sarṣapikā ṣaḍetā gṛhagolikāḥ\edlabel{SS.5.8.29-verse-1-12} \edtext{|}{
  \linenum{|\xlineref{SS.5.8.29-verse-1-12}}\lemma{gṛhagolikāḥ |}\Afootnote{gṛhago\uwave{li}kāḥ | K; gṛhagodhikāḥ | H.}
}
\pend

 
\pstart

                         \textsc{[ 5.8.29 verse 2]}
                        \caesura tābhir \edtext{daṣṭe}{
  \Afootnote{daṣṭa K.}
} daṃśatodo hṛtpīḍā dāha eva ca | \caesura \edtext{daṃśaśophaś}{
  \Afootnote{daṅśaśophaś H.}
} ca bhavati granthijanma ca dāruṇam\edlabel{SS.5.8.29-verse-2-14} \edtext{|}{
  \linenum{|\xlineref{SS.5.8.29-verse-2-14}}\lemma{dāruṇam |}\Afootnote{dāru\uwave{ṇaṃḥ} | K.}
}
\pend

 
\pstart
\edtext{}{
  \Afootnote{\textsc{[pre]} galagolikā śvetā, kṛṣṇā, raktarājī, raktamaṇḍalā, sarvaśvetā, sarṣapikety evaṃ ṣaṭ; tābhir daṣṭe sarṣapikāvarjaṃ dāhaśophakledā bhavanti, sarṣapikayā hṛdayapiḍā 'tisāraś ca, tāsu madhye sarṣapikā prāṇaharī || A.}
}
\pend

 
\pstart

                         \textsc{[ 5.8.30 verse 1]}
                        \caesura paruṣā kṛṣṇacitre ca kapilā pītikā tathā || \caesura raktā śvetāgnivarṇṇā ca \edtext{śatapādo}{
  \Afootnote{śatapādyo K.}
} 'ṣṭadhā smṛtāḥ |
 
\pend

 
\pstart

                         \textsc{[ 5.8.30 verse 2]}
                        \caesura tābhir daṣṭe\edlabel{SS.5.8.30-verse-2-2} \edtext{rujās}{
  \linenum{|\xlineref{SS.5.8.30-verse-2-2}}\lemma{daṣṭe rujās}\Afootnote{ddaṣṭarujās H.}
} tīvrā \edtext{daṃśaśophaś}{
  \Afootnote{daṅśaśophaś H.}
} ca dāruṇaḥ | \caesura daṃśe ca \edtext{piṭakotpattir}{
  \Afootnote{piṭako\uwave{tpāṭi}m H.}
} \edtext{mūrcchāṃ}{
  \Afootnote{mūrcchā H.}
} cāpi sudāruṇāḥ\edlabel{SS.5.8.30-verse-2-14} \edtext{|}{
  \linenum{|\xlineref{SS.5.8.30-verse-2-14}}\lemma{sudāruṇāḥ |}\Afootnote{sudāruṇā || H.}
}
\pend

 
\pstart
\edtext{}{
  \Afootnote{\textsc{[pre]} śatapadyas tu paruṣā, kṛṣṇā, citrā, kapilā, pītikā, raktā, śvetā, agniprabhā, ity aṣṭau; tābhir daṣṭe śopho vedanā dāhaś ca hṛdaye, śvetāgniprabhābhyām etad eva dāho mūrcchā cātimātraṃ śvetapiḍakotpattiś ca || A.}
}
\pend

 
\pstart

                         \textsc{[ 5.8.31 verse 1]}
                        \caesura śvetaś ca kṛṣṇavarṇṇaś ca \edtext{śaravarṇṇo}{
  \Afootnote{śaravarṇṇo \textbf{gnim} H.}
} 'yam aprabhaḥ \caesura kuharo haritaś cāpi bhṛkuṭī koṭikaś ca yaḥ | 
\pend

 
\pstart

                         \textsc{[ 5.8.31 verse 2]}
                        \caesura aṣṭāv ete \edtext{kīṭāsañjñā}{
  \Afootnote{kīṭasaṃjñā H.}
} dardurāḥ parikīrttitāḥ | \caesura tair daṣṭaḥ \edtext{kaṇḍusaṃyukto}{
  \Afootnote{kaṇḍūsaṃ° H.}
} haritaṃ mūrchito vamet | 
\pend

 
\pstart
\edtext{}{
  \Afootnote{\textsc{[pre]} maṇḍūkāḥ kṛṣṇaḥ, sāraḥ, kuhako, harito, rakto, yavavarṇābho, bhṛkuṭī, koṭikaś cety aṣṭau; tair daṣṭasya daṃśe kaṇḍūr bhavati pītaphenāgamaś ca vaktrāt, bhṛkuṭīkoṭikābhyām etad eva dāhaś chardir mūrcchā cātimātram || A.}
}
\pend

 
\pstart

                         \textsc{[ 5.8.31 add]}
                        \caesura jalaukāḥ ṣaṭ samākhyātāḥ salakṣaṇacikitsitāḥ | \caesura ahikutthuḥ kutthukaś ca vṛttaśūkas tathaiva ca || 
\pend

 
\pstart

                         \textsc{[ 5.8.32 verse]}
                        \caesura trayo viśvambharāḥ proktāḥ dāhajvararujāvahāḥ\edlabel{SS.5.8.32-verse-4} \edtext{| \caesura}{
  \linenum{|\xlineref{SS.5.8.32-verse-4}}\lemma{dāhajvararujāvahāḥ | }\Afootnote{°jāpahāḥ ||  H.}
} \edtext{tair}{
  \Afootnote{tai K.}
} daṣṭamātre śvayathur ādaṃśe kaṇḍur eva ca | 
\pend

 
\pstart
\edtext{}{
  \Afootnote{\textsc{[pre]} viśvambharābhir daṣṭe daṃśaḥ sarṣapākārābhiḥ piḍakābhiḥ sarujābhiś cīyate, śītajvarārtaś ca puruṣo bhavati || A.}
}
\pend

 
\pstart
\edtext{}{
  \Afootnote{\textsc{[pre]} ahiṇḍukābhir daṣṭe todadāhakaṇḍuśvayathavo bhavanti mohaś ca; kaṇḍūmakābhir daṣṭe pītāṅgaś chardyatīsārajvarādibhir abhihanyate; śūkavṛntābhir daṣṭe kaṇḍūkoṭhāḥ pravardhante śūkaṃ cātra lakṣyate || A.}
}
\pend

 
\pstart

                         \textsc{[ 5.8.34 verse 1]}
                        \caesura phenāgamo 'tisāraś ca \edtext{koṭhajanmaṃ\edlabel{SS.5.8.34-verse-1-4}}{
  \Afootnote{koṭhajanme K.}
} \edtext{ca}{
  \linenum{|\xlineref{SS.5.8.34-verse-1-4}}\lemma{koṭhajanmaṃ ca}\Afootnote{koṭhajanmeva H.}
} dāruṇam |
                        \caesura
 samvāhikā \edtext{sthūlaśīrṣā}{
  \Afootnote{gamvā° Nep.}
} brāhmaṇy aṅgulikā tathā |
 
\pend

 
\pstart

                         \textsc{[ 5.8.34 verse 2]}
                        \caesura \edtext{vivarṇṇā kapilā
}{
  \Afootnote{\textsc{[add]} \textbf{karṇṇilā granthāntare} H.}
} \edtext{cāpi}{
  \Afootnote{kapi\uuline{kā}lā K.}
} ṣaṭ \edtext{proktās}{
  \Afootnote{\textbf{ṣaṭ} K.}
} tu pipīlikāḥ | \caesura\edlabel{SS.5.8.34-verse-2-7} \edtext{tābhir}{
  \linenum{|\xlineref{SS.5.8.34-verse-2-7}}\lemma{|  tābhir}\Afootnote{°kāḥ |  K.}
} daṣṭe rujā \edtext{dāhaḥ}{
  \Afootnote{rujās H.}
} \edtext{kaṇḍuśvayathur}{
  \Afootnote{tīvrā H.}
} \edtext{eva}{
  \Afootnote{śophaś H.}
} \edtext{ca}{
  \Afootnote{\textsc{[om]} H.}
} | \caesura\edlabel{SS.5.8.34-verse-2-15} \edtext{viśeṣeṇa}{
  \linenum{|\xlineref{SS.5.8.34-verse-2-15}}\lemma{|  viśeṣeṇa}\Afootnote{\textsc{[om]} H.}
} daṃśaty\edlabel{SS.5.8.34-verse-2-17} \edtext{etāḥ}{
  \linenum{|\xlineref{SS.5.8.34-verse-2-17}}\lemma{daṃśaty etāḥ}\Afootnote{cādañśamaṇḍale | H.}
} netrayor\edlabel{SS.5.8.34-verse-2-19} netravallabhāḥ | 
\pend

 
\pstart
\edtext{}{
  \Afootnote{\textsc{[pre]} pipīlikāḥ sthūlaśīrṣā, saṃvāhikā, brahmaṇikā, aṅgulikā; kapilikā, citravarṇeti ṣaṭ; tābhir daṣṭe daṃśe śvayathur agnisparśavad dāhaśophau bhavataḥ || A.}
}
\pend

 
\pstart
\edtext{}{
  \Afootnote{\textsc{[pre]} dāhacoṣau ca niyatau vahni\textsc{(l. 5)}r eva ca tāpite ||  kāntārikā \textbf{ca} kṛṣṇā ca piṅgalā samadhūlikā | H.}
}
\pend

 
\pstart
\edtext{}{
  \Afootnote{\textsc{[pre]} makṣikāḥ kāntārikā, kṛṣṇā, piṅgalā, madhūlikā, kāṣāyī, sthālikety evaṃ ṣaṭ; tābhir daṣṭasya kaṇḍuśophadāharujo bhavanti, sthālikākāṣāyībhyām etad eva śyāvapiḍakotpattir upadravāś ca jvarādayo bhavanti, kāṣāyī sthālikā ca prāṇahare || A.}
}
\pend

 
\pstart

                         \textsc{[ 5.8.36 verse 1]}
                        \caesura maṇḍalaḥ \edtext{pārvataś}{
  \Afootnote{parvvataś H.}
} caiva kṛṣṇaḥ \edtext{sāmudra}{
  \Afootnote{samudra H.}
} eva ca | 
\pend

 
\pstart

                         \textsc{[ 5.8.36 verse 2]}
                        \caesura maśako hastināmā ca maśakāḥ pañcakīrttitāḥ | \caesura tair daṣṭe \edtext{roṣasaṃyuktaṃ}{
  \Afootnote{°yuktāṃ K.}
} śūnam ādaṃśamaṇḍalam | 
\pend

 
\pstart

                         \textsc{[ 5.8.36 verse 3]}
                        \caesura vedanā rāgabahulaṃ kaṇḍūyuktaṃ kṣaraty asṛk | 
\pend

 
\pstart
\edtext{}{
  \Afootnote{\textsc{[pre]} maśakāḥ sāmudraḥ, parimaṇḍalo, hastimaśakaḥ, kṛṣṇaḥ, pārvatīya iti pañca; tair daṣṭasya tīvrā kaṇḍūrdaṃśaśophaś ca, pārvatīyas tu kīṭaiḥ prāṇaharais tulyalakṣaṇaḥ || A.}
}
\pend

 
\pstart
\edtext{
                         \textsc{[1938 ed. 5.8.38]}
                        \caesura}{
  \Afootnote{\textsc{[pre]} bhavanti cātra  A.}
} godherakaḥ \edtext{sthālakā}{
  \Afootnote{sthālikā A.}
} ca ye ca śvetāgnisaprabhe | \caesura bhṛkuṭī koṭikaś \edtext{caiva}{
  \Afootnote{cāpi H.}
} na sidhyanty ekajātiṣu | 
\pend

 
\pstart
\edtext{}{
  \Afootnote{\textsc{[pre]} śavamūtrapurīṣais tu saviṣair avamarśanāt |  syuḥ kaṇḍūdāhakoṭhāruḥpiḍakātodavedanāḥ || A.}
}
\pend

 
\pstart
\edtext{}{
  \Afootnote{\textsc{[pre]} prakledavāṃs tathā srāvo bhṛśaṃ saṃpācayet tvacam |  digdhaviddhakriyās tatra yathāvad avacārayet || A.}
}
\pend

 
\pstart
\edtext{}{
  \Afootnote{\textsc{[pre]} nāvasannaṃ na cotsannam atisaṃram bhaved anam |  daṃśādau viparītārti kīṭadaṣṭaṃ subādhakam || A.}
}
\pend

 
\pstart

                         \textsc{[1938 ed. 5.8.42]}
                        \caesura \edtext{kīṭair}{
  \Afootnote{\textsc{[om]} A.}
} daṣṭān \edtext{ugraviṣaiḥ}{
  \Afootnote{\textsc{[add]} kīṭaiḥ A.}
} sarpavat samupācaret | \caesura \edtext{trividhānān}{
  \Afootnote{trividhānāṃ A.}
} tu \edtext{śeṣāṇāṃ}{
  \Afootnote{pūrveṣāṃ A.}
} traividhyaṃ\edlabel{SS.5.8.42-10} \edtext{bhavati}{
  \linenum{|\xlineref{SS.5.8.42-10}}\lemma{traividhyaṃ bhavati}\Afootnote{traividhyena A.}
} kriyā \edtext{|}{
  \Afootnote{\textsc{[add]} hitāḥ || A.}
}
\pend

 
\pstart

                         \textsc{[1938 ed. 5.8.43ab]}
                        \caesura \edtext{svedāṃ}{
  \Afootnote{svedam A; svedān H.}
} \edtext{bahuprakārāṃś\edlabel{SS.5.8.43-2}}{
  \Afootnote{°rām̐ś H.}
} ca \edtext{yuñjyād}{
  \linenum{|\xlineref{SS.5.8.43-2}}\lemma{bahuprakārāṃś\ldots yuñjyād}\Afootnote{coṣṇam atrāvacārayet |  A.}
\lemma{yuñjyād}  \Afootnote{yu\uwave{ñjyā}v H.}
} anyatra mūrcchitāt\edlabel{SS.5.8.43-6} \edtext{|
}{
  \linenum{|\xlineref{SS.5.8.43-6}}\lemma{mūrcchitāt |}\Afootnote{mūrcchitān | H.}
\lemma{|}  \Afootnote{\textsc{[add]} daṃśāt pākakothaprapīḍitāt || A.}
}
\pend

 
\pstart

                         \textsc{[1938 ed. 5.8.44ab]}
                        \caesura viṣaghnañ ca vidhiṃ \edtext{kuryāt}{
  \Afootnote{sarvaṃ A.}
} \edtext{kuryāt\edlabel{SS.5.8.44ab-5}}{
  \Afootnote{\textsc{[om]} K.}
} \edtext{saṃśodhanāni}{
  \linenum{|\xlineref{SS.5.8.44ab-5}}\lemma{kuryāt saṃśodhanāni}\Afootnote{bahuśaḥ śo° A.}
} ca | 
\pend

 
\pstart
\edtext{}{
  \Afootnote{\textsc{[pre]} śirīṣakaṭukākuṣṭhavacārajanisaindhavaiḥ || A.}
}
\pend

 
\pstart
\edtext{}{
  \Afootnote{\textsc{[pre]} kṣīramajjavasāsarpiḥśuṇṭhīpippalidāruṣu |  utkārikā sthirādau vā sukṛtā svedane hitā || A.}
}
\pend

 
\pstart
\edtext{}{
  \Afootnote{\textsc{[pre]} na svedayeta cādaṃśaṃ dhūmaṃ vakṣyāmi vṛścike |  agadānekajātīṣu pravakṣyāmi pṛthak pṛthak || A.}
}
\pend

 
\pstart
\edtext{}{
  \Afootnote{\textsc{[pre]} kuṣṭhaṃ vakraṃ vacā pāṭhā bilvamūlaṃ suvarcikā |  gṛhadhūmaṃ haridre dve trikaṇṭakaviṣe hitāḥ || A.}
}
\pend

 
\pstart
\edtext{}{
  \Afootnote{\textsc{[pre]} rajanyāgāradhūmaś ca vakraṃ kuṣṭhaṃ palāśajam |  galagolikadaṣṭānāmagado viṣanāśanaḥ || A.}
}
\pend

 
\pstart
\edtext{}{
  \Afootnote{\textsc{[pre]} kuṅkumaṃ tagaraṃ śigru padmakaṃ rajanīdvayam |  agado jalapiṣṭo 'yaṃ śatapadviṣanāśanaḥ || A.}
}
\pend

 
\pstart
\edtext{}{
  \Afootnote{\textsc{[pre]} meṣaśṛṅgī vacā pāṭhā niculo rohiṇī jalam |  sarvamaṇḍūkadaṣṭānāmagado 'yaṃ viṣāpahaḥ || A.}
}
\pend

 
\pstart
\edtext{}{
  \Afootnote{\textsc{[pre]} dhavāśvagandhātibalābalāsātiguhāguhāḥ |  viśvambharābhidaṣṭānāmagado 'yaṃ viṣāpahaḥ || A.}
}
\pend

 
\pstart
\edtext{}{
  \Afootnote{\textsc{[pre]} śirīṣaṃ tagaraṃ kuṣṭhaṃ śāliparṇī sahā niśe |  ahiṇḍukābhir daṣṭānāmagado viṣanāśanaḥ || A.}
}
\pend

 
\pstart
\edtext{}{
  \Afootnote{\textsc{[pre]} kaṇḍūmakābhir daṣṭānāṃ rātrau śītāḥ kriyā hitāḥ |  divā te naiva sidhyanti sūryaraśmibalārditāḥ || A.}
}
\pend

 
\pstart
\edtext{}{
  \Afootnote{\textsc{[pre]} vakraṃ kuṣṭhamapāmārgaḥ śūkavṛntaviṣe 'gadaḥ |  bhṛṅgasvarasapiṣṭā vā kṛṣṇavalmīkamṛttikā || A.}
}
\pend

 
\pstart
\edtext{}{
  \Afootnote{\textsc{[pre]} pipīlikābhir daṣṭānāṃ makṣikāmaśakais tathā |  gomūtreṇa yuto lepaḥ kṛṣṇavalmīkamṛttikā || A.}
}
\pend

 
\pstart
\edtext{}{
  \Afootnote{\textsc{[pre]} nakhāvaghṛṣṭasaṃjāte śophe bhṛṅgaraso hitaḥ |  pratisūryakadaṣṭānāṃ sarpadaṣṭavadāceret | A.}
}
\pend

 
\pstart

                         \textsc{[1938 ed. 5.8.56ef]}
                        \caesura trividhā vṛścikāḥ proktā mandamadhyamahāviṣāḥ | 
\pend

 
\pstart
\edtext{}{
  \Afootnote{\textsc{[pre]} gośakṛtkothajā mandā madhyāḥ kāṣṭheṣṭikodbhavāḥ | A.}
}
\pend

 
\pstart

                         \textsc{[1938 ed. 5.8.57cd]}
                        \caesura sarpakothodbhavās tīkṣṇā \edtext{digdhadaṣṭaṃ\edlabel{SS.5.8.57cd-3}}{
  \Afootnote{ye cānye A.}
} \edtext{viṣair}{
  \linenum{|\xlineref{SS.5.8.57cd-3}}\lemma{digdhadaṣṭaṃ viṣair}\Afootnote{digdhadaṃṣṭraviṣair H.}
\lemma{viṣair}  \Afootnote{viṣasaṃbhavāḥ || A.}
} hate\edlabel{SS.5.8.57cd-5} \edtext{|}{
  \linenum{|\xlineref{SS.5.8.57cd-5}}\lemma{hate |}\Afootnote{\textsc{[om]} A.}
}
\pend

 
\pstart

                         \textsc{[1938 ed. 5.8.58]}
                        \caesura \edtext{kothe}{
  \Afootnote{\textsc{[om]} A.}
} \edtext{madhyā}{
  \Afootnote{mandā A; madhye Nep.}
} gavādīnāṃ\edlabel{SS.5.8.58-3} \edtext{śakṛtkothe}{
  \linenum{|\xlineref{SS.5.8.58-3}}\lemma{gavādīnāṃ śakṛtkothe}\Afootnote{dvādaśa A.}
\lemma{śakṛtkothe}  \Afootnote{śa\uwave{kṛt}kothe H.}
} \edtext{varāḥ}{
  \Afootnote{madhyās A.}
} smṛtāḥ |
                        \caesura\edlabel{SS.5.8.58-6}
 \edtext{saptaviṃśatir}{
  \linenum{|\xlineref{SS.5.8.58-6}}\lemma{smṛtāḥ |  saptaviṃśatir}\Afootnote{tu trayaḥ A.}
\lemma{saptaviṃśatir}  \Afootnote{\textsc{[add]} pañcadaśottamāḥ |  A.}
} \edtext{evaite\edlabel{SS.5.8.58-8}}{
  \Afootnote{daśa viṃ° ity A.}
} \edtext{saṅkhyayā}{
  \linenum{|\xlineref{SS.5.8.58-8}}\lemma{evaite saṅkhyayā}\Afootnote{°śatirevaite K.}
\lemma{saṅkhyayā}  \Afootnote{ete A.}
} \edtext{parikīrttitāḥ}{
  \Afootnote{saṃkhyāyāḥ H.}
} | 
\pend

 
\pstart

                         \textsc{[1938 ed. 5.8.59]}
                        \caesura \edtext{kṛṣṇaḥ}{
  \Afootnote{kṛṣṇa K.}
} śyāvaḥ karburo \edtext{romaśaś}{
  \Afootnote{pāṇḍuvarṇo A.}
} \edtext{ca \caesura}{
  \Afootnote{\textsc{[om]} A.}
} gomūtrābhaḥ \edtext{paruṣo}{
  \Afootnote{karkaśo A.}
} \edtext{mecakaś}{
  \Afootnote{dakaś K; modakaś H.}
} ca |
                        \caesura
 śveto \edtext{rakto}{
  \Afootnote{pīto A.}
} \edtext{romaśīrṣogradhūmaḥ \caesura}{
  \Afootnote{dhūmro A.}
} \edtext{sarve}{
  \Afootnote{romaśaḥ A; °dhūmraḥ Nep.}
} 'py\edlabel{SS.5.8.59-14} \edtext{ete}{
  \linenum{|\xlineref{SS.5.8.59-14}}\lemma{'py ete}\Afootnote{śvetenodareṇeti A.}
} mandaviṣā\edlabel{SS.5.8.59-16} \edtext{matās}{
  \linenum{|\xlineref{SS.5.8.59-16}}\lemma{mandaviṣā matās}\Afootnote{\textsc{[om]} A.}
} \edtext{tu}{
  \Afootnote{mandāḥ || A.}
} |\edlabel{SS.5.8.59-19} 
\pend

 
\pstart
\edtext{}{
  \Afootnote{\textsc{[pre]} yuktāś caite vṛścikāḥ pucchadeśe syur bhūyobhiḥ parvabhiś cetarebhyaḥ | A.}
}
\pend

 
\pstart

                         \textsc{[1938 ed. 5.8.60cd]}
                        \caesura ebhir daṣṭe vedanā \edtext{vepathuś}{
  \Afootnote{vethuś K.}
} ca \caesura \edtext{gātrastabdhaḥ}{
  \Afootnote{gātrastambhaḥ A.}
} kṛṣṇaraktāgamaś ca | 
\pend

 
\pstart

                         \textsc{[1938 ed. 5.8.61]}
                        \caesura \edtext{śākhāviddhe}{
  \Afootnote{śākhādaṣṭe A.}
} \edtext{vedanāñ}{
  \Afootnote{vedanā A.}
} \edtext{cordhvam eti
                        \caesura
}{
  \Afootnote{corddham H.}
} daṃśasvedo \edtext{mukhaśophaś}{
  \Afootnote{dāhasvedau A; daṃśa\textbf{s tīvrā}s vedo H.}
} \edtext{ca}{
  \Afootnote{daṃśaśopho A; mukhaśopha\textbf{ś} H.}
} \edtext{tīvraḥ\edlabel{SS.5.8.61-7}}{
  \Afootnote{\textsc{[om]} A.}
} | \caesura\edlabel{SS.5.8.61-8} \edtext{raktaṃ}{
  \linenum{|\xlineref{SS.5.8.61-7}}\lemma{tīvraḥ\ldots raktaṃ}\Afootnote{catīvraḥ ||  H.}
  \linenum{|\xlineref{SS.5.8.61-8}}\lemma{|  raktaṃ}\Afootnote{jvaraś A.}
\lemma{raktaṃ}  \Afootnote{\textsc{[add]} ca |  A.}
} \edtext{pītaṃ}{
  \Afootnote{raktaḥ A.}
} \edtext{kapilaṃ}{
  \Afootnote{pītaḥ A.}
} codaraṃ\edlabel{SS.5.8.61-12} \edtext{tu \caesura}{
  \linenum{|\xlineref{SS.5.8.61-12}}\lemma{codaraṃ tu }\Afootnote{kāpilenodareṇa A.}
} \edtext{dhūmro}{
  \Afootnote{sarve A; ca H.}
} \edtext{varṇṇas}{
  \Afootnote{dhūmrāḥ A.}
} \edtext{tatra}{
  \Afootnote{parvabhiś A.}
} yo\edlabel{SS.5.8.61-17} \edtext{madhyavīryāḥ}{
  \linenum{|\xlineref{SS.5.8.61-17}}\lemma{yo madhyavīryāḥ}\Afootnote{ca tribhiḥ A.}
} |\edlabel{SS.5.8.61-19} 
\pend

 
\pstart
\edtext{}{
  \Afootnote{\textsc{[pre]} ete mūtroccārapūtyaṇḍajātā madhyā jñeyās triprakāroragāṇām |  yasyaiteṣām anvayādyaḥ prasūto doṣotpattiṃ tat svarūpāṃ sa kuryāt || A.}
}
\pend

 
\pstart

                         \textsc{[1938 ed. 5.8.63]}
                        \caesura jihvāśopho \edtext{rasanasyopaghāto \caesura}{
  \Afootnote{rasanasyāpa° H; bhojanasyāvarodho A.}
} mūrchā cogrā madhyaviṣābhidaṣṭe\edlabel{SS.5.8.63-5} \edtext{| \caesura}{
  \linenum{|\xlineref{SS.5.8.63-5}}\lemma{madhyaviṣābhidaṣṭe | }\Afootnote{madhyavīryābhi° A.}
} śvetaś citraḥ \edtext{śabalo}{
  \Afootnote{śyāmalo A.}
} lohitābhaḥ \caesura \edtext{kṛṣṇaḥ}{
  \Afootnote{raktaḥ A.}
} \edtext{śyāvaḥ}{
  \Afootnote{śveto A; śyāva\uwave{ś} K.}
} \edtext{śvetanīlodarau}{
  \Afootnote{śveta° K; raktanī° A.}
} ca | 
\pend

 
\pstart
\edtext{}{
  \Afootnote{\textsc{[pre]} pīto 'rakto nīlapīto 'paras tu rakto nīlo nīlaśuklas tathā ca | A.}
}
\pend

 
\pstart

                         \textsc{[1938 ed. 5.8.64cd]}
                        \caesura rakto \edtext{babhruḥ}{
  \Afootnote{vabhru K.}
} pūrvavad ekaparvā\edlabel{SS.5.8.64cd-4} \edtext{| \caesura}{
  \linenum{|\xlineref{SS.5.8.64cd-4}}\lemma{ekaparvā | }\Afootnote{caika° A.}
\lemma{| }  \Afootnote{\textsc{[add]} yaś A.}
} \edtext{pūrvā}{
  \Afootnote{cāparvā A.}
} \edtext{cāpi}{
  \Afootnote{\textsc{[om]} A.}
} parvaṇī dve ca yasya | 
\pend

 
\pstart

                         \textsc{[1938 ed. 5.8.65ab]}
                        \caesura \edtext{nānāvarṇṇā}{
  \Afootnote{nānārūpā A.}
} \edtext{rūpataś}{
  \Afootnote{varṇataś A.}
} cāpi ghorāḥ | \caesura \edtext{jñeyā}{
  \Afootnote{jñeyāś A.}
} hy\edlabel{SS.5.8.65ab-7} \edtext{ete}{
  \linenum{|\xlineref{SS.5.8.65ab-7}}\lemma{hy ete}\Afootnote{caite A.}
} vṛścikāḥ prāṇanāśāḥ\edlabel{SS.5.8.65ab-10} \edtext{||}{
  \linenum{|\xlineref{SS.5.8.65ab-10}}\lemma{prāṇanāśāḥ ||}\Afootnote{prāṇacaurāḥ | A.}
}
\pend

 
\pstart
\edtext{}{
  \Afootnote{\textsc{[pre]} janmaiteṣāṃ sarpakothāt pradiṣṭaṃ  dehebhyo vā ghātitānāṃ viṣeṇa || A.}
}
\pend

 
\pstart

                         \textsc{[1938 ed. 5.8.66]}
                        \caesura ebhir daṣṭe \edtext{viṣavegapravṛttiḥ \caesura}{
  \Afootnote{sarpave° A.}
} sphoṭotpattir \edtext{jvaradāhau}{
  \Afootnote{bhrāntidāhau A.}
} \edtext{bhramaś}{
  \Afootnote{jvaraś A.}
} ca | \caesura khebhyaḥ kṛṣṇaṃ śoṇitaṃ \edtext{cātitīvraṃ \caesura}{
  \Afootnote{cati° K; yāti tīvraṃ A.}
} \edtext{tataḥ}{
  \Afootnote{tasmāt A; \textbf{tataḥ}\uuline{srāvaḥ} H.}
} prāṇais \edtext{tyājyate}{
  \Afootnote{tyajyate A.}
} \edtext{kṣipram}{
  \Afootnote{śīghram A.}
} eva | 
\pend

 
\pstart

                         \textsc{[1938 ed. 5.8.67ab]}
                        \caesura ugramadhyaviṣair \edtext{daṣṭāṃś}{
  \Afootnote{daṣṭaṃ A; ddaṣṭām̐ś H.}
} cikitset sarpadaṣṭavat | 
\pend

 
\pstart

                         \textsc{[1938 ed. 5.8.70]}
                        \caesura \edtext{daṃśamandaviṣāṇān}{
  \Afootnote{daṃśo \uuline{manda}\textbf{madhya}vi° H; daṃśaṃ mandaviṣāṇāṃ A.}
} tu cakratailena secayet | \caesura \edtext{vidārigandhāditailena}{
  \Afootnote{vidārīgaṇasiddhena A.}
} sukhoṣṇenāthavā punaḥ | 
\pend

 
\pstart

                         \textsc{[1938 ed. 5.8.67cd]}
                        \caesura \edtext{ādaṃśaṃ}{
  \Afootnote{ādaṃ\textbf{\uwave{śaṃ}} K.}
} sveditaṃ \edtext{cūrṇṇaiḥ}{
  \Afootnote{cūrṇṇaiṃḥ K.}
} pracchitam pratisārayet | 
\pend

 
\pstart

                         \textsc{[1938 ed. 5.8.68]}
                        \caesura rajanīsaindhavavyoṣaśirīṣaphalapuṣpajaiḥ | \caesura mātuluṃgāmlagomūtrapiṣṭañ ca surasāgrajaṃ | 
\pend

 
\pstart

                         \textsc{[1938 ed. 5.8.69]}
                        \caesura \edtext{lepe}{
  \Afootnote{\textsc{[add]} svede A.}
} sukhoṣṇañ ca \edtext{tathā}{
  \Afootnote{\textsc{[om]} A.}
} gomayaṃ hitam ucyate\edlabel{SS.5.8.69-7} \edtext{|| \caesura}{
  \linenum{|\xlineref{SS.5.8.69-7}}\lemma{ucyate || }\Afootnote{iṣyate | A.}
} \edtext{sarpiḥ}{
  \Afootnote{pāne A.}
} kṣaudrayutaṃ \edtext{pāne}{
  \Afootnote{sarpiḥ A.}
} kṣīram vā vahuśarkaraṃ\edlabel{SS.5.8.69-14} \edtext{|}{
  \linenum{|\xlineref{SS.5.8.69-14}}\lemma{vahuśarkaraṃ |}\Afootnote{bahu° A.}
}
\pend

 
\pstart

                         \textsc{[1938 ed. SS.5.8.71cd]}
                        \caesura guḍodakaṃ vā \edtext{suhitaṃ}{
  \Afootnote{suhimaṃ A.}
} caturjātakavāsitaṃ\edlabel{SS.5.8.71cd-4} \edtext{|}{
  \linenum{|\xlineref{SS.5.8.71cd-4}}\lemma{caturjātakavāsitaṃ |}\Afootnote{cāturjātakasaṃyutam || A; °tavāsitaṃ | K.}
}
\pend

 
\pstart
\edtext{}{
  \Afootnote{\textsc{[pre]} pānam asmai pradātavyaṃ kṣīraṃ vā saguḍaṃ himam | A.}
}
\pend

 
\pstart

                         \textsc{[1938 ed. SS.5.8.72cd]}
                        \caesura śikhikukkuṭabarhāṇi saindhavaṃ \edtext{tailam}{
  \Afootnote{tailasarpiṣī || A.}
} eva\edlabel{SS.5.8.72cd-4} ca \edtext{|}{
  \linenum{|\xlineref{SS.5.8.72cd-4}}\lemma{eva\ldots |}\Afootnote{\textsc{[om]} A.}
}
\pend

 
\pstart

                         \textsc{[1938 ed. 5.8.71ab]}
                        \caesura kuryāc cotkārikāsvedaṃ viṣaghnair upanāhanaiḥ\edlabel{SS.5.8.71ab-4} \edtext{|}{
  \linenum{|\xlineref{SS.5.8.71ab-4}}\lemma{upanāhanaiḥ |}\Afootnote{°hayet | A.}
}
\pend

 
\pstart

                         \textsc{[1938 ed. 5.8.73]}
                        \caesura \edtext{dhūpo}{
  \Afootnote{dhūmo A; dhūpo\textbf{mo} H.}
} hanti prayukto \edtext{'yaṃ}{
  \Afootnote{tu A.}
} śīghraṃ vṛścikajaṃ viṣam |
                        \caesura
 kusuṃbhapuṣpaṃ \edtext{rajanī}{
  \Afootnote{kurumbha° H.}
} niṣyā \edtext{vā}{
  \Afootnote{niśā A.}
} kṣaudrakaṃ \edtext{tṛṇam}{
  \Afootnote{kodravaṃ A.}
} | 
\pend

 
\pstart

                         \textsc{[1938 ed. 5.8.74]}
                        \caesura ebhir ghṛtāktair dhūpas tu pāyudeśaprayojitaḥ\edlabel{SS.5.8.74-5} \edtext{|| \caesura}{
  \linenum{|\xlineref{SS.5.8.74-5}}\lemma{pāyudeśaprayojitaḥ || }\Afootnote{pāyudeśe pra° A.}
} nāśayed āśu kīṭotthaṃ \edtext{vṛścikasya}{
  \Afootnote{vṛścikasyā Nep.}
} ca yad viṣam |
 
\pend

 
\pstart

                         \textsc{[1938 ed. 5.8.75]}
                        \caesura lūtāviṣaṃ ghoratamaṃ durvijñeyatamañ ca yat\edlabel{SS.5.8.75-5} \edtext{| \caesura}{
  \linenum{|\xlineref{SS.5.8.75-5}}\lemma{yat | }\Afootnote{tat |  A.}
} duścikitsyatamañ cāpi \edtext{bhiṣagbhir}{
  \Afootnote{bhiṣagbhim H.}
} mandabuddhibhiḥ | 
\pend

 
\pstart

                         \textsc{[1938 ed. 5.8.76]}
                        \caesura saviṣaṃ nirviṣañ \edtext{cedam}{
  \Afootnote{caitad A.}
} ity evaṃ saviśaṃkite\edlabel{SS.5.8.76-6} \edtext{| \caesura}{
  \linenum{|\xlineref{SS.5.8.76-6}}\lemma{saviśaṃkite | }\Afootnote{pariśaṅ° A.}
} viṣaghnam eva karttavyam avirodhi yad auṣadham | 
\pend

 
\pstart

                         \textsc{[1938 ed. 5.8.77]}
                        \caesura agadānāṃ hi saṃyogo \edtext{viṣaduṣṭasya}{
  \Afootnote{viṣajuṣṭasya A; viṣayuktasya H.}
} yujyate\edlabel{SS.5.8.77-5} \edtext{| \caesura}{
  \linenum{|\xlineref{SS.5.8.77-5}}\lemma{yujyate | }\Afootnote{yuṃjyate ||  H.}
} nirviṣe mānave yukto 'gadaḥ saṃpadyate gadaḥ\edlabel{SS.5.8.77-12} \edtext{|}{
  \linenum{|\xlineref{SS.5.8.77-12}}\lemma{gadaḥ |}\Afootnote{'sukham || A; gataḥ | H.}
}
\pend

 
\pstart

                         \textsc{[1938 ed. 5.8.78]}
                        \caesura tasmāt sarvaprayatnena jñātavyo viṣaniścayaḥ || \caesura \edtext{ajñātvā}{
  \Afootnote{ajñātvād Nep.}
} viṣasadbhāvaṃ bhiṣag vyāpādayen naram ||
 
\pend

 
\pstart

                         \textsc{[1938 ed. 5.8.79]}
                        \caesura \edtext{yadvat}{
  \Afootnote{\textsc{[om]} A.}
} \edtext{prasūtena}{
  \Afootnote{prodbhidyamānas tu A.}
} \edtext{navāṃkureṇa \caesura}{
  \Afootnote{yathāṅku° A.}
} na \edtext{vyaktajāti}{
  \Afootnote{vyaktajātiḥ A.}
} \edtext{pratibhāti}{
  \Afootnote{pravibhāti A.}
} vṛkṣaḥ |
                        \caesura
 tadvad durālakṣyatamaṃ hi tāsāṃ \caesura viṣaṃ śarīre \edtext{pravikīrṇṇamātraṃ}{
  \Afootnote{rśarīre K.}
} | 
\pend

 
\pstart

                         \textsc{[1938 ed. 5.8.80]}
                        \caesura \edtext{īṣatsakaṇḍū}{
  \Afootnote{°kaṇḍu A.}
} pracalañ ca\edlabel{SS.5.8.80-3} \edtext{koṭham \caesura}{
  \linenum{|\xlineref{SS.5.8.80-3}}\lemma{ca koṭham }\Afootnote{sakoṭham A.}
} avyaktavarṇṇaṃ prathame 'hani syāt | \caesura anteṣu śūnaṃ \edtext{parinimna}{
  \Afootnote{parini H.}
} madhyam \caesura \edtext{pravyaktavarṇṇañ}{
  \Afootnote{avya° H; pravyaktarūpaṃ A.}
} ca dine dvitīye | 
\pend

 
\pstart

                         \textsc{[1938 ed. 5.8.81]}
                        \caesura tryaheṇa tad darśayatīha \edtext{daṃśaṃ \caesura}{
  \Afootnote{rūpaṃ A.}
} viṣaṃ caturthe 'hani \edtext{kopam}{
  \Afootnote{kopa K.}
} eti\edlabel{SS.5.8.81-9} \edtext{| \caesura}{
  \linenum{|\xlineref{SS.5.8.81-9}}\lemma{eti | }\Afootnote{meti |  K.}
} ato 'dhike 'hni prakaroti \edtext{janto \caesura}{
  \Afootnote{jantor A H.}
} viṣaprakopaprabhavān vikārān | 
\pend

 
\pstart

                         \textsc{[1938 ed. 5.8.82]}
                        \caesura ṣaṣṭhe dine \edtext{viprasṛtan}{
  \Afootnote{viprasṛtaṃ A.}
} tu sarvān \caesura marmapradeśān bhṛśam āvṛṇoti | \caesura tat \edtext{saptame}{
  \Afootnote{sa\textsc{(l. 2)}me K.}
} 'tyarthaparītagātraṃ \caesura vyāpādayet martyam atipravṛddhaṃ | 
\pend

 
\pstart

                         \textsc{[1938 ed. 5.8.83]}
                        \caesura yās\edlabel{SS.5.8.83-1} \edtext{tīkṣṇacaṇḍograviṣā}{
  \linenum{|\xlineref{SS.5.8.83-1}}\lemma{yās tīkṣṇacaṇḍograviṣā}\Afootnote{yāstīkṣṇa° Nep.}
} hi lūtās \caesura\edlabel{SS.5.8.83-4} \edtext{tāḥ}{
  \linenum{|\xlineref{SS.5.8.83-4}}\lemma{lūtās  tāḥ}\Afootnote{lūtāstāḥ Nep.}
} saptarātreṇa naraṃ nihanyuḥ | \caesura ato 'dhikenāpi nihanyur anyā \caesura \edtext{yeṣāṃ}{
  \Afootnote{yāsāṃ A.}
} viṣaṃ \edtext{madhyamavīryam\edlabel{SS.5.8.83-16}}{
  \Afootnote{°vīrya\textsc{(l. 2)}ryam H.}
} uktaṃ \edtext{|}{
  \linenum{|\xlineref{SS.5.8.83-16}}\lemma{madhyamavīryam\ldots |}\Afootnote{°vīryamuktaṃ | K.}
}
\pend

 
\pstart

                         \textsc{[1938 ed. 5.8.84]}
                        \caesura yāsāṃ\edlabel{SS.5.8.84-1} \edtext{kanīyo}{
  \linenum{|\xlineref{SS.5.8.84-1}}\lemma{yāsāṃ kanīyo}\Afootnote{yā sāṅka° H.}
} viṣavīryam\edlabel{SS.5.8.84-3} \edtext{uktaṃ \caesura}{
  \linenum{|\xlineref{SS.5.8.84-3}}\lemma{viṣavīryam uktaṃ }\Afootnote{viṣavīryamuktaṃ K; viṣavīryayuktaṃ H.}
} tāḥ pakṣamātreṇa vināśayanti | \caesura tasmāt prayatnaṃ bhiṣag\edlabel{SS.5.8.84-11} atra \edtext{kuryād \caesura}{
  \linenum{|\xlineref{SS.5.8.84-11}}\lemma{bhiṣag\ldots kuryād }\Afootnote{bhi\textsc{(l. 3)}ṣagatraku° K.}
} ā daṃśapātād viṣaghātavegaiḥ\edlabel{SS.5.8.84-16} \edtext{|
}{
  \linenum{|\xlineref{SS.5.8.84-16}}\lemma{viṣaghātavegaiḥ |}\Afootnote{viṣaghātiyogaiḥ || A; °ta\uuline{ve}\textbf{yo}rgaiḥ || H.}
}
\pend

 
\pstart

                         \textsc{[1938 ed. 5.8.85]}
                        \caesura \edtext{viṣan}{
  \Afootnote{viṣaṃ A; viṣaṃn K.}
} tu lālānakhamūtradaṃṣṭrā \caesura rajaḥ purīṣair atha cendriyeṇa | \caesura saptaprakāraṃ visṛjanti lūtās \caesura tad ugramadhyāvaravīryam\edlabel{SS.5.8.85-13} uktaṃ \edtext{||}{
  \linenum{|\xlineref{SS.5.8.85-13}}\lemma{ugramadhyāvaravīryam\ldots ||}\Afootnote{°vīryayuktam || A.}
}
\pend

 
\pstart

                         \textsc{[1938 ed. 5.8.86]}
                        \caesura \edtext{koṭhaṃ}{
  \Afootnote{sakaṇḍukoṭhaṃ A; koṭhan H.}
} \edtext{sakaṇḍūsthiram}{
  \Afootnote{sthiram A; sagaṇḍasthiram K.}
} alpamūlaṃ \caesura lālākṛtaṃ mandarujaṃ\edlabel{SS.5.8.86-5} vadanti \edtext{| \caesura}{
  \linenum{|\xlineref{SS.5.8.86-5}}\lemma{mandarujaṃ\ldots | }\Afootnote{°jaṃ\textsc{(l. 4)}vadanti ||  H.}
} \edtext{coṣaś}{
  \Afootnote{śophaś A; \uuline{coṣa}\textbf{śvāsa}ś H.}
} ca kaṇḍūś ca \edtext{pulāyikāś}{
  \Afootnote{pulālikā A.}
} ca \caesura dhūmāyanaṃ caiva nakhāgradaṃśe | 
\pend

 
\pstart

                         \textsc{[1938 ed. 5.8.87]}
                        \caesura \edtext{daṃśe}{
  \Afootnote{daṃśaṃ A; daṅśe H.}
} tu mūtreṇa sakṛṣṇamadhyam | \caesura saraktaparyantam \edtext{avaihi}{
  \Afootnote{avehi A.}
} dīrṇṇam | \caesura daṃṣṭrābhir ugraṃ kaṭhinaṃ vivarṇṇam | \caesura \edtext{jānīṣva}{
  \Afootnote{jānīhi A.}
} daṃśaṃ sthiramaṇḍalañ ca | 
\pend

 
\pstart

                         \textsc{[1938 ed. 5.8.88]}
                        \caesura rajaḥpurīṣendriyajañ \edtext{ca}{
  \Afootnote{hi A.}
} viddhi \caesura sphoṭaṃ prapakvāmalapīlupāṇḍum\edlabel{SS.5.8.88-5} \edtext{| \caesura}{
  \linenum{|\xlineref{SS.5.8.88-5}}\lemma{prapakvāmalapīlupāṇḍum | }\Afootnote{vipa° A; prapakvāṃ ma° H.}
} etāvad etat samudāhṛtaṃ \edtext{te \caesura}{
  \Afootnote{tu A.}
} vakṣyāmi lūtāprabhavaṃ pramāṇam\edlabel{SS.5.8.88-13} \edtext{|}{
  \linenum{|\xlineref{SS.5.8.88-13}}\lemma{pramāṇam |}\Afootnote{purāṇam || A; pramānaṃ | K.}
}
\pend

 
\pstart

                         \textsc{[1938 ed. 5.8.89]}
                        \caesura sāmānyato daṣṭam asādhyasādhyaṃ \caesura cikitsitaṃ cāpi \edtext{viśeṣaṇañ}{
  \Afootnote{yathāviśeṣam | A.}
} ca\edlabel{SS.5.8.89-7} \edtext{||}{
  \linenum{|\xlineref{SS.5.8.89-7}}\lemma{ca ||}\Afootnote{\textsc{[om]} A.}
}
\pend

 
\pstart

                         \textsc{[1938 ed. 5.8.90]}
                        \caesura viśvāmitro nṛpavaraḥ kadācid ṛṣisattamaṃ\edlabel{SS.5.8.90-4} \edtext{| \caesura}{
  \linenum{|\xlineref{SS.5.8.90-4}}\lemma{ṛṣisattamaṃ | }\Afootnote{°tta\textbf{maṃ}  K.}
} vaśiṣṭhaṃ kopayāmāsa gatvāśramapadaṃ kila | 
\pend

 
\pstart

                         \textsc{[1938 ed. 5.8.91]}
                        \caesura kupitasya munes tasya lalāṭāt svedabindavaḥ | \caesura \edtext{niyetur}{
  \Afootnote{apatan A.}
} darśanād eva \edtext{raves}{
  \Afootnote{sves A.}
} tatsamavarcasaḥ\edlabel{SS.5.8.91-11} \edtext{|}{
  \linenum{|\xlineref{SS.5.8.91-11}}\lemma{tatsamavarcasaḥ |}\Afootnote{tatsamatejasaḥ || A.}
}
\pend

 
\pstart

                         \textsc{[1938 ed. 5.8.92ab]}
                        \caesura \edtext{lūne}{
  \Afootnote{\textsc{[om]} A.}
} tṛṇe \edtext{maharṣiṇā}{
  \Afootnote{\textsc{[add]} lūne A.}
} \edtext{dhenvarthe}{
  \Afootnote{dhenvarthaṃ A; \uwave{ven}(\textbf{dhe})nvarthe K; \uwave{ven}(\textbf{dhe})ṇvarthe H.}
} sambhṛte 'pi ca\edlabel{SS.5.8.92ab-7} \edtext{|
}{
  \linenum{|\xlineref{SS.5.8.92ab-7}}\lemma{ca |}\Afootnote{caḥ | K.}
}
\pend

 
\pstart
\edtext{}{
  \Afootnote{\textsc{[pre]} tato jātās tv imā ghorā nānārūpā mahāviṣāḥ A.}
}
\pend

 
\pstart

                         \textsc{[1938 ed. 5.8.92ef]} apakārāya varttante nṛpaśāsanavāhane\edlabel{SS.5.8.92ef-3} \edtext{|}{
  \linenum{|\xlineref{SS.5.8.92ef-3}}\lemma{nṛpaśāsanavāhane |}\Afootnote{nṛpasādhana° A; °hanaiḥ || H.}
}
\pend

 
\pstart

                         \textsc{[1938 ed. 5.8.93]}
                        \caesura \edtext{yasmāl}{
  \Afootnote{yasmāḥ K.}
} \edtext{lūnaṃ}{
  \Afootnote{lūtan H.}
} tṛṇaṃ \edtext{prāptā}{
  \Afootnote{prāptān K; prāptāt H.}
} munes \edtext{te}{
  \Afootnote{\textsc{[om]} A.}
} svedabindavaḥ\edlabel{SS.5.8.93-7} \edtext{| \caesura}{
  \linenum{|\xlineref{SS.5.8.93-7}}\lemma{svedabindavaḥ | }\Afootnote{prasve° A.}
} tasmāl lūtā\edlabel{SS.5.8.93-10} \edtext{vibhāvyante}{
  \linenum{|\xlineref{SS.5.8.93-10}}\lemma{lūtā vibhāvyante}\Afootnote{lūteti bhāṣyante A.}
} saṃkhyayā tāś ca ṣoḍaśaḥ\edlabel{SS.5.8.93-15} \edtext{|}{
  \linenum{|\xlineref{SS.5.8.93-15}}\lemma{ṣoḍaśaḥ |}\Afootnote{ṣoḍaśa || A H.}
}
\pend

 
\pstart

                         \textsc{[1938 ed. 5.8.94]}
                        \caesura kṛcchrasādhyās tathāsādhyā lūtās tu dvividhāḥ smṛtāḥ | \caesura tāsām aṣṭau kṛcchrasādhyā varjyās tāvatya eva tu || 
\pend

 
\pstart

                         \textsc{[1938 ed. 5.8.95]}
                        \caesura \edtext{trimaṇḍalā}{
  \Afootnote{trimaṇḍala K; trimaṇḍalas H.}
} tathā \edtext{svetā}{
  \Afootnote{śvetā A.}
} kapilā pītikā tathā | \caesura \edtext{alamūtraviṣe}{
  \Afootnote{mala° H; ālamūtraviṣā A.}
} raktā kasanā cāṣṭamī smṛtāḥ\edlabel{SS.5.8.95-12} \edtext{||}{
  \linenum{|\xlineref{SS.5.8.95-12}}\lemma{smṛtāḥ ||}\Afootnote{smṛtā || A.}
}
\pend

 
\pstart

                         \textsc{[1938 ed. 5.8.96]}
                        \caesura tābhir \edtext{daṣṭe}{
  \Afootnote{ddṣṭe H.}
} śiroduḥkham \edtext{ādaṃśe}{
  \Afootnote{\textsc{[om]} A.}
} \edtext{kaṇḍur}{
  \Afootnote{kaṇḍūr A.}
} \edtext{eva}{
  \Afootnote{daṃśe A.}
} ca \edtext{| \caesura}{
  \Afootnote{\textsc{[add]} vedanā |  A.}
} bhavanti ca viśeṣeṇa gadāḥ śleṣmikavātikāḥ\edlabel{SS.5.8.96-13} \edtext{|}{
  \linenum{|\xlineref{SS.5.8.96-13}}\lemma{śleṣmikavātikāḥ |}\Afootnote{ślaiṣmi° A.}
}
\pend

 
\pstart

                         \textsc{[1938 ed. 5.8.97]}
                        \caesura sauvarṇṇikā lājavarṇṇā jāliny eṇīpadī tathā | \caesura \edtext{kṛṣṇāgnimukhyau}{
  \Afootnote{kṛṣṇāgnivarṇā A; °mukhyā H.}
} kākāṇḍā \edtext{mālāguṇy}{
  \Afootnote{mālāguṇā A.}
} \edtext{aṣṭamī}{
  \Afootnote{\textsc{[om]} a° A.}
} smṛtāḥ\edlabel{SS.5.8.97-11} \edtext{|}{
  \linenum{|\xlineref{SS.5.8.97-11}}\lemma{smṛtāḥ |}\Afootnote{tathā || A.}
}
\pend

 
\pstart

                         \textsc{[1938 ed. 5.8.98]}
                        \caesura tābhir daṣṭe daṃśakothaḥ pravṛttiḥ kṣatajasya ca | \caesura jvaro dāho 'tisāraś ca gadāḥ syuś ca tridoṣajāḥ | 
\pend

 
\pstart

                         \textsc{[1938 ed. 5.8.99]}
                        \caesura \edtext{piṭakā}{
  \Afootnote{piḍakā A H.}
} vividhākārā maṇḍalāni mahānti ca || \caesura \edtext{śophā}{
  \Afootnote{\textsc{[om]} A.}
} \edtext{mahānto}{
  \Afootnote{mahā\textsc{(l. 6)}to H.}
} \edtext{mṛdavo}{
  \Afootnote{mṛduvo K; \textsc{[add]} śophā A.}
} \edtext{raktāḥ}{
  \Afootnote{raktā H.}
} śyāvāś calās tathā | 
\pend

 
\pstart

                         \textsc{[1938 ed. 5.8.100]}
                        \caesura sāmānyaṃ \edtext{sarvalūtānām}{
  \Afootnote{°tā\textbf{nā}m K.}
} etad ādaṃśalakṣaṇam | \caesura viśeṣalakṣaṇaṃ tāsāṃ vakṣyāmi sacikitsitam | 
\pend

 
\pstart

                         \textsc{[1938 ed. 5.8.101]}
                        \caesura trimaṇḍalāyā \edtext{bahalaṃ}{
  \Afootnote{\textsc{[om]} A.}
} \edtext{daṃśaḥ}{
  \Afootnote{daṃśe 'sṛk A.}
} \edtext{kṛṣṇaṃ}{
  \Afootnote{kṛṣṇaḥ H.}
} \edtext{kṣaraty}{
  \Afootnote{sravati A.}
} asṛk\edlabel{SS.5.8.101-6} \edtext{| \caesura}{
  \linenum{|\xlineref{SS.5.8.101-6}}\lemma{asṛk | }\Afootnote{dīryate |  A.}
} bādhiryaṃ kaluṣā dṛṣṭis tathā dāhaś ca netrayoḥ | 
\pend

 
\pstart

                         \textsc{[1938 ed. 5.8.102]}
                        \caesura tatrārkamūlaṃ rajanī nākulī pṛśniparṇṇikā || \caesura nastaḥ\edlabel{SS.5.8.102-6} \edtext{karmaṇi}{
  \linenum{|\xlineref{SS.5.8.102-6}}\lemma{nastaḥ karmaṇi}\Afootnote{pānaka° A.}
} \edtext{śasyante}{
  \Afootnote{\textsc{[om]} H.}
} \edtext{pādābhyaṅgāñjaneṣu}{
  \Afootnote{nasyālepāñja° A.}
} ca | 
\pend

 
\pstart

                         \textsc{[1938 ed. 5.8.103]}
                        \caesura \edtext{śvetāyāḥ}{
  \Afootnote{śvetayā H.}
} piḍakā daṃśe śvetā kaṇḍūmatī bhavet | \caesura dāhamūrcchājvaravatī visarpakledarukkarī || 
\pend

 
\pstart

                         \textsc{[1938 ed. 5.8.104]}
                        \caesura tatra candanarāsnailahareṇunalavañjulāḥ\edlabel{SS.5.8.104-2} \edtext{| \caesura}{
  \linenum{|\xlineref{SS.5.8.104-2}}\lemma{candanarāsnailahareṇunalavañjulāḥ | }\Afootnote{candanarāsnailāha° A H.}
} \edtext{kuṣṭhalāmajjakaṃ}{
  \Afootnote{kuṣṭhaṃ lā° A H.}
} vakraṃ naladaṃ cāgado hitaḥ | 
\pend

 
\pstart

                         \textsc{[1938 ed. 5.8.105]}
                        \caesura ādaṃśe \edtext{piṭakā}{
  \Afootnote{piḍakā A; piṭakās K; piḍakās H.}
} tāmrā \edtext{sthirā}{
  \Afootnote{\textsc{[om]} A.}
} \edtext{kapilayā}{
  \Afootnote{kapilāyāḥ sthirā A.}
} bhavet | \caesura śiraso gauravaṃ \edtext{dāho}{
  \Afootnote{\textsc{[add]} timiraṃ A.}
} \edtext{bhavej}{
  \Afootnote{bhrama A.}
} \edtext{jantoś}{
  \Afootnote{\textsc{[om]} A.}
} \edtext{ca}{
  \Afootnote{eva A.}
} netrayoḥ\edlabel{SS.5.8.105-14} \edtext{||}{
  \linenum{|\xlineref{SS.5.8.105-14}}\lemma{netrayoḥ ||}\Afootnote{ca || A.}
}
\pend

 
\pstart

                         \textsc{[1938 ed. 5.8.106]}
                        \caesura \edtext{padma}{
  \Afootnote{tatra A.}
} padmaka \edtext{kuṣṭhaila}{
  \Afootnote{kuṣṭhailā A H.}
} karañja kakubha tvacaḥ | \caesura sthirā\edlabel{SS.5.8.106-8} \edtext{kampilyāpāmārga}{
  \linenum{|\xlineref{SS.5.8.106-8}}\lemma{sthirā kampilyāpāmārga}\Afootnote{sthirārkaparṇy apā° A.}
\lemma{kampilyāpāmārga}  \Afootnote{kampilyapā° K.}
} \edtext{dūrvā}{
  \Afootnote{durvvā H.}
} \edtext{brāhmyau}{
  \Afootnote{brāhmyo A; vrahmyo H.}
} viṣāpahāḥ\edlabel{SS.5.8.106-12} \edtext{|
}{
  \linenum{|\xlineref{SS.5.8.106-12}}\lemma{viṣāpahāḥ |}\Afootnote{viṣāpahā || H.}
}
\pend

 
\pstart

                         \textsc{[1938 ed. 5.8.107]}
                        \caesura ādaṃśe \edtext{piṭakā}{
  \Afootnote{pītikāyās A.}
} \edtext{pītā}{
  \Afootnote{tu piḍakā A.}
} \edtext{pītayā}{
  \Afootnote{pītikā A.}
} \edtext{jāyate}{
  \Afootnote{\textsc{[om]} A.}
} sthirā | \caesura tathā\edlabel{SS.5.8.107-8} \edtext{cchardijvaraḥ}{
  \linenum{|\xlineref{SS.5.8.107-8}}\lemma{tathā cchardijvaraḥ}\Afootnote{bhavec chardir jvaraḥ A.}
\lemma{cchardijvaraḥ}  \Afootnote{ccharddirjvaraḥ H.}
} \edtext{śūlo}{
  \Afootnote{śūlaṃ mūrdhni A.}
} rakte \edtext{syātāñ}{
  \Afootnote{tathā A.}
} \edtext{ca}{
  \Afootnote{\textsc{[om]} A.}
} locane\edlabel{SS.5.8.107-14} \edtext{|}{
  \linenum{|\xlineref{SS.5.8.107-14}}\lemma{locane |}\Afootnote{'kṣiṇī || A.}
}
\pend

 
\pstart

                         \textsc{[1938 ed. 5.8.108]}
                        \caesura \edtext{tatreṣṭhāḥ}{
  \Afootnote{tatreṣṭāḥ A; tatre\textsc{(l. 5)}ṣṭhā H.}
} kakubhośīramuñjābalvajavañjalāḥ\edlabel{SS.5.8.108-2} \edtext{| \caesura}{
  \linenum{|\xlineref{SS.5.8.108-2}}\lemma{kakubhośīramuñjābalvajavañjalāḥ | }\Afootnote{kuṭajośīratuṅgapadmakavañjulāḥ |  A.}
} kuśakāśavaṃśakiṇihīśirīṣakakubhatvacaḥ\edlabel{SS.5.8.108-4} \edtext{|}{
  \linenum{|\xlineref{SS.5.8.108-4}}\lemma{kuśakāśavaṃśakiṇihīśirīṣakakubhatvacaḥ |}\Afootnote{kuśakāśavaśa° H; śirīṣakiṇihīśelukadambaka° A; °hī śi\textsc{(l. 5)}rīṣakakubhatvacaḥ | K.}
}
\pend

 
\pstart

                         \textsc{[1938 ed. 5.8.109]}
                        \caesura \edtext{raktamaṇḍalavaddaṃśe}{
  \Afootnote{°ḍanibhe daṃśe A.}
} \edtext{piṭakāḥ}{
  \Afootnote{piḍakāḥ A.}
} sarṣapā\edlabel{SS.5.8.109-3} iva | \caesura \edtext{dūyate}{
  \Afootnote{jāyante A.}
} tāluśoṣaś ca dāhaś cālaviṣānvite\edlabel{SS.5.8.109-10} \edtext{|
}{
  \linenum{|\xlineref{SS.5.8.109-3}}\lemma{sarṣapā\ldots |}\Afootnote{sarṣapānvite | H.}
  \linenum{|\xlineref{SS.5.8.109-10}}\lemma{cālaviṣānvite |}\Afootnote{cala° K; °ṣārdite || A.}
}
\pend

 
\pstart

                         \textsc{[1938 ed. 5.8.110]}
                        \caesura tatra priyaṅguhrīverakuṣṭhalāmajjakāni\edlabel{SS.5.8.110-2} vā \edtext{| \caesura}{
  \linenum{|\xlineref{SS.5.8.110-2}}\lemma{priyaṅguhrīverakuṣṭhalāmajjakāni\ldots | }\Afootnote{°hrīberakuṣṭhalāmajjavañjulāḥ |  A.}
} \edtext{agadaḥ}{
  \Afootnote{aguruḥ H.}
} śatapuṣpā ca sapippalavaṭāṅkurāḥ | 
\pend

 
\pstart

                         \textsc{[1938 ed. 5.8.111]}
                        \caesura \edtext{pūtimūtraviṣādaṃśo}{
  \Afootnote{pūtir mū° A.}
} visarpī kṛṣṇaśoṇitaḥ | \caesura kāsaśvāsavamīmūrcchājvaradāhasamanvitaḥ | 
\pend

 
\pstart

                         \textsc{[1938 ed. 5.8.112]}
                        \caesura manaḥśilālamadhukakuṣṭha padmakacandanaiḥ\edlabel{SS.5.8.112-2} \edtext{| \caesura}{
  \linenum{|\xlineref{SS.5.8.112-2}}\lemma{padmakacandanaiḥ | }\Afootnote{candanapadmakaiḥ |  madhumiśraiḥ A.}
} \edtext{lāmajjakayutais}{
  \Afootnote{salāmajjair agadas A.}
} tatra \edtext{viṣanāśaḥ}{
  \Afootnote{\textsc{[om]} A.}
} prakīrttitaḥ\edlabel{SS.5.8.112-7} \edtext{|}{
  \linenum{|\xlineref{SS.5.8.112-7}}\lemma{prakīrttitaḥ |}\Afootnote{\textsc{[om]} pra° A.}
}
\pend

 
\pstart

                         \textsc{[1938 ed. 5.8.113]}
                        \caesura \edtext{daṃśaḥ}{
  \Afootnote{\textsc{[om]} A.}
} \edtext{sapāṇḍupiṭako}{
  \Afootnote{āpāṇḍupiḍako daṃśo A.}
} dāhakledasamanvitaḥ\edlabel{SS.5.8.113-3} \edtext{| \caesura}{
  \linenum{|\xlineref{SS.5.8.113-3}}\lemma{dāhakledasamanvitaḥ | }\Afootnote{dāhaḥ kle° H.}
} \edtext{raktayā}{
  \Afootnote{raktāyā A.}
} raktaparyanto vijñeyaś coṣasaṃyutaḥ\edlabel{SS.5.8.113-8} \edtext{|}{
  \linenum{|\xlineref{SS.5.8.113-8}}\lemma{coṣasaṃyutaḥ |}\Afootnote{raktasaṃyutaḥ || A.}
}
\pend

 
\pstart

                         \textsc{[1938 ed. 5.8.114]}
                        \caesura \edtext{cikitsā}{
  \Afootnote{kāryas A; cikitsān H.}
} \edtext{tatra}{
  \Afootnote{tatrāgadas A.}
} hrīveracandanośīrapadmakaiḥ\edlabel{SS.5.8.114-3} \edtext{| \caesura}{
  \linenum{|\xlineref{SS.5.8.114-3}}\lemma{hrīveracandanośīrapadmakaiḥ | }\Afootnote{hrīveraś can° H; toyacan° A.}
} \edtext{karttavyārjunaśelubhyāṃ}{
  \Afootnote{tathaivārju° A.}
} tvagbhir \edtext{āmrātakasya}{
  \Afootnote{(\textbf{āmrātaka})sya K.}
} ca | 
\pend

 
\pstart

                         \textsc{[1938 ed. 5.8.115]}
                        \caesura picchilaṃ \edtext{kasanādaṃśo}{
  \Afootnote{°daṃśād A; °daṃśā K; °daṃ(\textbf{śo}) H.}
} \edtext{rudhiraṃ}{
  \Afootnote{(\textbf{rudhi})raṃ H.}
} śītalaṃ sravet | \caesura \edtext{śvāsakāsau}{
  \Afootnote{kāsaśvāsau A.}
} ca \edtext{tatroktaṃ}{
  \Afootnote{(\textbf{tatroktaṃ}) K.}
} raktalūtācikitsitam | 
\pend

 
\pstart
\edtext{}{
  \Afootnote{\textsc{[pre]} purīṣagandhiralpāsṛk kṛṣṇāyā daṃśa eva tu |  jvaramūrcchāvamīdāhakāsaśvāsasamanvitaḥ || A.}
}
\pend

 
\pstart
\edtext{}{
  \Afootnote{\textsc{[pre]} tatrailāvakrasarpākṣīgandhanākulicandanaiḥ |  mahāsugandhisāhitaiḥ pratyākhyāyāgadaḥ smṛtaḥ || A.}
}
\pend

 
\pstart
\edtext{}{
  \Afootnote{\textsc{[pre]} daṃśe dāho 'gnivaktrāyāḥ srāvo 'tyarthaṃ jvaras tathā |  coṣakaṇḍūromaharṣā dāhavisphoṭasaṃyutaḥ || A.}
}
\pend

 
\pstart
\edtext{}{
  \Afootnote{\textsc{[pre]} kṛṣṇāpraśamanaṃ cātra pratyākhyāya prayojayet |  sārivośīrayaṣṭyāhvacandanotpalapadmakam || A.}
}
\pend

 
\pstart

                         \textsc{[1938 ed. 5.8.120]}
                        \caesura \edtext{sarveṣām}{
  \Afootnote{sarvāsām A.}
} eva yuñjīta \edtext{viṣe}{
  \Afootnote{viśe K.}
} śleṣmātakatvacam\edlabel{SS.5.8.120-5} \edtext{| \caesura}{
  \linenum{|\xlineref{SS.5.8.120-5}}\lemma{śleṣmātakatvacam | }\Afootnote{śleṣmānta° H.}
} \edtext{dhīraḥ}{
  \Afootnote{bhiṣak A.}
} \edtext{sarvavikāreṣu}{
  \Afootnote{sarvaprakāreṇa A.}
} tathā cākṣīvapippalam | 
\pend

 
\pstart

                         \textsc{[1938 ed. 5.8.121]}
                        \caesura kṛcchrasādhyā\edlabel{SS.5.8.121-1} \edtext{viṣā}{
  \linenum{|\xlineref{SS.5.8.121-1}}\lemma{kṛcchrasādhyā viṣā}\Afootnote{kṛcchrasādhyaviṣā A.}
} hy aṣṭau \edtext{lūtāḥ}{
  \Afootnote{\textsc{[om]} A; lūtā H.}
} \edtext{proktā}{
  \Afootnote{\textsc{[add]} dve ca A.}
} yathāgamaṃ\edlabel{SS.5.8.121-7} \edtext{| \caesura}{
  \linenum{|\xlineref{SS.5.8.121-7}}\lemma{yathāgamaṃ | }\Afootnote{yadṛcchayā |  A; yathākramaṃ ||  H.}
} \edtext{avārya}{
  \Afootnote{\uwave{(\textbf{avārya})} K; \textsc{(gap of 2, blank)}ra H.}
} viṣavīryāṇāṃ lakṣaṇāni nibodha me |
 
\pend

 
\pstart

                         \textsc{[1938 ed. 5.8.122]}
                        \caesura \edtext{dhyāmaḥ}{
  \Afootnote{dhyātaḥ Nep.}
} \edtext{sauvarṇṇikādaṃśaḥ}{
  \Afootnote{sauparṇṇi° H.}
} \edtext{sapheno}{
  \Afootnote{sapheno\uwave{} K.}
} matsyagandhikaḥ |
                        \caesura\edlabel{SS.5.8.122-4}
 \edtext{kāsaśvāso}{
  \linenum{|\xlineref{SS.5.8.122-4}}\lemma{matsyagandhikaḥ |  kāsaśvāso}\Afootnote{matsyagandhakaḥ |  A; °dhikaḥ |  K.}
} \edtext{jvaras}{
  \Afootnote{\textsc{[om]} kāsa° kāso A.}
} tṛṣṇā mūrcchā cātra sudāruṇāḥ |\edlabel{SS.5.8.122-11} 
\pend

 
\pstart
\edtext{
                         \textsc{[1938 ed. 5.8.123]}
                        \caesura}{
  \Afootnote{\textsc{[pre]} ādaṃśe lājavarṇāyā A.}
} \edtext{dhyāmaṃ}{
  \Afootnote{dhyāmaḥ Nep.}
} \edtext{pūti}{
  \Afootnote{pūtiḥ H.}
} \edtext{sraved}{
  \Afootnote{\textsc{[add]} asṛk |  A.}
} raktam\edlabel{SS.5.8.123-4} \edtext{ādāṃśe}{
  \Afootnote{ā\uwave{dāṃśe} K; ādaṃśe H.}
} lājavarṇṇayā |
                        \caesura
 \edtext{dāho}{
  \linenum{|\xlineref{SS.5.8.123-4}}\lemma{raktam\ldots dāho}\Afootnote{\textsc{[om]} A.}
} mūrcchātisārau \edtext{ca}{
  \Afootnote{mūrcchā 'tisāraś A; mūrcchā ca sārau H.}
} śiroduḥkha \edtext{ca}{
  \Afootnote{śiroduḥkhaṃ A; śiroduḥkhañ H.}
} jāyate | 
\pend

 
\pstart

                         \textsc{[1938 ed. 5.8.124]}
                        \caesura ghoro daṃśaś \edtext{ca}{
  \Afootnote{tu A.}
} jālinyā \edtext{rājimān}{
  \Afootnote{rājimāṇ K.}
} avadīryate\edlabel{SS.5.8.124-6} \edtext{| \caesura}{
  \linenum{|\xlineref{SS.5.8.124-6}}\lemma{avadīryate | }\Afootnote{avakīryate |  A.}
} stambhaḥ śvāsas tamovṛddhis tāluśoṣaś ca tatkṛtaḥ\edlabel{SS.5.8.124-13} \edtext{|}{
  \linenum{|\xlineref{SS.5.8.124-13}}\lemma{tatkṛtaḥ |}\Afootnote{jāyate || A.}
}
\pend

 
\pstart

                         \textsc{[1938 ed. 5.8.125]}
                        \caesura \edtext{eṇīpadyā}{
  \Afootnote{eṇīpadyās A; eṇīpadā Nep.}
} \edtext{mahādāho}{
  \Afootnote{tathā A.}
} \edtext{daṃśaḥ}{
  \Afootnote{\textsc{[add]} bhavet A.}
} kṛṣṇatilākṛtiḥ |
                        \caesura
 tṛṣṇāmūrcchājvaracchardiḥ \edtext{śvāsakāsasamanvitaḥ\edlabel{SS.5.8.125-6}}{
  \Afootnote{°raśccharddiḥ | H.}
} |\edlabel{SS.5.8.125-7} 
\pend

 
\pstart

                         \textsc{[1938 ed. 5.8.125.add-1]}
                        \caesura \edtext{kṛṣṇayā}{
  \Afootnote{\uwave{kṛ}° K; kṛṣṇa\uwave{yā} H.}
} \edtext{kṛṣṇaparyanto\edlabel{SS.5.8.125.add-1-2}}{
  \Afootnote{kṛṣṇapa° K.}
} \edtext{nimnamadhyo}{
  \linenum{|\xlineref{SS.5.8.125.add-1-2}}\lemma{kṛṣṇaparyanto nimnamadhyo}\Afootnote{hṛṣṇaparyanto\textsc{(l. 1)}nim° H.}
} 'ticoṣavān | \caesura pāṇḍumūrcchāvamīdāhaḥ śvāsakāsasamanvitaḥ\edlabel{SS.5.8.125.add-1-7} \edtext{|}{
  \linenum{|\xlineref{SS.5.8.125.add-1-7}}\lemma{śvāsakāsasamanvitaḥ |}\Afootnote{śvasa° H.}
}
\pend

 
\pstart

                         \textsc{[1938 ed. 5.8.125.add-2]}
                        \caesura daṃśo 'gnimukhyo vijñeyo dagdhaḥ sphoṭāḥ savedanaḥ | \caesura coṣakaṇḍūromaharṣo dāhajvaranipīḍataḥ\edlabel{SS.5.8.125.add-2-9} \edtext{|}{
  \linenum{|\xlineref{SS.5.8.125.add-2-9}}\lemma{dāhajvaranipīḍataḥ |}\Afootnote{°rasamanvitaḥ |\textsc{(l. 2)} H.}
}
\pend

 
\pstart

                         \textsc{[1938 ed. 5.8.126]}
                        \caesura daṃśaḥ kākāṇḍikādaṣṭe pāṇḍurakto 'tivedanaḥ | \caesura \edtext{hikkākāsas\edlabel{SS.5.8.126-6}}{
  \Afootnote{\textsc{[om]} A.}
} tṛṣāmūrcchānidrāhṛdrogapīḍitaḥ\edlabel{SS.5.8.126-7} \edtext{|}{
  \linenum{|\xlineref{SS.5.8.126-6}}\lemma{hikkākāsas\ldots |}\Afootnote{hikvākāsatṛṣāmūrcchā ni° H.}
  \linenum{|\xlineref{SS.5.8.126-7}}\lemma{tṛṣāmūrcchānidrāhṛdrogapīḍitaḥ |}\Afootnote{tṛṇmūrcchāśvāsahṛdrogahikkākāsāḥ syur ucchritāḥ || A.}
}
\pend

 
\pstart

                         \textsc{[1938 ed. 5.8.127]}
                        \caesura rakto \edtext{daṃśo}{
  \Afootnote{mālāguṇādaṃśo A.}
} \edtext{dhūmagandhir}{
  \Afootnote{dhūmagandho A; madhūgandhir K.}
} mālāguṇyātivedanaḥ\edlabel{SS.5.8.127-4} \edtext{| 
                        \caesura
}{
  \linenum{|\xlineref{SS.5.8.127-4}}\lemma{mālāguṇyātivedanaḥ |  }\Afootnote{\textsc{[om]} mālāguṇyā° A; °ṇyādivedanaḥ ||  H.}
} vidīryate\edlabel{SS.5.8.127-6} \edtext{ca}{
  \linenum{|\xlineref{SS.5.8.127-6}}\lemma{vidīryate ca}\Afootnote{\textsc{[om]} A.}
} \edtext{bahudhā}{
  \Afootnote{\textsc{[add]} ca viśīryeta A.}
} dāhamūrcchājvarānvitaḥ | 
\pend

 
\pstart

                         \textsc{[1938 ed. 5.8.128]}
                        \caesura \edtext{asādhyānām}{
  \Afootnote{asādhyāsv A.}
} api \edtext{bhiṣak}{
  \Afootnote{abhihitaṃ A; bhiṣaṣ H.}
} prayuñjīta\edlabel{SS.5.8.128-4} cikitsitam \edtext{|| \caesura}{
  \linenum{|\xlineref{SS.5.8.128-4}}\lemma{prayuñjīta\ldots || }\Afootnote{pratyākhyāyāśu yojayet |  A.}
} \edtext{doṣocchrāyaviśeṣeṇa}{
  \Afootnote{doṣocchrāyo vi° K.}
} cchedakarmavivarjitam\edlabel{SS.5.8.128-8} \edtext{|}{
  \linenum{|\xlineref{SS.5.8.128-8}}\lemma{cchedakarmavivarjitam |}\Afootnote{dāhacchedavi° A.}
}
\pend

 
\pstart

                         \textsc{[1938 ed. 5.8.129]}
                        \caesura sādhyābhir \edtext{atha}{
  \Afootnote{ābhir A.}
} lūtābhir \edtext{daṣṭamātrasya}{
  \Afootnote{ddṣṭa\textsc{(l. 4)}mā° H.}
} dehinaḥ | \caesura vṛddhipatreṇa \edtext{matimāṃ}{
  \Afootnote{matimān A H.}
} samyag ādaṃśam uddharet | 
\pend

 
\pstart

                         \textsc{[SS.5.8.129.add-1]}
                        \caesura jamboṣṭhena sutaptena dahed ākaravāraṇāt | 
\pend

 
\pstart
\edtext{}{
  \Afootnote{\textsc{[pre]} amarmaṇi vidhānajño varjitasya jvarādibhiḥ |  daṃśasyotkartanaṃ kuyād alpaśvayathukasya ca || A.}
}
\pend

 
\pstart

                         \textsc{[1938 ed. 5.8.131ab]}
                        \caesura madhusaindhavasaṃyuktair \edtext{agadair}{
  \Afootnote{agardai K.}
} lepayet tataḥ |
 
\pend

 
\pstart
\edtext{}{
  \Afootnote{\textsc{[pre]} priyaṅgurajanīkuṣṭhasamaṅgāmadhukais tathā || A.}
}
\pend

 
\pstart
\edtext{}{
  \Afootnote{\textsc{[pre]} sārivāṃ madhukaṃ drākṣāṃ payasyāṃ kṣīramoraṭam |  vidārīgokṣurakṣaudramadhukaṃ pāyayeta vā || A.}
}
\pend

 
\pstart

                         \textsc{[1938 ed. 5.8.133ab]}
                        \caesura kṣīriṇāṃ tvakkaṣāyeṇa kusumbhamadhusaindhavaiḥ\edlabel{SS.5.8.133ab-3} \edtext{|}{
  \linenum{|\xlineref{SS.5.8.133ab-3}}\lemma{kusumbhamadhusaindhavaiḥ |}\Afootnote{suśītena A.}
\lemma{|}  \Afootnote{\textsc{[add]} ca secayet | A.}
}
\pend

 
\pstart
\edtext{}{
  \Afootnote{\textsc{[pre]} upadravān yathādoṣaṃ viṣaghnair eva sādhayet || A.}
}
\pend

 
\pstart
\edtext{}{
  \Afootnote{\textsc{[pre]} nasyāñjanābhyañjanapānadhūmaṃ tathā 'vapīḍaṃ kavalagrahaṃ ca |  saṃśodhanaṃ cobhayataḥ pragāḍhaṃ kuryātsirāmokṣaṇam eva cātra || A.}
}
\pend

 
\pstart
\edtext{}{
  \Afootnote{\textsc{[pre]} kīṭadaṣṭavraṇān sarvānahidaṣṭavraṇān api|  ādāhapākāttān sarvāñcikitsedduṣṭavadbhiṣag || A.}
}
\pend

 
\pstart
\edtext{}{
  \Afootnote{\textsc{[pre]} vinivṛtte tataḥ śophe karṇikāpātanaṃ hitam |  nimbapatraṃ trivṛddantī kusumbhaṃ kusumaṃ madhu || A.}
}
\pend

 
\pstart

                         \textsc{[1938 ed. 5.8.137]}
                        \caesura \edtext{kiṇva}{
  \Afootnote{\textsc{[om]} A.}
} \edtext{guggula}{
  \Afootnote{gugguluḥ saindhavaṃ A; guggulu H.}
} \edtext{godantapārāvatamalair}{
  \Afootnote{kiṇvaṃ varcaḥ pārāvatasya A.}
} api\edlabel{SS.5.8.137-4} \edtext{| \caesura}{
  \linenum{|\xlineref{SS.5.8.137-4}}\lemma{api | }\Afootnote{ca |  A.}
} viṣavṛddhikarañ cānnaṃ hitvā \edtext{saṃbhojanaṃ}{
  \Afootnote{sabho° H.}
} hitam | 
\pend

 
\pstart

                         \textsc{[1938 ed. 5.8.138]}
                        \caesura \edtext{viṣebhyaḥ}{
  \Afootnote{visebhyaḥ H.}
} khalu sarvebhyaḥ karṇṇikām arujāṃ sthirām | \caesura pracchayitvā \edtext{madhuyutaiḥ}{
  \Afootnote{madhūnmiśraiḥ A.}
} śodhanīyair upācaret | 
\pend

 
\pstart

                         \textsc{[1938 ed. 5.8.138 add]}
                        \caesura sadāhapākāny annena cikitsed \edtext{dṛṣṭavān}{
  \Afootnote{dṛṣṭavā K.}
} bhiṣak | 
\pend

 
\pstart

                         \textsc{[1938 ed. 5.8.139]}
                        \caesura saptaṣaṣṭhasya kīṭānāṃ śatasyaitad vibhāgaśaḥ | \caesura daṣṭalakṣaṇam ākhyātaṃ \edtext{cikitsāñ}{
  \Afootnote{cikitsā A.}
} cāpy ataḥ\edlabel{SS.5.8.139-10} param \edtext{|}{
  \linenum{|\xlineref{SS.5.8.139-10}}\lemma{ataḥ\ldots |}\Afootnote{anantaram \uline{A} \uline{H}.}
}
\pend

 
\pstart

                         \textsc{[1938 ed. 5.8.140]}
                        \caesura saviṃśam adhyāyaśatam etad uktaṃ vibhāgaśaḥ | \caesura ihoddiṣṭān anirdiṣṭān \edtext{arthān}{
  \Afootnote{arthā K.}
} vakṣyāmi cottare\edlabel{SS.5.8.140-11} \edtext{||}{
  \linenum{|\xlineref{SS.5.8.140-11}}\lemma{cottare ||}\Afootnote{athottare || A.}
}
\pend

 
\pstart

                         \textsc{[1938 ed. 5.8.140 add 1]}
                        \caesura śāstraṃ śāstrasamutpattiṃ\edlabel{SS.5.8.140.add-1-2} \edtext{vyādhikāryabālābalaṃ \caesura}{
  \linenum{|\xlineref{SS.5.8.140.add-1-2}}\lemma{śāstrasamutpattiṃ vyādhikāryabālābalaṃ }\Afootnote{°tpattivyādhikāryavalāvalaṃ ||  H.}
} sūtrabhūtaṃ samāsena ślokasthānaṃ pracakṣate\edlabel{SS.5.8.140.add-1-7} \edtext{||}{
  \linenum{|\xlineref{SS.5.8.140.add-1-7}}\lemma{pracakṣate ||}\Afootnote{pravakṣyate | H.}
}
\pend

 
\pstart

                         \textsc{[1938 ed. 5.8.140 add 2]}
                        \caesura doṣāhārāpacāraiś ca sāgantuvyādhilakṣaṇam\edlabel{SS.5.8.140.add-2-3} \edtext{| \caesura}{
  \linenum{|\xlineref{SS.5.8.140.add-2-3}}\lemma{sāgantuvyādhilakṣaṇam | }\Afootnote{\uwave{sāgantu}vyā° H.}
} avasthālakṣaṇañ caiva nidānaṃ sthānam ucyate || 
\pend

 
\pstart

                         \textsc{[1938 ed. 5.8.140 add 3]}
                        \caesura sambhavaś caiva dehasya dhāturindriyamarmasu |
                        \caesura
 sirādīnāñ ca sarveṣāṃ śārīre kathitam mayā | 
\pend

 
\pstart

                         \textsc{[1938 ed. 5.8.140 add 4]}
                        \caesura \edtext{yathāsthānopadiṣṭānāṃ}{
  \Afootnote{°pa\uwave{} \uwave{}diṣṭānām H.}
} viditānāñ\edlabel{SS.5.8.140.add-4-2} ca lakṣaṇaiḥ |
                        \caesura
 \edtext{vyādhīnāṃ}{
  \linenum{|\xlineref{SS.5.8.140.add-4-2}}\lemma{viditānāñ\ldots vyādhīnāṃ}\Afootnote{vi\textsc{(l. 4)}ditā\textsc{(gap)}kṣaṇaiḥ ||  H.}
} \edtext{sādhanaṃ}{
  \Afootnote{(From 157r)\textsc{(l. 1)}vyādhīnā K.}
} śāstre cikitsitam iti smṛtam ||\edlabel{SS.5.8.140.add-4-11} 
\pend

 
\pstart

                         \textsc{[1938 ed. 5.8.140 add 5]}
                        \caesura sthāvare jaṅgame caiva \edtext{viṣe}{
  \Afootnote{viśe K.}
} hitavikalpanam | \caesura sādhanaṃ caiva \edtext{kārtsnye}{
  \Afootnote{kārtsnyena H.}
} kalpasthānaṃ tad ucyate || 
\pend

 
\pstart

                         \textsc{[1938 ed. 5.8.140 add 6]}
                        \caesura sāhasro vistaraḥ pūrvaṃ prajāpatimukhodbhavaḥ | \caesura \edtext{saviṃśadadhyāyaśataṃ}{
  \Afootnote{saviṅśamadhyā° H.}
} mayā \edtext{vatsa}{
  \Afootnote{va\textbf{\uwave{tsa}} K.}
} prakīrttitam || 
\pend

 
\pstart

                         \textsc{[1938 ed. 5.8.141]}
                        \caesura sanātanatvāt vedānām akṣaratvāt tathaiva ca | \caesura \edtext{dṛṣṭādṛṣṭaphalatvāc}{
  \Afootnote{tathā dṛ° A.}
} ca hitatvāc \edtext{cāpi}{
  \Afootnote{api A.}
} dehinām | 
\pend

 
\pstart

                         \textsc{[1938 ed. 5.8.142]}
                        \caesura vāksamūhārthavistārāt pūjitatvāc ca dehiṣu\edlabel{SS.5.8.142-4} \edtext{| \caesura}{
  \linenum{|\xlineref{SS.5.8.142-4}}\lemma{dehiṣu | }\Afootnote{dehibhiḥ |  A; dehinaḥ |  H.}
} cikitsitāt \edtext{puṇyatamaṃ}{
  \Afootnote{pūṇya° H.}
} na kiñcid\edlabel{SS.5.8.142-9} \edtext{api}{
  \linenum{|\xlineref{SS.5.8.142-9}}\lemma{kiñcid api}\Afootnote{kiñcidapi K.}
} suśruta\edlabel{SS.5.8.142-11} \edtext{|}{
  \linenum{|\xlineref{SS.5.8.142-11}}\lemma{suśruta |}\Afootnote{śuśrumaḥ || A H; suśrutaḥ | K.}
}
\pend

 
\pstart

                         \textsc{[1938 ed. 5.8.143]}
                        \caesura ṛṣer indraprabhāvasya\edlabel{SS.5.8.143-2} tasmād amṛtajanmanaḥ \edtext{| \caesura}{
  \linenum{|\xlineref{SS.5.8.143-2}}\lemma{indraprabhāvasya\ldots | }\Afootnote{°vasyāmṛtayoner A.}
\lemma{| }  \Afootnote{\textsc{[add]} bhiṣag guroḥ |  A.}
} \edtext{dhārayitvedam}{
  \Afootnote{dhārayitvā tu A.}
} \edtext{amalaṃ}{
  \Afootnote{vimalaṃ A.}
} matam paramasammatam | \caesura \edtext{uktācārasamācāraḥ}{
  \Afootnote{uktāhārasamācāra iha A; °cārāḥ H.}
} \edtext{pretya}{
  \Afootnote{pratya H.}
} \edtext{ceha}{
  \Afootnote{\textsc{[om]} A; veha H.}
} ca nandati\edlabel{SS.5.8.143-15} \edtext{|}{
  \linenum{|\xlineref{SS.5.8.143-15}}\lemma{nandati |}\Afootnote{modate || A.}
}
\pend

 
\pstart

                         \textsc{[1938 ed. 5.8.143 add 1]}
                        \caesura śeṣāṇām api tantrāṇāṃ yuktijño lokabāndhavaḥ | 
\pend

 
\pstart

                         \textsc{[1938 ed. 5.8.143 add 2]}
                        \caesura yat kiṃcid ābādhakaran tad yasmāc chalyasaṃjñitaṃ | \caesura vyāptāny aṅgāny 
                            atas
                            ato
                         tena śalyajñānena 
                            sūriṇā
                            bhūriṇā
                         |
 
\pend

 
\pstart

                         \textsc{[1938 ed. 5.8.143 add 3]}
                        \caesura ataś cāsya viśeṣeṇa gatir na pratiṣidhyate | \caesura yathā svaviṣayasthasya rājño balavato gatiḥ |
 
\pend

 
\pstart

                         \textsc{[1938 ed. 5.8.143 add 4]}
                        \caesura \edtext{upadravāṇāṃ}{
  \Afootnote{°vānāṃ K.}
} nirdeśo nidānaṃ \edtext{vyañjanāni}{
  \Afootnote{vyaja° H.}
} ca | \caesura jvarādīnāñ cikitsārtham uttaran tantram ucyate || 
\pend

 
\pstart

                         \textsc{[1938 ed. 5.8.143 add 5]}
                        \caesura \edtext{bhavati}{
  \Afootnote{bhavanti H.}
} cātra || \caesura idan tu yaḥ pañcasu sanniveśitam | \caesura saviṃśadadhyāyaśataṃ sahottaram | \caesura paṭhet sa rājño 'rhati vaidyapūjitaḥ | \caesura kriyām prayoktuṃ \edtext{bhiṣag 
                            āgatakramaḥ
                            āgamajñaḥ
                        }{
  \Afootnote{bhiṣa\uuline{bhā} \textbf{g} H.}
} ||\edlabel{SS.5.8.143.add-5-22}
 
\pend

 
\pstart

                         \textsc{[1938 ed. 5.8.143 add 6]}
                        \caesura annarakṣā sthāvaraviṣaṃ jaṅgamañ ca viṣaṃ tathā | \caesura \edtext{sarpadaṣṭaviṣajñānaṃ}{
  \Afootnote{°viṣājñanaṃ K.}
} sarpadaṣṭacikitsitam || \caesura mūṣikā dundubhiś caiva kīṭākalpena cāṣṭamaḥ ||
 
\pend

 
\pstart
\edtext{
                         \textsc{[Chapter colophon]}
                        \caesura}{
  \Afootnote{\textsc{[pre]} iti A.}
} sauśrute\edlabel{SS.5.8.trailer-1} \edtext{śalyatantre}{
  \linenum{|\xlineref{SS.5.8.trailer-1}}\lemma{sauśrute śalyatantre}\Afootnote{suśrutasaṃhitāyāṃ A.}
} \edtext{kalpasthānaṃ}{
  \Afootnote{kalpasthāne kīṭakalpo A.}
} samāptam\edlabel{SS.5.8.trailer-4} || ❈ \edtext{||}{
  \linenum{|\xlineref{SS.5.8.trailer-4}}\lemma{samāptam\ldots ||}\Afootnote{nāmāṣṭamo A; samāptaḥ || ❈ || K.}
\lemma{||}  \Afootnote{\textsc{[add]} 'dhyāyaḥ ||8 || A.}
}
\pend

 

\endnumbering
\endgroup
\end{document}