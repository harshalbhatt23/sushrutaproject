% !TeX root = incremental_SS_Translation.tex

\chapter{Śārīrasthāna 1:  On the consideration of all beings}



\section{Literature} 

Meulenbeld offered an annotated overview of this chapter and a
bibliography of earlier scholarship to
2002 and, in his notes, citations of the parallel passages in the 
\CS.\fvolcite{IA}[243]{meul-hist}   The short account of Sāṅkhya philosophy 
offered in this chapter of the \SS\ is several times characterized as being special 
to physicians.\footnote{3.1.11 \dev{vaidyake tu} ``but in medicine\ldots''; 
3.1.13 \dev{cikitsite} ``in medicine''; 3.1.16 \dev{āyurvedaśāstreṣu} ``in the 
treatises about medicine\ldots ''; 3.1.16 \dev{sa eṣa 
karmapuruṣaścikitsādhikṛtaḥ} ``it is this agentic person that medicine is 
concerned with.''} Recent overviews of the classical theory include those of 
\textcites{ruzs-2025}[ch.\,22]{adam-2022}[\S2.4]{chat-2021}.
\citet{comb-2011} studied the specifically Ayurvedic interpretations of Indian 
philosophical ideas.
%\citet[chs 6--8]{das-2003} also
%studied topics of this chapter and in chapter 13 provided an overview of
%the conceptual background of ayurveda on the topics discussed in this
%chapter.  

\section{Translation}

\begin{translation}
    
    \item [1] 
    
    So, now we shall explain anatomy chapter that is a reflection about all 
    beings.\footnote{The Nepalese version has nouns in apposition 
    (“-\dev{cintā} $\leftrightarrow$ \dev{śārīram}”).  The vulgate makes this a 
    single 
    karmadhāraya compound that is slightly easier to parse.} 
    
\item[3]

%The cause of all beings, called “the unmanifest,” is without a cause,
%is characterised by sattva, rajas and tamas,  has eight forms and
%is the reason for the appearance  of this whole world.

That which is called “the unmanifest” is the causeless cause of all living beings,  
having the characteristics of sattva, rajas and tamas, having eight forms, and 
being the reason for the appearance of this whole world.
    
It is the single basis of the many \se{kṣetrajña}{knowers of the
    field}, just as the ocean is to the beings whose \se{ojas}{power} is
water.\footnote{The Nepalese version differs from the vulgate here,
    introducing the word “power.”  This substantially changes the sense of
    the passage.  If the Nepalese witnesses did not agree, it would be
    tempting to emend the text to \dev{udakaja-} (creatures) ``born of
    water.''}
    
\item[3.1.4]

From that unmanifest, the Mahat arises, having exactly the same
properties. \footnote{In classical Sāṅkhya theory, \dev{mahat} is a
    synonym for \dev{buddhi}, ``intellect.''  In the present passage, this
    identity is not explicit; rather, it is a cosmological entity.  In the
    cosmology of the \emph{Pātañjalayogaśāstra}, it is pure being, 
    \dev{sattāmātra} 2.19 \citep[85]{agas-1904}, it is
    also sometimes designated as the great \dev{ātman} ``great self'' in
    the sense of a universal being.} %
    From that Mahat, which has those same properties, arises the
    Ahaṅkāra, having exactly the same characteristics.\footnote{The
        Ahaṅkāra ``I-maker'' is the principle of individual identity.}  It
        has three aspects: \se{vaikārika}{mutable}, \se{taijasa}{fiery}
        and \se{bhūtādi}{elemental}.\footnote{These technical terms occur
            in the \emph{Sāṅkhakārikā} 22 as synonyms for Ahaṅkāra
            \parencites[46--47]{sast-1948}[187--188]{wezl-1998}. They also
            occur in the Purāṇic cosmogonies.}
            
            From that mutable Ahaṅkāra the 
            eleven \se{indriya}{faculties} arise, with the very same
            characteristics. It is as follows: ear, skin, eye, tongue, nose,
            speech, hand, genitals, anus, feet and mind.  Amongst these, the first 
            five are the faculties of \se{buddhi}{cognition}; the next five are the 
            faculties of \se{karma}{action}.  The mind has properties of both.
            
            From the Ahaṅkāra as \se{bhūtādi}{starting point for the
    elements}, arise the five \se{tanmātra}{bare entities},
with exactly the same characteristics.\footnote{Earlier,
    the Ahaṅkāra was said to have three aspects, so we would
    here expect a description of the \se{taijasa}{fiery}
    aspect.  But the Nepalese version goes straight to the
    \se{bhūtādi}{elemental} aspect.  The vulgate text inserts
    the fiery aspect alongside the elemental as if it were
    similar in all respects (\dev{taijasasahāya}).}  It is as
    follows: bare sound, bare touch, bare form, bare taste,
    bare smell.\footnote{Or, ``the essence of sound,'' etc.}
    
    From these \se{bhūta}{elements} come space, air, fire, water and earth; 
    from these come sound, touch, form, taste and smell, with the same 
    distinctions.  In this way these twenty-four \sepl{tattva}{principle} have 
    been explained. 
            
    
\end{translation}
